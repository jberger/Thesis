%This work is licensed under the Creative Commons Attribution-NonCommercial-NoDerivs 3.0 United States License. To view a copy of this license, visit http://creativecommons.org/licenses/by-nc-nd/3.0/us/ or send a letter to Creative Commons, 444 Castro Street, Suite 900, Mountain View, California, 94041, USA.

\section{Model of Michalik and Sipe}

This section introduces the Analytic Gaussian model of Michalik and Sipe \cite{michalik_analytic_2006}.

\subsection{Formal Evolution} \label{sec:formal_evolution}

For a classical set of $N$ particles (labelled by the index $i$), distributed 6D space (3 position, 3 momentum) by 
\begin{gather}
  f_{\smallD}(\vec{r}, \vec{p}; t) = 
  \sum\limits^{N}_{i=1} \delta(\vec{r}_{i} - \vec{r}) \delta(\vec{p}_{i} - \vec{p}) \\
  \text{which satisfies} \quad \int d^{3}x\,d^{3}p \; f_{\smallD}(\vec{r}, \vec{p}; t) = N
\end{gather}
of mass $m$ which interact with some potential $V(\vec{r}_{i} - \vec{r}_{j})$, their motion is governed by the equations
\begin{subequations}\label{eq:std_motion}
\begin{gather}
  \frac{d\vec{r}_{i}(t)}{dt} = \frac{\vec{p}_{i}(t)}{m}\\
  \frac{d\vec{p}_{i}(t)}{dt} = -\sum\limits_{j} \frac{d}{d\vec{r}_{i}} V(\vec{r}_{i} - \vec{r}_{j})
\end{gather}
\end{subequations}
Now consider the time evolution of $f_{\smallD}$
\begin{equation}
  \frac{\partial f_{\smallD}}{\partial t} = 
  \frac{d\vec{r}}{dt} \frac{\partial f_{\smallD}}{\partial \vec{r}} 
  + \frac{\partial f_{\smallD}}{\partial \vec{p}} \frac{d\vec{p}}{dt}
\end{equation}
but with \ref{eq:std_motion} the final discreet evolution equation is
%TODO find missing minus sign!
\begin{equation}
  \frac{d}{dt} f_{\smallD}(\vec{r}, \vec{p}; t) =
  -\frac{\vec{p}}{m} \frac{\partial}{\partial \vec{r}} f_{\smallD}(\vec{r}, \vec{p}; t)
  + \frac{\partial}{\partial \vec{p}} f_{\smallD} (\vec{r}, \vec{p}; t)
  \frac{\partial}{\partial \vec{r}} \iint d\vecprime{r} d\vecprime{p} V(\vec{r} - \vecprime{r}) f_{\smallD}(\vecprime{r},\vecprime{p};t)
\end{equation}
%TODO ensemble av and mean-field
Final equation:
\begin{equation} \label{eq:formal_evolution}
  \frac{d}{dt} f(\vec{r}, \vec{p}; t) =
  -\frac{\vec{p}}{m} \frac{\partial}{\partial \vec{r}} f(\vec{r}, \vec{p}; t)
  + \frac{\partial}{\partial \vec{p}} f (\vec{r}, \vec{p}; t)
  \frac{\partial}{\partial \vec{r}} \iint d\vecprime{r} d\vecprime{p} V(\vec{r} - \vecprime{r}) f(\vecprime{r},\vecprime{p};t)
\end{equation}

\subsection{Evolution of a Gaussian Distribution}

The formal evolution explored in \ref{sec:formal_evolution} allows for any distribution function.
For Ultrafast Electron Microscopy, the electron pulse is generated by a laser pulse, therefore the electron pulse can be thought to be a Gaussian, mimicing the laser pulse for which generated it.
How correct this assumption is will be explored in \ref{sec:initial_conditions}.
In this section we consider a Gaussian distribution function, generally of the form
\begin{equation}
  f(\vec{r}, \vec{p}; t) = C(t)\exp \left[ - \Gamma(\vec{r}, \vec{p}; t) \right]
\end{equation}
where in this case we choose the argument (i.e. the function $\Gamma$) as
\begin{equation}
  \Gamma(\vec{r}, \vec{p}; t) =
  \frac{x^2 + y^2}{2 \sigma_{\smallT}} + \frac{z^2}{2 \sigma_{z}}
  + \frac{
    [p_x - (\gamma_{\smallT}/\sigma_{\smallT}) x ]^2 
    + [p_y - (\gamma_{\smallT}/\sigma_{\smallT}) y ]^2
  }{2 \eta_{\smallT}}
  + \frac{ [p_z - (\gamma_{z}/\sigma_{z}) z ]^2 }{2 \eta_{z}}
\end{equation}
This parameterization defines the spatial uncertainty $\sigma$, the local momentum uncertainty $\eta$, and the momentum chirp $\gamma$ in each spatial direction.
The ``chirp'' of a pulse quantifies the difference in local momentum from one side of the pulse to the other.
For example, in a pulse which is normally (positively) chirped longitudinally, the front of the pulse with be moving faster than the back.

The normalization constant, is found by integrating the entire distribution and knowing that this will contain all $N$ electrons
\begin{equation}
  \int f(\vec{r}, \vec{p}; t) d\vec{r} d\vec{p} = N
\end{equation}
is
\begin{equation}
  C(t) = \frac{N}{(2\pi)^3} 
  \left( 
    \frac{1}{\sigma_{\smallT}^2\eta_{\smallT}^2\sigma_{z}\eta_{z}}
  \right)^{\frac{1}{2}}
\end{equation}
These equations may be viewed from a matrix perspective if we define the ``coordinate vector''
\begin{equation} \label{eq:coordinate_vector}
u_i = \{x, p_x, y, p_y, z, p_z\}\text{ .}
\end{equation}
In this space, \ref{eq:formal_evolution} becomes
\begin{equation}
  \Gamma(\vec{r}, \vec{p}; t) = \frac{1}{2}\sum\limits_{ij} A_{ij} u_i u_j
\end{equation}
where
\begin{equation}
  A = 
  \begin{pmatrix}
    a_{\smallT} & 0 & 0 \\
    0 & a_{\smallT} & 0 \\
    0 & 0 & a_{z}
  \end{pmatrix}
  \qquad \text{and} \qquad
  a_{\alpha} = 
  \begin{pmatrix}
    1/\sigma_{\alpha} + \gamma_{\alpha}^2/(\eta_{\alpha} \sigma_{\alpha}) & \gamma_{\alpha}/(\eta_{\alpha} \sigma_{\alpha}) \\
    \gamma_{\alpha}/(\eta_{\alpha} \sigma_{\alpha}) & 1/\eta_{\alpha}
  \end{pmatrix}
\end{equation}
with ${\scriptstyle \alpha} = \smallT \text{ or } z$. This form allows the isolation of any of the six parameters using
\begin{equation}
  (A^{-1})_{ij} = \frac{1}{N} \int f(\vec{r}, \vec{p}; t) d\vec{r} d\vec{p}
\end{equation}
which yields
\begin{equation}
  A = 
  \begin{pmatrix}
    a_{\smallT}^{-1} & 0 & 0 \\
    0 & a_{\smallT}^{-1} & 0 \\
    0 & 0 & a_{z}^{-1}
  \end{pmatrix}
  \qquad \text{and} \qquad
  a^{-1}_{\alpha} = 
  \begin{pmatrix}
    \sigma_{\alpha} & \gamma_{\alpha} \\
    \gamma_{\alpha} & \eta_{\alpha} + \gamma_{\alpha}^2/\sigma_{\alpha}
  \end{pmatrix}
\end{equation}
Taking the time derivative of the components of $A^{-1}$
\begin{equation} \label{eq:dainvdt}
  \frac{d}{dt} a^{-1}_{\alpha} = 
  \begin{pmatrix}
    \dfrac{d\sigma_{\alpha}}{dt} & \dfrac{d\gamma_{\alpha}}{dt} \\
    \dfrac{d\gamma_{\alpha}}{dt} & \dfrac{d\eta_{\alpha}}{dt} + 2\dfrac{\gamma_{\alpha}}{\sigma_{\alpha}}\dfrac{d\gamma_{\alpha}}{dt}- \dfrac{\gamma^{2}_{\alpha}}{\sigma^{2}_{\alpha}}\dfrac{d\sigma_{\alpha}}{dt}
  \end{pmatrix}
\end{equation}
These components, when viewed in the context of \ref{eq:formal_evolution}, form the left hand side (LHS) of the differential equations governing the evolution of $\sigma_{\alpha}$, $\gamma_{\alpha}$, and $\eta_{\alpha}$, which will be established in the upcoming sections.

