The development of laboratory-scale instrumentation for the direct visualization of fundamental ultrafast (i.e., sub-nanosecond) events with nano-scale (even atomic) spatial resolution is a current active area of research~\cite{king_ultrafast_2005}.
In many areas of physics, chemistry and materials science a space-time resolution in the sub-1nm$\cdot$ps range is ideally required to allow for the observation of the fastest structural events.
A promising approach is to employ ultrashort pulses of electrons generated by a laser-driven 10-100keV photoelectron gun to probe the dynamics of the specimen that has been perturbed by a synchronized laser pulse~\cite{king_ultrafast_2005}.
Ultrafast electron diffraction (UED) or femtosecond electron diffraction (FED)~\cite{srinivasan_ultrafast_2003,williamson_clocking_1997,cao_femtosecond_2003} is an example of this type of technique and has been used successfully to elucidate atomic-scale dynamics in melting~\cite{cao_femtosecond_2003,sciaini_electronic_2009,siwick_atomic-level_2003}, chemical bond breaking~\cite{zewail_laser_1988}, etc.
Direct time-resolved imaging of similar events can be accomplished using transmission electron microscopy (TEM).
To date, such dynamic transmission electron microscopes (DTEMs)~\cite{bostanjoglo_tracing_1980,domer_high-speed_2003} have been developed for both single-shot imaging with $\sim$$10^{5}$nm$\cdot$ps space-time resolution (e.g., 30ns electron pulses and 5-10nm spatial resolution~\cite{lagrange_single-shot_2006,armstrong_practical_2007}) and $\sim$1nm$\cdot$ps space-time resolution ($\sim$100fs with a few nanometer resolution~\cite{park_direct_2009}) in the multi-shot data acquisition mode of operation.
Both of these instruments employ retro-fitted standard electron microscope columns and are limited in performance mainly by space-charge effects.

Improvements in the performance of DTEMs, especially ultrafast electron microscopes (UEMs), and UED systems will likely require the implementation of techniques that offset or compensate for deleterious space-charge effects~\cite{armstrong_prospects_2007,lobastov_four-dimensional_2005}.
Such performance optimization efforts will require efficient models of electron pulse propagation dynamics that include the ability to simulate the action of electron optical elements (including the photoelectron gun itself) on the pulse.
Indeed $2^{\text{nd}}$ generation UED instruments are now beginning to employ both magnetic lenses and RF cavities~\cite{oudheusden_electron_2007,veisz_hybrid_2007} to optimize the electron pulse delivery to the specimen; that is, compensate for global space-charge effects in both the transverse and longitudinal pulse dimensions, respectively.
Similarly, future UEM column designs are expected to use RF pulse compression cavities while circumventing high charge density beam cross-overs (i.e., beam foci).

In this thesis, I will present an extension of the work of Michalik and Sipe~\cite{michalik_analytic_2006,michalik_erratum:_2008}, describing electron bunch propagation in an mean-field self-similar Analytic Gaussian (AG) pulse treatment, to include the influence of linear external forces due to electron optical elements.
The resulting computationally efficient propagation analysis can then be used to model, and hence design, UEM columns and UED systems in a straightforward manner.
For completeness, in Chapter \ref{chap:ms_model} I will introduce, in MKS units, the essential features of the AG model.
In Chapter \ref{chap:extension}, I present some optimizations to the AG model, and develop useful initial conditions representing single-photon photoemission from a typical metal photocathode.
I then present my extensions to the Michalik and Sipe AG model which permit the inclusion of generic external forces.
In order to adapt this extended AG model for simulating pulses in a realistic UEM column, I then present specific contributions that arise from magnetic lenses, RF cavities and DC accelerators.
I would like to note that the analyses presented are performed in the non-relativistic limit, which is a reasonable approximation for the typical 20-200keV electron energies employed in UED and electron microscopy.

Since the extension to the AG model is valid only within the limits of the analytical method itself, in particular, its mean internal space-charge field and self-similar Gaussian approximations~\cite{michalik_analytic_2006}, the extension reflects a first-order (i.e., linear force) analysis of the effects of electron optics upon electron pulse propagation.
Nonetheless, for free-space propagation, the AG model of charge bunch dynamics has already been shown to be very consistent with full Monte Carlo (i.e., particle tracking) simulations for a wide variety of electron pulse shapes~\cite{michalik_analytic_2006,michalik_evolution_2009}, including the uniform ellipsoid~\cite{luiten_how_2004}.
This successful benchmarking is due primarily to the versatility of the AG model which results from its use of transverse and longitudinal pulse position and momentum variances.
Consequently, the AG approach is applicable to both the single electron per pulse limit~\cite{lobastov_four-dimensional_2005}, where momentum variances determine the pulse evolution and the model is exact (obeying Gaussian optics), and the high charge density limit in which space-charge effects dominate~\cite{luiten_how_2004,siwick_ultrafast_2002,cao_femtosecond_2003}.
It is this versatility combined with its computational efficiency that makes the presented extended AG model particularly suitable for rapid initial assessments of pulsed electron microscope column designs and electron pulse delivery systems in UED experiments.
%Verification of the validity (and determination of the limits) of the extended AG model will, of course, require future comparison with both experiment and more complete simulations of electron pulse propagation dynamics (e.g., full particle tracking models) that include nonlinear forces, for both the intra-pulse space-charge field and the description of aberrations in electron optics.

In Chapter \ref{chap:model_results}, I apply the AG model to draw some initial insight into the behaviors of ultrafast electron pulses in a UEM column.
Then in Chapter \ref{chap:considerations}, I apply some well known results from the field of electron microcopy, namely the Rose criterion~\cite{rose_television_1948}, an ultrafast analog to the Child-Langmuir law~\cite{child_discharge_1911,langmuir_effect_1923,valfells_effects_2002}, and the conservation of transverse emittance~\cite{jensen_emittance_2010} (each of which will be introduced) to arrive at two design goals for building a UEM which will produce high-resolution images.
First, that the electron pulses should be emitted from a large-area source, and second, that every effort must be made to reduce the initial rms transverse momentum (i.e., the emission cone angle).

All of these considerations motivate the construction of the prototype UEM column being built by our group at UIC.
In Chapter \ref{chap:prototype} I detail the design and fabrication of large-aperture electron-optical elements, which can readily admit these large width electron pulses; included among these are the accelerator and magnetic lenses.
The, before I conclude, in \ref{chap:photocathode}, I highlight the photocathode engineering work which has been undertaken in an attempt to reduce the rms transverse momentum.
I will present several different photoemission processes, some of which are yielding very promising results.
In fact, our recent work may even challenge our current understanding of photoemission.

