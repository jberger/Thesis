%This work is licensed under the Creative Commons Attribution-NonCommercial-NoDerivs 3.0 United States License. To view a copy of this license, visit http://creativecommons.org/licenses/by-nc-nd/3.0/us/ or send a letter to Creative Commons, 444 Castro Street, Suite 900, Mountain View, California, 94041, USA.

\subsection{Determining Appropriate Initial Conditions} \label{sec:initial_conditions}
For a DTEM or UED apparatus, where the electron pulse is produced in a laser-driven photo-electron gun, the implementation of the AG model of Michalik and Sipe \cite{michalik_analytic_2006} requires knowledge of the initial ($ t = 0 $) bunch parameters immediately after generation at the photocathode surface inside the acceleration region of the gun.
For the purposes of this paper, we will make a simple approximation to the behavior of the gun in order to provide generic initial conditions for the electron pulse.

For single-photon photoemission process with a spatially cylindrically symmetric incident Gaussian laser beam with Gaussian temporal irradiance profile 
\begin{equation}\label{eq:laser_form}
I ( x , y , t ) = I_{0} \exp \left ( - \frac{ ( x^{2} +y^{2} ) }{ w^{2} } - \frac{ t^{ 2 } }{ \tau^{ 2 } } \right ) \text{ ,}
\end{equation}
where $w$ is the half-width 1/e maximum (HW1/eM) of the spatial irradiance distribution, $\tau$ is the HW1/e pulse duration and $I_{0}$ is the peak pulse irradiance, the electron pulse's initial spatial variance is
\begin{equation}
\sigma_{\smallT } ( 0 ) = ( \Delta x )^{2} = ( \Delta y)^{2} = \frac{ w^{2} }{ 2 } \text{ .}
\end{equation}
For the longitudinal direction, and in the limit where dispersion effects in the acceleration region of the electron gun can be neglected, we may write
\begin{equation} \label{eq:initial_sigma_z}
\sigma_{ z } ( 0 ) = ( \Delta z)^{2} = \frac{ ( v_{{ \scriptscriptstyle 0}} \tau )^{2} }{ 2 } \text{ ,}
\end{equation}
with the velocity of the pulse in the lab frame $v_{{ \scriptscriptstyle 0}} = \sqrt{ 2 e V / m_{e} d } $, where $ V $ and $ d $ are the potential and cathode-anode separation of the gun's DC acceleration region respectively.
For spatially-uniform photoemission from a planar photocathode, we have $ \gamma_{\smallT} ( 0 ) = 0 $; i.e., there is no initial spatial momentum chirp in the electron bunch. Immediately after photo-generation, however, it is strictly inaccurate to say that $ \gamma_{ z } ( 0 ) = 0 $.
This is because the excess photoemission energy $ \Delta E = \hbar \omega - \Phi $, associated with the electron emission from a material with work function $\Phi$ using photons of energy $ \hbar \omega $ (which gives the maximum initial velocity of an electron, $ v_{max} = \sqrt{ 2 \Delta E/m_{e} } $), can produce an initial momentum chirp.
Nonetheless, the initial non-zero nature of $ \gamma_{z} $ is quickly dominated by the evolution of the pulse, so this approximation will suffice in the region far from the gun.

The single-photon photoemission analysis of Jensen \textit{et al.}\cite{jensen_theoretical_2006} indicates that for a metallic photocathode the transverse momentum variance at $ t = 0 $ is given by
\begin{equation} \label{eq:jensen_eta}
( \Delta p_{x} )^{2} = ( \Delta p_{y} )^{2} = \eta_{ \smallT } ( 0 ) \approx \frac{ m_{e} \epsilon_{\smallF} ( \hbar \omega - \Phi ) } { \hbar \omega + \epsilon_{\smallF} } \; \text{,}
\end{equation}
in the limit where the Fermi energy $ \epsilon_{\smallF} $ of the metal photocathode and excess photoemission energy, $ \Delta E = \hbar \omega - \Phi $ are significantly greater that the thermal energy $ k_{B} T $ of the electrons in the photocathode -- a condition that is generally met in laser-driven DC photoelectron guns.
From a simplistic viewpoint, only the electrons generated with positive longitudinal momentum will contribute to the pulse since those electrons generated with negative momentum do not leave the photocathode.
This implies that the uncertainty in the longitudinal direction is half that of the transverse directions; that is to say
\begin{equation}
( \Delta p_{ z } )^{2} = \left ( \frac{\Delta p_{ x , y }}{ 2 } \right )^{2} \text{,}
\end{equation}
so that
\begin{equation}
\eta_{ z } ( 0 ) = \frac{ \eta_{ \smallT } ( 0 )}{ 4 } \; \text{.}
\end{equation}

These initial conditions at $t=0$ for an incident Gaussian laser pulse of the form given in \ref{eq:laser_form}, will be used for the remainder of this paper.
\begin{subequations} \label{eq:summary}
\begin{gather}
\sigma_{ \smallT } ( 0 ) = \frac{ w^{ 2 } }{ 2 } \\
\sigma_{ z } ( 0 ) = \frac{ ( v_{{ \scriptscriptstyle 0}} \tau )^{ 2 } }{ 2 } \\
%alternative formulation + \left ( \frac{ e E_{ DC } \tau^{ 2 } }{ 4 m } \right )^{ 2 } \\
\gamma_{ \smallT } ( 0 ) = 0 \\
\gamma_{ z } ( 0 ) = 0 \\
\eta_{ \smallT } ( 0 ) = \frac{ m_{e} \epsilon_{ F } ( \hbar \omega - \Phi ) }{ 3 ( \hbar \omega + \epsilon_{ \smallF } ) } \\
\eta_{ z } ( 0 ) = \frac{ \eta_{ \smallT } ( 0 ) }{ 4 }
\end{gather}
\end{subequations}

We recognize that these initial conditions for an electron pulse generated by a laser-driven DC gun are idealized in that they neglect several important effects; in particular, the interplay of dispersive ($ \eta_{i}(0) \neq 0 $) and space-charge effects in the acceleration region of the photoelectron gun and transverse and longitudinal effects due to the divergence of the DC field in the vicinity of the gun anode.\cite{berger_dc_2009,togawa_ceb6_2007}
Such effects and the limits of the validity of \ref{eq:initial_sigma_z}, within the context of the AG model, will be presented elsewhere as they require \textit{a priori} this presented extension of the AG model.
However, \ref{eq:summary} also represent a defined set of initial conditions that allows the effect of electron optics (i.e., magnetic lenses and RF cavities) on electron pulse propagation to be more readily understood.
Moreover, they reflect the fact that the majority of DC photoelectron guns are driven by laser pulses with spatial and temporal profiles that are close to Gaussian.\cite{williamson_clocking_1997,sciaini_electronic_2009}
Only for specialized cases, such as the generation of uniform ellipsoid electron bunch shapes, is spatial and temporal manipulation of the incident laser pulse considered.\cite{luiten_how_2004,li_generating_2008}
