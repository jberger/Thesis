%This work is licensed under the Creative Commons Attribution-NonCommercial-NoDerivs 3.0 United States License. To view a copy of this license, visit http://creativecommons.org/licenses/by-nc-nd/3.0/us/ or send a letter to Creative Commons, 444 Castro Street, Suite 900, Mountain View, California, 94041, USA.

\section{Semiconductor Photocathodes}

\subsection{Theoretical Basis}

\subsubsection{Band Structure}

\subsubsection{Population of Upper Conduction Band}

\subsubsection{Absorption into Upper Conduction Band}

In order to quantify the behavior of emission from this upper conduction band it is necessary to calculate the absorption $\alpha_0$ into this band.
The text `Optoelectronics' [cite] has a formula for absorption in terms of the energy gap $E_g$, the effective masses and a few optical properties.
\begin{equation} \label{eq:absorption_optoelectronics}
  \alpha_0(\omega) = \frac{ q^2 x_{vc}^2 (2 m_r)^{3/2} }{ \lambda \varepsilon_0 \hbar^3 n_{op} } \sqrt{\hbar\omega - E_g}
\end{equation}
Here $\omega$ and $\lambda$ are the frequency and wavelength of the incoming photons respectively.
$m_r$ is the reduced mass of the system given by
\begin{equation}
  \frac{1}{m_r} = \frac{1}{m_c} + \frac{1}{m_v}
\end{equation}
in terms of the effective mass of the valence holes ($m_v$) and conduction electrons ($m_c$).
The quantity $x_{vc}^2$ is
\begin{equation}
  x_{vc}^2 = \frac{1}{3} \frac{\hbar^2}{E_g^2} \frac{E_p}{m_0}
\end{equation}
$m_0$ is the standard electron mass. $E_p$ is the Kane energy which is normally calculated from the matrix elements of the band structure.
It can be estimated using [Optoelectronics] (5.C.19)
\begin{equation}
  \frac{m_c}{m_0} = \left( 1 + \frac{E_p}{E_g} \right)^{-1}
\end{equation}
A further approximation is made owing to the difficulty in finding accurate data on these upper conduction bands.
Visually one can recognize that the curvature of the valence band this upper conduction band are comparable, meaning that approximating $ m_c \approx m_v $ is reasonable for order of magnitude calculations. With this approximation, \ref{eq:absorption_optoelectronics} becomes 
\begin{equation}
    \alpha_0(\omega) = \frac{ q^2 x_{vc}^2 m_v^{3/2} }{ \lambda \varepsilon_0 \hbar^3 n_{op} } \sqrt{\hbar\omega - E_g}
\end{equation}

\begin{table} 
  \begin{tabular}{|l|l|c|c|c|}
      \hline
    Material  & UC Band & $m_v$ $(m_0)$ & $E_g$ (eV) & $n_{op}$ \\
      \hline\hline
    \ce{GaSb} & $\Gamma(8c)$ & 0.28 & 3.77 & 1.45 \\
    \ce{InSb} & $\Gamma(8c)$ & 0.45 & 3.45 & 1.40 \\
      \hline
  \end{tabular}
  \caption{Data and results from analysis of the absorption into upper conduction bands of useful III-V semiconductors.}
  \label{table:absorption}
\end{table}
Using data collected from [Springer Ref], \ref{table:absorption} shows the results of these calculations for useful III-V semiconductors.


