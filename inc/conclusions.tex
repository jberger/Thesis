%This work is licensed under the Creative Commons Attribution-NonCommercial-NoDerivs 3.0 United States License. To view a copy of this license, visit http://creativecommons.org/licenses/by-nc-nd/3.0/us/ or send a letter to Creative Commons, 444 Castro Street, Suite 900, Mountain View, California, 94041, USA.

Ultrafast Electron Microscopy is an exciting new field of intrumentation research which promises to open our eyes to the dynamics occurring in nanoscale processes.

During my time at UIC I have worked to advance this technology, I have developed a model and implemented it, and I have  both the theoretical (simulation/modeling) and experimental (prototyping and photocathode engineering) fronts.

\section{Modeling}

In this thesis I have presented the extended Analytic Gaussian (AG) model, an effiecient numerical model based on the work of Michalik and Sipe \cite{michalik_analytic_2006}, but extended so as to be useful in modeling the dynamics of ultrafast electron pulses in microscope columns.
For pulses containing $N$ carriers of charge $e$ and mass $m$, the model is simply these four equations,
\begin{subequations}
\begin{gather}
  \frac{d\sigma_{\alpha}}{dt} = \frac{2\gamma_{\alpha}}{m} \\
  \frac{d\gamma_{\alpha}}{dt} = \frac{ \Gamma_{\alpha}^2 + \gamma_{\alpha}^2 }{\sigma_{\alpha} m}
    + \frac{N e^2}{4\pi\varepsilon_0} \frac{1}{6 \sqrt{\sigma_{\alpha}\pi}} L_{\alpha}(\xi) + \sigma_{\alpha} \sum\limits_{f} M_{\alpha}^{\lbrace f \rbrace} \label{eq:ag_model_gamma_conc} \,\text{.}
\end{gather}
\end{subequations}
where the subscript $\alpha$ denotes the directionality of the dynamics, either $\smallT$ for the transverse or $z$ for the longitudinal.
At any time $t$, the $\sigma_{\alpha}$ quantities characterize the spatial variance of the pulse and $\gamma_{\alpha}$ represent the momentum chirp across the pulse.
A third parameter, the local momentum variance $\eta_{\alpha}$, is also encoded in these equations.
By Liouville's theorem, which the AG model explicitly satisfies, we know that the product of the spatial variance $\sigma_{\alpha}$ and the local momentum variance $\eta_{\alpha}$ is constant throughout propagation (for perfect lensing and without clipping the puse), so we define the constant,
\begin{equation} \label{eq:define_big_gamma_conc}
  \Gamma_{\alpha}^2 \equiv \sigma_{\alpha} \eta_{\alpha} \,\text{.}
\end{equation}
By evaluating this constant once, presumably from initial conditions, and later applying \ref{eq:define_big_gamma_conc} we can get the extract these $\eta_{\alpha}$ parameters while only needing to compute the other four.
The parameter $\xi$ characterizes the ellipticity of the pulse,
\begin{equation}
  \xi \equiv \sqrt{\dfrac{\sigma_{z}}{\sigma_{\smallT}}} \,\text{,}
\end{equation}
and the family of $L$ functions (defined in \ref{eq:ag_l_family}) are smooth and well behaved functions for all physically meaningful $\xi$.
Finally the $M_{\alpha}^{\lbrace f \rbrace}$ quantities define the $\alpha$ diretion influence of the $f^{\text{th}}$ external force.

In UEM, the initial pulse conditions are of notable importance for the performance of the instrument.
Therefore I have presented a simplistic but useful set of initial conditions for the AG model.
First lets assume we photocathode which has a work function $\Phi$, the energy an electron must absorb before it is freed to leave the material.
Lets assume that the photocathode is in an acceleration region, having an applied electric field $E_{DC}$.
Finally, lets assume we have a laser pulse whose intensity $I$ is of the form,
\begin{equation}
  I ( x , y , t ) = I_{0} \exp \left ( - \frac{ ( x^{2} +y^{2} ) }{ w^{2} } - \frac{ t^{ 2 } }{ \tau^{ 2 } } \right ) \punct{,}
\end{equation}
where $w$ and $\tau$ are the HW1/eM laser pulse width and duration respectively, and the photon energy $\hbar \omega$ is greater than $\Phi$, thus permitting single-photon photoemission.
Ejected electrons will then carry an excess photoemission energy $\Delta E = \hbar \omega - \Phi$, so that the maximum velocity that an ejected electron may have is $ v_{max} = \sqrt{ 2 \Delta E / m } $.
Given these generic assumptions, the resulting initial conditions are simply,
\begin{subequations}
  \begin{gather}
    \sigma_{ \smallT } ( 0 ) \approx \frac{ w^{ 2 } }{ 2 } \\
    \sigma_{ z } ( 0 ) \approx \frac{ ( v_{max} \tau )^{ 2 } }{ 2 } + \left ( \frac{ e E_{ DC } \tau^{ 2 } }{ 4 m } \right )^{ 2 } \\
    \gamma_{ \smallT } ( 0 ) \approx 0 \\
    \gamma_{ z } ( 0 ) \approx \sqrt{ \sigma_{ z } ( 0 ) } \left ( m v_{max} + \frac{ e E_{DC} \tau }{ \sqrt{ 2 } } \right ) \\
    \eta_{ \smallT } ( 0 ) \approx \frac{ m }{ 3 } ( \hbar \omega - \Phi ) \\
    \eta_{ z } ( 0 ) \approx \frac{ \eta_{ \smallT } ( 0 ) }{ 4 } \punct{.}
  \end{gather}
\end{subequations}
In Section \ref{sec:bin_model} I have also presented a binning model which, when fully investigated, may eventually be employed rather than this simplisitc set of initial conditions.

Most UEM systems will employ magnetic lenses and DC accelerators; there is also increasing interest in using RF cavities for pulse compression.
The AG model can model the influence of these external forces via the $M$ parameters in \ref{eq:ag_model_gamma_conc}, which are in terms of the position of the center of the pulse $z^{\prime}$ and its velocity $v_{\smallzero}$.
The each magnetic lens, positioned at $z_{mag}^{\prime}$ and of length $L_{mag}$, can be represented simply as 
\begin{equation}
  M_{\smallT}^{mag} = M_{\smallT}^{mag}(I) \exp \left [ - \left (  \frac{ z^{\prime} - z_{mag}^{\prime} }{ L_{mag} / 2 } \right )^{ 2 } \right ] \,\text{,}
\end{equation}
where $M_{\smallT}^{mag}(I)$ is a lens-specific characterization of the magnetic field strength (versus current $I$).
A single TM$_{010}$-mode cylindrical RF cavity, positioned at $z_{RF}^{\prime}$, having length $L_{RF}$, and oscillating at frequency $\Omega$ with field strength $E_{0}$ and phase shift $\phi$, may be expressed as
\begin{gather}
  M_{z}^{RF} = \left [ \frac{ e \Omega E_{0} }{ v_{\smallzero} } \cos \left ( \frac{ \Omega z^{\prime} }{ v_{\smallzero} } + \phi \right ) \right ] \exp \left [ - \left (  \frac{ z^{\prime} - z_{RF}^{\prime} }{ L_{RF} / 2 } \right )^{ 2 } \right ] \\
  \begin{aligned}
  M_{\smallT}^{RF} &= e E_{0} \left [ \frac{ v_{\smallzero} \Omega }{c^{2}} \cos \left ( \frac{ \Omega z^{\prime} }{ v_{\smallzero} } + \phi \right ) + \frac{2 n}{L_{RF}} \left ( \frac{z^{\prime} - z^{\prime}_{RF}}{L_{RF} / 2} \right )^{2 n - 1} \right. \\ & \times \left. \sin \left ( \frac{ \Omega z^{\prime} }{ v_{\smallzero} } + \phi \right )
   \right ] \exp \left [ - \left (  \frac{ z^{\prime} - z_{RF}^{\prime} }{ L_{RF} / 2 } \right )^{ 2 } \right ] \,\text{.}
  \end{aligned}
\end{gather}
Additionally, since an RF cavity can act as a pulse accelerator an additional force term
\begin{equation}
  F_{z}^{RF} = e E_{0} \sin \left ( \frac{ \Omega z^{\prime} }{ v_{\smallzero} } + \phi \right ) \exp \left [ - \left (  \frac{ z^{\prime} - z_{RF}^{\prime} }{ L_{RF} / 2 } \right )^{ 2 } \right ] \, \text{,}
\end{equation}
should be added to the equation of motion governing the propagation of the pulse down the column.
Finally, if we assume that an acceleration region, with cathode at $z^{\prime} = 0$, anode at $z^{\prime}_A$, and acceleration potential $V$, has an on-axis force of the form
\begin{equation}
  F_z(0,z^{\prime}) = \frac{eV}{2z^{\prime}_A} \left( 1 - \tanh \left( \frac{ z^{\prime} - z^{\prime}_A }{ z^{\prime}_A / s } \right) \right) \,\text{,}
\end{equation}
where $s$ is a parameter that quantifies the sharpness of the field drop at the anode, then accelerator's influence on the pulse dynamics are given by
\begin{gather}
  M_{\smallT} = -\frac{eVs}{4z^{\prime 2}_A} \operatorname{sech}^2 \left( \frac{z^{\prime} - z^{\prime}_A }{ z^{\prime}_A / s } \right) \\
  M_z = \frac{eVs}{2z^{\prime 2}_A} \operatorname{sech}^2 \left( \frac{z^{\prime} - z^{\prime}_A }{ z^{\prime}_A / s } \right) \, \text{.}
\end{gather}

I have written an implementation of this model for the Perl programming language and released to the Comprehensive Perl Archive Network (CPAN) under the name \verb!Physics::UEMColumn! for use by the UEM community (\url{https://metacpan.org/module/Physics::UEMColumn}).
This model is fully object-oriented, which makes writing simulations remarkably easy for a user to read, write and change scripts.
Under the hood, though, it uses a custom set of C level bindings to the GNU Scientific Library (GSL), so that the numerical equation solving is fast and accurate.
I have found that this speed, combined with the dynamic flexibility offered by object-oriented Perl, has allowed me to easily write optimization scripts and generate figures which might have been nearly impossible previously; several of these have been shown in this thesis.

\section{UEM Design}

I have expounded on several factors that we feel are key concepts in designing a UEM.
I have used the above model to show that an electron pulse, whether oblate (disk-like) or prolate (cigar-like), its initial expansion will be primarily in the the shorter dimension, i.e. an electron pulse ``wants'' to be spherical.
The model predicts that the use of shorter focal-length magnetic lenses will result in higher fidelity focusing; such lenses limit the impulse delivered to the pulse by reducing the time of flight that the pulse experiences the higher charge densities inherent during focusing.
Additionally, we have seen that oblate pulses are preferred to prolate pulses in terms of focus fidelity.
The model predicts that when properly designed and driven, RF cavities can effectively recompress electron pulses which have expanded longitudinally.
Interestingly, the model suggests that the additional bandwidth generated by space-charge repulsion (especially in higher charge-density pulses) can allow for additional pulse compression even at one-to-one image to object distances.

The Rose criterion \cite{rose_television_1948} states that in order to have proper gray-scaling in a single-shot DTEM employing a 1k X 1k CCD camera, the pulse must contain at least $10^8$ electrons.
We have proposed that the term ``ultrafast'' be used in the case where an electron pulse may be completely contained inside the acceleration region of the column.
The reason for this designation is that such pulses are subject to a different form of Child-Langmuir law \cite{child_discharge_1911,langmuir_effect_1923,valfells_effects_2002}, which limits its the number of electrons $N_{crit}$ that may be generated by a laser of HW1/eM width $w$ before the pulse is subject to deleterious distortions,
\begin{equation}
  N_{crit} = \frac{\pi \varepsilon_0 w^2 V_{DC}}{e d} \,\text{,}
\end{equation}
where $V_{DC}$ and $d$ are the accelerator voltage and length respectively.
In order to satisfy the both the Rose criterion and this ultrafast Child-Langmuir restriction, we expect that UEM instruments will need to employ a large-area emission source.
This has necessitated building a custom acceleration region based on the design of Togawa \textit{et al.} \cite{togawa_ceb6_2007} which boasts a moderately uniform acceleration field near the axis of the accelerator, as well as other large-aperture electron-optical elements for our prototype UEM column.
We have also designed and built a custom laser source capable of generating these electron pulses.

A common figure of merit for electron beams is its normalized transverse emittance \cite{jensen_emittance_2010}, $\varepsilon_{\smallT}$, which for the usual case of axially symmetric electron pulses has the form,
\begin{equation}
  \varepsilon_{\smallT} = \frac{1}{m c} \Delta x \Delta p_{\smallT} \,\text{,}
\end{equation}
where $\Delta x$ and $\Delta p_{\smallT}$ are the transverse spatial uncertainty and transverse momentum uncertainty, respectively.
This quantity is conserved under propagation through free space and perfect electron-optical, though it may increase for imperfect ones \cite{oshea_reversible_1998}.
The normalized transverse emittance limits the resolving capablility of the beam, so this quantity can be thought of as a measure of the beam quality, where a lower number is preferred \cite{berger_dc_2009}.
Since we have already established that a large area emission source must be used, the only avenue to reduce $\varepsilon_{\smallT}$ is to reduce the transverse momentum uncertainty, which is dependent on the photocathode and photoemission process employed.

\section{Photocathode Engineering}

The desire to reduce the transverse momentum uncertainty $\Delta p_{\smallT}$ --- in an effort to reduce the rms transverse emittance $\varepsilon_{\smallT}$ and thereby improve the resolving power of the electron pulse~\cite{berger_dc_2009} --- has driven us to explore many photocathodes and even several different photoemission processes.

