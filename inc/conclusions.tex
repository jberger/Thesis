%This work is licensed under the Creative Commons Attribution-NonCommercial-NoDerivs 3.0 United States License. To view a copy of this license, visit http://creativecommons.org/licenses/by-nc-nd/3.0/us/ or send a letter to Creative Commons, 444 Castro Street, Suite 900, Mountain View, California, 94041, USA.

Ultrafast Electron Microscopy is an exciting new field of intrumentation research which promises to open our eyes to the dynamics occurring in nanoscale processes.

During my time at UIC I have worked to advance this technology, on both the theoretical (simulation/modeling) and experimental (prototyping and photocathode engineering) fronts.

\section{Modeling}

In this thesis I have presented the extended Analytic Gaussian (AG) model, an effiecient numerical model based on the work of Michalik and Sipe \cite{michalik_analytic_2006}, but extended so as to be useful in modeling the dynamics of ultrafast electron pulses in microscope columns.
For pulses containing $N$ carriers of charge $e$ and mass $m$, the model is simply these four equations,
\begin{subequations}
\begin{gather}
  \frac{d\sigma_{\alpha}}{dt} = \frac{2\gamma_{\alpha}}{m} \\
  \frac{d\gamma_{\alpha}}{dt} = \frac{ \Gamma_{\alpha}^2 + \gamma_{\alpha}^2 }{\sigma_{\alpha} m}
    + \frac{N e^2}{4\pi\varepsilon_0} \frac{1}{6 \sqrt{\sigma_{\alpha}\pi}} L_{\alpha}(\xi) + \sigma_{\alpha} \sum\limits_{f} M_{\alpha}^{\lbrace f \rbrace} \,\text{.}
\end{gather}
\end{subequations}
where the subscript $\alpha$ denotes the directionality of the dynamics, either $\smallT$ for the transverse or $z$ for the longitudinal.
At any time $t$, the $\sigma_{\alpha}$ quantities characterize the spatial variance of the pulse and $\gamma_{\alpha}$ represent the momentum chirp across the pulse.
A third parameter, the local momentum variance $\eta_{\alpha}$, is also encoded in these equations.
By Liouville's theorem, which the AG model explicitly satisfies, we know that the product of the spatial variance $\sigma_{\alpha}$ and the local momentum variance $\eta_{\alpha}$ is constant throughout propagation (for perfect lensing and without clipping the puse), so we define the constant,
\begin{equation} \label{eq:define_big_gamma_conc}
  \Gamma_{\alpha}^2 \equiv \sigma_{\alpha} \eta_{\alpha} \,\text{.}
\end{equation}
By evaluating this constant once, presumably from initial conditions, and later applying \ref{eq:define_big_gamma_conc} we can get the extract these $\eta_{\alpha}$ parameters while only needing to compute the other four.
The parameter $\xi$ characterizes the ellipticity of the pulse,
\begin{equation}
  \xi \equiv \sqrt{\dfrac{\sigma_{z}}{\sigma_{\smallT}}} \,\text{,}
\end{equation}
and the family of $L$ functions (defined in \ref{eq:ag_l_family}) are smooth and well behaved functions for all physically meaningful $\xi$.
Finally the $M_{\alpha}^{\lbrace f \rbrace}$ quantities define the $\alpha$ diretion influence of the $f^{\text{th}}$ external force.

In UEM, the initial pulse conditions are of notable importance for the performance of the instrument.
Therefore I have presented a simplistic but useful set of initial conditions for the AG model.
First lets assume we photocathode which has a work function $\Phi$, the energy an electron must absorb before it is freed to leave the material.
Lets assume that the photocathode is in an acceleration region, having an applied electric field $E_{DC}$.
Finally, lets assume we have a laser pulse whose intensity $I$ is of the form,
\begin{equation}\label{eq:laser_form}
  I ( x , y , t ) = I_{0} \exp \left ( - \frac{ ( x^{2} +y^{2} ) }{ w^{2} } - \frac{ t^{ 2 } }{ \tau^{ 2 } } \right ) \punct{,}
\end{equation}
where $w$ and $\tau$ are the HW1/eM laser pulse width and duration respectively, and the photon energy $\hbar \omega$ is greater than $\Phi$, thus permitting single-photon photoemission.
Ejected electrons will then carry an excess photoemission energy $\Delta E = \hbar \omega - \Phi$, so that the maximum velocity that an ejected electron may have is $ v_{max} = \sqrt{ 2 \Delta E / m } $.
Given these generic assumptions, the resulting initial conditions are simply,
\begin{subequations}
  \begin{gather}
    \sigma_{ \smallT } ( 0 ) \approx \frac{ w^{ 2 } }{ 2 } \\
    \sigma_{ z } ( 0 ) \approx \frac{ ( v_{max} \tau )^{ 2 } }{ 2 } + \left ( \frac{ e E_{ DC } \tau^{ 2 } }{ 4 m } \right )^{ 2 } \\
    \gamma_{ \smallT } ( 0 ) \approx 0 \\
    \gamma_{ z } ( 0 ) \approx \sqrt{ \sigma_{ z } ( 0 ) } \left ( m v_{max} + \frac{ e E_{DC} \tau }{ \sqrt{ 2 } } \right ) \\
    \eta_{ \smallT } ( 0 ) \approx \frac{ m }{ 3 } ( \hbar \omega - \Phi ) \\
    \eta_{ z } ( 0 ) \approx \frac{ \eta_{ \smallT } ( 0 ) }{ 4 } \punct{.}
  \end{gather}
\end{subequations}
In Section \ref{sec:bin_model} I have also presented a binning model which, when fully investigated, may eventually be employed rather than this simplisitc set of initial conditions.
