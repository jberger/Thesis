%This work is licensed under the Creative Commons Attribution-NonCommercial-NoDerivs 3.0 United States License. To view a copy of this license, visit http://creativecommons.org/licenses/by-nc-nd/3.0/us/ or send a letter to Creative Commons, 444 Castro Street, Suite 900, Mountain View, California, 94041, USA.

\section{Two-photon Assisted Thermionic Emission} \label{sec:two_photon_thermionic}

Although single-photon photoemission is more efficient than multi-photon photoemission, the concept remains the same; an electron in the photocathode must absorb enough energy to exceed the vacuum level.
In multi-photon photoemission this energy is attained by absorbing multiple photons.
As this is more likely to occur when more photons are present, the efficiency, and thus the yield, is proportional to the laser intensity to the power of the number of photons required.
Two-photon photoemission, therefore, will have yield that is quadratic in laser intensity ($I^2$).

Two-photon assisted thermionic emission (2PTE) occurs when two photons provide nearly enough energy to emit a cold electron, and therefore a hotter electron, in the tail of the Fermi distribution, may have enough energy to be emitted~\cite{yen_thermally_1980}.
Of course this is even less efficient than direct two-photon photoemission, but as three-photon photoemission goes as $I^3$, 2PTE will still be the dominant photoemission mechanism.
For thermionic emission from electrons at a temperature $T_e$, the rms transverse momentum $\Delta p_{\smallT}$ is given by 
\begin{equation}
  \Delta p_{\smallT} = \sqrt{m k_B T_e} \,\text{,}
\end{equation}
where $k_B$ is the Boltzmann constant~\cite{dowell_quantum_2009,jensen_emittance_2010}.
Since a high laser intensity is required to drive 2PTE, the laser itself will heat the electrons, therefore, we expect the measured Fourier plane spot size to grow as incident laser pulse energy increases, assuming that the laser pulse duration is less than the cooling time of the electron distribution so that the laser is only heating the electrons and not the lattice, which is true in this case.

\begin{figure}
  \centering
  \input{au_plot}
  \caption[Measured Fourier plane electron beam size and yield vs incident laser pulse energy on gold photocathode]{
    Measured Fourier plane electron beam size (HWe$^{-1}$M; squares) and electron yield (circles) as a function of the incident $\sim$200fs, 523nm laser pulse energy for the 300nm-thick gold photocathode.
    The electron yield is proportional to the square of laser pulse energy (solid green line) and the dependence of the beam size with the laser pulse energy fits that predicted by a zero free parameter electron heating model (solid red line; dashed lines represent $\pm$10\% error in the laser spot size).
    A representative raw Fourier plane beam image is also shown.
  }
  \label{fig:gold-emission}
\end{figure}

Again using the experimental technique outlined in Section \ref{sec:photocathode-method}, we monitored the momentum distribution (Fourier plane spot size) for electrons emitted by 2PTE from gold using the $\sim$200fs green (523nm) $p$-polarized laser pulses.
Due to the 60$^{\circ}$ angle of incidence, the circular laser beam is focused by the 30cm focal length lens to an elliptical 50x100$\mu$m (half-width 1/e maximum (HWe$^{-1}$M) of the field) spot size on the photocathode surface.
The results, the measured Fourier plane beam size and electron yield, are displayed in \ref{fig:gold-emission} as a function of the incident laser pulse energy, corrected to account for 10\% optical loss (mainly from the uncoated vacuum system windows).

The quadratic dependence of the electron yield on pulse energy clearly indicates a two-photon emission process.
This is expected since an excitation energy of two green photons ($\hbar \omega$ = 2.37eV) will be required to overcome the reported $\Phi$ = 4.69eV effective work function of a thick gold film contaminated with adsorbed water~\cite{monjushiro_ultraviolet_1991} or ensure that at least the tail of the electron Fermi distribution has sufficient energy to overcome the 5.1($\pm$0.1)eV work function of a clean Au surface~\cite{eastman_photoelectric_1970}.
Moreover, when $2\hbar \omega \approx \Phi$, the dependence of the yield on the square of the electron's excess energy above the work function~\cite{monjushiro_ultraviolet_1991} ensures that predominantly only the high energy (Boltzmann) tail of the Fermi distribution contributes to the emission.
Consequently, the observed increase in $\Delta p_{\smallT}$ with incident pulse energy must be due to a heating effect; specifically, laser heating of the free electron Fermi gas as two-photon excitation is instantaneous and the $\sim$200fs green laser pulse duration excludes any coupling to the lattice which occurs on the time scale of a few picoseconds~\cite{chen_semiclassical_2006}.
The solid red line in \ref{fig:gold-emission} is the predicted variation derived using a simple zero free parameter model of this effect.
Using the known optical properties of gold~\cite{johnson_optical_1972}, which give a reflectance $R_p$ = 0.53 and an absorption depth of 20nm, and the temperature-dependent heat capacity of a free electron Fermi gas, the model evaluates the average temperature of the laser-heated electron gas that may be two-photon excited assuming that only electrons within a few nanometers of the surface can be emitted, as the mean free path for electrons in Au is $\sim$4nm~\cite{seah_quantitative_1979}.
This average electron temperature $T_e$ is then used in the AG model simulation of the experiment (\ref{fig:transverse-measurement}) to determine the expected Fourier plane spot size on the YAG scintillator; that is, using an initial $\Delta p_{\smallT} = \sqrt{m k_B T_e}$.  The dashed red lines indicate the expected trends for a $\pm$10\% error in the laser spot size $\Delta x_0$ incident on the gold cathode surface.
The close agreement between the data and the simulation strongly supports our interpretation of the emission mechanism.
