%This work is licensed under the Creative Commons Attribution-NonCommercial-NoDerivs 3.0 United States License. To view a copy of this license, visit http://creativecommons.org/licenses/by-nc-nd/3.0/us/ or send a letter to Creative Commons, 444 Castro Street, Suite 900, Mountain View, California, 94041, USA.

\section{Nanopatterned (Plasmonic) Photocathodes}

In the search for alternative mechanisms to reduce the intrinsic divergence of the electron beam upon photoemission, a possible area of interest was Plasmon-Assisted Photoemission (PAPE). 
This avenue of investigation was motivated by Zawadzka et al \cite{zawadzka_evanescent_2001} who indicate that when driving a surface plasmon on gold they witnessed a reduction in the emission angle of photoemitted electrons.
In their paper, the authors mount a gold foil on a prism; the laser is incident on the gold through the prism.
This geometry is called the Kretschmann geometry.

While this geometry is common, due to fears about the lack of heat conduction through the glass and the power needed to generate a sufficient electron beam current, that UEM applications should explore the ``grating coupling'' geometry.
In this geometry, the plasmon is driven on a periodic surface rather than a flat surface attached to glass.
Futher, in this geometry the back surface is now free to be attached to a heat sink.

%\subsection{Driving the Surface Plasmon}

As the name suggests, a plasmon is the quasiparticle of plasma oscillation.
Electrons in a material can couple with electromagnetic fields, in this case a laser field, and oscillate accordingly.
When this occurs at the surface of a material, the resultant quasiparticles are called ``surface plasmon-polaritons.''
Since these will be the only plasmons discussed in this paper, the word ``plasmon'' will be used interchangeably for the name ``surface plasmon-polaritons.''
%TODO should this be formalized, either one way (SPP) or the other

The mathematical formalism of surface plasmon-polaritons is a lengthy subject and well described elsewhere \cite{cottam_introduction_2004,concepts_2002}.

Every material has an intrinsic plasma frequency given by 
\begin{equation}
  \omega_{P} = \sqrt{\frac{\rho_{e} e^2}{\varepsilon_{0} m^*}}
\end{equation}
where $\rho_{e}$ is the electron density and $m*$ is the effective mass of the material.
And in the most simple (undamped) approximations, the dielectric function of the material may be written in terms of this constant.
\begin{equation} \label{eq:dielectric_as_omega}
  \varepsilon(\omega) = 1 - \frac{ \omega_{P}^2 }{ \omega^2 }
\end{equation}

At the interface of a vacuum and such a simple material material, contstraints on the material parameters which may allow surface plasmons result in the condition that the dielectric constant at the oscillation frequency of the surface plasmon $ \varepsilon(\omega_{SP}) = -1 $.
Substituting this result into \ref{eq:dielectric_as_omega} show that this frequency of oscillation is simply related to intrinsic plasma frequency by
\begin{equation}
  \omega_{SP} = \omega_{P} / \sqrt{2}
\end{equation}

Further, plasmons carry momentum whose magnitude is given by
\begin{equation}
  k_{P}^2 = \frac{\omega^2}{c^2} \left( \frac{\varepsilon_{1} \varepsilon_{2}}{\varepsilon_{1} + \varepsilon_{2}} \right)
\end{equation}
but of course for a vacuum interface this is simplified to 
\begin{equation}
  k_{SP}^2 = \frac{\omega^2}{c^2} \left( \frac{ \varepsilon }{ 1 + \varepsilon } \right)
\end{equation}

In the grating coupling scheme (shown in \ref{fig:plasmon_schematic}) a laser of wavelength $\lambda$ is incident on a material with a grating of period $d$, making an angle $\theta$ from the surface normal. A surface plasmon is driven when the portion of the laser wave-vector parallel to the plasmon ($k_{Lx}$) and a fourier component ($k_{G} = 2 \pi / d$) of the grating matches the wave-vector of the surface plasmon
\begin{equation}
  k_{SP} = k_{Lx} + n k_{G}
\end{equation}
where $n$ here is an integer.
Which reduces to 
\begin{equation} \label{eq:plasmon_condition}
  \sqrt{ \frac{ \varepsilon }{ 1 + \varepsilon } } = \sin \theta + \frac{ n \lambda }{ d }
\end{equation}
For a given grating period, \ref{eq:plasmon_condition} determines the angle of incidence that the laser must make with the surface in order to drive the plasmon.
Further, it is used to design the grating, allowing for angles that the microscope column will accept.

\begin{figure}
  \centering
  \begin{tikzpicture}

\draw [orange!70,fill=orange!30]
  (0,0)
  \foreach \x in {1,2,...,5}{
    -- ++(0.25, 0)
    -- ++(0, 0.25)
    -- ++(0.5, 0)
      coordinate [pos=0.5] (mid-\x)
    -- ++(0, -0.25)
    -- ++(0.25, 0)
  }
  -- ++(0,-1)
  -- ++(-5,0)
  -- cycle
;

\draw [dashed]
  (mid-3)
  -- ++(0,3)
    node [left=0.3, pos=0.4] {$\theta$}
;

\draw [green, thick,->]
  (0,3)
  -- (mid-3)
;
\draw [green, thick,->]
  (mid-3)
  -- (5,3)
;

\node at (0.5, -0.45) {$\varepsilon$};

\draw [thick, |-|]
  ($(mid-4)+(0,-0.7)$)
  -- ++ (1, 0)
    node [fill=orange!30, pos=0.5] {$d$}
;
  

\end{tikzpicture}

  \caption{Schematic diagram of a grating coupling geometry for driving a surface plasmon}
  \label{fig:plasmon_schematic}
\end{figure}

\subsection{Commerical Grating on Glass}

For proof-of-concept purposes the first experiment was run on a commercially available gold foil on glass holographic grating (750 lines/mm).
Like the Kretschmann geometry, a foil on glass grating is vulnerable to heating problems; a plasmon creates a large amount of heat but the glass is a poor heat conductor.
The aforementioned laser (emitting $\sim$200fs pulses of green light at 30MW/cm$^2$) was used to drive the plasmon.

The input laser angle was slowly changed until the plasmon oscillation is visible.
At the instant that the plasmon started, a bright flash was visible on the scintillator screen then disappeared.

After this, the grating showed visual signs of damage, both on the surface and in the reflected light pattern.
\ref{fig:grating-damage} shows the results of Scanning Electron Microscopy (SEM) on the grating.
The damage pattern is round and of the size of the laser spot used to drive the plasmon.
The extent of the damage indicates a large temperature increase, which given the other evidence of the test, can be assumed to be from the sudden appearance of a surface plasmon.

Though this conclusively shows that a plasmon was driven successfully, the heat generated by the plasmon oscillation destroyed the grating.
Further this experiment validated the decision not to pursue a Kretschmann geometry for fear of heat damage at these high powers.

\begin{figure}
  \centering
  \includegraphics{damage.png}
  \caption{Damage to commerical gold-on-glass holographic grating.}
  \label{fig:grating-damage}
\end{figure}

\subsection{Sinusoidal Grating}

After the destruction of the gold foil on glass grating, new gratings were designed to have a higher thermal conductivity.
The second generation grating was constructed from a silicon substract then coated with a gold film.
The thermal conductivity of silicon is approximately 100 times that of glass. %TODO reference needed?

Collaborators at Argonne National Laboratory's Electron Microscopy Center, Material Science Division used a defocused Gallium focused ion beam (FIB) (Zeiss 1540XB) to iteratively mill an approximately sinusoidal shape into the surface of a silicon wafer.
Several regions were created each 100x100$\mu$m in area, having a $\sim$1$\mu$m period and $\sim$10\% modulation depth.
Given a laser wavelength of 523nm ($\varepsilon_{Au} \approx –3.95$ \cite{johnson_optical_1972}), this choice of periodicity corresponds by \ref{eq:plasmon_condition} to an incidence angle of $\sim39^{\circ}$, which is within the acceptance range of our experimental setup (35-75$^{\circ}$). %TODO point to Togawa section
An SEM image of one such region is shown in \ref{fig:fib-si-sem}.

\begin{figure}
  \centering
  \includegraphics{HighMagSEM.jpg}
  \caption{
    SEM image of sinusoidal grating on silicon.
    Pattern milled by focused ion beam milling at Argonne National Lab.
    Sample was later coated with 300 nm gold film.
  }
  \label{fig:fib-si-sem}
\end{figure}

Using the Varian e-beam deposition chamber at the UIC Nano Core Facility (NCF), I deposited a thin layer of chromium as a binding agent, then a 300nm film of gold.
This thickness of gold is several times thicker than the skin depth of green light on gold, therefore the laser radiation cannot interact with the silicon.

\begin{figure}
  \centering
  \begin{tikzpicture}
  \inputdata{power_seq_data}
\end{tikzpicture}

  \caption{}
  \label{fig:pape}
\end{figure}

% Content taken from 2010 ANL report
Strong enhanced phototemission was observed for only a narrow $\pm$5mrad range of incidence angles $\theta$ around the expected plasmon resonance angle of 39$^{\circ}$.
Moreover, this enhanced laser-driven electron emission was only observed for $p$-polarized incident 523nm femtosecond laser radiation.
These two properties are known characteristics of plasmon-assisted photoemission.

As seen in \ref{fig:pape}, at low incident laser intensities, below about 25MW/cm$^2$, the emission from the nano-patterned photocathode at the plasmon resonance condition is very similar to that observed from a flat gold surface neighboring the FIB-milled region.
Specifically, the strength of both emissions have a quadratic dependence on the incident laser intensity.
This is consistent with a two-photon field emission process --- it requires the absorption of two 2.37eV (green) photons for an electron to acquire sufficient energy to allow any significant field emission as gold has a work function of 5.1eV.
Moreover, both emissions have similar momentum distributions \fixme{(images A and E)}, again indicating a similar emission process.
The only difference is that the emission from the nano-patterned region is 3-4 times more efficient.
This is readily explained by the fact that the periodic surface modulation results in a local enhancement of the $\sim$1MV/m DC gun field at the peaks of the modulation which, through the Schottky effect, then presents a reduced barrier for field emission.

Above $\sim$32MW/cm$^2$, there is a rapid and dramatic increase in the emission efficiency for the nano-patterned photocathode.
This can be explained by a suppression of the barrier to photoemission below the two-photon excitation energy, thus allowing direct two-photon photoemission.
This is likely caused by the laser-excited surface plasmon since a similar increase from the neighboring flat gold surface is not present.
The observed distinct changes in the momentum distributions \fixme{(images B and C)} in this laser intensity range support this argument.
In \fixme{image B}, at a laser intensity of $\sim$30MW/cm$^2$, there is already a distinct broadening of the momentum distribution in the direction parallel to the wavevector $k_G$ of the sinusoidal modulation on the photocathode (i.e., perpendicular to the FIB-milled grooves). 
This broadening becomes the dominant feature of the electron emission above $\sim$32MW/cm$^2$; indeed, it becomes so severe that at laser intensities above $\sim$37MW/cm$^2$ the 6mm-diameter bore of the magnetic lenses becomes a strong limiting aperture for the resulting beam --- as is clearly seen in \fixme{image D} and by the cut-off in the detected electron signal (YAG fluorescence).

\subsection{Sinusoidal Grating, Rotated}

\subsection{Trenched Grating}
