\section{Future Prospects} \label{sec:photocathode_future}

The philosophical argument presented in Section \ref{sec:este} for using the effective mass $m^*$ in the expressions for the transverse momentum variance $\Delta p_{\smallT}$ simply follows from conservation of energy and momentum at the barrier and actually does not depend on the ``excited state'' nature nor the thermionic emission process.
For the example of two-photon assisted thermionic emission (2PTE) from gold presented in Section \ref{sec:two_photon_thermionic}, the close agreement between the data and the simulation is supported by the fact that the effective mass of a free electron in Au is equal to its rest mass $m$ \cite{johnson_optical_1972} --- the value employed in that analysis.

For direct single-photon photoemission, we would therefore also expect to require that the expression describing the rms transverse momentum (\ref{eq:jensen_eta}, Refs. \cite{dowell_quantum_2009} and \cite{jensen_emittance_2010}) to be written as $\Delta p_{\smallT} = \sqrt{m^* ( \hbar \omega - \Phi ) / 3 }$.
Efforts are currently underway to determine if this is indeed the case using a variety of metal photocathodes and the $\sim$4ps, 261nm laser radiation source.
In an upcoming publication, we will show that the dependence of $\Delta p_{\smallT}$ on the electron effective mass may be extended to simple planar metal photocathodes --- thus potentially allowing for the future development or discovery of a robust, high brightness photocathode material.
For a selection of single-crystal and polycrystalline photocathodes, we will show that the simulation, incorporating $m^*$, is in good agreement with our direct measurements of $\Delta p_{\smallT}$.
We will also discuss the effect of laser-induced heating on $\Delta p_{\smallT}$ of the electron distribution in the photocathode during photoemission, which may limit the attainable pulsed electron source brightness at the desired high peak pulse current densities.
