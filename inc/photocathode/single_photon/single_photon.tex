%This work is licensed under the Creative Commons Attribution-NonCommercial-NoDerivs 3.0 United States License. To view a copy of this license, visit http://creativecommons.org/licenses/by-nc-nd/3.0/us/ or send a letter to Creative Commons, 444 Castro Street, Suite 900, Mountain View, California, 94041, USA.

\section{Single-photon Photoemission from Flat Metal Photocathodes} \label{sec:single_photon}

Single-photon photoemission is the most common photoemission process and the easiest to explain.
As described by Einstein in his famous paper on ``the photoelectric effect'' \cite{einstein_uber_1905}, as it is colloquially known, if an electron absorbs a photon whose energy is sufficient to promote the electron above the local vacuum level, it becomes free to leave the host material.
The material dependent minimum energy needed to eject an electron is termed its ``work function'' $\Phi$.
Any additional energy above the work function remains with the ejected electron, usually in the form of kinetic energy, and is called the ``excess photoemission energy'' $\Delta E$; for single photon photoemission, using a laser of frequency $\omega$ (photon energy $\hbar \omega$), this is simply,
\begin{equation}
  \Delta E = \hbar \omega - \Phi \,\text{.}
\end{equation}
The efficiency of photoemission $\eta_{{\scriptscriptstyle PE}}$, when $\Delta E$ is small compared to $\Phi$ may be written as, %TODO ref
\begin{equation}
  \eta_{{\scriptscriptstyle PE}} = A ( 1 - R ) \Delta E^2 \,\text{,}
\end{equation}
where $A$ is a material paramenter (typically of the order of $\sim 10^{-4} \text{eV}^{-2}$ %TODO ref
), and $R$ is the photocathode surface reflectivity, which is itself a function of the polarization, frequency and angle of the incident light.
%TODO calculation of $\eta_{PE}$?

\begin{figure}
  \centering
  \begin{tikzpicture}[gnuplot]
%% generated with GNUPLOT 4.6p0 (Lua 5.1; terminal rev. 99, script rev. 100)
%% Mon 08 Apr 2013 02:01:49 PM CDT
\path (0.000,0.000) rectangle (8.750,6.125);
\gpcolor{color=gp lt color border}
\gpsetlinetype{gp lt border}
\gpsetlinewidth{1.00}
\draw[gp path] (1.504,0.985)--(1.684,0.985);
\node[gp node right] at (1.320,0.985) { 0};
\draw[gp path] (1.504,1.780)--(1.684,1.780);
\node[gp node right] at (1.320,1.780) { 0.2};
\draw[gp path] (1.504,2.575)--(1.684,2.575);
\node[gp node right] at (1.320,2.575) { 0.4};
\draw[gp path] (1.504,3.371)--(1.684,3.371);
\node[gp node right] at (1.320,3.371) { 0.6};
\draw[gp path] (1.504,4.166)--(1.684,4.166);
\node[gp node right] at (1.320,4.166) { 0.8};
\draw[gp path] (1.504,4.961)--(1.684,4.961);
\node[gp node right] at (1.320,4.961) { 1};
\draw[gp path] (1.504,5.756)--(1.684,5.756);
\node[gp node right] at (1.320,5.756) { 1.2};
\draw[gp path] (1.504,0.985)--(1.504,1.165);
\draw[gp path] (1.504,5.756)--(1.504,5.576);
\node[gp node center] at (1.504,0.677) { 0};
\draw[gp path] (3.062,0.985)--(3.062,1.165);
\draw[gp path] (3.062,5.756)--(3.062,5.576);
\node[gp node center] at (3.062,0.677) { 0.05};
\draw[gp path] (4.620,0.985)--(4.620,1.165);
\draw[gp path] (4.620,5.756)--(4.620,5.576);
\node[gp node center] at (4.620,0.677) { 0.1};
\draw[gp path] (6.177,0.985)--(6.177,1.165);
\draw[gp path] (6.177,5.756)--(6.177,5.576);
\node[gp node center] at (6.177,0.677) { 0.15};
\draw[gp path] (7.735,0.985)--(7.735,1.165);
\draw[gp path] (7.735,5.756)--(7.735,5.576);
\node[gp node center] at (7.735,0.677) { 0.2};
\draw[gp path] (1.504,5.756)--(1.504,0.985)--(7.735,0.985)--(7.735,5.756)--cycle;
\node[gp node center,rotate=-270] at (0.246,3.370) {HW1/eM Pulse Width (mm)};
\node[gp node center,rotate=-270] at (8.042,3.370) {Electron Yield (a.u.)};
\node[gp node center] at (4.619,0.215) {UV Laser Pulse Energy (nJ)};
\gpcolor{rgb color={0.000,0.000,0.000}}
\gpsetlinetype{gp lt plot 4}
\draw[gp path] (2.701,4.900)--(2.701,5.756);
\draw[gp path] (2.611,4.900)--(2.791,4.900);
\draw[gp path] (2.611,5.756)--(2.791,5.756);
\draw[gp path] (3.452,4.713)--(3.452,5.541);
\draw[gp path] (3.362,4.713)--(3.542,4.713);
\draw[gp path] (3.362,5.541)--(3.542,5.541);
\draw[gp path] (4.294,4.780)--(4.294,5.623);
\draw[gp path] (4.204,4.780)--(4.384,4.780);
\draw[gp path] (4.204,5.623)--(4.384,5.623);
\draw[gp path] (5.161,4.743)--(5.161,5.579);
\draw[gp path] (5.071,4.743)--(5.251,4.743);
\draw[gp path] (5.071,5.579)--(5.251,5.579);
\draw[gp path] (5.985,4.778)--(5.985,5.620);
\draw[gp path] (5.895,4.778)--(6.075,4.778);
\draw[gp path] (5.895,5.620)--(6.075,5.620);
\draw[gp path] (6.704,4.797)--(6.704,5.644);
\draw[gp path] (6.614,4.797)--(6.794,4.797);
\draw[gp path] (6.614,5.644)--(6.794,5.644);
\draw[gp path] (7.262,4.792)--(7.262,5.638);
\draw[gp path] (7.172,4.792)--(7.352,4.792);
\draw[gp path] (7.172,5.638)--(7.352,5.638);
\draw[gp path] (7.614,4.785)--(7.614,5.630);
\draw[gp path] (7.524,4.785)--(7.704,4.785);
\draw[gp path] (7.524,5.630)--(7.704,5.630);
\draw[gp path] (7.705,4.795)--(7.705,5.642);
\draw[gp path] (7.615,4.795)--(7.795,4.795);
\draw[gp path] (7.615,5.642)--(7.795,5.642);
\draw[gp path] (7.735,4.781)--(7.735,5.624);
\draw[gp path] (7.645,4.781)--(7.825,4.781);
\draw[gp path] (7.645,5.624)--(7.825,5.624);
\gpsetpointsize{4.00}
\gppoint{gp mark 5}{(2.701,5.335)}
\gppoint{gp mark 5}{(3.452,5.127)}
\gppoint{gp mark 5}{(4.294,5.201)}
\gppoint{gp mark 5}{(5.161,5.161)}
\gppoint{gp mark 5}{(5.985,5.199)}
\gppoint{gp mark 5}{(6.704,5.221)}
\gppoint{gp mark 5}{(7.262,5.215)}
\gppoint{gp mark 5}{(7.614,5.207)}
\gppoint{gp mark 5}{(7.705,5.219)}
\gppoint{gp mark 5}{(7.735,5.202)}
\gpsetlinetype{gp lt plot 6}
\draw[gp path] (2.701,0.985)--(2.701,1.861);
\draw[gp path] (2.611,0.985)--(2.791,0.985);
\draw[gp path] (2.611,1.861)--(2.791,1.861);
\draw[gp path] (3.452,1.000)--(3.452,2.267);
\draw[gp path] (3.362,1.000)--(3.542,1.000);
\draw[gp path] (3.362,2.267)--(3.542,2.267);
\draw[gp path] (4.294,1.903)--(4.294,3.170);
\draw[gp path] (4.204,1.903)--(4.384,1.903);
\draw[gp path] (4.204,3.170)--(4.384,3.170);
\draw[gp path] (5.161,2.815)--(5.161,4.081);
\draw[gp path] (5.071,2.815)--(5.251,2.815);
\draw[gp path] (5.071,4.081)--(5.251,4.081);
\draw[gp path] (5.985,3.149)--(5.985,4.415);
\draw[gp path] (5.895,3.149)--(6.075,3.149);
\draw[gp path] (5.895,4.415)--(6.075,4.415);
\draw[gp path] (6.704,3.500)--(6.704,4.766);
\draw[gp path] (6.614,3.500)--(6.794,3.500);
\draw[gp path] (6.614,4.766)--(6.794,4.766);
\draw[gp path] (7.262,3.761)--(7.262,5.028);
\draw[gp path] (7.172,3.761)--(7.352,3.761);
\draw[gp path] (7.172,5.028)--(7.352,5.028);
\draw[gp path] (7.614,3.982)--(7.614,5.249);
\draw[gp path] (7.524,3.982)--(7.704,3.982);
\draw[gp path] (7.524,5.249)--(7.704,5.249);
\draw[gp path] (7.705,4.171)--(7.705,5.438);
\draw[gp path] (7.615,4.171)--(7.795,4.171);
\draw[gp path] (7.615,5.438)--(7.795,5.438);
\draw[gp path] (7.735,4.136)--(7.735,5.402);
\draw[gp path] (7.645,4.136)--(7.825,4.136);
\draw[gp path] (7.645,5.402)--(7.825,5.402);
\gppoint{gp mark 7}{(2.701,1.228)}
\gppoint{gp mark 7}{(3.452,1.634)}
\gppoint{gp mark 7}{(4.294,2.536)}
\gppoint{gp mark 7}{(5.161,3.448)}
\gppoint{gp mark 7}{(5.985,3.782)}
\gppoint{gp mark 7}{(6.704,4.133)}
\gppoint{gp mark 7}{(7.262,4.395)}
\gppoint{gp mark 7}{(7.614,4.615)}
\gppoint{gp mark 7}{(7.705,4.804)}
\gppoint{gp mark 7}{(7.735,4.769)}
\gpcolor{rgb color={0.000,1.000,0.000}}
\gpsetlinetype{gp lt plot 0}
\draw[gp path] (1.504,0.985)--(1.567,1.023)--(1.630,1.061)--(1.693,1.098)--(1.756,1.136)%
  --(1.819,1.174)--(1.882,1.212)--(1.945,1.250)--(2.008,1.287)--(2.070,1.325)--(2.133,1.363)%
  --(2.196,1.401)--(2.259,1.439)--(2.322,1.476)--(2.385,1.514)--(2.448,1.552)--(2.511,1.590)%
  --(2.574,1.627)--(2.637,1.665)--(2.700,1.703)--(2.763,1.741)--(2.826,1.779)--(2.889,1.816)%
  --(2.952,1.854)--(3.015,1.892)--(3.077,1.930)--(3.140,1.968)--(3.203,2.005)--(3.266,2.043)%
  --(3.329,2.081)--(3.392,2.119)--(3.455,2.157)--(3.518,2.194)--(3.581,2.232)--(3.644,2.270)%
  --(3.707,2.308)--(3.770,2.346)--(3.833,2.383)--(3.896,2.421)--(3.959,2.459)--(4.022,2.497)%
  --(4.085,2.535)--(4.147,2.572)--(4.210,2.610)--(4.273,2.648)--(4.336,2.686)--(4.399,2.724)%
  --(4.462,2.761)--(4.525,2.799)--(4.588,2.837)--(4.651,2.875)--(4.714,2.912)--(4.777,2.950)%
  --(4.840,2.988)--(4.903,3.026)--(4.966,3.064)--(5.029,3.101)--(5.092,3.139)--(5.154,3.177)%
  --(5.217,3.215)--(5.280,3.253)--(5.343,3.290)--(5.406,3.328)--(5.469,3.366)--(5.532,3.404)%
  --(5.595,3.442)--(5.658,3.479)--(5.721,3.517)--(5.784,3.555)--(5.847,3.593)--(5.910,3.631)%
  --(5.973,3.668)--(6.036,3.706)--(6.099,3.744)--(6.162,3.782)--(6.224,3.820)--(6.287,3.857)%
  --(6.350,3.895)--(6.413,3.933)--(6.476,3.971)--(6.539,4.009)--(6.602,4.046)--(6.665,4.084)%
  --(6.728,4.122)--(6.791,4.160)--(6.854,4.197)--(6.917,4.235)--(6.980,4.273)--(7.043,4.311)%
  --(7.106,4.349)--(7.169,4.386)--(7.231,4.424)--(7.294,4.462)--(7.357,4.500)--(7.420,4.538)%
  --(7.483,4.575)--(7.546,4.613)--(7.609,4.651)--(7.672,4.689)--(7.735,4.727);
\gpcolor{color=gp lt color border}
\gpsetlinetype{gp lt border}
\draw[gp path] (1.504,5.756)--(1.504,0.985)--(7.735,0.985)--(7.735,5.756)--cycle;
%% coordinates of the plot area
\gpdefrectangularnode{gp plot 1}{\pgfpoint{1.504cm}{0.985cm}}{\pgfpoint{7.735cm}{5.756cm}}
\end{tikzpicture}
%% gnuplot variables

  \caption[Single-photon photoemission from Tungsten]{
    Single-photon photoemission from Tungsten.
    The good linear fit (green) to the relative electron yield (squares) indicates a single-photon process.
    The Fourier plane spot size (squares), related to the initial transverse momentum variance, do not depend on laser pulse energy, as expected.
  }
  \label{fig:single_photon_tungsten}
\end{figure}

An example of single-photon photoemission is presented in \ref{fig:single_photon_tungsten}, wherein a sample of polished tungsten (W), having a work function $\Phi \approx $ 4.3-4.5 eV \cite{yen_thermally_1980}, is used as a photocathode.
When paired with our $\hbar \omega = $ 4.75 eV photon energy laser (see Section \ref{sec:laser}) the excess photoemission energy $\Delta E \approx $ 0.35 eV.
The linear dependence of the electron yield (circles) on the laser pulse energy is indicative of a single photon process, as expected since each absorbed photon may eject one electron.
In this experiment, the yield is determined by integrating the total counts recorded on the CCD detector and therefore is only a relative measurement.
The fact that the Fourier plane spot-size (squares), directly related the initial transverse momentum variance $\Delta p_{\smallT}$, is unchanged over the full range of laser power is also expected, since by $\Delta p_{\smallT} = \sqrt{ m \Delta E / 3 } $ \cite{dowell_quantum_2009}, $\Delta p_{\smallT}$ only depends on $\Delta E$.

In order to improve performance, one must be able to minimize $\Delta p_{\smallT}$ without negatively affecting the photoemission efficiency, and thus electron yield.
For single-photon photoemission, although the polarization-dependent reflectivity can be tweaked to improve performance slightly \cite{berger_dc_2009}, due to the quadratic dependence of the efficiency on excess photoemission energy \cite{shalaev_electron_1994}, any reduction in $\Delta p_{\smallT}$ gained by reducing $\Delta E$ will be mitigated by a large drop in electron yield.
For this reason, as it stands currently and as we have already seen in Section \ref{sec:must_reduce_transverse_momentum}, flat metal photocathodes employing single-photon photoemission are not likely to be useful for UEM.
However, as I will show in Section \ref{sec:photocathode_future}, this does not quite tell the complete story.

