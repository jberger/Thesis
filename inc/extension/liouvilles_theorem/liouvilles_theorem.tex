%This work is licensed under the Creative Commons Attribution-NonCommercial-NoDerivs 3.0 United States License. To view a copy of this license, visit http://creativecommons.org/licenses/by-nc-nd/3.0/us/ or send a letter to Creative Commons, 444 Castro Street, Suite 900, Mountain View, California, 94041, USA.

\section{Implications of Liouville's Theorem}

One of the most well known theorems in physics (and mathematics) is Liouville's Theorem.
This theorem can be summarized succinctly as a `the total phase space of a system is conserved.'
%TODO find a good reference for this

In the AG model this phase space product is simply $\sigma_{\alpha} \eta_{\alpha}$
To evaluate the effect of this theorem on these equations, consider the time derivative of this product
Using \ref{eq:ag_original} it is easy to show that
\begin{gather}
  \frac{d}{dt} (\sigma_{\alpha} \eta_{\alpha}) = \sigma_{\alpha} \frac{d \eta_{\alpha}}{dt} + \eta_{\alpha} \frac{d \sigma_{\alpha}}{dt} = 0\\
  \label{eq:define_big_gamma}
  \therefore \sigma_{\alpha} \eta_{\alpha} \equiv \Gamma_{\alpha}^2 = \text{Constant}
\end{gather}
showing that the Analytic Gaussian model explicitly obeys the Liouville Theorem.

Not only is this an interesting result, but it can be both instructive and computationally useful.
By rewriting \ref{eq:define_big_gamma} as $\eta_{\alpha} = \Gamma_{\alpha}^2 / \sigma_{\alpha}$, evaulating at some time (presumably $t=0$) and substituting into \ref{eq:ag_original}.
\begin{subequations} \label{eq:ag_big_gamma}
\begin{gather}
  \frac{d\sigma_{\alpha}}{dt} = \frac{2\gamma_{\alpha}}{m} \\
  \frac{d\gamma_{\alpha}}{dt} = \frac{ \Gamma_{\alpha}^2(0) + \gamma_{\alpha}^2 }{\sigma_{\alpha} m}
    + \frac{N e^2}{4\pi\varepsilon_0} \frac{1}{6 \sqrt{\sigma_{\alpha}\pi}} L_{\alpha}(\xi)
\end{gather}
\end{subequations}
these four equations now contain all of the information previously contained in six.
Having fewer equations, and fewer mathematical operations, makes this set of equations much less computationally intensive.
Further, now one can see that these $\Gamma$ constants can be thought of as some minimum/intrinsic action of the pulse. 
This will become more apparent when in future sections
%TODO include internal reference
it is shown that at the focal point of a lens, $\gamma_{\alpha}$ goes through zero.

In the transverse dimension, this is consistent with the definition of `coherent fluence' employed by Reed et al.\cite{reed_evolution_2009} for time-resolved electron microscopy and, of course, the spatial emittance of an electron beam \cite{jensen_theoretical_2006,siwick_ultrafast_2002}.
