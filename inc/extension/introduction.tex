In Chapter \ref{chap:ms_model}, I have introduced the AG model, originally presented by Michalik and Sipe~\cite{michalik_analytic_2006}.
Although it is a very impressive model and a very useful starting point, because it cannot include the influence of external forces on the pulse dynamics, it does not provide the complete picture of the evolution of an electron pulse through an entire microscope column.
Further, it does not contain a set of initial conditions which reflect a realistic photoemission process.

In the following chapter, I will derive a set of initial conditions which, though simplistic, should be sufficient to simulate single-photon photoemission from a metal, and which may additionally serve as a basis for future sets of initial conditions for other photoemission processes.
I will then present a model which builds on and extends the AG model to include the influence of external forces on pulses.
I will also include the mathematical representations of magnetic lenses, TM$_{010}$ RF cavities, and DC accelerators.
With these computational tools it is possible to form one model which has the ability to simulate a UEM column.
Indeed I have implemented this model in the Perl programming language, in a form generic enough to easily model many different electron column configurations and yet is computationally fast enough to be used to design and optimize a column.
