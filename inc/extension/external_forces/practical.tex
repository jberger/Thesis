%This work is licensed under the Creative Commons Attribution-NonCommercial-NoDerivs 3.0 United States License. To view a copy of this license, visit http://creativecommons.org/licenses/by-nc-nd/3.0/us/ or send a letter to Creative Commons, 444 Castro Street, Suite 900, Mountain View, California, 94041, USA.

\section{Practical Extension to Include Magnetic Lenses, RF Cavities and DC Accelerators}

The mathematical formalism developed in Section \ref{sec:external_forces} is only an abstract mechanism for including external forces.
To model electron pulses traveling through realistic electron microcope columns, these abstract mechanisms must be developed further to properly describe each specific electron-optical element.
In the following section, I demonstrate concrete applications of the extended AG model, including magnetic lenses, radio-frequency (RF) cavities (useful for both pulse compression and acceleration %TODO refs
) and DC acceleration regions (i.e. electron gun dynamics).

\subsection{Region of Influence}

%TODO perhaps this shouldn't be a redefinition, but some other function
Of course, physical electron-optical elements occupy only a finite space in the column, and thus their influence on the pulse should be limited to a region near the element itself.
To localize the mathematical result shown in Section \ref{sec:external_forces}, it is convenient to redefine a particular $M_{\alpha}$ to include a super-Gaussian envelope in $z^{\prime}$, the position of the peak of the pulse in the lab frame;
\begin{equation} \label{eq:reg_of_influence}
  M_{\alpha} \to M_{\alpha} \exp \left [ - \left (  \frac{ z^{\prime} - z_{lens}^{\prime} }{ L_{lens} / 2 } \right )^{ 2 n } \right ] \text{ ,}
\end{equation}
where $z_{lens}^{\prime}$ is the position of the center of the lens, $L_{lens}$ is the axial length of the lens and $n$ is the super-Gaussian order parameter.
The order parameter defines the ``sharpness'' of the edge of the region of influence with integer values ranging from 1 (a simple Gaussian) through $\infty$ (a top-hat).
We connect $z^{\prime}$ to $t$ through $ v_{\smallzero} $; the kinetic equations are simply defined for the position of the peak of the pulse.

\subsection{Magnetic Lenses} \label{sec:mag_lens_model}

A simple magnetic lens is just a solenoid whose axis is collinear with the axis of the microscope column.
%Initially one would think that electrons moving parallel to this axis would not be affected by the magnetic field lines in the solenoid, however the contribution of the fringing field must be taken into consideration.
As electrons propagate through the solenoid the fringing magnetic field imparts an azimuthal velocity.
This ``tumbling'' around the axis, along with the axial magnetic field, is what causes the inward-directed force. 

The full description of the motion of an electron (in cylindrical coordinates $r, \theta, z$) in an azimuthally symmetric magnetic field ($\vec{B} = B_r \hat{r} + B_z \hat{z}$) is easily derived \cite{el-kareh_electron_1970},
\begin{subequations} \label{eq:lens_eq_of_motion}
\begin{gather}
  m \ddot{r} = -q B_z r \dot{\theta} + m r \dot{\theta}^2                 \\
  \frac{d}{dt} ( m r^2 \dot{\theta} ) = q r \dot{r} B_z - q r B_r \dot{z} \\
  m \ddot{z} = q B_r r \dot{\theta}
\end{gather}
\end{subequations}
where each dot denotes a time derivative.
Attempting to further seperate these equations quickly highlights the tightly coupled nature of the motion of the electron in this system.
Compounding the problem, the full description of the off-axis field of a real (non-infinite) air-core solenoid is also surprisingly complicated.
Montgomery and Terrell \cite{montgomery_some_1961} provide a full treatment for many lens configurations, including a tractable treatment for a single current loop of radius $a$, again in cylindrical coordinates, 
\begin{subequations} \label{eq:field_of_loop}
\begin{gather}
  \frac{B_z}{\mu} = H_z = \frac{2I}{\sqrt{Q}} \left(   F(k) + \frac{ a^2 - r^2 - z^2 }{ (1-k^2) Q } E(k) \right) \\
  \frac{B_r}{\mu} = H_r = \frac{2I}{\sqrt{Q}} \left( - F(k) + \frac{ a^2 + r^2 + z^2 }{ (1-k^2) Q } E(k) \right) \\
  Q \equiv (a+r)^2 + z^2 \\
  k \equiv \sqrt{ 4 a r / Q }
\end{gather}
\end{subequations}
%%TODO I can't decide whether the Q and k definitions should be numbered or not
%where
%\begin{gather*}
%  Q \equiv (a+r)^2 + z^2 \\
%  k \equiv \sqrt{ 4 a r / Q }
%\end{gather*}
where $\mu$ is the local magnetic permability, $H$ is the auxiliary field, and $F(k)$ and $E(k)$ are the complete elliptic integrals of the first and second kind respectively.
In the vacuum, $\mu$ is simply the usual free space permeability $\mu_0$; in Section \ref{sec:mag_lens}, we will see how this permeability can be used to improve the performance of magnetic lenses.
An even more accurate solenoid can be modeled by appropriately summing the fields of many such loops.

For the AG model, a magnetic lens may be modeled using one linear force contribution in the transverse ($ \smallT $) direction.
Unfortunately, it is not practical to write an analytic form of the magnitude of this term.
As seen in \ref{eq:field_of_loop} the field distribution is surprisingly complicated to describe and the ``lensing'' action is in fact a compound force (first rotating the beam, then lensing).
I have found that in practice, characterizing a specific magnetic lens ($M_{\smallT}^{mag}(I)$ where $I$ is the current in the magnetic lens) is much easier than attempting to find a generic functional form.
This may be done in comparison to either experimental data or, as shown in \ref{fig:mag_lens_loops}, to numerical simulations based on \ref{eq:field_of_loop}.

\begin{figure}
  \centering
  \begin{tikzpicture}[gnuplot]
%% generated with GNUPLOT 4.6p0 (Lua 5.1; terminal rev. 99, script rev. 100)
%% Thu 07 Mar 2013 02:33:49 PM CST
\gpsolidlines
\path (0.000,0.000) rectangle (12.500,8.750);
\gpcolor{color=gp lt color border}
\gpsetlinetype{gp lt border}
\gpsetlinewidth{1.00}
\draw[gp path] (1.688,0.985)--(1.868,0.985);
\draw[gp path] (11.947,0.985)--(11.767,0.985);
\node[gp node right] at (1.504,0.985) { 0};
\draw[gp path] (1.688,2.218)--(1.868,2.218);
\draw[gp path] (11.947,2.218)--(11.767,2.218);
\node[gp node right] at (1.504,2.218) { 200};
\draw[gp path] (1.688,3.450)--(1.868,3.450);
\draw[gp path] (11.947,3.450)--(11.767,3.450);
\node[gp node right] at (1.504,3.450) { 400};
\draw[gp path] (1.688,4.683)--(1.868,4.683);
\draw[gp path] (11.947,4.683)--(11.767,4.683);
\node[gp node right] at (1.504,4.683) { 600};
\draw[gp path] (1.688,5.916)--(1.868,5.916);
\draw[gp path] (11.947,5.916)--(11.767,5.916);
\node[gp node right] at (1.504,5.916) { 800};
\draw[gp path] (1.688,7.148)--(1.868,7.148);
\draw[gp path] (11.947,7.148)--(11.767,7.148);
\node[gp node right] at (1.504,7.148) { 1000};
\draw[gp path] (1.688,8.381)--(1.868,8.381);
\draw[gp path] (11.947,8.381)--(11.767,8.381);
\node[gp node right] at (1.504,8.381) { 1200};
\draw[gp path] (1.688,0.985)--(1.688,1.165);
\draw[gp path] (1.688,8.381)--(1.688,8.201);
\node[gp node center] at (1.688,0.677) {-8};
\draw[gp path] (2.970,0.985)--(2.970,1.165);
\draw[gp path] (2.970,8.381)--(2.970,8.201);
\node[gp node center] at (2.970,0.677) {-6};
\draw[gp path] (4.253,0.985)--(4.253,1.165);
\draw[gp path] (4.253,8.381)--(4.253,8.201);
\node[gp node center] at (4.253,0.677) {-4};
\draw[gp path] (5.535,0.985)--(5.535,1.165);
\draw[gp path] (5.535,8.381)--(5.535,8.201);
\node[gp node center] at (5.535,0.677) {-2};
\draw[gp path] (6.818,0.985)--(6.818,1.165);
\draw[gp path] (6.818,8.381)--(6.818,8.201);
\node[gp node center] at (6.818,0.677) { 0};
\draw[gp path] (8.100,0.985)--(8.100,1.165);
\draw[gp path] (8.100,8.381)--(8.100,8.201);
\node[gp node center] at (8.100,0.677) { 2};
\draw[gp path] (9.382,0.985)--(9.382,1.165);
\draw[gp path] (9.382,8.381)--(9.382,8.201);
\node[gp node center] at (9.382,0.677) { 4};
\draw[gp path] (10.665,0.985)--(10.665,1.165);
\draw[gp path] (10.665,8.381)--(10.665,8.201);
\node[gp node center] at (10.665,0.677) { 6};
\draw[gp path] (11.947,0.985)--(11.947,1.165);
\draw[gp path] (11.947,8.381)--(11.947,8.201);
\node[gp node center] at (11.947,0.677) { 8};
\draw[gp path] (1.688,8.381)--(1.688,0.985)--(11.947,0.985)--(11.947,8.381)--cycle;
\node[gp node center,rotate=-270] at (0.246,4.683) {Radial Ray Position, Beam Width ($\mu$m)};
\node[gp node center] at (6.817,0.215) {Position in column (cm)};
\gpcolor{color=gp lt color 1}
\gpsetlinetype{gp lt plot 1}
\draw[gp path] (2.746,2.012)--(2.790,2.022)--(2.834,2.032)--(2.878,2.042)--(2.922,2.052)%
  --(2.966,2.063)--(3.009,2.073)--(3.053,2.083)--(3.097,2.093)--(3.141,2.103)--(3.185,2.113)%
  --(3.229,2.123)--(3.273,2.133)--(3.317,2.143)--(3.361,2.153)--(3.405,2.163)--(3.449,2.173)%
  --(3.493,2.183)--(3.537,2.193)--(3.580,2.204)--(3.624,2.214)--(3.668,2.224)--(3.712,2.234)%
  --(3.756,2.244)--(3.800,2.254)--(3.844,2.264)--(3.888,2.274)--(3.932,2.284)--(3.976,2.294)%
  --(4.020,2.304)--(4.064,2.314)--(4.108,2.324)--(4.151,2.334)--(4.195,2.344)--(4.239,2.355)%
  --(4.283,2.365)--(4.327,2.375)--(4.371,2.385)--(4.415,2.395)--(4.459,2.405)--(4.503,2.415)%
  --(4.547,2.425)--(4.591,2.435)--(4.635,2.445)--(4.678,2.455)--(4.722,2.465)--(4.766,2.475)%
  --(4.810,2.485)--(4.854,2.496)--(4.898,2.506)--(4.942,2.516)--(4.986,2.526)--(5.030,2.536)%
  --(5.074,2.546)--(5.118,2.556)--(5.162,2.566)--(5.206,2.576)--(5.249,2.586)--(5.293,2.596)%
  --(5.337,2.606)--(5.381,2.616)--(5.425,2.626)--(5.469,2.636)--(5.513,2.646)--(5.557,2.657)%
  --(5.601,2.667)--(5.645,2.677)--(5.689,2.687)--(5.733,2.697)--(5.777,2.707)--(5.820,2.717)%
  --(5.864,2.727)--(5.908,2.737)--(5.952,2.747)--(5.996,2.757)--(6.040,2.767)--(6.084,2.777)%
  --(6.128,2.787)--(6.172,2.797)--(6.216,2.807)--(6.260,2.817)--(6.304,2.826)--(6.348,2.836)%
  --(6.391,2.846)--(6.435,2.855)--(6.479,2.864)--(6.523,2.873)--(6.567,2.881)--(6.611,2.889)%
  --(6.655,2.896)--(6.699,2.901)--(6.743,2.906)--(6.787,2.908)--(6.831,2.909)--(6.875,2.908)%
  --(6.918,2.905)--(6.962,2.900)--(7.006,2.893)--(7.050,2.886)--(7.094,2.877)--(7.138,2.868)%
  --(7.182,2.858)--(7.226,2.848)--(7.270,2.838)--(7.314,2.827)--(7.358,2.816)--(7.402,2.805)%
  --(7.445,2.794)--(7.489,2.783)--(7.533,2.772)--(7.577,2.761)--(7.621,2.750)--(7.665,2.739)%
  --(7.709,2.728)--(7.753,2.716)--(7.797,2.705)--(7.841,2.694)--(7.885,2.683)--(7.929,2.672)%
  --(7.973,2.661)--(8.016,2.649)--(8.060,2.638)--(8.104,2.627)--(8.148,2.616)--(8.192,2.605)%
  --(8.236,2.593)--(8.280,2.582)--(8.324,2.571)--(8.368,2.560)--(8.412,2.549)--(8.456,2.537)%
  --(8.500,2.526)--(8.544,2.515)--(8.587,2.504)--(8.631,2.493)--(8.675,2.481)--(8.719,2.470)%
  --(8.763,2.459)--(8.807,2.448)--(8.851,2.437)--(8.895,2.425)--(8.939,2.414)--(8.983,2.403)%
  --(9.027,2.392)--(9.071,2.381)--(9.115,2.369)--(9.158,2.358)--(9.202,2.347)--(9.246,2.336)%
  --(9.290,2.325)--(9.334,2.313)--(9.378,2.302)--(9.422,2.291)--(9.466,2.280)--(9.510,2.269)%
  --(9.554,2.257)--(9.598,2.246)--(9.642,2.235)--(9.685,2.224)--(9.729,2.213)--(9.773,2.201)%
  --(9.817,2.190)--(9.861,2.179)--(9.905,2.168)--(9.949,2.157)--(9.993,2.145)--(10.037,2.134)%
  --(10.081,2.123)--(10.125,2.112)--(10.169,2.101)--(10.213,2.089)--(10.256,2.078)--(10.300,2.067)%
  --(10.344,2.056)--(10.388,2.045)--(10.432,2.033)--(10.476,2.022)--(10.520,2.011)--(10.564,2.000)%
  --(10.608,1.989)--(10.652,1.977)--(10.696,1.966)--(10.740,1.955)--(10.784,1.944)--(10.827,1.933)%
  --(10.871,1.921)--(10.915,1.910)--(10.959,1.899)--(11.003,1.888)--(11.047,1.877)--(11.091,1.865)%
  --(11.135,1.854)--(11.179,1.843)--(11.223,1.832)--(11.267,1.821)--(11.311,1.809)--(11.354,1.798)%
  --(11.398,1.787)--(11.442,1.776)--(11.486,1.765)--(11.530,1.753);
\draw[gp path] (2.746,2.012)--(2.790,2.002)--(2.834,1.992)--(2.878,1.982)--(2.922,1.972)%
  --(2.966,1.962)--(3.009,1.952)--(3.053,1.942)--(3.097,1.932)--(3.141,1.922)--(3.185,1.912)%
  --(3.229,1.901)--(3.273,1.891)--(3.317,1.881)--(3.361,1.871)--(3.405,1.861)--(3.449,1.851)%
  --(3.493,1.841)--(3.537,1.831)--(3.580,1.821)--(3.624,1.811)--(3.668,1.801)--(3.712,1.791)%
  --(3.756,1.781)--(3.800,1.771)--(3.844,1.761)--(3.888,1.750)--(3.932,1.740)--(3.976,1.730)%
  --(4.020,1.720)--(4.064,1.710)--(4.108,1.700)--(4.151,1.690)--(4.195,1.680)--(4.239,1.670)%
  --(4.283,1.660)--(4.327,1.650)--(4.371,1.640)--(4.415,1.630)--(4.459,1.620)--(4.503,1.609)%
  --(4.547,1.599)--(4.591,1.589)--(4.635,1.579)--(4.678,1.569)--(4.722,1.559)--(4.766,1.549)%
  --(4.810,1.539)--(4.854,1.529)--(4.898,1.519)--(4.942,1.509)--(4.986,1.499)--(5.030,1.489)%
  --(5.074,1.479)--(5.118,1.469)--(5.162,1.458)--(5.206,1.448)--(5.249,1.438)--(5.293,1.428)%
  --(5.337,1.418)--(5.381,1.408)--(5.425,1.398)--(5.469,1.388)--(5.513,1.378)--(5.557,1.368)%
  --(5.601,1.358)--(5.645,1.348)--(5.689,1.338)--(5.733,1.328)--(5.777,1.317)--(5.820,1.307)%
  --(5.864,1.297)--(5.908,1.287)--(5.952,1.277)--(5.996,1.267)--(6.040,1.257)--(6.084,1.247)%
  --(6.128,1.237)--(6.172,1.227)--(6.216,1.217)--(6.260,1.207)--(6.304,1.196)--(6.348,1.186)%
  --(6.391,1.176)--(6.435,1.166)--(6.479,1.156)--(6.523,1.146)--(6.567,1.135)--(6.611,1.125)%
  --(6.655,1.115)--(6.699,1.104)--(6.743,1.094)--(6.787,1.083)--(6.831,1.072)--(6.875,1.062)%
  --(6.918,1.051)--(6.962,1.040)--(7.006,1.029)--(7.050,1.018)--(7.094,1.007)--(7.138,0.996)%
  --(7.182,0.988)--(7.226,0.997)--(7.270,1.008)--(7.314,1.019)--(7.358,1.030)--(7.402,1.041)%
  --(7.445,1.052)--(7.489,1.063)--(7.533,1.074)--(7.577,1.085)--(7.621,1.096)--(7.665,1.107)%
  --(7.709,1.118)--(7.753,1.129)--(7.797,1.140)--(7.841,1.151)--(7.885,1.162)--(7.929,1.173)%
  --(7.973,1.184)--(8.016,1.195)--(8.060,1.206)--(8.104,1.217)--(8.148,1.228)--(8.192,1.239)%
  --(8.236,1.250)--(8.280,1.261)--(8.324,1.272)--(8.368,1.283)--(8.412,1.294)--(8.456,1.305)%
  --(8.500,1.316)--(8.544,1.327)--(8.587,1.338)--(8.631,1.349)--(8.675,1.360)--(8.719,1.371)%
  --(8.763,1.382)--(8.807,1.393)--(8.851,1.404)--(8.895,1.415)--(8.939,1.426)--(8.983,1.437)%
  --(9.027,1.448)--(9.071,1.459)--(9.115,1.470)--(9.158,1.481)--(9.202,1.492)--(9.246,1.503)%
  --(9.290,1.514)--(9.334,1.525)--(9.378,1.536)--(9.422,1.547)--(9.466,1.558)--(9.510,1.569)%
  --(9.554,1.580)--(9.598,1.591)--(9.642,1.602)--(9.685,1.613)--(9.729,1.624)--(9.773,1.635)%
  --(9.817,1.646)--(9.861,1.657)--(9.905,1.668)--(9.949,1.679)--(9.993,1.690)--(10.037,1.701)%
  --(10.081,1.712)--(10.125,1.723)--(10.169,1.735)--(10.213,1.746)--(10.256,1.757)--(10.300,1.768)%
  --(10.344,1.779)--(10.388,1.790)--(10.432,1.801)--(10.476,1.812)--(10.520,1.823)--(10.564,1.834)%
  --(10.608,1.845)--(10.652,1.856)--(10.696,1.867)--(10.740,1.878)--(10.784,1.889)--(10.827,1.900)%
  --(10.871,1.911)--(10.915,1.922)--(10.959,1.933)--(11.003,1.944)--(11.047,1.955)--(11.091,1.966)%
  --(11.135,1.977)--(11.179,1.988)--(11.223,1.999)--(11.267,2.010)--(11.311,2.021)--(11.354,2.032)%
  --(11.398,2.043)--(11.442,2.054)--(11.486,2.065)--(11.530,2.076);
\draw[gp path] (2.746,2.012)--(2.790,2.012)--(2.834,2.012)--(2.878,2.012)--(2.922,2.012)%
  --(2.966,2.012)--(3.009,2.012)--(3.053,2.012)--(3.097,2.012)--(3.141,2.012)--(3.185,2.012)%
  --(3.229,2.012)--(3.273,2.012)--(3.317,2.012)--(3.361,2.012)--(3.405,2.012)--(3.449,2.012)%
  --(3.493,2.012)--(3.537,2.012)--(3.580,2.012)--(3.624,2.012)--(3.668,2.012)--(3.712,2.012)%
  --(3.756,2.012)--(3.800,2.012)--(3.844,2.012)--(3.888,2.012)--(3.932,2.012)--(3.976,2.012)%
  --(4.020,2.012)--(4.064,2.012)--(4.108,2.012)--(4.151,2.012)--(4.195,2.012)--(4.239,2.012)%
  --(4.283,2.012)--(4.327,2.012)--(4.371,2.012)--(4.415,2.012)--(4.459,2.012)--(4.503,2.012)%
  --(4.547,2.012)--(4.591,2.012)--(4.635,2.012)--(4.678,2.012)--(4.722,2.012)--(4.766,2.012)%
  --(4.810,2.012)--(4.854,2.012)--(4.898,2.012)--(4.942,2.012)--(4.986,2.012)--(5.030,2.012)%
  --(5.074,2.012)--(5.118,2.012)--(5.162,2.012)--(5.206,2.012)--(5.249,2.012)--(5.293,2.012)%
  --(5.337,2.012)--(5.381,2.012)--(5.425,2.012)--(5.469,2.012)--(5.513,2.012)--(5.557,2.012)%
  --(5.601,2.012)--(5.645,2.012)--(5.689,2.012)--(5.733,2.012)--(5.777,2.012)--(5.820,2.012)%
  --(5.864,2.012)--(5.908,2.012)--(5.952,2.012)--(5.996,2.012)--(6.040,2.012)--(6.084,2.012)%
  --(6.128,2.012)--(6.172,2.012)--(6.216,2.012)--(6.260,2.012)--(6.304,2.011)--(6.348,2.011)%
  --(6.391,2.011)--(6.435,2.011)--(6.479,2.010)--(6.523,2.009)--(6.567,2.008)--(6.611,2.007)%
  --(6.655,2.005)--(6.699,2.003)--(6.743,2.000)--(6.787,1.996)--(6.831,1.991)--(6.875,1.985)%
  --(6.918,1.978)--(6.962,1.970)--(7.006,1.961)--(7.050,1.952)--(7.094,1.942)--(7.138,1.932)%
  --(7.182,1.921)--(7.226,1.911)--(7.270,1.900)--(7.314,1.889)--(7.358,1.878)--(7.402,1.867)%
  --(7.445,1.856)--(7.489,1.845)--(7.533,1.834)--(7.577,1.823)--(7.621,1.812)--(7.665,1.801)%
  --(7.709,1.790)--(7.753,1.779)--(7.797,1.768)--(7.841,1.757)--(7.885,1.746)--(7.929,1.735)%
  --(7.973,1.724)--(8.016,1.713)--(8.060,1.701)--(8.104,1.690)--(8.148,1.679)--(8.192,1.668)%
  --(8.236,1.657)--(8.280,1.646)--(8.324,1.635)--(8.368,1.624)--(8.412,1.613)--(8.456,1.602)%
  --(8.500,1.591)--(8.544,1.579)--(8.587,1.568)--(8.631,1.557)--(8.675,1.546)--(8.719,1.535)%
  --(8.763,1.524)--(8.807,1.513)--(8.851,1.502)--(8.895,1.491)--(8.939,1.480)--(8.983,1.468)%
  --(9.027,1.457)--(9.071,1.446)--(9.115,1.435)--(9.158,1.424)--(9.202,1.413)--(9.246,1.402)%
  --(9.290,1.391)--(9.334,1.380)--(9.378,1.369)--(9.422,1.357)--(9.466,1.346)--(9.510,1.335)%
  --(9.554,1.324)--(9.598,1.313)--(9.642,1.302)--(9.685,1.291)--(9.729,1.280)--(9.773,1.269)%
  --(9.817,1.258)--(9.861,1.246)--(9.905,1.235)--(9.949,1.224)--(9.993,1.213)--(10.037,1.202)%
  --(10.081,1.191)--(10.125,1.180)--(10.169,1.169)--(10.213,1.158)--(10.256,1.147)--(10.300,1.136)%
  --(10.344,1.124)--(10.388,1.113)--(10.432,1.102)--(10.476,1.091)--(10.520,1.080)--(10.564,1.069)%
  --(10.608,1.058)--(10.652,1.047)--(10.696,1.036)--(10.740,1.025)--(10.784,1.014)--(10.827,1.003)%
  --(10.871,0.992)--(10.915,0.991)--(10.959,1.001)--(11.003,1.012)--(11.047,1.023)--(11.091,1.034)%
  --(11.135,1.045)--(11.179,1.057)--(11.223,1.068)--(11.267,1.079)--(11.311,1.090)--(11.354,1.101)%
  --(11.398,1.112)--(11.442,1.123)--(11.486,1.134)--(11.530,1.145);
\draw[gp path] (2.746,3.039)--(2.790,3.050)--(2.834,3.060)--(2.878,3.070)--(2.922,3.080)%
  --(2.966,3.090)--(3.009,3.100)--(3.053,3.110)--(3.097,3.120)--(3.141,3.130)--(3.185,3.140)%
  --(3.229,3.150)--(3.273,3.160)--(3.317,3.170)--(3.361,3.180)--(3.405,3.190)--(3.449,3.201)%
  --(3.493,3.211)--(3.537,3.221)--(3.580,3.231)--(3.624,3.241)--(3.668,3.251)--(3.712,3.261)%
  --(3.756,3.271)--(3.800,3.281)--(3.844,3.291)--(3.888,3.301)--(3.932,3.311)--(3.976,3.321)%
  --(4.020,3.331)--(4.064,3.341)--(4.108,3.352)--(4.151,3.362)--(4.195,3.372)--(4.239,3.382)%
  --(4.283,3.392)--(4.327,3.402)--(4.371,3.412)--(4.415,3.422)--(4.459,3.432)--(4.503,3.442)%
  --(4.547,3.452)--(4.591,3.462)--(4.635,3.472)--(4.678,3.482)--(4.722,3.493)--(4.766,3.503)%
  --(4.810,3.513)--(4.854,3.523)--(4.898,3.533)--(4.942,3.543)--(4.986,3.553)--(5.030,3.563)%
  --(5.074,3.573)--(5.118,3.583)--(5.162,3.593)--(5.206,3.603)--(5.249,3.613)--(5.293,3.623)%
  --(5.337,3.633)--(5.381,3.644)--(5.425,3.654)--(5.469,3.664)--(5.513,3.674)--(5.557,3.684)%
  --(5.601,3.694)--(5.645,3.704)--(5.689,3.714)--(5.733,3.724)--(5.777,3.734)--(5.820,3.744)%
  --(5.864,3.754)--(5.908,3.764)--(5.952,3.774)--(5.996,3.784)--(6.040,3.794)--(6.084,3.804)%
  --(6.128,3.814)--(6.172,3.824)--(6.216,3.834)--(6.260,3.843)--(6.304,3.853)--(6.348,3.862)%
  --(6.391,3.872)--(6.435,3.881)--(6.479,3.889)--(6.523,3.897)--(6.567,3.905)--(6.611,3.911)%
  --(6.655,3.916)--(6.699,3.919)--(6.743,3.920)--(6.787,3.919)--(6.831,3.915)--(6.875,3.907)%
  --(6.918,3.897)--(6.962,3.884)--(7.006,3.869)--(7.050,3.852)--(7.094,3.834)--(7.138,3.814)%
  --(7.182,3.794)--(7.226,3.773)--(7.270,3.752)--(7.314,3.731)--(7.358,3.709)--(7.402,3.687)%
  --(7.445,3.665)--(7.489,3.643)--(7.533,3.621)--(7.577,3.598)--(7.621,3.576)--(7.665,3.554)%
  --(7.709,3.532)--(7.753,3.509)--(7.797,3.487)--(7.841,3.465)--(7.885,3.442)--(7.929,3.420)%
  --(7.973,3.398)--(8.016,3.375)--(8.060,3.353)--(8.104,3.331)--(8.148,3.308)--(8.192,3.286)%
  --(8.236,3.264)--(8.280,3.241)--(8.324,3.219)--(8.368,3.196)--(8.412,3.174)--(8.456,3.152)%
  --(8.500,3.129)--(8.544,3.107)--(8.587,3.085)--(8.631,3.062)--(8.675,3.040)--(8.719,3.018)%
  --(8.763,2.995)--(8.807,2.973)--(8.851,2.951)--(8.895,2.928)--(8.939,2.906)--(8.983,2.883)%
  --(9.027,2.861)--(9.071,2.839)--(9.115,2.816)--(9.158,2.794)--(9.202,2.772)--(9.246,2.749)%
  --(9.290,2.727)--(9.334,2.705)--(9.378,2.682)--(9.422,2.660)--(9.466,2.637)--(9.510,2.615)%
  --(9.554,2.593)--(9.598,2.570)--(9.642,2.548)--(9.685,2.526)--(9.729,2.503)--(9.773,2.481)%
  --(9.817,2.459)--(9.861,2.436)--(9.905,2.414)--(9.949,2.391)--(9.993,2.369)--(10.037,2.347)%
  --(10.081,2.324)--(10.125,2.302)--(10.169,2.280)--(10.213,2.257)--(10.256,2.235)--(10.300,2.213)%
  --(10.344,2.190)--(10.388,2.168)--(10.432,2.145)--(10.476,2.123)--(10.520,2.101)--(10.564,2.078)%
  --(10.608,2.056)--(10.652,2.034)--(10.696,2.011)--(10.740,1.989)--(10.784,1.967)--(10.827,1.944)%
  --(10.871,1.922)--(10.915,1.900)--(10.959,1.877)--(11.003,1.855)--(11.047,1.832)--(11.091,1.810)%
  --(11.135,1.788)--(11.179,1.765)--(11.223,1.743)--(11.267,1.721)--(11.311,1.698)--(11.354,1.676)%
  --(11.398,1.654)--(11.442,1.631)--(11.486,1.609)--(11.530,1.586);
\draw[gp path] (2.746,3.039)--(2.790,3.029)--(2.834,3.019)--(2.878,3.009)--(2.922,2.999)%
  --(2.966,2.989)--(3.009,2.979)--(3.053,2.969)--(3.097,2.959)--(3.141,2.949)--(3.185,2.939)%
  --(3.229,2.929)--(3.273,2.919)--(3.317,2.909)--(3.361,2.898)--(3.405,2.888)--(3.449,2.878)%
  --(3.493,2.868)--(3.537,2.858)--(3.580,2.848)--(3.624,2.838)--(3.668,2.828)--(3.712,2.818)%
  --(3.756,2.808)--(3.800,2.798)--(3.844,2.788)--(3.888,2.778)--(3.932,2.768)--(3.976,2.758)%
  --(4.020,2.747)--(4.064,2.737)--(4.108,2.727)--(4.151,2.717)--(4.195,2.707)--(4.239,2.697)%
  --(4.283,2.687)--(4.327,2.677)--(4.371,2.667)--(4.415,2.657)--(4.459,2.647)--(4.503,2.637)%
  --(4.547,2.627)--(4.591,2.617)--(4.635,2.606)--(4.678,2.596)--(4.722,2.586)--(4.766,2.576)%
  --(4.810,2.566)--(4.854,2.556)--(4.898,2.546)--(4.942,2.536)--(4.986,2.526)--(5.030,2.516)%
  --(5.074,2.506)--(5.118,2.496)--(5.162,2.486)--(5.206,2.476)--(5.249,2.466)--(5.293,2.455)%
  --(5.337,2.445)--(5.381,2.435)--(5.425,2.425)--(5.469,2.415)--(5.513,2.405)--(5.557,2.395)%
  --(5.601,2.385)--(5.645,2.375)--(5.689,2.365)--(5.733,2.355)--(5.777,2.345)--(5.820,2.335)%
  --(5.864,2.324)--(5.908,2.314)--(5.952,2.304)--(5.996,2.294)--(6.040,2.284)--(6.084,2.274)%
  --(6.128,2.264)--(6.172,2.254)--(6.216,2.243)--(6.260,2.233)--(6.304,2.223)--(6.348,2.213)%
  --(6.391,2.202)--(6.435,2.192)--(6.479,2.181)--(6.523,2.170)--(6.567,2.159)--(6.611,2.147)%
  --(6.655,2.135)--(6.699,2.122)--(6.743,2.108)--(6.787,2.094)--(6.831,2.078)--(6.875,2.061)%
  --(6.918,2.043)--(6.962,2.025)--(7.006,2.005)--(7.050,1.985)--(7.094,1.964)--(7.138,1.943)%
  --(7.182,1.921)--(7.226,1.900)--(7.270,1.878)--(7.314,1.856)--(7.358,1.834)--(7.402,1.812)%
  --(7.445,1.790)--(7.489,1.768)--(7.533,1.746)--(7.577,1.724)--(7.621,1.702)--(7.665,1.680)%
  --(7.709,1.658)--(7.753,1.635)--(7.797,1.613)--(7.841,1.591)--(7.885,1.569)--(7.929,1.547)%
  --(7.973,1.525)--(8.016,1.503)--(8.060,1.481)--(8.104,1.459)--(8.148,1.436)--(8.192,1.414)%
  --(8.236,1.392)--(8.280,1.370)--(8.324,1.348)--(8.368,1.326)--(8.412,1.304)--(8.456,1.282)%
  --(8.500,1.260)--(8.544,1.237)--(8.587,1.215)--(8.631,1.193)--(8.675,1.171)--(8.719,1.149)%
  --(8.763,1.127)--(8.807,1.105)--(8.851,1.083)--(8.895,1.061)--(8.939,1.039)--(8.983,1.017)%
  --(9.027,0.996)--(9.071,0.999)--(9.115,1.021)--(9.158,1.042)--(9.202,1.065)--(9.246,1.087)%
  --(9.290,1.109)--(9.334,1.131)--(9.378,1.153)--(9.422,1.175)--(9.466,1.197)--(9.510,1.219)%
  --(9.554,1.241)--(9.598,1.263)--(9.642,1.285)--(9.685,1.308)--(9.729,1.330)--(9.773,1.352)%
  --(9.817,1.374)--(9.861,1.396)--(9.905,1.418)--(9.949,1.440)--(9.993,1.462)--(10.037,1.484)%
  --(10.081,1.507)--(10.125,1.529)--(10.169,1.551)--(10.213,1.573)--(10.256,1.595)--(10.300,1.617)%
  --(10.344,1.639)--(10.388,1.661)--(10.432,1.684)--(10.476,1.706)--(10.520,1.728)--(10.564,1.750)%
  --(10.608,1.772)--(10.652,1.794)--(10.696,1.816)--(10.740,1.838)--(10.784,1.860)--(10.827,1.883)%
  --(10.871,1.905)--(10.915,1.927)--(10.959,1.949)--(11.003,1.971)--(11.047,1.993)--(11.091,2.015)%
  --(11.135,2.037)--(11.179,2.059)--(11.223,2.082)--(11.267,2.104)--(11.311,2.126)--(11.354,2.148)%
  --(11.398,2.170)--(11.442,2.192)--(11.486,2.214)--(11.530,2.236);
\draw[gp path] (2.746,3.039)--(2.790,3.039)--(2.834,3.039)--(2.878,3.039)--(2.922,3.039)%
  --(2.966,3.039)--(3.009,3.039)--(3.053,3.039)--(3.097,3.039)--(3.141,3.039)--(3.185,3.039)%
  --(3.229,3.039)--(3.273,3.039)--(3.317,3.039)--(3.361,3.039)--(3.405,3.039)--(3.449,3.039)%
  --(3.493,3.039)--(3.537,3.039)--(3.580,3.039)--(3.624,3.039)--(3.668,3.039)--(3.712,3.039)%
  --(3.756,3.039)--(3.800,3.039)--(3.844,3.039)--(3.888,3.039)--(3.932,3.039)--(3.976,3.039)%
  --(4.020,3.039)--(4.064,3.039)--(4.108,3.039)--(4.151,3.039)--(4.195,3.039)--(4.239,3.039)%
  --(4.283,3.039)--(4.327,3.039)--(4.371,3.039)--(4.415,3.039)--(4.459,3.039)--(4.503,3.039)%
  --(4.547,3.039)--(4.591,3.039)--(4.635,3.039)--(4.678,3.039)--(4.722,3.039)--(4.766,3.039)%
  --(4.810,3.039)--(4.854,3.039)--(4.898,3.039)--(4.942,3.039)--(4.986,3.039)--(5.030,3.039)%
  --(5.074,3.039)--(5.118,3.039)--(5.162,3.039)--(5.206,3.039)--(5.249,3.039)--(5.293,3.039)%
  --(5.337,3.039)--(5.381,3.039)--(5.425,3.039)--(5.469,3.039)--(5.513,3.039)--(5.557,3.039)%
  --(5.601,3.039)--(5.645,3.039)--(5.689,3.039)--(5.733,3.039)--(5.777,3.039)--(5.820,3.039)%
  --(5.864,3.039)--(5.908,3.039)--(5.952,3.039)--(5.996,3.039)--(6.040,3.039)--(6.084,3.039)%
  --(6.128,3.039)--(6.172,3.039)--(6.216,3.038)--(6.260,3.038)--(6.304,3.038)--(6.348,3.037)%
  --(6.391,3.037)--(6.435,3.036)--(6.479,3.035)--(6.523,3.034)--(6.567,3.032)--(6.611,3.029)%
  --(6.655,3.026)--(6.699,3.021)--(6.743,3.014)--(6.787,3.006)--(6.831,2.996)--(6.875,2.984)%
  --(6.918,2.970)--(6.962,2.955)--(7.006,2.937)--(7.050,2.919)--(7.094,2.899)--(7.138,2.879)%
  --(7.182,2.858)--(7.226,2.837)--(7.270,2.815)--(7.314,2.793)--(7.358,2.772)--(7.402,2.750)%
  --(7.445,2.728)--(7.489,2.706)--(7.533,2.684)--(7.577,2.661)--(7.621,2.639)--(7.665,2.617)%
  --(7.709,2.595)--(7.753,2.573)--(7.797,2.551)--(7.841,2.528)--(7.885,2.506)--(7.929,2.484)%
  --(7.973,2.462)--(8.016,2.440)--(8.060,2.417)--(8.104,2.395)--(8.148,2.373)--(8.192,2.351)%
  --(8.236,2.328)--(8.280,2.306)--(8.324,2.284)--(8.368,2.262)--(8.412,2.240)--(8.456,2.217)%
  --(8.500,2.195)--(8.544,2.173)--(8.587,2.151)--(8.631,2.129)--(8.675,2.106)--(8.719,2.084)%
  --(8.763,2.062)--(8.807,2.040)--(8.851,2.017)--(8.895,1.995)--(8.939,1.973)--(8.983,1.951)%
  --(9.027,1.929)--(9.071,1.906)--(9.115,1.884)--(9.158,1.862)--(9.202,1.840)--(9.246,1.817)%
  --(9.290,1.795)--(9.334,1.773)--(9.378,1.751)--(9.422,1.729)--(9.466,1.706)--(9.510,1.684)%
  --(9.554,1.662)--(9.598,1.640)--(9.642,1.618)--(9.685,1.595)--(9.729,1.573)--(9.773,1.551)%
  --(9.817,1.529)--(9.861,1.506)--(9.905,1.484)--(9.949,1.462)--(9.993,1.440)--(10.037,1.418)%
  --(10.081,1.395)--(10.125,1.373)--(10.169,1.351)--(10.213,1.329)--(10.256,1.306)--(10.300,1.284)%
  --(10.344,1.262)--(10.388,1.240)--(10.432,1.218)--(10.476,1.195)--(10.520,1.173)--(10.564,1.151)%
  --(10.608,1.129)--(10.652,1.107)--(10.696,1.084)--(10.740,1.062)--(10.784,1.040)--(10.827,1.018)%
  --(10.871,0.997)--(10.915,0.998)--(10.959,1.020)--(11.003,1.042)--(11.047,1.064)--(11.091,1.086)%
  --(11.135,1.108)--(11.179,1.130)--(11.223,1.153)--(11.267,1.175)--(11.311,1.197)--(11.354,1.219)%
  --(11.398,1.241)--(11.442,1.264)--(11.486,1.286)--(11.530,1.308);
\draw[gp path] (2.746,4.067)--(2.790,4.077)--(2.834,4.087)--(2.878,4.097)--(2.922,4.107)%
  --(2.966,4.117)--(3.009,4.127)--(3.053,4.137)--(3.097,4.147)--(3.141,4.157)--(3.185,4.167)%
  --(3.229,4.177)--(3.273,4.187)--(3.317,4.198)--(3.361,4.208)--(3.405,4.218)--(3.449,4.228)%
  --(3.493,4.238)--(3.537,4.248)--(3.580,4.258)--(3.624,4.268)--(3.668,4.278)--(3.712,4.288)%
  --(3.756,4.298)--(3.800,4.308)--(3.844,4.318)--(3.888,4.328)--(3.932,4.339)--(3.976,4.349)%
  --(4.020,4.359)--(4.064,4.369)--(4.108,4.379)--(4.151,4.389)--(4.195,4.399)--(4.239,4.409)%
  --(4.283,4.419)--(4.327,4.429)--(4.371,4.439)--(4.415,4.449)--(4.459,4.459)--(4.503,4.469)%
  --(4.547,4.479)--(4.591,4.490)--(4.635,4.500)--(4.678,4.510)--(4.722,4.520)--(4.766,4.530)%
  --(4.810,4.540)--(4.854,4.550)--(4.898,4.560)--(4.942,4.570)--(4.986,4.580)--(5.030,4.590)%
  --(5.074,4.600)--(5.118,4.610)--(5.162,4.620)--(5.206,4.630)--(5.249,4.641)--(5.293,4.651)%
  --(5.337,4.661)--(5.381,4.671)--(5.425,4.681)--(5.469,4.691)--(5.513,4.701)--(5.557,4.711)%
  --(5.601,4.721)--(5.645,4.731)--(5.689,4.741)--(5.733,4.751)--(5.777,4.761)--(5.820,4.771)%
  --(5.864,4.781)--(5.908,4.791)--(5.952,4.801)--(5.996,4.811)--(6.040,4.821)--(6.084,4.831)%
  --(6.128,4.841)--(6.172,4.851)--(6.216,4.860)--(6.260,4.870)--(6.304,4.879)--(6.348,4.889)%
  --(6.391,4.898)--(6.435,4.906)--(6.479,4.914)--(6.523,4.922)--(6.567,4.928)--(6.611,4.933)%
  --(6.655,4.936)--(6.699,4.937)--(6.743,4.935)--(6.787,4.930)--(6.831,4.921)--(6.875,4.907)%
  --(6.918,4.890)--(6.962,4.869)--(7.006,4.845)--(7.050,4.819)--(7.094,4.790)--(7.138,4.761)%
  --(7.182,4.730)--(7.226,4.698)--(7.270,4.666)--(7.314,4.634)--(7.358,4.601)--(7.402,4.568)%
  --(7.445,4.535)--(7.489,4.501)--(7.533,4.468)--(7.577,4.435)--(7.621,4.401)--(7.665,4.368)%
  --(7.709,4.334)--(7.753,4.301)--(7.797,4.267)--(7.841,4.234)--(7.885,4.200)--(7.929,4.167)%
  --(7.973,4.133)--(8.016,4.100)--(8.060,4.066)--(8.104,4.032)--(8.148,3.999)--(8.192,3.965)%
  --(8.236,3.932)--(8.280,3.898)--(8.324,3.864)--(8.368,3.831)--(8.412,3.797)--(8.456,3.764)%
  --(8.500,3.730)--(8.544,3.697)--(8.587,3.663)--(8.631,3.629)--(8.675,3.596)--(8.719,3.562)%
  --(8.763,3.529)--(8.807,3.495)--(8.851,3.461)--(8.895,3.428)--(8.939,3.394)--(8.983,3.361)%
  --(9.027,3.327)--(9.071,3.293)--(9.115,3.260)--(9.158,3.226)--(9.202,3.193)--(9.246,3.159)%
  --(9.290,3.126)--(9.334,3.092)--(9.378,3.058)--(9.422,3.025)--(9.466,2.991)--(9.510,2.958)%
  --(9.554,2.924)--(9.598,2.890)--(9.642,2.857)--(9.685,2.823)--(9.729,2.790)--(9.773,2.756)%
  --(9.817,2.722)--(9.861,2.689)--(9.905,2.655)--(9.949,2.622)--(9.993,2.588)--(10.037,2.554)%
  --(10.081,2.521)--(10.125,2.487)--(10.169,2.454)--(10.213,2.420)--(10.256,2.387)--(10.300,2.353)%
  --(10.344,2.319)--(10.388,2.286)--(10.432,2.252)--(10.476,2.219)--(10.520,2.185)--(10.564,2.151)%
  --(10.608,2.118)--(10.652,2.084)--(10.696,2.051)--(10.740,2.017)--(10.784,1.983)--(10.827,1.950)%
  --(10.871,1.916)--(10.915,1.883)--(10.959,1.849)--(11.003,1.816)--(11.047,1.782)--(11.091,1.748)%
  --(11.135,1.715)--(11.179,1.681)--(11.223,1.648)--(11.267,1.614)--(11.311,1.580)--(11.354,1.547)%
  --(11.398,1.513)--(11.442,1.480)--(11.486,1.446)--(11.530,1.412);
\draw[gp path] (2.746,4.067)--(2.790,4.057)--(2.834,4.047)--(2.878,4.036)--(2.922,4.026)%
  --(2.966,4.016)--(3.009,4.006)--(3.053,3.996)--(3.097,3.986)--(3.141,3.976)--(3.185,3.966)%
  --(3.229,3.956)--(3.273,3.946)--(3.317,3.936)--(3.361,3.926)--(3.405,3.916)--(3.449,3.906)%
  --(3.493,3.896)--(3.537,3.885)--(3.580,3.875)--(3.624,3.865)--(3.668,3.855)--(3.712,3.845)%
  --(3.756,3.835)--(3.800,3.825)--(3.844,3.815)--(3.888,3.805)--(3.932,3.795)--(3.976,3.785)%
  --(4.020,3.775)--(4.064,3.765)--(4.108,3.755)--(4.151,3.744)--(4.195,3.734)--(4.239,3.724)%
  --(4.283,3.714)--(4.327,3.704)--(4.371,3.694)--(4.415,3.684)--(4.459,3.674)--(4.503,3.664)%
  --(4.547,3.654)--(4.591,3.644)--(4.635,3.634)--(4.678,3.624)--(4.722,3.614)--(4.766,3.604)%
  --(4.810,3.593)--(4.854,3.583)--(4.898,3.573)--(4.942,3.563)--(4.986,3.553)--(5.030,3.543)%
  --(5.074,3.533)--(5.118,3.523)--(5.162,3.513)--(5.206,3.503)--(5.249,3.493)--(5.293,3.483)%
  --(5.337,3.473)--(5.381,3.463)--(5.425,3.452)--(5.469,3.442)--(5.513,3.432)--(5.557,3.422)%
  --(5.601,3.412)--(5.645,3.402)--(5.689,3.392)--(5.733,3.382)--(5.777,3.372)--(5.820,3.362)%
  --(5.864,3.352)--(5.908,3.341)--(5.952,3.331)--(5.996,3.321)--(6.040,3.311)--(6.084,3.301)%
  --(6.128,3.291)--(6.172,3.280)--(6.216,3.270)--(6.260,3.260)--(6.304,3.249)--(6.348,3.239)%
  --(6.391,3.228)--(6.435,3.217)--(6.479,3.206)--(6.523,3.194)--(6.567,3.182)--(6.611,3.169)%
  --(6.655,3.155)--(6.699,3.140)--(6.743,3.123)--(6.787,3.105)--(6.831,3.084)--(6.875,3.061)%
  --(6.918,3.036)--(6.962,3.009)--(7.006,2.981)--(7.050,2.951)--(7.094,2.921)--(7.138,2.889)%
  --(7.182,2.858)--(7.226,2.825)--(7.270,2.793)--(7.314,2.760)--(7.358,2.727)--(7.402,2.694)%
  --(7.445,2.661)--(7.489,2.628)--(7.533,2.595)--(7.577,2.562)--(7.621,2.529)--(7.665,2.495)%
  --(7.709,2.462)--(7.753,2.429)--(7.797,2.396)--(7.841,2.363)--(7.885,2.329)--(7.929,2.296)%
  --(7.973,2.263)--(8.016,2.230)--(8.060,2.196)--(8.104,2.163)--(8.148,2.130)--(8.192,2.097)%
  --(8.236,2.064)--(8.280,2.030)--(8.324,1.997)--(8.368,1.964)--(8.412,1.931)--(8.456,1.897)%
  --(8.500,1.864)--(8.544,1.831)--(8.587,1.798)--(8.631,1.764)--(8.675,1.731)--(8.719,1.698)%
  --(8.763,1.665)--(8.807,1.632)--(8.851,1.598)--(8.895,1.565)--(8.939,1.532)--(8.983,1.499)%
  --(9.027,1.465)--(9.071,1.432)--(9.115,1.399)--(9.158,1.366)--(9.202,1.332)--(9.246,1.299)%
  --(9.290,1.266)--(9.334,1.233)--(9.378,1.200)--(9.422,1.166)--(9.466,1.133)--(9.510,1.100)%
  --(9.554,1.067)--(9.598,1.034)--(9.642,1.002)--(9.685,1.005)--(9.729,1.037)--(9.773,1.070)%
  --(9.817,1.103)--(9.861,1.137)--(9.905,1.170)--(9.949,1.203)--(9.993,1.236)--(10.037,1.269)%
  --(10.081,1.303)--(10.125,1.336)--(10.169,1.369)--(10.213,1.402)--(10.256,1.435)--(10.300,1.469)%
  --(10.344,1.502)--(10.388,1.535)--(10.432,1.568)--(10.476,1.602)--(10.520,1.635)--(10.564,1.668)%
  --(10.608,1.701)--(10.652,1.735)--(10.696,1.768)--(10.740,1.801)--(10.784,1.834)--(10.827,1.868)%
  --(10.871,1.901)--(10.915,1.934)--(10.959,1.967)--(11.003,2.000)--(11.047,2.034)--(11.091,2.067)%
  --(11.135,2.100)--(11.179,2.133)--(11.223,2.167)--(11.267,2.200)--(11.311,2.233)--(11.354,2.266)%
  --(11.398,2.300)--(11.442,2.333)--(11.486,2.366)--(11.530,2.399);
\draw[gp path] (2.746,4.067)--(2.790,4.067)--(2.834,4.067)--(2.878,4.067)--(2.922,4.067)%
  --(2.966,4.067)--(3.009,4.067)--(3.053,4.067)--(3.097,4.067)--(3.141,4.067)--(3.185,4.067)%
  --(3.229,4.067)--(3.273,4.067)--(3.317,4.067)--(3.361,4.067)--(3.405,4.067)--(3.449,4.067)%
  --(3.493,4.067)--(3.537,4.067)--(3.580,4.067)--(3.624,4.067)--(3.668,4.067)--(3.712,4.067)%
  --(3.756,4.067)--(3.800,4.067)--(3.844,4.067)--(3.888,4.067)--(3.932,4.067)--(3.976,4.067)%
  --(4.020,4.067)--(4.064,4.067)--(4.108,4.067)--(4.151,4.067)--(4.195,4.067)--(4.239,4.067)%
  --(4.283,4.067)--(4.327,4.067)--(4.371,4.067)--(4.415,4.067)--(4.459,4.067)--(4.503,4.067)%
  --(4.547,4.067)--(4.591,4.067)--(4.635,4.067)--(4.678,4.067)--(4.722,4.067)--(4.766,4.067)%
  --(4.810,4.067)--(4.854,4.067)--(4.898,4.067)--(4.942,4.067)--(4.986,4.067)--(5.030,4.067)%
  --(5.074,4.067)--(5.118,4.067)--(5.162,4.067)--(5.206,4.067)--(5.249,4.067)--(5.293,4.067)%
  --(5.337,4.067)--(5.381,4.067)--(5.425,4.067)--(5.469,4.067)--(5.513,4.067)--(5.557,4.067)%
  --(5.601,4.067)--(5.645,4.067)--(5.689,4.066)--(5.733,4.066)--(5.777,4.066)--(5.820,4.066)%
  --(5.864,4.066)--(5.908,4.066)--(5.952,4.066)--(5.996,4.066)--(6.040,4.066)--(6.084,4.066)%
  --(6.128,4.066)--(6.172,4.065)--(6.216,4.065)--(6.260,4.065)--(6.304,4.064)--(6.348,4.064)%
  --(6.391,4.063)--(6.435,4.062)--(6.479,4.060)--(6.523,4.058)--(6.567,4.055)--(6.611,4.051)%
  --(6.655,4.046)--(6.699,4.039)--(6.743,4.029)--(6.787,4.017)--(6.831,4.002)--(6.875,3.984)%
  --(6.918,3.963)--(6.962,3.939)--(7.006,3.913)--(7.050,3.885)--(7.094,3.856)--(7.138,3.825)%
  --(7.182,3.794)--(7.226,3.762)--(7.270,3.730)--(7.314,3.697)--(7.358,3.664)--(7.402,3.631)%
  --(7.445,3.598)--(7.489,3.565)--(7.533,3.532)--(7.577,3.499)--(7.621,3.465)--(7.665,3.432)%
  --(7.709,3.399)--(7.753,3.366)--(7.797,3.332)--(7.841,3.299)--(7.885,3.265)--(7.929,3.232)%
  --(7.973,3.199)--(8.016,3.165)--(8.060,3.132)--(8.104,3.099)--(8.148,3.065)--(8.192,3.032)%
  --(8.236,2.998)--(8.280,2.965)--(8.324,2.932)--(8.368,2.898)--(8.412,2.865)--(8.456,2.832)%
  --(8.500,2.798)--(8.544,2.765)--(8.587,2.731)--(8.631,2.698)--(8.675,2.665)--(8.719,2.631)%
  --(8.763,2.598)--(8.807,2.565)--(8.851,2.531)--(8.895,2.498)--(8.939,2.464)--(8.983,2.431)%
  --(9.027,2.398)--(9.071,2.364)--(9.115,2.331)--(9.158,2.297)--(9.202,2.264)--(9.246,2.231)%
  --(9.290,2.197)--(9.334,2.164)--(9.378,2.131)--(9.422,2.097)--(9.466,2.064)--(9.510,2.030)%
  --(9.554,1.997)--(9.598,1.964)--(9.642,1.930)--(9.685,1.897)--(9.729,1.863)--(9.773,1.830)%
  --(9.817,1.797)--(9.861,1.763)--(9.905,1.730)--(9.949,1.697)--(9.993,1.663)--(10.037,1.630)%
  --(10.081,1.596)--(10.125,1.563)--(10.169,1.530)--(10.213,1.496)--(10.256,1.463)--(10.300,1.429)%
  --(10.344,1.396)--(10.388,1.363)--(10.432,1.329)--(10.476,1.296)--(10.520,1.263)--(10.564,1.229)%
  --(10.608,1.196)--(10.652,1.163)--(10.696,1.129)--(10.740,1.096)--(10.784,1.063)--(10.827,1.030)%
  --(10.871,0.999)--(10.915,1.010)--(10.959,1.042)--(11.003,1.075)--(11.047,1.108)--(11.091,1.142)%
  --(11.135,1.175)--(11.179,1.208)--(11.223,1.242)--(11.267,1.275)--(11.311,1.309)--(11.354,1.342)%
  --(11.398,1.375)--(11.442,1.409)--(11.486,1.442)--(11.530,1.475);
\draw[gp path] (2.746,5.094)--(2.790,5.104)--(2.834,5.114)--(2.878,5.124)--(2.922,5.134)%
  --(2.966,5.144)--(3.009,5.154)--(3.053,5.164)--(3.097,5.174)--(3.141,5.185)--(3.185,5.195)%
  --(3.229,5.205)--(3.273,5.215)--(3.317,5.225)--(3.361,5.235)--(3.405,5.245)--(3.449,5.255)%
  --(3.493,5.265)--(3.537,5.275)--(3.580,5.285)--(3.624,5.295)--(3.668,5.305)--(3.712,5.315)%
  --(3.756,5.325)--(3.800,5.336)--(3.844,5.346)--(3.888,5.356)--(3.932,5.366)--(3.976,5.376)%
  --(4.020,5.386)--(4.064,5.396)--(4.108,5.406)--(4.151,5.416)--(4.195,5.426)--(4.239,5.436)%
  --(4.283,5.446)--(4.327,5.456)--(4.371,5.466)--(4.415,5.476)--(4.459,5.487)--(4.503,5.497)%
  --(4.547,5.507)--(4.591,5.517)--(4.635,5.527)--(4.678,5.537)--(4.722,5.547)--(4.766,5.557)%
  --(4.810,5.567)--(4.854,5.577)--(4.898,5.587)--(4.942,5.597)--(4.986,5.607)--(5.030,5.617)%
  --(5.074,5.627)--(5.118,5.638)--(5.162,5.648)--(5.206,5.658)--(5.249,5.668)--(5.293,5.678)%
  --(5.337,5.688)--(5.381,5.698)--(5.425,5.708)--(5.469,5.718)--(5.513,5.728)--(5.557,5.738)%
  --(5.601,5.748)--(5.645,5.758)--(5.689,5.768)--(5.733,5.778)--(5.777,5.788)--(5.820,5.798)%
  --(5.864,5.808)--(5.908,5.818)--(5.952,5.828)--(5.996,5.838)--(6.040,5.848)--(6.084,5.858)%
  --(6.128,5.868)--(6.172,5.877)--(6.216,5.887)--(6.260,5.897)--(6.304,5.906)--(6.348,5.915)%
  --(6.391,5.924)--(6.435,5.932)--(6.479,5.939)--(6.523,5.946)--(6.567,5.952)--(6.611,5.955)%
  --(6.655,5.957)--(6.699,5.955)--(6.743,5.950)--(6.787,5.941)--(6.831,5.927)--(6.875,5.907)%
  --(6.918,5.883)--(6.962,5.854)--(7.006,5.821)--(7.050,5.785)--(7.094,5.746)--(7.138,5.706)%
  --(7.182,5.665)--(7.226,5.622)--(7.270,5.579)--(7.314,5.536)--(7.358,5.492)--(7.402,5.448)%
  --(7.445,5.403)--(7.489,5.359)--(7.533,5.314)--(7.577,5.270)--(7.621,5.225)--(7.665,5.180)%
  --(7.709,5.135)--(7.753,5.091)--(7.797,5.046)--(7.841,5.001)--(7.885,4.956)--(7.929,4.911)%
  --(7.973,4.866)--(8.016,4.821)--(8.060,4.776)--(8.104,4.732)--(8.148,4.687)--(8.192,4.642)%
  --(8.236,4.597)--(8.280,4.552)--(8.324,4.507)--(8.368,4.462)--(8.412,4.417)--(8.456,4.372)%
  --(8.500,4.327)--(8.544,4.283)--(8.587,4.238)--(8.631,4.193)--(8.675,4.148)--(8.719,4.103)%
  --(8.763,4.058)--(8.807,4.013)--(8.851,3.968)--(8.895,3.923)--(8.939,3.878)--(8.983,3.834)%
  --(9.027,3.789)--(9.071,3.744)--(9.115,3.699)--(9.158,3.654)--(9.202,3.609)--(9.246,3.564)%
  --(9.290,3.519)--(9.334,3.474)--(9.378,3.429)--(9.422,3.384)--(9.466,3.340)--(9.510,3.295)%
  --(9.554,3.250)--(9.598,3.205)--(9.642,3.160)--(9.685,3.115)--(9.729,3.070)--(9.773,3.025)%
  --(9.817,2.980)--(9.861,2.935)--(9.905,2.890)--(9.949,2.846)--(9.993,2.801)--(10.037,2.756)%
  --(10.081,2.711)--(10.125,2.666)--(10.169,2.621)--(10.213,2.576)--(10.256,2.531)--(10.300,2.486)%
  --(10.344,2.441)--(10.388,2.396)--(10.432,2.352)--(10.476,2.307)--(10.520,2.262)--(10.564,2.217)%
  --(10.608,2.172)--(10.652,2.127)--(10.696,2.082)--(10.740,2.037)--(10.784,1.992)--(10.827,1.947)%
  --(10.871,1.903)--(10.915,1.858)--(10.959,1.813)--(11.003,1.768)--(11.047,1.723)--(11.091,1.678)%
  --(11.135,1.633)--(11.179,1.588)--(11.223,1.543)--(11.267,1.498)--(11.311,1.454)--(11.354,1.409)%
  --(11.398,1.364)--(11.442,1.319)--(11.486,1.274)--(11.530,1.229);
\draw[gp path] (2.746,5.094)--(2.790,5.084)--(2.834,5.074)--(2.878,5.064)--(2.922,5.054)%
  --(2.966,5.044)--(3.009,5.033)--(3.053,5.023)--(3.097,5.013)--(3.141,5.003)--(3.185,4.993)%
  --(3.229,4.983)--(3.273,4.973)--(3.317,4.963)--(3.361,4.953)--(3.405,4.943)--(3.449,4.933)%
  --(3.493,4.923)--(3.537,4.913)--(3.580,4.903)--(3.624,4.893)--(3.668,4.882)--(3.712,4.872)%
  --(3.756,4.862)--(3.800,4.852)--(3.844,4.842)--(3.888,4.832)--(3.932,4.822)--(3.976,4.812)%
  --(4.020,4.802)--(4.064,4.792)--(4.108,4.782)--(4.151,4.772)--(4.195,4.762)--(4.239,4.752)%
  --(4.283,4.741)--(4.327,4.731)--(4.371,4.721)--(4.415,4.711)--(4.459,4.701)--(4.503,4.691)%
  --(4.547,4.681)--(4.591,4.671)--(4.635,4.661)--(4.678,4.651)--(4.722,4.641)--(4.766,4.631)%
  --(4.810,4.621)--(4.854,4.611)--(4.898,4.601)--(4.942,4.590)--(4.986,4.580)--(5.030,4.570)%
  --(5.074,4.560)--(5.118,4.550)--(5.162,4.540)--(5.206,4.530)--(5.249,4.520)--(5.293,4.510)%
  --(5.337,4.500)--(5.381,4.490)--(5.425,4.480)--(5.469,4.470)--(5.513,4.459)--(5.557,4.449)%
  --(5.601,4.439)--(5.645,4.429)--(5.689,4.419)--(5.733,4.409)--(5.777,4.399)--(5.820,4.389)%
  --(5.864,4.379)--(5.908,4.369)--(5.952,4.358)--(5.996,4.348)--(6.040,4.338)--(6.084,4.328)%
  --(6.128,4.318)--(6.172,4.307)--(6.216,4.297)--(6.260,4.286)--(6.304,4.276)--(6.348,4.265)%
  --(6.391,4.254)--(6.435,4.243)--(6.479,4.231)--(6.523,4.219)--(6.567,4.206)--(6.611,4.191)%
  --(6.655,4.176)--(6.699,4.158)--(6.743,4.138)--(6.787,4.115)--(6.831,4.090)--(6.875,4.061)%
  --(6.918,4.029)--(6.962,3.994)--(7.006,3.957)--(7.050,3.918)--(7.094,3.877)--(7.138,3.836)%
  --(7.182,3.794)--(7.226,3.751)--(7.270,3.707)--(7.314,3.664)--(7.358,3.620)--(7.402,3.576)%
  --(7.445,3.532)--(7.489,3.488)--(7.533,3.443)--(7.577,3.399)--(7.621,3.355)--(7.665,3.310)%
  --(7.709,3.266)--(7.753,3.222)--(7.797,3.177)--(7.841,3.133)--(7.885,3.089)--(7.929,3.044)%
  --(7.973,3.000)--(8.016,2.955)--(8.060,2.911)--(8.104,2.867)--(8.148,2.822)--(8.192,2.778)%
  --(8.236,2.733)--(8.280,2.689)--(8.324,2.645)--(8.368,2.600)--(8.412,2.556)--(8.456,2.511)%
  --(8.500,2.467)--(8.544,2.423)--(8.587,2.378)--(8.631,2.334)--(8.675,2.289)--(8.719,2.245)%
  --(8.763,2.201)--(8.807,2.156)--(8.851,2.112)--(8.895,2.067)--(8.939,2.023)--(8.983,1.979)%
  --(9.027,1.934)--(9.071,1.890)--(9.115,1.845)--(9.158,1.801)--(9.202,1.757)--(9.246,1.712)%
  --(9.290,1.668)--(9.334,1.623)--(9.378,1.579)--(9.422,1.535)--(9.466,1.490)--(9.510,1.446)%
  --(9.554,1.401)--(9.598,1.357)--(9.642,1.313)--(9.685,1.268)--(9.729,1.224)--(9.773,1.179)%
  --(9.817,1.135)--(9.861,1.091)--(9.905,1.047)--(9.949,1.005)--(9.993,1.015)--(10.037,1.058)%
  --(10.081,1.102)--(10.125,1.147)--(10.169,1.191)--(10.213,1.235)--(10.256,1.280)--(10.300,1.324)%
  --(10.344,1.368)--(10.388,1.413)--(10.432,1.457)--(10.476,1.502)--(10.520,1.546)--(10.564,1.590)%
  --(10.608,1.635)--(10.652,1.679)--(10.696,1.724)--(10.740,1.768)--(10.784,1.812)--(10.827,1.857)%
  --(10.871,1.901)--(10.915,1.946)--(10.959,1.990)--(11.003,2.034)--(11.047,2.079)--(11.091,2.123)%
  --(11.135,2.168)--(11.179,2.212)--(11.223,2.256)--(11.267,2.301)--(11.311,2.345)--(11.354,2.390)%
  --(11.398,2.434)--(11.442,2.478)--(11.486,2.523)--(11.530,2.567);
\draw[gp path] (2.746,5.094)--(2.790,5.094)--(2.834,5.094)--(2.878,5.094)--(2.922,5.094)%
  --(2.966,5.094)--(3.009,5.094)--(3.053,5.094)--(3.097,5.094)--(3.141,5.094)--(3.185,5.094)%
  --(3.229,5.094)--(3.273,5.094)--(3.317,5.094)--(3.361,5.094)--(3.405,5.094)--(3.449,5.094)%
  --(3.493,5.094)--(3.537,5.094)--(3.580,5.094)--(3.624,5.094)--(3.668,5.094)--(3.712,5.094)%
  --(3.756,5.094)--(3.800,5.094)--(3.844,5.094)--(3.888,5.094)--(3.932,5.094)--(3.976,5.094)%
  --(4.020,5.094)--(4.064,5.094)--(4.108,5.094)--(4.151,5.094)--(4.195,5.094)--(4.239,5.094)%
  --(4.283,5.094)--(4.327,5.094)--(4.371,5.094)--(4.415,5.094)--(4.459,5.094)--(4.503,5.094)%
  --(4.547,5.094)--(4.591,5.094)--(4.635,5.094)--(4.678,5.094)--(4.722,5.094)--(4.766,5.094)%
  --(4.810,5.094)--(4.854,5.094)--(4.898,5.094)--(4.942,5.094)--(4.986,5.094)--(5.030,5.094)%
  --(5.074,5.094)--(5.118,5.094)--(5.162,5.094)--(5.206,5.094)--(5.249,5.094)--(5.293,5.094)%
  --(5.337,5.094)--(5.381,5.094)--(5.425,5.094)--(5.469,5.094)--(5.513,5.094)--(5.557,5.094)%
  --(5.601,5.094)--(5.645,5.094)--(5.689,5.094)--(5.733,5.094)--(5.777,5.094)--(5.820,5.094)%
  --(5.864,5.093)--(5.908,5.093)--(5.952,5.093)--(5.996,5.093)--(6.040,5.093)--(6.084,5.093)%
  --(6.128,5.093)--(6.172,5.092)--(6.216,5.092)--(6.260,5.091)--(6.304,5.091)--(6.348,5.090)%
  --(6.391,5.089)--(6.435,5.087)--(6.479,5.085)--(6.523,5.082)--(6.567,5.079)--(6.611,5.073)%
  --(6.655,5.066)--(6.699,5.057)--(6.743,5.044)--(6.787,5.028)--(6.831,5.008)--(6.875,4.984)%
  --(6.918,4.956)--(6.962,4.924)--(7.006,4.889)--(7.050,4.851)--(7.094,4.812)--(7.138,4.771)%
  --(7.182,4.729)--(7.226,4.687)--(7.270,4.644)--(7.314,4.600)--(7.358,4.556)--(7.402,4.512)%
  --(7.445,4.468)--(7.489,4.424)--(7.533,4.379)--(7.577,4.335)--(7.621,4.290)--(7.665,4.246)%
  --(7.709,4.201)--(7.753,4.157)--(7.797,4.112)--(7.841,4.068)--(7.885,4.023)--(7.929,3.979)%
  --(7.973,3.934)--(8.016,3.889)--(8.060,3.845)--(8.104,3.800)--(8.148,3.756)--(8.192,3.711)%
  --(8.236,3.666)--(8.280,3.622)--(8.324,3.577)--(8.368,3.533)--(8.412,3.488)--(8.456,3.443)%
  --(8.500,3.399)--(8.544,3.354)--(8.587,3.309)--(8.631,3.265)--(8.675,3.220)--(8.719,3.176)%
  --(8.763,3.131)--(8.807,3.086)--(8.851,3.042)--(8.895,2.997)--(8.939,2.953)--(8.983,2.908)%
  --(9.027,2.863)--(9.071,2.819)--(9.115,2.774)--(9.158,2.729)--(9.202,2.685)--(9.246,2.640)%
  --(9.290,2.596)--(9.334,2.551)--(9.378,2.506)--(9.422,2.462)--(9.466,2.417)--(9.510,2.372)%
  --(9.554,2.328)--(9.598,2.283)--(9.642,2.239)--(9.685,2.194)--(9.729,2.149)--(9.773,2.105)%
  --(9.817,2.060)--(9.861,2.016)--(9.905,1.971)--(9.949,1.926)--(9.993,1.882)--(10.037,1.837)%
  --(10.081,1.792)--(10.125,1.748)--(10.169,1.703)--(10.213,1.659)--(10.256,1.614)--(10.300,1.569)%
  --(10.344,1.525)--(10.388,1.480)--(10.432,1.436)--(10.476,1.391)--(10.520,1.346)--(10.564,1.302)%
  --(10.608,1.257)--(10.652,1.213)--(10.696,1.168)--(10.740,1.124)--(10.784,1.079)--(10.827,1.035)%
  --(10.871,0.997)--(10.915,1.027)--(10.959,1.071)--(11.003,1.115)--(11.047,1.160)--(11.091,1.204)%
  --(11.135,1.249)--(11.179,1.293)--(11.223,1.338)--(11.267,1.383)--(11.311,1.427)--(11.354,1.472)%
  --(11.398,1.516)--(11.442,1.561)--(11.486,1.606)--(11.530,1.650);
\draw[gp path] (2.746,6.121)--(2.790,6.131)--(2.834,6.141)--(2.878,6.151)--(2.922,6.161)%
  --(2.966,6.171)--(3.009,6.182)--(3.053,6.192)--(3.097,6.202)--(3.141,6.212)--(3.185,6.222)%
  --(3.229,6.232)--(3.273,6.242)--(3.317,6.252)--(3.361,6.262)--(3.405,6.272)--(3.449,6.282)%
  --(3.493,6.292)--(3.537,6.302)--(3.580,6.312)--(3.624,6.322)--(3.668,6.333)--(3.712,6.343)%
  --(3.756,6.353)--(3.800,6.363)--(3.844,6.373)--(3.888,6.383)--(3.932,6.393)--(3.976,6.403)%
  --(4.020,6.413)--(4.064,6.423)--(4.108,6.433)--(4.151,6.443)--(4.195,6.453)--(4.239,6.463)%
  --(4.283,6.474)--(4.327,6.484)--(4.371,6.494)--(4.415,6.504)--(4.459,6.514)--(4.503,6.524)%
  --(4.547,6.534)--(4.591,6.544)--(4.635,6.554)--(4.678,6.564)--(4.722,6.574)--(4.766,6.584)%
  --(4.810,6.594)--(4.854,6.604)--(4.898,6.614)--(4.942,6.625)--(4.986,6.635)--(5.030,6.645)%
  --(5.074,6.655)--(5.118,6.665)--(5.162,6.675)--(5.206,6.685)--(5.249,6.695)--(5.293,6.705)%
  --(5.337,6.715)--(5.381,6.725)--(5.425,6.735)--(5.469,6.745)--(5.513,6.755)--(5.557,6.765)%
  --(5.601,6.775)--(5.645,6.785)--(5.689,6.795)--(5.733,6.805)--(5.777,6.815)--(5.820,6.825)%
  --(5.864,6.835)--(5.908,6.845)--(5.952,6.855)--(5.996,6.865)--(6.040,6.875)--(6.084,6.885)%
  --(6.128,6.895)--(6.172,6.904)--(6.216,6.914)--(6.260,6.923)--(6.304,6.932)--(6.348,6.941)%
  --(6.391,6.950)--(6.435,6.958)--(6.479,6.965)--(6.523,6.971)--(6.567,6.975)--(6.611,6.978)%
  --(6.655,6.978)--(6.699,6.974)--(6.743,6.966)--(6.787,6.952)--(6.831,6.933)--(6.875,6.907)%
  --(6.918,6.875)--(6.962,6.838)--(7.006,6.796)--(7.050,6.750)--(7.094,6.702)--(7.138,6.651)%
  --(7.182,6.599)--(7.226,6.546)--(7.270,6.491)--(7.314,6.437)--(7.358,6.382)--(7.402,6.326)%
  --(7.445,6.270)--(7.489,6.215)--(7.533,6.159)--(7.577,6.103)--(7.621,6.046)--(7.665,5.990)%
  --(7.709,5.934)--(7.753,5.878)--(7.797,5.822)--(7.841,5.765)--(7.885,5.709)--(7.929,5.653)%
  --(7.973,5.597)--(8.016,5.540)--(8.060,5.484)--(8.104,5.428)--(8.148,5.371)--(8.192,5.315)%
  --(8.236,5.259)--(8.280,5.202)--(8.324,5.146)--(8.368,5.090)--(8.412,5.033)--(8.456,4.977)%
  --(8.500,4.921)--(8.544,4.864)--(8.587,4.808)--(8.631,4.752)--(8.675,4.695)--(8.719,4.639)%
  --(8.763,4.583)--(8.807,4.526)--(8.851,4.470)--(8.895,4.414)--(8.939,4.357)--(8.983,4.301)%
  --(9.027,4.245)--(9.071,4.188)--(9.115,4.132)--(9.158,4.076)--(9.202,4.019)--(9.246,3.963)%
  --(9.290,3.907)--(9.334,3.850)--(9.378,3.794)--(9.422,3.737)--(9.466,3.681)--(9.510,3.625)%
  --(9.554,3.568)--(9.598,3.512)--(9.642,3.456)--(9.685,3.399)--(9.729,3.343)--(9.773,3.287)%
  --(9.817,3.230)--(9.861,3.174)--(9.905,3.118)--(9.949,3.061)--(9.993,3.005)--(10.037,2.949)%
  --(10.081,2.892)--(10.125,2.836)--(10.169,2.780)--(10.213,2.723)--(10.256,2.667)--(10.300,2.611)%
  --(10.344,2.554)--(10.388,2.498)--(10.432,2.442)--(10.476,2.385)--(10.520,2.329)--(10.564,2.273)%
  --(10.608,2.216)--(10.652,2.160)--(10.696,2.104)--(10.740,2.047)--(10.784,1.991)--(10.827,1.935)%
  --(10.871,1.878)--(10.915,1.822)--(10.959,1.766)--(11.003,1.709)--(11.047,1.653)--(11.091,1.597)%
  --(11.135,1.540)--(11.179,1.484)--(11.223,1.428)--(11.267,1.371)--(11.311,1.315)--(11.354,1.259)%
  --(11.398,1.203)--(11.442,1.146)--(11.486,1.090)--(11.530,1.035);
\draw[gp path] (2.746,6.121)--(2.790,6.111)--(2.834,6.101)--(2.878,6.091)--(2.922,6.081)%
  --(2.966,6.071)--(3.009,6.061)--(3.053,6.051)--(3.097,6.041)--(3.141,6.030)--(3.185,6.020)%
  --(3.229,6.010)--(3.273,6.000)--(3.317,5.990)--(3.361,5.980)--(3.405,5.970)--(3.449,5.960)%
  --(3.493,5.950)--(3.537,5.940)--(3.580,5.930)--(3.624,5.920)--(3.668,5.910)--(3.712,5.900)%
  --(3.756,5.890)--(3.800,5.879)--(3.844,5.869)--(3.888,5.859)--(3.932,5.849)--(3.976,5.839)%
  --(4.020,5.829)--(4.064,5.819)--(4.108,5.809)--(4.151,5.799)--(4.195,5.789)--(4.239,5.779)%
  --(4.283,5.769)--(4.327,5.759)--(4.371,5.749)--(4.415,5.738)--(4.459,5.728)--(4.503,5.718)%
  --(4.547,5.708)--(4.591,5.698)--(4.635,5.688)--(4.678,5.678)--(4.722,5.668)--(4.766,5.658)%
  --(4.810,5.648)--(4.854,5.638)--(4.898,5.628)--(4.942,5.618)--(4.986,5.608)--(5.030,5.598)%
  --(5.074,5.587)--(5.118,5.577)--(5.162,5.567)--(5.206,5.557)--(5.249,5.547)--(5.293,5.537)%
  --(5.337,5.527)--(5.381,5.517)--(5.425,5.507)--(5.469,5.497)--(5.513,5.487)--(5.557,5.477)%
  --(5.601,5.466)--(5.645,5.456)--(5.689,5.446)--(5.733,5.436)--(5.777,5.426)--(5.820,5.416)%
  --(5.864,5.406)--(5.908,5.396)--(5.952,5.385)--(5.996,5.375)--(6.040,5.365)--(6.084,5.355)%
  --(6.128,5.344)--(6.172,5.334)--(6.216,5.324)--(6.260,5.313)--(6.304,5.302)--(6.348,5.291)%
  --(6.391,5.280)--(6.435,5.268)--(6.479,5.256)--(6.523,5.243)--(6.567,5.229)--(6.611,5.213)%
  --(6.655,5.196)--(6.699,5.176)--(6.743,5.153)--(6.787,5.126)--(6.831,5.096)--(6.875,5.061)%
  --(6.918,5.021)--(6.962,4.979)--(7.006,4.933)--(7.050,4.884)--(7.094,4.834)--(7.138,4.782)%
  --(7.182,4.729)--(7.226,4.675)--(7.270,4.621)--(7.314,4.567)--(7.358,4.512)--(7.402,4.457)%
  --(7.445,4.401)--(7.489,4.346)--(7.533,4.291)--(7.577,4.235)--(7.621,4.180)--(7.665,4.124)%
  --(7.709,4.069)--(7.753,4.013)--(7.797,3.957)--(7.841,3.902)--(7.885,3.846)--(7.929,3.791)%
  --(7.973,3.735)--(8.016,3.679)--(8.060,3.624)--(8.104,3.568)--(8.148,3.512)--(8.192,3.457)%
  --(8.236,3.401)--(8.280,3.345)--(8.324,3.290)--(8.368,3.234)--(8.412,3.179)--(8.456,3.123)%
  --(8.500,3.067)--(8.544,3.012)--(8.587,2.956)--(8.631,2.900)--(8.675,2.845)--(8.719,2.789)%
  --(8.763,2.733)--(8.807,2.678)--(8.851,2.622)--(8.895,2.566)--(8.939,2.511)--(8.983,2.455)%
  --(9.027,2.399)--(9.071,2.344)--(9.115,2.288)--(9.158,2.233)--(9.202,2.177)--(9.246,2.121)%
  --(9.290,2.066)--(9.334,2.010)--(9.378,1.954)--(9.422,1.899)--(9.466,1.843)--(9.510,1.787)%
  --(9.554,1.732)--(9.598,1.676)--(9.642,1.620)--(9.685,1.565)--(9.729,1.509)--(9.773,1.454)%
  --(9.817,1.398)--(9.861,1.342)--(9.905,1.287)--(9.949,1.231)--(9.993,1.176)--(10.037,1.120)%
  --(10.081,1.065)--(10.125,1.012)--(10.169,1.021)--(10.213,1.074)--(10.256,1.130)--(10.300,1.185)%
  --(10.344,1.240)--(10.388,1.296)--(10.432,1.352)--(10.476,1.407)--(10.520,1.463)--(10.564,1.519)%
  --(10.608,1.574)--(10.652,1.630)--(10.696,1.685)--(10.740,1.741)--(10.784,1.797)--(10.827,1.852)%
  --(10.871,1.908)--(10.915,1.964)--(10.959,2.019)--(11.003,2.075)--(11.047,2.131)--(11.091,2.186)%
  --(11.135,2.242)--(11.179,2.298)--(11.223,2.353)--(11.267,2.409)--(11.311,2.464)--(11.354,2.520)%
  --(11.398,2.576)--(11.442,2.631)--(11.486,2.687)--(11.530,2.743);
\draw[gp path] (2.746,6.121)--(2.790,6.121)--(2.834,6.121)--(2.878,6.121)--(2.922,6.121)%
  --(2.966,6.121)--(3.009,6.121)--(3.053,6.121)--(3.097,6.121)--(3.141,6.121)--(3.185,6.121)%
  --(3.229,6.121)--(3.273,6.121)--(3.317,6.121)--(3.361,6.121)--(3.405,6.121)--(3.449,6.121)%
  --(3.493,6.121)--(3.537,6.121)--(3.580,6.121)--(3.624,6.121)--(3.668,6.121)--(3.712,6.121)%
  --(3.756,6.121)--(3.800,6.121)--(3.844,6.121)--(3.888,6.121)--(3.932,6.121)--(3.976,6.121)%
  --(4.020,6.121)--(4.064,6.121)--(4.108,6.121)--(4.151,6.121)--(4.195,6.121)--(4.239,6.121)%
  --(4.283,6.121)--(4.327,6.121)--(4.371,6.121)--(4.415,6.121)--(4.459,6.121)--(4.503,6.121)%
  --(4.547,6.121)--(4.591,6.121)--(4.635,6.121)--(4.678,6.121)--(4.722,6.121)--(4.766,6.121)%
  --(4.810,6.121)--(4.854,6.121)--(4.898,6.121)--(4.942,6.121)--(4.986,6.121)--(5.030,6.121)%
  --(5.074,6.121)--(5.118,6.121)--(5.162,6.121)--(5.206,6.121)--(5.249,6.121)--(5.293,6.121)%
  --(5.337,6.121)--(5.381,6.121)--(5.425,6.121)--(5.469,6.121)--(5.513,6.121)--(5.557,6.121)%
  --(5.601,6.121)--(5.645,6.121)--(5.689,6.121)--(5.733,6.121)--(5.777,6.121)--(5.820,6.121)%
  --(5.864,6.121)--(5.908,6.120)--(5.952,6.120)--(5.996,6.120)--(6.040,6.120)--(6.084,6.120)%
  --(6.128,6.120)--(6.172,6.119)--(6.216,6.119)--(6.260,6.118)--(6.304,6.117)--(6.348,6.116)%
  --(6.391,6.115)--(6.435,6.113)--(6.479,6.110)--(6.523,6.107)--(6.567,6.102)--(6.611,6.096)%
  --(6.655,6.087)--(6.699,6.075)--(6.743,6.059)--(6.787,6.039)--(6.831,6.014)--(6.875,5.984)%
  --(6.918,5.948)--(6.962,5.908)--(7.006,5.864)--(7.050,5.818)--(7.094,5.768)--(7.138,5.717)%
  --(7.182,5.664)--(7.226,5.611)--(7.270,5.557)--(7.314,5.502)--(7.358,5.447)--(7.402,5.392)%
  --(7.445,5.337)--(7.489,5.281)--(7.533,5.225)--(7.577,5.170)--(7.621,5.114)--(7.665,5.058)%
  --(7.709,5.002)--(7.753,4.946)--(7.797,4.891)--(7.841,4.835)--(7.885,4.779)--(7.929,4.723)%
  --(7.973,4.667)--(8.016,4.611)--(8.060,4.555)--(8.104,4.499)--(8.148,4.443)--(8.192,4.387)%
  --(8.236,4.331)--(8.280,4.275)--(8.324,4.220)--(8.368,4.164)--(8.412,4.108)--(8.456,4.052)%
  --(8.500,3.996)--(8.544,3.940)--(8.587,3.884)--(8.631,3.828)--(8.675,3.772)--(8.719,3.716)%
  --(8.763,3.660)--(8.807,3.604)--(8.851,3.548)--(8.895,3.492)--(8.939,3.436)--(8.983,3.380)%
  --(9.027,3.324)--(9.071,3.268)--(9.115,3.213)--(9.158,3.157)--(9.202,3.101)--(9.246,3.045)%
  --(9.290,2.989)--(9.334,2.933)--(9.378,2.877)--(9.422,2.821)--(9.466,2.765)--(9.510,2.709)%
  --(9.554,2.653)--(9.598,2.597)--(9.642,2.541)--(9.685,2.485)--(9.729,2.429)--(9.773,2.373)%
  --(9.817,2.317)--(9.861,2.261)--(9.905,2.206)--(9.949,2.150)--(9.993,2.094)--(10.037,2.038)%
  --(10.081,1.982)--(10.125,1.926)--(10.169,1.870)--(10.213,1.814)--(10.256,1.758)--(10.300,1.702)%
  --(10.344,1.646)--(10.388,1.590)--(10.432,1.534)--(10.476,1.478)--(10.520,1.422)--(10.564,1.367)%
  --(10.608,1.311)--(10.652,1.255)--(10.696,1.199)--(10.740,1.143)--(10.784,1.088)--(10.827,1.033)%
  --(10.871,1.003)--(10.915,1.053)--(10.959,1.108)--(11.003,1.164)--(11.047,1.220)--(11.091,1.275)%
  --(11.135,1.331)--(11.179,1.387)--(11.223,1.443)--(11.267,1.499)--(11.311,1.555)--(11.354,1.611)%
  --(11.398,1.667)--(11.442,1.723)--(11.486,1.779)--(11.530,1.835);
\draw[gp path] (2.746,7.148)--(2.790,7.158)--(2.834,7.168)--(2.878,7.179)--(2.922,7.189)%
  --(2.966,7.199)--(3.009,7.209)--(3.053,7.219)--(3.097,7.229)--(3.141,7.239)--(3.185,7.249)%
  --(3.229,7.259)--(3.273,7.269)--(3.317,7.279)--(3.361,7.289)--(3.405,7.299)--(3.449,7.309)%
  --(3.493,7.319)--(3.537,7.330)--(3.580,7.340)--(3.624,7.350)--(3.668,7.360)--(3.712,7.370)%
  --(3.756,7.380)--(3.800,7.390)--(3.844,7.400)--(3.888,7.410)--(3.932,7.420)--(3.976,7.430)%
  --(4.020,7.440)--(4.064,7.450)--(4.108,7.460)--(4.151,7.471)--(4.195,7.481)--(4.239,7.491)%
  --(4.283,7.501)--(4.327,7.511)--(4.371,7.521)--(4.415,7.531)--(4.459,7.541)--(4.503,7.551)%
  --(4.547,7.561)--(4.591,7.571)--(4.635,7.581)--(4.678,7.591)--(4.722,7.601)--(4.766,7.611)%
  --(4.810,7.622)--(4.854,7.632)--(4.898,7.642)--(4.942,7.652)--(4.986,7.662)--(5.030,7.672)%
  --(5.074,7.682)--(5.118,7.692)--(5.162,7.702)--(5.206,7.712)--(5.249,7.722)--(5.293,7.732)%
  --(5.337,7.742)--(5.381,7.752)--(5.425,7.762)--(5.469,7.772)--(5.513,7.782)--(5.557,7.792)%
  --(5.601,7.803)--(5.645,7.813)--(5.689,7.823)--(5.733,7.833)--(5.777,7.843)--(5.820,7.853)%
  --(5.864,7.863)--(5.908,7.872)--(5.952,7.882)--(5.996,7.892)--(6.040,7.902)--(6.084,7.912)%
  --(6.128,7.922)--(6.172,7.931)--(6.216,7.941)--(6.260,7.950)--(6.304,7.959)--(6.348,7.967)%
  --(6.391,7.976)--(6.435,7.983)--(6.479,7.990)--(6.523,7.995)--(6.567,7.999)--(6.611,8.000)%
  --(6.655,7.998)--(6.699,7.992)--(6.743,7.981)--(6.787,7.963)--(6.831,7.939)--(6.875,7.907)%
  --(6.918,7.868)--(6.962,7.822)--(7.006,7.771)--(7.050,7.716)--(7.094,7.657)--(7.138,7.595)%
  --(7.182,7.532)--(7.226,7.468)--(7.270,7.402)--(7.314,7.336)--(7.358,7.270)--(7.402,7.203)%
  --(7.445,7.136)--(7.489,7.068)--(7.533,7.001)--(7.577,6.933)--(7.621,6.866)--(7.665,6.798)%
  --(7.709,6.730)--(7.753,6.662)--(7.797,6.595)--(7.841,6.527)--(7.885,6.459)--(7.929,6.391)%
  --(7.973,6.323)--(8.016,6.255)--(8.060,6.188)--(8.104,6.120)--(8.148,6.052)--(8.192,5.984)%
  --(8.236,5.916)--(8.280,5.848)--(8.324,5.780)--(8.368,5.712)--(8.412,5.644)--(8.456,5.576)%
  --(8.500,5.509)--(8.544,5.441)--(8.587,5.373)--(8.631,5.305)--(8.675,5.237)--(8.719,5.169)%
  --(8.763,5.101)--(8.807,5.033)--(8.851,4.965)--(8.895,4.897)--(8.939,4.829)--(8.983,4.761)%
  --(9.027,4.694)--(9.071,4.626)--(9.115,4.558)--(9.158,4.490)--(9.202,4.422)--(9.246,4.354)%
  --(9.290,4.286)--(9.334,4.218)--(9.378,4.150)--(9.422,4.082)--(9.466,4.014)--(9.510,3.947)%
  --(9.554,3.879)--(9.598,3.811)--(9.642,3.743)--(9.685,3.675)--(9.729,3.607)--(9.773,3.539)%
  --(9.817,3.471)--(9.861,3.403)--(9.905,3.335)--(9.949,3.267)--(9.993,3.199)--(10.037,3.132)%
  --(10.081,3.064)--(10.125,2.996)--(10.169,2.928)--(10.213,2.860)--(10.256,2.792)--(10.300,2.724)%
  --(10.344,2.656)--(10.388,2.588)--(10.432,2.520)--(10.476,2.452)--(10.520,2.385)--(10.564,2.317)%
  --(10.608,2.249)--(10.652,2.181)--(10.696,2.113)--(10.740,2.045)--(10.784,1.977)--(10.827,1.909)%
  --(10.871,1.841)--(10.915,1.773)--(10.959,1.706)--(11.003,1.638)--(11.047,1.570)--(11.091,1.502)%
  --(11.135,1.434)--(11.179,1.366)--(11.223,1.298)--(11.267,1.231)--(11.311,1.163)--(11.354,1.095)%
  --(11.398,1.030)--(11.442,1.017)--(11.486,1.081)--(11.530,1.149);
\draw[gp path] (2.746,7.148)--(2.790,7.138)--(2.834,7.128)--(2.878,7.118)--(2.922,7.108)%
  --(2.966,7.098)--(3.009,7.088)--(3.053,7.078)--(3.097,7.068)--(3.141,7.058)--(3.185,7.048)%
  --(3.229,7.038)--(3.273,7.028)--(3.317,7.017)--(3.361,7.007)--(3.405,6.997)--(3.449,6.987)%
  --(3.493,6.977)--(3.537,6.967)--(3.580,6.957)--(3.624,6.947)--(3.668,6.937)--(3.712,6.927)%
  --(3.756,6.917)--(3.800,6.907)--(3.844,6.897)--(3.888,6.887)--(3.932,6.876)--(3.976,6.866)%
  --(4.020,6.856)--(4.064,6.846)--(4.108,6.836)--(4.151,6.826)--(4.195,6.816)--(4.239,6.806)%
  --(4.283,6.796)--(4.327,6.786)--(4.371,6.776)--(4.415,6.766)--(4.459,6.756)--(4.503,6.746)%
  --(4.547,6.736)--(4.591,6.725)--(4.635,6.715)--(4.678,6.705)--(4.722,6.695)--(4.766,6.685)%
  --(4.810,6.675)--(4.854,6.665)--(4.898,6.655)--(4.942,6.645)--(4.986,6.635)--(5.030,6.625)%
  --(5.074,6.615)--(5.118,6.605)--(5.162,6.594)--(5.206,6.584)--(5.249,6.574)--(5.293,6.564)%
  --(5.337,6.554)--(5.381,6.544)--(5.425,6.534)--(5.469,6.524)--(5.513,6.514)--(5.557,6.504)%
  --(5.601,6.494)--(5.645,6.484)--(5.689,6.473)--(5.733,6.463)--(5.777,6.453)--(5.820,6.443)%
  --(5.864,6.433)--(5.908,6.423)--(5.952,6.413)--(5.996,6.402)--(6.040,6.392)--(6.084,6.382)%
  --(6.128,6.371)--(6.172,6.361)--(6.216,6.350)--(6.260,6.340)--(6.304,6.329)--(6.348,6.318)%
  --(6.391,6.306)--(6.435,6.294)--(6.479,6.281)--(6.523,6.268)--(6.567,6.253)--(6.611,6.236)%
  --(6.655,6.217)--(6.699,6.194)--(6.743,6.168)--(6.787,6.137)--(6.831,6.102)--(6.875,6.060)%
  --(6.918,6.014)--(6.962,5.963)--(7.006,5.908)--(7.050,5.850)--(7.094,5.790)--(7.138,5.728)%
  --(7.182,5.664)--(7.226,5.600)--(7.270,5.534)--(7.314,5.469)--(7.358,5.403)--(7.402,5.336)%
  --(7.445,5.270)--(7.489,5.203)--(7.533,5.137)--(7.577,5.070)--(7.621,5.003)--(7.665,4.936)%
  --(7.709,4.869)--(7.753,4.802)--(7.797,4.735)--(7.841,4.669)--(7.885,4.602)--(7.929,4.535)%
  --(7.973,4.468)--(8.016,4.401)--(8.060,4.334)--(8.104,4.267)--(8.148,4.200)--(8.192,4.133)%
  --(8.236,4.066)--(8.280,3.999)--(8.324,3.932)--(8.368,3.865)--(8.412,3.798)--(8.456,3.731)%
  --(8.500,3.664)--(8.544,3.597)--(8.587,3.530)--(8.631,3.463)--(8.675,3.396)--(8.719,3.329)%
  --(8.763,3.262)--(8.807,3.195)--(8.851,3.128)--(8.895,3.061)--(8.939,2.994)--(8.983,2.927)%
  --(9.027,2.860)--(9.071,2.793)--(9.115,2.726)--(9.158,2.659)--(9.202,2.592)--(9.246,2.525)%
  --(9.290,2.458)--(9.334,2.391)--(9.378,2.324)--(9.422,2.257)--(9.466,2.190)--(9.510,2.123)%
  --(9.554,2.056)--(9.598,1.990)--(9.642,1.923)--(9.685,1.856)--(9.729,1.789)--(9.773,1.722)%
  --(9.817,1.655)--(9.861,1.588)--(9.905,1.521)--(9.949,1.454)--(9.993,1.387)--(10.037,1.320)%
  --(10.081,1.253)--(10.125,1.186)--(10.169,1.120)--(10.213,1.054)--(10.256,1.003)--(10.300,1.055)%
  --(10.344,1.121)--(10.388,1.187)--(10.432,1.254)--(10.476,1.321)--(10.520,1.388)--(10.564,1.455)%
  --(10.608,1.522)--(10.652,1.589)--(10.696,1.655)--(10.740,1.722)--(10.784,1.789)--(10.827,1.856)%
  --(10.871,1.923)--(10.915,1.990)--(10.959,2.057)--(11.003,2.124)--(11.047,2.191)--(11.091,2.258)%
  --(11.135,2.325)--(11.179,2.392)--(11.223,2.459)--(11.267,2.526)--(11.311,2.593)--(11.354,2.660)%
  --(11.398,2.727)--(11.442,2.794)--(11.486,2.861)--(11.530,2.928);
\draw[gp path] (2.746,7.148)--(2.790,7.148)--(2.834,7.148)--(2.878,7.148)--(2.922,7.148)%
  --(2.966,7.148)--(3.009,7.148)--(3.053,7.148)--(3.097,7.148)--(3.141,7.148)--(3.185,7.148)%
  --(3.229,7.148)--(3.273,7.148)--(3.317,7.148)--(3.361,7.148)--(3.405,7.148)--(3.449,7.148)%
  --(3.493,7.148)--(3.537,7.148)--(3.580,7.148)--(3.624,7.148)--(3.668,7.148)--(3.712,7.148)%
  --(3.756,7.148)--(3.800,7.148)--(3.844,7.148)--(3.888,7.148)--(3.932,7.148)--(3.976,7.148)%
  --(4.020,7.148)--(4.064,7.148)--(4.108,7.148)--(4.151,7.148)--(4.195,7.148)--(4.239,7.148)%
  --(4.283,7.148)--(4.327,7.148)--(4.371,7.148)--(4.415,7.148)--(4.459,7.148)--(4.503,7.148)%
  --(4.547,7.148)--(4.591,7.148)--(4.635,7.148)--(4.678,7.148)--(4.722,7.148)--(4.766,7.148)%
  --(4.810,7.148)--(4.854,7.148)--(4.898,7.148)--(4.942,7.148)--(4.986,7.148)--(5.030,7.148)%
  --(5.074,7.148)--(5.118,7.148)--(5.162,7.148)--(5.206,7.148)--(5.249,7.148)--(5.293,7.148)%
  --(5.337,7.148)--(5.381,7.148)--(5.425,7.148)--(5.469,7.148)--(5.513,7.148)--(5.557,7.148)%
  --(5.601,7.148)--(5.645,7.148)--(5.689,7.148)--(5.733,7.148)--(5.777,7.148)--(5.820,7.148)%
  --(5.864,7.148)--(5.908,7.148)--(5.952,7.147)--(5.996,7.147)--(6.040,7.147)--(6.084,7.147)%
  --(6.128,7.146)--(6.172,7.146)--(6.216,7.145)--(6.260,7.145)--(6.304,7.144)--(6.348,7.143)%
  --(6.391,7.141)--(6.435,7.139)--(6.479,7.136)--(6.523,7.131)--(6.567,7.126)--(6.611,7.118)%
  --(6.655,7.107)--(6.699,7.093)--(6.743,7.074)--(6.787,7.050)--(6.831,7.020)--(6.875,6.984)%
  --(6.918,6.941)--(6.962,6.893)--(7.006,6.840)--(7.050,6.783)--(7.094,6.724)--(7.138,6.662)%
  --(7.182,6.599)--(7.226,6.534)--(7.270,6.469)--(7.314,6.403)--(7.358,6.337)--(7.402,6.270)%
  --(7.445,6.204)--(7.489,6.137)--(7.533,6.070)--(7.577,6.003)--(7.621,5.935)--(7.665,5.868)%
  --(7.709,5.801)--(7.753,5.734)--(7.797,5.666)--(7.841,5.599)--(7.885,5.532)--(7.929,5.464)%
  --(7.973,5.397)--(8.016,5.330)--(8.060,5.262)--(8.104,5.195)--(8.148,5.128)--(8.192,5.060)%
  --(8.236,4.993)--(8.280,4.925)--(8.324,4.858)--(8.368,4.791)--(8.412,4.723)--(8.456,4.656)%
  --(8.500,4.589)--(8.544,4.521)--(8.587,4.454)--(8.631,4.386)--(8.675,4.319)--(8.719,4.252)%
  --(8.763,4.184)--(8.807,4.117)--(8.851,4.049)--(8.895,3.982)--(8.939,3.915)--(8.983,3.847)%
  --(9.027,3.780)--(9.071,3.712)--(9.115,3.645)--(9.158,3.578)--(9.202,3.510)--(9.246,3.443)%
  --(9.290,3.376)--(9.334,3.308)--(9.378,3.241)--(9.422,3.173)--(9.466,3.106)--(9.510,3.039)%
  --(9.554,2.971)--(9.598,2.904)--(9.642,2.836)--(9.685,2.769)--(9.729,2.702)--(9.773,2.634)%
  --(9.817,2.567)--(9.861,2.500)--(9.905,2.432)--(9.949,2.365)--(9.993,2.297)--(10.037,2.230)%
  --(10.081,2.163)--(10.125,2.095)--(10.169,2.028)--(10.213,1.960)--(10.256,1.893)--(10.300,1.826)%
  --(10.344,1.758)--(10.388,1.691)--(10.432,1.624)--(10.476,1.556)--(10.520,1.489)--(10.564,1.422)%
  --(10.608,1.354)--(10.652,1.287)--(10.696,1.220)--(10.740,1.153)--(10.784,1.086)--(10.827,1.021)%
  --(10.871,1.025)--(10.915,1.089)--(10.959,1.156)--(11.003,1.223)--(11.047,1.291)--(11.091,1.358)%
  --(11.135,1.425)--(11.179,1.493)--(11.223,1.560)--(11.267,1.627)--(11.311,1.695)--(11.354,1.762)%
  --(11.398,1.829)--(11.442,1.897)--(11.486,1.964)--(11.530,2.032);
\gpcolor{color=gp lt color 0}
\gpsetlinetype{gp lt plot 0}
\gpsetlinewidth{3.00}
\draw[gp path] (2.746,4.067)--(2.791,4.067)--(2.836,4.067)--(2.880,4.067)--(2.925,4.067)%
  --(2.970,4.067)--(3.015,4.067)--(3.059,4.067)--(3.104,4.067)--(3.149,4.067)--(3.194,4.067)%
  --(3.239,4.067)--(3.283,4.067)--(3.328,4.067)--(3.373,4.067)--(3.418,4.068)--(3.463,4.068)%
  --(3.507,4.068)--(3.552,4.068)--(3.597,4.068)--(3.642,4.068)--(3.686,4.068)--(3.731,4.069)%
  --(3.776,4.069)--(3.821,4.069)--(3.866,4.069)--(3.910,4.069)--(3.955,4.070)--(4.000,4.070)%
  --(4.045,4.070)--(4.090,4.070)--(4.134,4.071)--(4.179,4.071)--(4.224,4.071)--(4.269,4.071)%
  --(4.314,4.072)--(4.358,4.072)--(4.403,4.072)--(4.448,4.072)--(4.493,4.073)--(4.537,4.073)%
  --(4.582,4.073)--(4.627,4.074)--(4.672,4.074)--(4.717,4.074)--(4.761,4.075)--(4.806,4.075)%
  --(4.851,4.076)--(4.896,4.076)--(4.941,4.076)--(4.985,4.077)--(5.030,4.077)--(5.075,4.078)%
  --(5.120,4.078)--(5.164,4.078)--(5.209,4.079)--(5.254,4.079)--(5.299,4.080)--(5.344,4.080)%
  --(5.388,4.081)--(5.433,4.081)--(5.478,4.082)--(5.523,4.082)--(5.568,4.083)--(5.612,4.083)%
  --(5.657,4.084)--(5.702,4.084)--(5.747,4.085)--(5.791,4.085)--(5.836,4.086)--(5.881,4.086)%
  --(5.926,4.087)--(5.971,4.087)--(6.015,4.088)--(6.060,4.089)--(6.105,4.089)--(6.150,4.090)%
  --(6.195,4.090)--(6.239,4.091)--(6.284,4.092)--(6.329,4.092)--(6.374,4.093)--(6.418,4.094)%
  --(6.463,4.094)--(6.508,4.095)--(6.553,4.096)--(6.598,4.096)--(6.642,4.097)--(6.687,4.098)%
  --(6.732,4.098)--(6.777,4.099)--(6.822,4.094)--(6.866,4.063)--(6.911,4.030)--(6.956,3.996)%
  --(7.001,3.963)--(7.046,3.929)--(7.090,3.895)--(7.135,3.862)--(7.180,3.828)--(7.225,3.795)%
  --(7.269,3.762)--(7.314,3.728)--(7.359,3.695)--(7.404,3.661)--(7.449,3.628)--(7.493,3.594)%
  --(7.538,3.561)--(7.583,3.528)--(7.628,3.494)--(7.673,3.461)--(7.717,3.427)--(7.762,3.394)%
  --(7.807,3.361)--(7.852,3.328)--(7.896,3.294)--(7.941,3.261)--(7.986,3.228)--(8.031,3.195)%
  --(8.076,3.161)--(8.120,3.128)--(8.165,3.095)--(8.210,3.062)--(8.255,3.029)--(8.300,2.996)%
  --(8.344,2.963)--(8.389,2.930)--(8.434,2.897)--(8.479,2.864)--(8.523,2.831)--(8.568,2.798)%
  --(8.613,2.765)--(8.658,2.732)--(8.703,2.700)--(8.747,2.667)--(8.792,2.634)--(8.837,2.602)%
  --(8.882,2.569)--(8.927,2.537)--(8.971,2.504)--(9.016,2.472)--(9.061,2.440)--(9.106,2.408)%
  --(9.150,2.375)--(9.195,2.343)--(9.240,2.311)--(9.285,2.280)--(9.330,2.248)--(9.374,2.216)%
  --(9.419,2.185)--(9.464,2.153)--(9.509,2.122)--(9.554,2.091)--(9.598,2.060)--(9.643,2.030)%
  --(9.688,1.999)--(9.733,1.969)--(9.778,1.939)--(9.822,1.909)--(9.867,1.880)--(9.912,1.851)%
  --(9.957,1.822)--(10.001,1.794)--(10.046,1.766)--(10.091,1.738)--(10.136,1.712)--(10.181,1.685)%
  --(10.225,1.660)--(10.270,1.635)--(10.315,1.611)--(10.360,1.588)--(10.405,1.566)--(10.449,1.546)%
  --(10.494,1.526)--(10.539,1.509)--(10.584,1.492)--(10.628,1.478)--(10.673,1.466)--(10.718,1.455)%
  --(10.763,1.447)--(10.808,1.442)--(10.852,1.438)--(10.897,1.438)--(10.942,1.440)--(10.987,1.444)%
  --(11.032,1.451)--(11.076,1.460)--(11.121,1.471)--(11.166,1.485)--(11.211,1.500)--(11.255,1.517)%
  --(11.300,1.536)--(11.345,1.556)--(11.390,1.577)--(11.435,1.599)--(11.479,1.623)--(11.524,1.647)%
  --(11.569,1.672)--(11.614,1.698)--(11.659,1.725)--(11.703,1.752);
\gpcolor{color=gp lt color border}
\gpsetlinetype{gp lt border}
\gpsetlinewidth{1.00}
\draw[gp path] (1.688,8.381)--(1.688,0.985)--(11.947,0.985)--(11.947,8.381)--cycle;
%% coordinates of the plot area
\gpdefrectangularnode{gp plot 1}{\pgfpoint{1.688cm}{0.985cm}}{\pgfpoint{11.947cm}{8.381cm}}
\end{tikzpicture}
%% gnuplot variables

  \caption[Comparison of the AG model and a kinematic model of a magnetic lens]{
    Comparison of electron rays tracked kinematically through a realistic magnetic lens simulation (green) and an AG model simulation of a Guassian electron beam (red) of a similar HW1/eM beam width $ w = 500 \mu\text{m} $.
  }
  \label{fig:mag_lens_loops}
\end{figure}

For localizing the region of influence of the lens, I have generally modelled a thin magnetic lens as having a super-Gaussian order value $n=1$, a simple Gaussian; of course this may depend on the specifics of the lens.
To simulate the effect of magnetic lenses in a UEM by the extended AG model, for each magnetic lens, one adds a term like
\begin{equation}
  M_{\smallT}^{mag} = M_{\smallT}^{mag}(I) \exp \left [ - \left (  \frac{ z^{\prime} - z_{mag}^{\prime} }{ L_{mag} / 2 } \right )^{ 2 } \right ] \text{ ,}
\end{equation}
to the set of $M_{\smallT}$ parameters, where $z_{mag}^{\prime}$ and $L_{mag}$ are the position and length of the lens in question respectively.
%TODO include discussion and plot of beam size versus aberrations

\ref{fig:mag_lens_loops} shows electron rays tracked kinematically (by employing \ref{eq:lens_eq_of_motion}) through a realistic magnetic lens simulation (shown in green).
The kinematic model employs ten individual current loops generating magnetic fields described by \ref{eq:field_of_loop}; these loops occupy a total length of 635$\mu$m at position $z=0$.
These loops each carry a current of 45A; this approximates a more realistic stack of 10 loops carrying 4.5A, thus simulating a  100 turn solenoid in a 10x10 array.
Each ray bunch consists of one ray parallel to the axis of propagation and one each having a longitudinal velocity of $1 \times 10^{8}$ m/s and an inward or outward transverse velocity related to an excess photoemission energy $\Delta E \approx $ 0.1eV.
The bunches are distributed evenly from $r=0$ to $r = 1$mm.
These are compared with an AG model simulation of a Guassian electron beam of HW1/eM beam width $ w = 500 \mu\text{m} $ having an initial transverse momentum variance associated with $\Delta E$.

The AG model very accurately mimics the behavior of the kinematic model for an appropriate choice of the free parameter $M_{\smallT}^{mag}$.
In fact, when the rays are weighted by a Gaussian of width $w$ the minimum ``radius'' of the beam is 100$\mu$m, while the AG model has a minimum HW1/eM width of 73$\mu$m.
(The kinematic model actually overestimates the width as there are two rays at the outer convergence zone for every one which converges at the axis.) %TODO footnote?

\subsection{RF Cavities} \label{rf_cav_model}

In the longitudinal ($ z $) direction, a resonant radio-frequency (RF) cavity can be used in a UEM column or UED experiment \cite{oudheusden_electron_2007,fill_sub-fs_2006} to act as a pulse compression or `temporal focusing' element.
The model of the RF cavity is different from the magnetic lens in two distinct ways.
First, the RF cavity contains several different fields and thus contributes several different linear force terms to the overall equation set.
Second, though there are more terms, their magnitudes are easily connected to the physical characteristics of the cavity.

Although there are many possible resonant cavity structures,\cite{oudheusden_electron_2007,humphries_principles_1986} for simplicity, this analysis will consider only the evacuated cylindrical (or `pillbox') $\text{TM}_{010}$ cavity for which the electrons propagate down the axis.\cite{fill_sub-fs_2006,humphries_principles_1986}
The known mathematical forms of the electric and magnetic components of this resonant mode together with a super-Gaussian envelope function (\ref{eq:reg_of_influence}) allow the three main effects of a pill-box RF cavity element on the electron pulse to be described within the approximation of the linear force extension to the AG model: the primary action of the pulse compression (or acceleration and expansion dependent upon the relative RF phase), and the transverse lens effects due to the magnetic component of the $\text{TM}_{010}$ resonance and the electric RF cavity fringe fields at its entrance and exit apertures.\cite{kim_rf_1989}

The spatial dependence of the axial RF electric field experienced by the electron pulse in a TM$_{010}$ cavity of radius $a$ may be written as
\begin{equation} \label{eq:RF_Efield}
  E ( r , z , z^{\prime} ) = E_{0} \sin \left ( \frac{ \Omega ( z^{\prime} - z ) }{ v_{\smallzero} } + \phi \right )\operatorname{J_{0}} \left ( \frac{ k r }{ a } \right ) \text{,}
\end{equation}
where $ E_{ 0 } $ is the oscillating field amplitude, $\Omega$ is the RF frequency and $\phi$ is a phase factor determining whether the RF cavity acts as a pulse compressor $ ( \phi \approx 0 ) $ or an electron accelerator $ ( \phi \approx \pm \pi/2 ) $.
The coordinate $z^{\prime} $ describes the position of the center of the pulse in the RF cavity, which extends from $ z^{\prime} = - d / 2 $ to $ z^{\prime} = d / 2 $, and $z$ is the position of any electron in the pulse with respect to the center of the pulse.
For the $\text{TM}_{010}$ oscillation mode, the RF frequency is given by
\begin{equation}
  \Omega = 2.405 \frac{ c }{a} \text{,}
\end{equation}
where $c$ is the speed of light in a vacuum and 2.405 is the first zero of the $\operatorname{ J_{0} }$ Bessel function.

For the usual case of a spatially compact electron pulse with respect to the dimension of the RF cavity (i.e., $ \Omega z \ll v_{\smallzero} $ and $ \sqrt{\sigma_{\smallT}} \ll a $), \ref{eq:RF_Efield} can be rewritten as
\begin{equation} \label{eq:RF_Efield_approx}
  E ( r , z , z^{\prime} ) = E_{0} \biggl [ \sin \left ( \frac{ \Omega z^{\prime} }{ v_{\smallzero} } + \phi \right ) - \frac{ \Omega z }{ v_{\smallzero} } \cos \left ( \frac{ \Omega z^{\prime} }{ v_{\smallzero} } + \phi \right ) \biggr ] \text{,}
\end{equation}
which demonstrates that the RF cavity acts both as a pulse acceleration (first term on RHS of \ref{eq:RF_Efield_approx}) and a pulse compression or expansion (second term on RHS of \eqref{eq:RF_Efield_approx}) device. %TODO clarify this sentance and the next
The linear term (second) then gives the desired form (see \ref{eq:linear_force}) for the force manipulating the temporal dynamics and characteristics during electron pulse compression;
\begin{equation} \label{eq:RF_Force_z}
  \boldsymbol{F}_{ z , RF }^{linear} = - \left [ \frac{ e \Omega E_{0} }{ v_{\smallzero} } \cos \left ( \frac{ \Omega z^{\prime} }{ v_{\smallzero} } + \phi \right ) \right ] z \hat{\operatorname{ \boldsymbol{z} }} \text{.}
\end{equation}

By considering the impulse exerted by the RF cavity field on the electrons in the pulse, it is straightforward to show that pulse compression (when $\phi = 0$) is maximized for $ \Omega d / (2 v_{\smallzero} ) = \pi /2 $; that is, the time of flight of the electron pulse through the RF cavity of length $ d $ is equal to half the RF period.
Such a calculation assumes that the pulse velocity $v_{\smallzero}$ is constant; so that, the condition that $ \phi = 0 $ when the pulse is at the center of the RF cavity (i.e., when the pulse is at $ z^{\prime} = 0 $) is readily met.
As suggested by the acceleration term in \ref{eq:RF_Efield_approx}, this is only a good approximation if the RF field amplitude $E_{0}$ is sufficiently small to not affect the velocity of the peak of the pulse as it propagates through the cavity.
As a result, in practice, the phase $\phi$ of the RF field is adjusted to optimize the performance of the cavity.

The two main effects contributing to the transverse lens of a $\text{TM}_{010}$-mode RF cavity, its magnetic field component and electric fringe field aperture effects, may also be included within the linear force approximation.
The azimuthal magnetic component of the resonant mode oscillates 90$^{\circ}$ out of phase with the electric component and with a $\operatorname{ J_{1} }$ Bessel radial form; i.e., $ \operatorname{ J_{1} }( k r / a ) \sim k r / a $ for small $r$.
Symmetry and Gauss' Law dictate that in the vicinity of the axial center of the entrance and exit apertures of the RF cavity there exists a radial component of the oscillating electric field determined by $ E_{r} = -\left ( r / 2 \right ) \left ( \partial E_{z} / \partial z \right )_{r=0} $,\cite{kim_rf_1989} where the form of the axial electric field for $\text{TM}_{010}$ oscillation (\ref{eq:RF_Efield}) implies that, in the context of this analysis, its axial gradient is solely dependent upon the envelope function for the RF cavity (\ref{eq:reg_of_influence}).
In practice, of course, the exact form of $E_{r}(z)$ is strongly tied to the size and shape of the RF cavity apertures (see for example $1 \frac{1}{2}$-cell RF photo-gun designs \cite{mcdonald_design_1988}). 
For the case of an evacuated cavity, these two contributions result in a net force,
\begin{equation} \label{eq:RF_Force_r}
  \boldsymbol{F}_{ r , RF } = e E_{0} \left [ 
    \frac{ v_{\smallzero} \Omega }{c^{2}} \cos \left ( \frac{ \Omega z^{\prime} }{ v_{\smallzero} } + \phi \right ) + \frac{2 n}{d} \left ( \frac{z^{\prime} - z^{\prime}_{RF}}{d / 2} \right )^{2 n - 1} \sin \left ( \frac{ \Omega z^{\prime} }{ v_{\smallzero} } + \phi \right )
   \right ] r \hat{\operatorname{ \boldsymbol{r} }} \text{.}
\end{equation}
In obtaining \ref{eq:RF_Force_r}, which is also of the form for use in \ref{eq:linear_force}, we have again used an arbitrary super-Gaussian of order $n$ as a region of influence envelope (see \ref{eq:reg_of_influence}) with $L_{RF} \equiv d$, the RF cavity length.
%TODO examples shown below?
For the examples shown below under the pulse compression conditions ($\phi = 0$ when the pulse is at $z^{\prime} = 0$), both effects are defocusing for the electron pulse and the magnetic contribution is generally smaller than the transverse lenses associated with the RF cavity apertures when $ d = \pi v_{\smallzero} / \Omega $,\cite{kim_rf_1989} even though its magnitude is maximized for $\phi = 0$.
Other effects, such as magnetic fringe field aperture effects, could also be included, but are significantly weaker.

To summarize, the additional terms are which are added to model the internal pulse dynamics arising from a single TM$_{010}$-mode cylindrical RF cavity are
\begin{gather}
  M_{z}^{RF} = \left [ \frac{ e \Omega E_{0} }{ v_{\smallzero} } \cos \left ( \frac{ \Omega z^{\prime} }{ v_{\smallzero} } + \phi \right ) \right ] \exp \left [ - \left (  \frac{ z^{\prime} - z_{RF}^{\prime} }{ L_{RF} / 2 } \right )^{ 2 } \right ] \\
  \begin{aligned}
  M_{\smallT}^{RF} = & e E_{0} \left [ \frac{ v_{\smallzero} \Omega }{c^{2}} \cos \left ( \frac{ \Omega z^{\prime} }{ v_{\smallzero} } + \phi \right ) + \frac{2 n}{d} \left ( \frac{z^{\prime} - z^{\prime}_{RF}}{d / 2} \right )^{2 n - 1} \right. \\ \times & \left. \sin \left ( \frac{ \Omega z^{\prime} }{ v_{\smallzero} } + \phi \right )
   \right ] \exp \left [ - \left (  \frac{ z^{\prime} - z_{RF}^{\prime} }{ L_{RF} / 2 } \right )^{ 2 } \right ]
  \end{aligned}
\end{gather}
Additionally, since an RF cavity can act as a pulse accelerator (as seen in the leading term of \ref{eq:RF_Efield_approx}) an additional force term
\begin{equation}
  F_{z}^{RF} = e E_{0} \sin \left ( \frac{ \Omega z^{\prime} }{ v_{\smallzero} } + \phi \right ) \exp \left [ - \left (  \frac{ z^{\prime} - z_{RF}^{\prime} }{ L_{RF} / 2 } \right )^{ 2 } \right ] \, \text{,}
\end{equation}
should be added to the equation of motion governing the propagation of the pulse down the column.
% excluded example section from JAP paper as it will need new graphics
