\section{Column Design and Construction}

Apart from the laser system, the majority of an Ultrafast Electron Microscope column is either vacuum hardware or contained within the vacuum system.
It is common in the UEM field to modify an existing Transmission Electron Microscope column, adapting it for laser-stimulated electron generation.
The system presented in the following section has been designed and built either from scatch or from commercially available parts.
The reasons for this are two-fold, the ease of prototyping inherent in using replacable parts and the physical limitations our unique choices have imposed on our beam profile.

The vacuum chamber itself is almost entirely 4.5 in standard ConFlat hardware.
Much has been purchased from off-the-shelf hardware suppliers.
More complicated chamber components have been purchased from Kimball Physics Inc.

Kimball's line of Multi-CF hardware has been especially useful in building the prototype column.
The main chamber is a 4.5 in ``spherical cube,'' having six 4.5 in ConFlat ports on a cube, with eight 1.33 in ports on the corners.
All ports on Multi-CF hardware also contain a set of grooves, designed as mounting points for a system of erector-set-like equipment called ``eV Parts.''
This system is used to mount the accelerator anode and the the deflector plates which are both to be described later.

