%This work is licensed under the Creative Commons Attribution-NonCommercial-NoDerivs 3.0 United States License. To view a copy of this license, visit http://creativecommons.org/licenses/by-nc-nd/3.0/us/ or send a letter to Creative Commons, 444 Castro Street, Suite 900, Mountain View, California, 94041, USA.

\section{Yb:KGW Laser System}
\subsection{Introduction}
% Draws from:
% High-power, femtosecond, thermal-lens-shaped Yb:KGW oscillator
% Joel A. Berger, Michael J. Greco, and W. Andreas Schroeder
% 9 June 2008 / Vol. 16, No. 12 / OPTICS EXPRESS

% page 8630
Ytterbium-doped potassium gadolinium tungstate (\ce{Yb$:$KGd(WO4)2} or Yb:KGW) is becoming a widely used solid-state gain medium for ultrafast laser systems.
As with other Yb-doped laser crystals [1], this is primarily due to the gain medium’s advantageous absorption properties for direct laser diode pumping at 980nm, its broad emission bandwidth around 1040nm, and consequent small ($\sim$ 6\%) quantum defect that results in a relatively low thermal load. 
Mode-locked (ML), diode-pumped, Yb:KGW laser oscillators generating pulses with sub-picosecond durations at average output powers in excess of 1W have been demonstrated [2-5] and are now available commercially [6,7].
Such lasers have used dichroic mirrors [2,4,8,9], polarization-coupling [5], or more complex optical geometries (e.g., thin disk lasers [6,10]) to pump the laser crystal in an efficient manner.

% page 8631
The spectroscopic properties of the biaxial Yb:KGW laser crystal [11] indicate that the optimum diode pumping arrangement is for 980nm radiation polarized parallel to the optical $N_m$-axis since the crystal has the largest absorption cross-section in this configuration and, hence, the smallest absorption saturation intensity to drive the gain medium into a quasi 4-level system or to achieve transparency at the emission wavelength [1].
However, as is the case for Yb-doped potassium yttrium tungstate (Yb:KYW) [12,13], mode-locked Yb:KGW lasers should ideally oscillate with radiation polarized parallel to the optical $N_p$-axis to access the broadest gain bandwidth around 1040nm, where the emission cross-section is comparable to that along the optical $N_m$-axis [5,11]. This optimal requirement for orthogonal linear pump and laser polarizations led Holtom [5] to employ a polarization-coupling scheme to diode pump longitudinally a 10W, sub-500fs, Yb:KGW laser.

%TODO is this first sentance correct or should I use "I" voice?
In this section, a high-power, femtosecond Yb:KGW oscillator is presented that also accesses the broader emission bandwidth parallel to the optical $N_p$-axis (parallel to the crystallographic $b$-axis), yet employs a simpler geometry for polarized diode-pumping on the strongest 980nm absorption line parallel to the optical $N_m$-axis (rotated by $\sim\,20^\circ$ from the crystallographic $a$-axis [11,14,15]).
The laser head design is based on the thermal lens shaping (TLS) technique developed in-house for astigmatism compensation in diode-pumped Nd-doped lasers with Brewster-cut gain media [16].
It employs a simple single layer \ce{SiO2} anti-reflection coating on the gain medium to provide minimal loss for $p$-polarized intracavity 1040nm laser radiation and less than 0.5\% surface reflection for orthogonal $s$-polarized 980nm pump light.
The result is a simple and robust, soliton mode-locked and directly diode-pumped, solid-state laser oscillator delivering over 4W of average output power in a 63MHz train of pulses with a duration less than 300fs.
Frequency doubling of this laser output in a 2mm Brewster-cut Lithium triborate (LBO) crystal provides 1.65W of average power at 520nm.

The Watt-level green output power from the frequency-doubled sub-picosecond Yb:KGW oscillator corresponds to a visible peak laser pulse power in excess of 100kW, which is well-suited for further frequency conversion through harmonic generation or pumping of an ultrashort pulse optical parametric oscillator.
%TODO change voice/check ->
To our knowledge, the generation of subpicosecond green peak pulse powers greater than 100kW by frequency doubling the output of a laser oscillator has only been demonstrated for high-power thin-disk [17,18] and cavity-dumped [19] Yb-doped solid-state lasers, although externally-doubled commercial Ytterbium femtosecond oscillators [6,7,20,21] and oscillators with rod-like Yb-fiber gain media [22] can be expected to yield comparable peak green pulse powers.
%TODO replace to "This output will then be use to ..." and link to something from previous
%In our case, the visible femtosecond pulses will be used as the driving laser radiation source for an ultrafast electron microscope [23] – a technology that seeks to combine the sub-nanometer spatial resolution of electron microscopy with the sub-picosecond temporal resolution afforded by today’s ultrashort pulse laser systems.
%Critical to this technology is likely to be the development of a femtosecond photo-electron gun with a low spatial emittance [25].
%The coherent control of photoemission through visible laser-driven plasmon excitations on large-area (a few mm2), gold or silver, nano-patterned photocathodes may provide a mechanism by which this can be achieved [25].

\subsection{Laser Head Design}

The design of the laser head follows the example of Holtom [5], which ensures efficient pumping and access to the broadest emission bandwidth of Yb:KGW, while also employing the thermal lens shaping (TLS) technique to compensate for astigmatism [16].
To ensure efficient operation of a Yb:KGW laser, the dominant absorption feature at 980nm for radiation polarized parallel to the optical $N_m$-axis ($\sim\,20^\circ$ from the crystallographic a-axis) should be pumped using high brightness diode lasers.
On the other hand, as with \ce{Yb$:$KY(WO4)2} [12,13], the broadest emission bandwidth for ultrashort pulse generation occurs for radiation polarized parallel to the optical $N_p$-axis (parallel to the crystallographic $b$-axis [5]) in this biaxial material. 
%page 8632
Instead of using a polarization-coupled pumping geometry [5] to ensure these optimum pump-lasing conditions, we employ the non-Brewster crystal geometry and cut shown in Fig. 1(a), where a single 193nm-thick \ce{SiO2} anti-reflection coating is applied to both polished 3x10mm Yb:KGW crystal faces.
At an angle of incidence of 32.3$^\circ$, this coating theoretically has zero reflection loss for $p$-polarized 1040nm laser radiation parallel to the $b$-axis propagating perpendicular to the $a$-$b$ plane of the crystal (Fig. 1(b)); i.e., the preferred optical $N_p$-axis with refractive index $n_p$ = 1.98 [11,15].
The \ce{SiO2} coating also has an appropriate bandwidth for femtosecond laser operation in excess of 100nm (to the
0.1\% reflection points) for $p$-polarized radiation centered at 1040nm.
On the other hand, the $\sim$ 0.4\% reflection loss per surface for $s$-polarized 1040nm radiation is sufficient to suppress $s$-polarized oscillation in a laser resonator.
In addition, the same anti-reflection coating has less than 0.5\% reflection loss for $s$-polarized 980nm pump radiation (polarized parallel to the $a$-axis of the Yb:KGW crystal) at incidence angles between 21 and 39$^\circ$ (Fig. 1(b)).
Thus, in this Yb:KGW laser crystal geometry, the largest absorption cross-section associated with the optical $N_m$-axis is accessed with about 94\% efficiency ($\cos 20^\circ = 0.94$) by the $s$-polarized pump radiation.

% insert figure 1 here

Efficient pumping is achieved with two, 35W, dual-axis collimated and TM-polarized laser diode arrays operating at 980nm (HLU35C10x5-980 from LIMO GmbH [26]).
The $\sim$3nm emission linewidth of the two laser diode arrays is well matched to the 3.7nm fullwidth half-maximum (FWHM) 980nm absorption line of Yb:KGW [8,11,27].
Moreover, the two dual-axis collimated laser diodes have full-angle $1/e^2$ intensity beam divergences of $\theta_y \approx$ 4.4mrad and $\theta_x \approx$ 6.8mrad in the horizontal ($x$-direction parallel to the emitter array) and vertical ($y$) directions respectively. 
With 100mm focal length lenses, this allows the pump radiation to be focused to a pump spot diameter 2W $\approx$ 400$\mu$m, yielding a maximum incident pump irradiance of $\sim$28kW/cm$^2$ from one laser diode array.
The Yb concentration was chosen to be 2\% atomic doping to ensure that more than 90\% of the pump radiation was absorbed over the 3mm Yb:KGW crystal thickness.
With the employed counter-propagating pump geometry, the resultant maximum average pump irradiance over the crystal length is then roughly 3 times the $\sim$6kW/cm$^2$ saturation intensity of the dominant 980nm absorption line [28].
The laser crystal was obtained from NovaPhase [29].

To ensure a stigmatic laser resonator, we also employ the thermal lens shaping (TLS) technique [16] to compensate for the astigmatism induced by the non-normal incidence geometry of the laser crystal. 
For the employed non-Brewster crystal cut (Fig. 1(a)), a ray
%page 8633
transfer matrix analysis (with the $x$ direction parallel to the crystallographic $b$-axis and the $y$
direction parallel to the $a$-axis) reveals that the required ellipticity for a stigmatic thermal lens
generated by a parabolic temperature distribution, $T(x, y) = T_0 - \frac{1}{2} ( A_\smallT x^2 + B_\smallT y^2 )$, where $T_0$
is the on-axis temperature, is given by
%TODO check W symbol consistent with the rest of the paper
\begin{equation} \label{eq:laser_ratio_ba}
  \frac{B_\smallT}{A_\smallT} = 
  \frac{ \cos^2 \theta_2 \left( \dfrac{dn}{dT} \right) + 2 \Delta \alpha_\smallT \left( n \cos\theta_2 - \cos\theta_1 \right) }
       { \cos^2 \theta_1 \left[ \left( \dfrac{dn}{dT} \right) + 2 \Delta \alpha_\smallT \left( n \cos\theta_2 - \cos\theta_1 \right) \right] }
  \quad\text{.}
\end{equation}
Here, $\frac{dn}{dT}$ and $n$ are the thermo-optic coefficient and refractive index of the medium respectively (in this case for radiation polarized along the optical $N_p$-axis), $\theta_1$ is the 32.3$^\circ$ angle of incidence, $\theta_2$ = 15.7$^\circ$ is the angle of refraction (as $n_p$ = 1.98 [11,15]), and $\Delta$ is the fraction of the crystal length $l$ contributing to the bowing of the crystal faces under thermal
expansion with coefficient $\alpha T$ [16].
Generally, $\Delta \approx W l$ , where $W$ is the transverse size (radius) of the pumped region [30].
In our case, $l \approx$ 3mm and $W \approx$ 0.2mm, so that $\Delta < 0.1$, implying that the thermal duct due to a non-zero $\frac{dn}{dT}$ primarily determines the required $B_T$/$A_T$ ratio.
In other words, for the potassium gadolinium tungstate (\ce{KGd(WO4)2}) crystal geometry of Fig. 1(a), we expect from \ref{eq:laser_ratio_ba} to require $\dfrac{B_\smallT}{A_\smallT} \approx \dfrac{ \cos^2 \theta_2 }{ \cos^2 \theta_1 } = 1.3$.

The spatial pump distribution needed to generate the required $B_T/A_T$ ratio can be found using a cosine Fourier series solution to the thermal diffusion equation for our two-dimensional problem,
\begin{equation} \label{eq:heat_eqn}
  \kappa \nabla^2 T(x,y) = Q(x,y)
  \quad\text{,}
\end{equation}
where $\kappa$ is the thermal diffusion coefficient.
Thermal diffusion in the third $z$-direction has been neglected due to the relatively uniform longitudinal heat deposition resulting from our counter-propagating pump geometry [16].
The specifications of our dual-axis collimated diodes [26] indicate that heat source can be written as
\begin{equation}
  Q(x,y) = Q_0 \exp \left[ - \frac{x^2}{W_x^2} - \left( \frac{y^2}{W_y^2} \right)^3 \right]
  \quad\text{;}
\end{equation}
that is, the focused pump radiation is described accurately by a Gaussian irradiance distribution in the horizontal ($x$) direction with half-width 1/e maximum (HW1/eM) spot size $W_x$ and a super-Gaussian of order 3 in the vertical ($y$) direction with spot size $W_y$.
\ref{eq:heat_eqn} can then be solved using the appropriate boundary conditions for our 3x10mm crystal cross-section for the employed TLS technique [16]; namely, heat removal only through the top crystal face implies $T(x,y = \pm 1.5\text{mm}) = 0$ (or arbitrary constant) and $\dfrac{\partial T}{\partial x} = 0$ at $ x = \pm 5.0$mm for any $y$.
For a fixed vertical pump spot size of $W_y = 0.2$mm, our analysis indicates that in order to achieve the required ratio $B_T / A_T = 1.3$ in the parabolic approximation for $T(x,y)$ about $(x,y) = (0,0)$ a horizontal pump spot size $W_x = 0.3$mm is needed; in other words, the required $Q(x,y)$ heat source spot size ratio (or ellipticity) $W_x / W_y = 1.5$.
This analytical result for the central region around the laser beam axis is shown in Fig. 2; displayed are the
%page 8634
horizontal ($y = 0$, Fig. 2(a)) and vertical ($x = 0$, Fig. 2(b)) sections, respectively, through the $Q(x,y)$ and $T(x,y)$ distributions.
Also shown are the parabolic approximations to the pump-induced temperature distribution (dashed lines) and representative TEM$_{00}$ Gaussian laser modes (shaded area) in the Yb:KGW crystal with a HW1/eM field mode sizes of $w_y = 160\mu$m and $w_x = 182 \mu$m (14\% larger due to refraction (Fig. 1(a)).
This mode size ensures efficient pump-probe overlap while minimizing the loss due to ground state absorption in the unpumped regions of the Yb-doped gain medium [1]; in particular, in the vertical direction (Fig. 2(b)) where $w_y$ greater than about $0.8 W_y$ leads to significant absorptive loss as the extrema of the laser mode extend beyond the sharply-bounded super-Gaussian pump region.

% insert figure 2

The analysis clearly shows that the TEM$_{00}$ laser mode will experience a near perfect pump-induced thermal lens upon propagation through the gain medium --- the deviation of $T(x,y)$ from the parabolic approximation, and hence a perfect dominant thermo-optic (GRIN) duct, being insignificant over the size of the laser mode in both the horizontal and vertical directions.
Moreover, the fact that $W_x \approx 1.6 w_x$ implies that some degree of horizontal spatial walk-off between the pump and laser beams can be tolerated in this TLS geometry.
For example, when the pump and laser beams are exactly overlapped at the center of the Yb:KGW crystal, an external angle of 0.1rad (= 5.7$^\circ$) would generate an effective spatial walk-off of only about 75$\mu$m over our 3mm gain crystal length.
Clearly, this can be readily accommodated by the $W_x = 0.3$mm horizontal pump spot size when $w_x \approx 180\mu$m.
However, even though $W_x > w_x$, the Gaussian (rather than super-Gaussian) horizontal pump mode profile is still expected to assist in ensuring TEM$_{00}$-mode operation since the net horizontal gain profile (gain minus absorptive loss) will fall off more rapidly than the pump profile for a quasi-three level Yb-doped gain medium [1].

We note that the above thermal lens analysis neglects the anisotropy of the thermal conductivity $\kappa$ in potassium gadolinium tungstate (\ce{KGd(WO4)2}) [11,14] and stress-induced refractive index changes [31].
For the employed crystal cut in our TLS geometry (Fig. 1(a)), the anisotropy in $\kappa$ [14,27] is not expected to have a significant effect; the laser head geometry dictates that the dominant pump-induced heat conduction will be in the saggital ($y$) direction along the crystal $a$-axis where $\kappa \approx 2.6$W/(mK) (heat extraction only through the top and bottom 3x10mm crystal faces), so that the 46\% larger value of $\kappa$ along the $b$-axis ($x$) direction at 3.8W/(mK) should not have a large influence on the heat conduction.
The refractive index
%page 8635
changes due to stress in the pumped gain medium are difficult to quantify since the substantial variations in the reported values of the thermo-optic coefficients in KGW make it impossible to extract either the sign or magnitude of the stress dependence of the refractive index [21].
Moreover, nearly all analyses of thermal lensing effects Nd:KGW and Yb:KGW lasers have been performed for laser radiation polarized along the optical $N_m$-axis, which accesses the largest emission cross-section for both Nd- and Yb-doping [31,34].
Consequently, for our laser polarized along the optical $N_p$-axis [5], there is insufficient information to include accurately the effects of stress-induced refractive index changes.

\subsection{The Yb:KGW laser cavity}

The diode-pumped TLS Yb:KGW laser head is placed in the 2.35m-long, asymmetric z-fold cavity shown in Fig. 3(a).
The two focusing gain section mirrors each have a group velocity dispersion (GVD) of -1300($\pm$150) fs$^2$ (Layertec GmbH [32]), a radius of curvature $R$ = 50cm, and are positioned 29.5cm from the gain medium.
The 5$^\circ$ angle of incidence on the gain section mirrors results in minimal astigmatism, although any net round-trip cavity astigmatism can be compensated for by the TLS technique.
The longer 75cm arm of the resonator is terminated by a concave ($R$ = 1m and angle of incidence $< 2^\circ$) high reflector
% insert figure 3
%page 8636
focusing the intracavity radiation on a saturable Bragg reflector (SBR) with a 1\% reflectivity modulation depth at 1040nm (BATOP GmbH [33]) positioned a distance $z$ from the focusing mirror.
The shorter 63cm arm is terminated by a plane output coupler with a 7\% transmission at 1040nm.
As a free-running oscillator (SBR focusing section replaced by flat high reflector at $d_2$ = 75cm), the laser produced over 6W of TEM$_{00}$ ($M^2 < 1.2$) output power.

The stability analysis of our diode-pumped Yb:KGW laser (Fig. 3(b)), clearly indicates that the resonator configuration is unstable for thermal focal lengths less than about $-6m^{-1}$ (diopters) and is unlikely to oscillate for negative thermal focal lengths as this requires a vertical TEM$_{00}$-mode size $w_y$ greater than 180$\mu$m in the Yb:KGW gain medium --- a value that would result in strong absorptive losses due to the restrictive $W_y = 200\mu$m super-Gaussian vertical pump beam size in the Yb-doped crystal.
The clear implication is therefore that the net thermal lens due to the dominant thermal duct is positive; that is, the sum of the refractive index change due to the thermo-optic coefficient and stress for 1040nm laser radiation polarized along the optical $N_p$-axis (crystal $b$-axis) is positive.
This result is in apparent contradiction to some recent measurements of relatively large negative thermo-optic coefficients in Yb:KGW [11], but is consistent with other determinations of the pump-induced thermal focal length for laser oscillation polarized parallel to the optical $N_p$-axis [5,34].

\subsection{Mode-locked laser operation}

