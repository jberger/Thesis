%This work is licensed under the Creative Commons Attribution-NonCommercial-NoDerivs 3.0 United States License. To view a copy of this license, visit http://creativecommons.org/licenses/by-nc-nd/3.0/us/ or send a letter to Creative Commons, 444 Castro Street, Suite 900, Mountain View, California, 94041, USA.

\section{The Analytic Gaussian Model} \label{sec:ms_model}

This section introduces the Analytic Gaussian model of Michalik and Sipe \cite{michalik_analytic_2006}.
%TODO more here (obviously), comparison with GPT, extended in ch. 3, etc

\subsection{Formal Evolution} \label{sec:formal_evolution}

Before one may describe the evolution of a pulse of electrons, one must first learn to describe the pulse.
Rather than record the position of each electron, commonly one writes a distribution function for the electrons.
A distribution function describes the location of a number of objects in a certain space.
This space may be a 1D line, a 3D space, or even a 6D phase space (three spatial dimensions and three momentum dimensions).
The functional form may be chosen to appropriately characterize the objects in a natural way, however one property is true for all distribution functions: when integrated over all relevant space, all objects should be accounted for.
In $n$ dimensional space, with $N$ objects, this normalization condition on a distribution function $f$ can be written mathematically as
\begin{equation}
N = \int\limits_{\text{all space}} \!\!\! f(\vec{x}) \; d^{n}x
\end{equation}
which for 6D phase space is written as
\begin{equation} \label{eq:normalization}
N = \int f (\vec{r}, \vec{p}; t) \; d^{3}x\,d^{3}p
\end{equation}
This notation suggests to the reader that the distribution function is implicitly a function of time as well as of the spatial and momentum coordinates; for most circumstances however, the total number of particles $N$ will not change with time.

For a classical set of $N$ discreet particles (labelled by the index $i$), distributed in 6D space (3 position, 3 momentum) the distribution function is 
\begin{equation}
  f_{\smallD}(\vec{r}, \vec{p}; t) = 
  \sum\limits^{N}_{i=1} \delta(\vec{r}_{i} - \vec{r}) \delta(\vec{p}_{i} - \vec{p})
\end{equation}
which automatically satisfies \ref{eq:normalization}.
If the particles have mass $m$ which interact with some potential $V(\vec{r}_{i} - \vec{r}_{j})$, their motion is governed by the equations
\begin{subequations}\label{eq:std_motion}
\begin{gather}
  \frac{d\vec{r}_{i}(t)}{dt} = \frac{\vec{p}_{i}(t)}{m}\\
  \frac{d\vec{p}_{i}(t)}{dt} = -\sum\limits_{j} \frac{d}{d\vec{r}_{i}} V(\vec{r}_{i} - \vec{r}_{j}) \label{eq:std_motion_v}
\end{gather}
\end{subequations}
Now let us consider the time evolution of $f_{\smallD}$;
\begin{equation}
  \frac{\partial f_{\smallD}}{\partial t} = 
  \frac{d\vec{r}}{dt} \frac{\partial f_{\smallD}}{\partial \vec{r}} 
  + \frac{\partial f_{\smallD}}{\partial \vec{p}} \frac{d\vec{p}}{dt}
\end{equation}
but with \ref{eq:std_motion} the final discreet evolution equation is
\begin{equation} \label{eq:evolution_discreet}
  \frac{d}{dt} f_{\smallD}(\vec{r}, \vec{p}; t) =
  -\frac{\vec{p}}{m} \frac{\partial}{\partial \vec{r}} f_{\smallD}(\vec{r}, \vec{p}; t)
  + \frac{\partial}{\partial \vec{p}} f_{\smallD} (\vec{r}, \vec{p}; t)
  \frac{\partial}{\partial \vec{r}} \iint d\vecprime{r} d\vecprime{p} V(\vec{r} - \vecprime{r}) f_{\smallD}(\vecprime{r},\vecprime{p};t)
\end{equation}
where the extra minus sign comes from the change of perspective of \ref{eq:std_motion_v} from the frame of the individual particle $i$ to the observer frame.

This analysis can be extended to continuous distributions by taking the ensemble average over \ref{eq:evolution_discreet}.
In order to simplify the resulting equation to a useful result, one applies the usual mean-field approximation
\begin{equation}
  \left < f_{\smallD} (\vec{r}, \vec{p}; t) f_{\smallD} (\vecprime{r}, \vecprime{p}; t) \right > = \left < f_{\smallD} (\vec{r}, \vec{p}; t) \right > \left < f_{\smallD} (\vecprime{r}, \vecprime{p}; t) \right >
\end{equation}
so that the formal evolution of a continuous distribution is given by
\begin{equation} \label{eq:formal_evolution}
  \frac{d}{dt} f(\vec{r}, \vec{p}; t) =
  -\frac{\vec{p}}{m} \frac{\partial}{\partial \vec{r}} f(\vec{r}, \vec{p}; t)
  + \frac{\partial}{\partial \vec{p}} f (\vec{r}, \vec{p}; t)
  \frac{\partial}{\partial \vec{r}} \iint d\vecprime{r} d\vecprime{p} V(\vec{r} - \vecprime{r}) f(\vecprime{r},\vecprime{p};t)
\end{equation}
Although this equation looks very similar to \ref{eq:evolution_discreet}, notice that it contains approximations, mean-field that has been seen and in practice the self-similarity of the distribution function, that the discreet discription did not.

\subsection{Evolution of a Gaussian Distribution}

The formal evolution explored in Section \ref{sec:formal_evolution} allows for any distribution function.
For Ultrafast Electron Microscopy, the electron pulse is generated by a laser pulse, therefore the electron pulse can be thought to be a Gaussian in space and time, mimicing the laser pulse which generated it.
How correct this assumption is will be explored in \ref{sec:initial_conditions}.
In this section we consider a Gaussian distribution function, generally of the form
\begin{equation}
  f(\vec{r}, \vec{p}; t) = C(t)\exp \left[ - G(\vec{r}, \vec{p}; t) \right]
\end{equation}
The analysis presented here will consider a cylindrically symmetric pulse.
For the remainder of this analysis, the subscripts $\smallT$ (transverse) and $z$ (longitudinal) will accompany many quantites, reflecting the congruence of $x$ and $y$.
The subscript $\alpha$ will refer to either $\smallT$ or $z$.
In order to model several important physical characteristics of the pulse, we choose the argument (i.e. the function $G$) as
\begin{equation} \label{eq:gaussian_exponent}
  G(\vec{r}, \vec{p}; t) =
  \frac{x^2 + y^2}{2 \sigma_{\smallT}} + \frac{z^2}{2 \sigma_{z}}
  + \frac{
    [p_x - (\gamma_{\smallT}/\sigma_{\smallT}) x ]^2 
    + [p_y - (\gamma_{\smallT}/\sigma_{\smallT}) y ]^2
  }{2 \eta_{\smallT}}
  + \frac{ [p_z - (\gamma_{z}/\sigma_{z}) z ]^2 }{2 \eta_{z}}
\end{equation}
This parameterization defines the spatial variance $\sigma_{\alpha}$, the local momentum variance $\eta_{\alpha}$, and the momentum chirp $\gamma_{\alpha}$ in each spatial direction $\alpha$.
The ``chirp'' of a pulse quantifies the difference in average local momentum from one side of the pulse to the other.
For example, in a pulse which is normally (positively) chirped longitudinally, electrons near the front of the pulse will be moving faster (on average) than those in the back.
The normalization constant $C(t)$, for this distributions of $N$ total particles, is found by applying equation \ref{eq:normalization}:
\begin{equation}
  C(t) = \frac{N}{(2\pi)^3} 
  \left( 
    \frac{1}{\sigma_{\smallT}^2\eta_{\smallT}^2\sigma_{z}\eta_{z}}
  \right)^{\frac{1}{2}}
\end{equation}

To help simplify what is already becoming a complex problem, these equations may be viewed from a matrix perspective if we define the ``coordinate vector''
\begin{equation} \label{eq:coordinate_vector}
u_i = \{x, p_x, y, p_y, z, p_z\}\text{ .}
\end{equation}
In this space, \ref{eq:gaussian_exponent} becomes
\begin{equation}
  G(\vec{r}, \vec{p}; t) = \frac{1}{2}\sum\limits_{ij} A_{ij} u_i u_j
\end{equation}
where the 9 x 9 matrix $A$ is defined as
\begin{equation}
  A = 
  \begin{pmatrix}
    a_{\smallT} & 0 & 0 \\
    0 & a_{\smallT} & 0 \\
    0 & 0 & a_{z}
  \end{pmatrix}
  \qquad \text{and} \qquad
  a_{\alpha} = 
  \begin{pmatrix}
    1/\sigma_{\alpha} + \gamma_{\alpha}^2/(\eta_{\alpha} \sigma_{\alpha}) & \gamma_{\alpha}/(\eta_{\alpha} \sigma_{\alpha}) \\
    \gamma_{\alpha}/(\eta_{\alpha} \sigma_{\alpha}) & 1/\eta_{\alpha}
  \end{pmatrix}
\end{equation}
and again ${\scriptstyle \alpha} = \smallT \text{ or } z$. 
A nice property of Gaussians of this form is the convenient ``Multidimensional Gaussian Integral'' given by
\begin{equation} \label{eq:gaussian_integral}
  (A^{-1})_{ij} = \frac{1}{N} \int u_i u_j f(\vec{r}, \vec{p}; t) d\vec{r} d\vec{p}
\end{equation}
which is a common form of integral analysis in Quantum Field Theory (see for example Peskin and Schroeder Equation 9.70). 
Application of this integral yields another 9 x 9 matrix
\begin{equation}
  A = 
  \begin{pmatrix}
    a_{\smallT}^{-1} & 0 & 0 \\
    0 & a_{\smallT}^{-1} & 0 \\
    0 & 0 & a_{z}^{-1}
  \end{pmatrix}
  \qquad \text{and} \qquad
  a^{-1}_{\alpha} = 
  \begin{pmatrix}
    \sigma_{\alpha} & \gamma_{\alpha} \\
    \gamma_{\alpha} & \eta_{\alpha} + \gamma_{\alpha}^2/\sigma_{\alpha}
  \end{pmatrix}
\end{equation}

Since time derivative of \ref{eq:gaussian_integral} is simply
\begin{equation} \label{eq:gaussian_integral_dt}
  \frac{d}{dt}(A^{-1})_{ij} = \frac{1}{N} \int u_i u_j \frac{df(\vec{r}, \vec{p}; t)}{dt} d\vec{r} d\vec{p}
\end{equation}
and the time derivative of the components of $A^{-1}$ are
\begin{equation} \label{eq:dainvdt}
  \frac{d}{dt} a^{-1}_{\alpha} = 
  \begin{pmatrix}
    \dfrac{d\sigma_{\alpha}}{dt} & \dfrac{d\gamma_{\alpha}}{dt} \\
    \dfrac{d\gamma_{\alpha}}{dt} & \dfrac{d\eta_{\alpha}}{dt} + 2\dfrac{\gamma_{\alpha}}{\sigma_{\alpha}}\dfrac{d\gamma_{\alpha}}{dt}- \dfrac{\gamma^{2}_{\alpha}}{\sigma^{2}_{\alpha}}\dfrac{d\sigma_{\alpha}}{dt}
  \end{pmatrix}
\end{equation}
these components, when viewed in the context of \ref{eq:formal_evolution}, form the left hand side (LHS) of the system of differential equations governing the evolution of $\sigma_{\alpha}$, $\gamma_{\alpha}$, and $\eta_{\alpha}$, which will be established in the upcoming sections.

\subsection{Generating the Differential Equations}

Inserting \ref{eq:formal_evolution} into \ref{eq:gaussian_integral_dt} yields two integral terms, one related to the propagation of a non-charged Gaussian beam, the other adds the influence of the internal force due to Coulomb repulsion for an electron pulse.
These terms will be denoted $K^{flow}$ and $K^{force}$ respectively.
\begin{equation} \label{eq:gaussian_integral_dt_k}
  \frac{d}{dt}(A^{-1})_{ij} = K^{flow}_{ij} + K^{int}_{ij}
\end{equation}
Collecting and expanding terms where possible gives these forms of the $K$ terms
\begin{gather}
  K^{flow}_{ij} \equiv -\frac{1}{N} \int u_i u_j \frac{\vec{p}}{m} \frac{\partial f}{\partial \vec{r}} d\vec{r} d\vec{p} \\
  \label{eq:Kforce}
  K^{force}_{ij} \equiv \frac{1}{N} \int u_i u_j \frac{\partial f}{\partial \vec{p}} \frac{\partial \Phi(\vec{r};t)}{\partial \vec{r}} d\vec{r} d\vec{p} \\
  n(\vec{r};t) \equiv \int d\vec{p} f(\vec{r},\vec{p};t) 
    = \frac{N}{(2\pi)^{3/2} \sigma_{\smallT} \sqrt{\sigma_{z}}} \exp \left( -\frac{x^2+y^2}{2\sigma_{T}} - \frac{z^2}{2\sigma_{z}} \right) \\
    \label{eq:potential_integral}
    \Phi(\vec{r};t) \equiv \int d\vecprime{r} V(\vec{r} - \vecprime{r}) n(\vecprime{r}; t)
\end{gather}
Obviously the $K^{flow}$ term is far easier to evaluate;
\begin{equation}
  K^{flow} = 
  \begin{pmatrix}
    k^{flow}_{\smallT} & 0 & 0 \\
    0 & k^{flow}_{\smallT} & 0 \\
    0 & 0 & k^{flow}_{z}
  \end{pmatrix}
  \quad \text{and} \quad
  k^{flow}_{\alpha} = 
  \begin{pmatrix}
    \dfrac{2\sigma_{\alpha}}{m} & \dfrac{1}{m} \left(\eta_{\alpha} + \dfrac{\gamma_{\alpha}^2}{\sigma_{\alpha}} \right) \\
    \dfrac{1}{m} \left(\eta_{\alpha} + \dfrac{\gamma_{\alpha}^2}{\sigma_{\alpha}} \right) & 0
  \end{pmatrix}
\end{equation}

The $K^{force}$ integrals, related to the internal force of the pulse's Coulombic repulsion, are far more complicated to evaluate.
For more information about the derivation, see \cite{michalik_analytic_2006}. %TODO link to an appendix with Bessel etc for Kforce
The final matrices are
\begin{equation}
  K^{force} = 
  \begin{pmatrix}
    \hat{k}^{force}_{\smallT} & 0 & 0 \\
    0 & \hat{k}^{force}_{\smallT} & 0 \\
    0 & 0 & \hat{k}^{force}_{z}
  \end{pmatrix}
\end{equation}
where
\begin{equation}
  \hat{k}^{force}_{\alpha} = 
  \frac{N e^2}{4\pi\varepsilon_0} \frac{1}{2\pi^{1/2}\sigma_{\alpha}^{1/2}} L_{\alpha}(\xi)
  \begin{pmatrix}
    0 & 1 \\
    1 & \dfrac{2 \gamma_{\alpha}}{\sigma_{\alpha}}
  \end{pmatrix}
\end{equation}
The parameter $\xi$ quantifies the spatial ellipticity of the electron pulse, defined as
\begin{equation} \label{eq:define_xi}
  \xi \equiv \sqrt{\dfrac{\sigma_{z}}{\sigma_{\smallT}}}
\end{equation}
[Note that the original paper of Michalik and Sipe (Reference \cite{michalik_analytic_2006}) contained an error which incorrectly defined $\xi$; this was corrected in an erratum (Reference \cite{michalik_erratum:_2008}).]
The family of $L$ functions are
\begin{subequations}
  \begin{gather}
    L_{\smallT}(\xi) = \frac{3}{2} \left( L(\xi) + \frac{\xi^2 L(\xi) - \xi}{ 1 - \xi^2 } \right) \\
    L_{z}(\xi) = \frac{ 3 \xi^2 }{ \xi^2 - 1} \left( \xi L(\xi) - 1 \right) \\
    L(\xi) = \frac{1}{2} \int_0^{\pi} \frac{1}{1+\xi\sin(\theta)} d\theta \label{eq:ag_l}
  \end{gather}
\end{subequations}
The integral function $L(\xi)$ (\ref{eq:ag_l}) can be further evaluated to the piecewise function
\begin{equation}
  L(\xi) = 
  \begin{cases}
    \frac{ \arcsin \sqrt{1-\xi^2} }{ \sqrt{1-\xi^2} } & 0 \le \xi \le 1 \\
    \frac{ \ln\left( \xi + \sqrt{\xi^2 - 1} \right) }{ \sqrt{\xi^2 - 1} } & \xi \ge 1
  \end{cases}
\end{equation}
which, although defined piecewise, is a smooth and well behaved function for all physically meaningful $\xi$.

\subsection{The Differential Equation System}

When these $K$ matrices are summed (\ref{eq:gaussian_integral_dt_k}) to form the right hand side (RHS) of the differential equations and compared to LHS derived above (\ref{eq:dainvdt}), the non-zero elements finally yield a system of differential equations which govern the dynamics of these Gaussian beam parameters.
\begin{subequations} \label{eq:ag_original}
\begin{gather}
  \frac{d\sigma_{\alpha}}{dt} = \frac{2\gamma_{\alpha}}{m} \\
  \frac{d\gamma_{\alpha}}{dt} = \frac{1}{m} \left(\eta_{\alpha} + \frac{\gamma_{\alpha}^2}{\sigma_{\alpha}} \right) 
    + \frac{N e^2}{4\pi\varepsilon_0} \frac{1}{6 \sqrt{\sigma_{\alpha}\pi}} L_{\alpha}(\xi)\\
  \frac{d\eta_{\alpha}}{dt} = - \frac{2 \gamma_{\alpha} \eta_{\alpha}}{m \sigma_{\alpha}} 
\end{gather}
\end{subequations}
These equations are the crux of the model presented by Michalik and Sipe for the time evolution of a Gaussian electron pulse propagating through free space (no external forces).
They form the basis upon which much of the work in the following chapters is founded.

\subsection{Validity and Limitations}

This AG formalism of Michalik and Sipe \cite{michalik_analytic_2006} employs a mean-field approximation in the evolution of the dynamical effect of the electron pulse's internal space-charge field.
This is of course not strictly accurate for a Gaussian charge distribution and the error leads to a distortion of the pulse shape.
The ``self-similar'' nature of the model, the fact that the overall shape of the pulse cannot change, prevents inclusion of these distortions.
Even so, the AG model of charge bunch dynamics has been benchmarked against particle tracking simulations for a number of electron pulse shapes \cite{michalik_analytic_2006,michalik_evolution_2009}, including the uniform ellipsoid which explicitly features a linear internal space-charge field \cite{luiten_how_2004}.
%TODO add the error found in these benchmarks
