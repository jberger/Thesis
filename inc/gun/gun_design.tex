\section{Accelerator Design} \label{sec:gun_design}

The acceleration region of our prototype UEM system has several requirement which are unique in comparison to standard electron guns.
Primarily, given the limitations on the beam discussed earlier, %TODO link to discussion of Childs law beam size limitation
the employ laser pulse must be a relatively large area.
This concern leads to several additional criteria.
To accomodate this large beam, the anode aperture must be similarly large; in fact, this will be larger than the emission source due to beam expansion during acceleration.
Also, in transit from the cathode to the anode, this large beam is more susceptable to non-uniformities in the accelerating electic field given that it will span a wider proportion of the accelerator than thin beams used in standard TEM.

\begin{figure}
  \centering
  \begin{tikzpicture}
  \node {\includegraphics{gunfield.png}};
\end{tikzpicture}

  \caption{
    Schematic of the employed Togawa inspired cathode ($-V$) and anode (0V) pair.
    The emission source is inserted behind the cathode Wehnelt.
    Also shown are simulated equipotential lines generated by FEM modeling.
    The separation of the cathode and anode is adjustable to allow a greater range of input laser angles if needed. 
  }
  \label{fig:gun-field}
\end{figure}

These requirements lead to the fabrication of cathode/Wehnelt and anode pair based on the design of Togawa \textit{et al.} \cite{togawa_ceb6_2007}.
A schematic of this design, overlayed by a Finite Element Model (FEM) simulation of the electic field, is shown in \ref{fig:gun-field}.
The schematic shows that the aperture concerns are easily satisfied.
From the field equipotentials, it is clear that this design has a consistently flat electric field for large displacements off the central axis.

Of course, as derived in Section \ref{sec:gun_model}, the change of electric field strength near the anode will cause some divergence of the electic field.
This field will cause a divergence of the beam.
Future work will include investigating the designs of Butler %TODO reference
which may be able to build in an intrinsic lensing to compensate for this effect.

%TODO comment on alignment
%TODO comment on high voltage
%TODO comment on EV parts etc

Re-alignment of cathode and anode is sometimes necessary to direct the emitted beam down the center axis of the column.
To facilitate this, the prototype column has the photocathode and Wehnelt mounted on the end of the high voltage feedthrough, which is in turn mounted on a two axis alignment mechanism.
Although this mechanism, called a ``port aligner,'' is typically used to couple two non-colinear vacuum ports, for use in this instrument it has been modified to be able to manually adjust the aligner during use.
In contrast to this flexibility, the anode is mounted to column using Kimball Physics's ``eV Parts'' mounting systems.
These strictly manufactured mounting hardwares ensure that the anode is properly aligned with the remainder of the microscope column.
Thus, by iteratively repositioning the input laser and then the gun alignment, the beam may be walked to pass as near as possible down the axis of the column.

