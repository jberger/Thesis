\subsection{Acceleration Region} \label{sec:gun_model}

In the acceleration region of a UEM column, a constant electric field is applied to the newly created electron pulse, increasing its velocity until it achieves its ultimate velocity just upon leaving the region at the anode aperture.
As seen in Section \ref{sec:const_force}, this constant electric field does not affect the intrapulse dynamics, however the distortion in the electric field near this aperture causes the anode itself to act as a negative transverse lens. %TODO reference
Additionally, the longitudinal electric field stregth sharply decreases near the anode.
As the front of the pulse sees this change before the back, this changing field also acts as a pulse compression element.

Since the aforementioned effects are localized in the area very near the anode, the following analysis is only asserted for pulses that are of approximately the same length as the acceleration region or shorter.
This assumption is very reasonable for UEM and for some of the shorter-pulse DTEM systems.
Though these effects are expected for longer pulses, they are unlikely to be predominant factors in the dynamics.

To model an acceleration region, one must assume a form for the accelerating field.
From that form one may derive the terms governing the pulse dynamics. Initially, assume that the on-axis field is purely in the longitudinal direction, which may be thought of as an implication of radial symmetry.
Mathematically, that is
\begin{equation}
  \vec{F}(r^{\prime}=0,z^{\prime}) = F_z (0,z^{\prime}) \hat{z} = e E_z (0,z^{\prime}) \hat{z}
\end{equation}
where as before, the primed coordinates are the location of the center of the pulse.

To first model the longitudinal (compression) action at the anode, %TODO ref
consider electrons that are a distance $z$ ahead or behind of the center of the pulse and so feel a slightly different force, 
\begin{equation}
  F_z(0,z^{\prime} - z) \approx F_z(0,z^{\prime} - z) + \frac{\partial F_z}{\partial z^{\prime}} z
\end{equation}
Again, the constant leading term does not contribute to the intrapulse dynamic, however the second term is of the proper form to say that in this region the anode's longitudinal compression strength $M_z$ is given by
\begin{equation}
  M_z = \frac{\partial F_z}{\partial z^{\prime}} = e \frac{\partial E_z}{\partial z^{\prime}} \, \text{.}
\end{equation}

To model the radial action (negative transverse lensing), we can expand the radial force $F_r$ for electrons that are a distance $r$ radially outward from the center of the pulse,
\begin{equation}
  F_r(r,z^{\prime}) \approx F_r(r,z^{\prime}) + \frac{\partial F_r}{\partial r^{\prime}} r
\end{equation}
Without an analytic form for $F$, we can appeal to Gauss' law (in absense of external charge)
\begin{gather}
  \frac{\partial F_x}{\partial x^{\prime}} + \frac{\partial F_y}{\partial y^{\prime}} + \frac{\partial F_z}{\partial z^{\prime}} = 2 \frac{\partial F_{\smallT}}{\partial r^{\prime}} + \frac{\partial F_z}{\partial z^{\prime}} = 0 \\
  \Rightarrow \frac{\partial F_{\smallT}}{\partial r^{\prime}} = - \frac{1}{2} \frac{\partial F_z}{\partial z^{\prime}}
\end{gather} 
to see that the transverse lens traverse lens strength $M_{\smallT}$ from the anode is simply,
\begin{equation}
  M_{\smallT} = -\frac{1}{2}\frac{\partial F_z}{\partial z^{\prime}} \, \text{.}
\end{equation}

We may choose to assume that the on-axis field is defined as a smoothed step down function
\begin{equation} \label{eq:anode_tanh}
  F_z(0,z^{\prime}) = \frac{eV}{2z^{\prime}_A} \left( 1 - \tanh \left( \frac{ z^{\prime} - z^{\prime}_A }{ z^{\prime}_A / s } \right) \right)
\end{equation}
where $V$ is the accelerating potential, $z^{\prime}_A$ is the position of the anode (the cathode assumed to be at $z^{\prime} = 0$) and $s$ is a constant quantifying the ``sharpness'' of the fall-off of the electric field.
Indeed, this form is analgous to the super-Gaussian envalope (\ref{eq:reg_of_influence}), however it is open ended on the cathode side, which is necessary for the electrons to be accelerated away from the photocathode.
Given this form the ``lensing'' strengths are then
\begin{gather}
  M_{\smallT} = -\frac{eVs}{4z^{\prime 2}_A} \operatorname{sech}^2 \left( \frac{z^{\prime} - z^{\prime}_A }{ z^{\prime}_A / s } \right) \\
  M_z = \frac{eVs}{2z^{\prime 2}_A} \operatorname{sech}^2 \left( \frac{z^{\prime} - z^{\prime}_A }{ z^{\prime}_A / s } \right) \, \text{.}
\end{gather}
As with the RF cavity, one needs to include the on-axis force (\ref{eq:anode_tanh}) in the overall pulse propagation kinematics.

\begin{figure}
  \centerline{
    \subfloat[][]{
      \label{fig:field_on_axis}
      \begin{tikzpicture}
        \begin{tikzpicture}[gnuplot]
%% generated with GNUPLOT 4.6p0 (Lua 5.1; terminal rev. 99, script rev. 100)
%% Fri 22 Mar 2013 10:33:44 AM CDT
\path (0.000,0.000) rectangle (8.750,8.750);
\gpcolor{color=gp lt color border}
\gpsetlinetype{gp lt border}
\gpsetlinewidth{1.00}
\draw[gp path] (1.504,0.985)--(1.684,0.985);
\draw[gp path] (8.197,0.985)--(8.017,0.985);
\node[gp node right] at (1.320,0.985) { 0};
\draw[gp path] (1.504,2.042)--(1.684,2.042);
\draw[gp path] (8.197,2.042)--(8.017,2.042);
\node[gp node right] at (1.320,2.042) { 0.5};
\draw[gp path] (1.504,3.098)--(1.684,3.098);
\draw[gp path] (8.197,3.098)--(8.017,3.098);
\node[gp node right] at (1.320,3.098) { 1};
\draw[gp path] (1.504,4.155)--(1.684,4.155);
\draw[gp path] (8.197,4.155)--(8.017,4.155);
\node[gp node right] at (1.320,4.155) { 1.5};
\draw[gp path] (1.504,5.211)--(1.684,5.211);
\draw[gp path] (8.197,5.211)--(8.017,5.211);
\node[gp node right] at (1.320,5.211) { 2};
\draw[gp path] (1.504,6.268)--(1.684,6.268);
\draw[gp path] (8.197,6.268)--(8.017,6.268);
\node[gp node right] at (1.320,6.268) { 2.5};
\draw[gp path] (1.504,7.324)--(1.684,7.324);
\draw[gp path] (8.197,7.324)--(8.017,7.324);
\node[gp node right] at (1.320,7.324) { 3};
\draw[gp path] (1.504,8.381)--(1.684,8.381);
\draw[gp path] (8.197,8.381)--(8.017,8.381);
\node[gp node right] at (1.320,8.381) { 3.5};
\draw[gp path] (1.504,0.985)--(1.504,1.165);
\draw[gp path] (1.504,8.381)--(1.504,8.201);
\node[gp node center] at (1.504,0.677) { 0};
\draw[gp path] (2.843,0.985)--(2.843,1.165);
\draw[gp path] (2.843,8.381)--(2.843,8.201);
\node[gp node center] at (2.843,0.677) { 2};
\draw[gp path] (4.181,0.985)--(4.181,1.165);
\draw[gp path] (4.181,8.381)--(4.181,8.201);
\node[gp node center] at (4.181,0.677) { 4};
\draw[gp path] (5.520,0.985)--(5.520,1.165);
\draw[gp path] (5.520,8.381)--(5.520,8.201);
\node[gp node center] at (5.520,0.677) { 6};
\draw[gp path] (6.858,0.985)--(6.858,1.165);
\draw[gp path] (6.858,8.381)--(6.858,8.201);
\node[gp node center] at (6.858,0.677) { 8};
\draw[gp path] (8.197,0.985)--(8.197,1.165);
\draw[gp path] (8.197,8.381)--(8.197,8.201);
\node[gp node center] at (8.197,0.677) { 10};
\draw[gp path] (1.504,8.381)--(1.504,0.985)--(8.197,0.985)--(8.197,8.381)--cycle;
\node[gp node center,rotate=-270] at (0.246,4.683) {Electric field (kV/mm)};
\node[gp node center] at (4.850,0.215) {Position in column (mm)};
\gpcolor{rgb color={1.000,0.000,0.000}}
\gpsetlinetype{gp lt plot 0}
\draw[gp path] (1.504,7.659)--(1.572,7.659)--(1.639,7.659)--(1.707,7.659)--(1.774,7.659)%
  --(1.842,7.659)--(1.910,7.659)--(1.977,7.659)--(2.045,7.659)--(2.112,7.658)--(2.180,7.658)%
  --(2.248,7.658)--(2.315,7.657)--(2.383,7.657)--(2.450,7.657)--(2.518,7.656)--(2.586,7.655)%
  --(2.653,7.655)--(2.721,7.654)--(2.789,7.653)--(2.856,7.651)--(2.924,7.650)--(2.991,7.648)%
  --(3.059,7.646)--(3.127,7.644)--(3.194,7.641)--(3.262,7.637)--(3.329,7.633)--(3.397,7.628)%
  --(3.465,7.623)--(3.532,7.616)--(3.600,7.608)--(3.667,7.599)--(3.735,7.589)--(3.803,7.576)%
  --(3.870,7.561)--(3.938,7.544)--(4.005,7.523)--(4.073,7.499)--(4.141,7.471)--(4.208,7.438)%
  --(4.276,7.399)--(4.343,7.355)--(4.411,7.303)--(4.479,7.242)--(4.546,7.173)--(4.614,7.092)%
  --(4.681,7.000)--(4.749,6.895)--(4.817,6.775)--(4.884,6.640)--(4.952,6.488)--(5.020,6.320)%
  --(5.087,6.133)--(5.155,5.930)--(5.222,5.709)--(5.290,5.472)--(5.358,5.222)--(5.425,4.960)%
  --(5.493,4.690)--(5.560,4.415)--(5.628,4.139)--(5.696,3.865)--(5.763,3.597)--(5.831,3.339)%
  --(5.898,3.093)--(5.966,2.862)--(6.034,2.646)--(6.101,2.448)--(6.169,2.268)--(6.236,2.105)%
  --(6.304,1.959)--(6.372,1.829)--(6.439,1.714)--(6.507,1.613)--(6.574,1.525)--(6.642,1.448)%
  --(6.710,1.382)--(6.777,1.324)--(6.845,1.275)--(6.912,1.232)--(6.980,1.195)--(7.048,1.164)%
  --(7.115,1.137)--(7.183,1.115)--(7.251,1.095)--(7.318,1.079)--(7.386,1.064)--(7.453,1.052)%
  --(7.521,1.042)--(7.589,1.034)--(7.656,1.026)--(7.724,1.020)--(7.791,1.015)--(7.859,1.010)%
  --(7.927,1.006)--(7.994,1.003)--(8.062,1.000)--(8.129,0.998)--(8.197,0.996);
\gpcolor{rgb color={0.000,1.000,0.000}}
\draw[gp path] (1.504,7.660)--(1.572,7.660)--(1.639,7.660)--(1.707,7.660)--(1.774,7.660)%
  --(1.842,7.660)--(1.910,7.660)--(1.977,7.660)--(2.045,7.660)--(2.112,7.660)--(2.180,7.660)%
  --(2.248,7.660)--(2.315,7.660)--(2.383,7.660)--(2.450,7.660)--(2.518,7.660)--(2.586,7.660)%
  --(2.653,7.660)--(2.721,7.660)--(2.789,7.660)--(2.856,7.660)--(2.924,7.660)--(2.991,7.660)%
  --(3.059,7.660)--(3.127,7.660)--(3.194,7.660)--(3.262,7.660)--(3.329,7.660)--(3.397,7.660)%
  --(3.465,7.659)--(3.532,7.659)--(3.600,7.659)--(3.667,7.659)--(3.735,7.659)--(3.803,7.659)%
  --(3.870,7.658)--(3.938,7.658)--(4.005,7.657)--(4.073,7.656)--(4.141,7.654)--(4.208,7.652)%
  --(4.276,7.649)--(4.343,7.644)--(4.411,7.638)--(4.479,7.630)--(4.546,7.619)--(4.614,7.603)%
  --(4.681,7.580)--(4.749,7.550)--(4.817,7.507)--(4.884,7.450)--(4.952,7.370)--(5.020,7.264)%
  --(5.087,7.120)--(5.155,6.932)--(5.222,6.687)--(5.290,6.378)--(5.358,6.000)--(5.425,5.554)%
  --(5.493,5.049)--(5.560,4.508)--(5.628,3.957)--(5.696,3.425)--(5.763,2.938)--(5.831,2.513)%
  --(5.898,2.158)--(5.966,1.871)--(6.034,1.645)--(6.101,1.473)--(6.169,1.342)--(6.236,1.246)%
  --(6.304,1.174)--(6.372,1.122)--(6.439,1.084)--(6.507,1.056)--(6.574,1.036)--(6.642,1.022)%
  --(6.710,1.012)--(6.777,1.004)--(6.845,0.999)--(6.912,0.995)--(6.980,0.992)--(7.048,0.990)%
  --(7.115,0.989)--(7.183,0.988)--(7.251,0.987)--(7.318,0.986)--(7.386,0.986)--(7.453,0.986)%
  --(7.521,0.985)--(7.589,0.985)--(7.656,0.985)--(7.724,0.985)--(7.791,0.985)--(7.859,0.985)%
  --(7.927,0.985)--(7.994,0.985)--(8.062,0.985)--(8.129,0.985)--(8.197,0.985);
\gpcolor{rgb color={0.000,0.000,1.000}}
\draw[gp path] (1.504,7.660)--(1.572,7.660)--(1.639,7.660)--(1.707,7.660)--(1.774,7.660)%
  --(1.842,7.660)--(1.910,7.660)--(1.977,7.660)--(2.045,7.660)--(2.112,7.660)--(2.180,7.660)%
  --(2.248,7.660)--(2.315,7.660)--(2.383,7.660)--(2.450,7.660)--(2.518,7.660)--(2.586,7.660)%
  --(2.653,7.660)--(2.721,7.660)--(2.789,7.660)--(2.856,7.660)--(2.924,7.660)--(2.991,7.660)%
  --(3.059,7.660)--(3.127,7.660)--(3.194,7.660)--(3.262,7.660)--(3.329,7.660)--(3.397,7.660)%
  --(3.465,7.660)--(3.532,7.660)--(3.600,7.660)--(3.667,7.660)--(3.735,7.660)--(3.803,7.660)%
  --(3.870,7.660)--(3.938,7.660)--(4.005,7.660)--(4.073,7.660)--(4.141,7.660)--(4.208,7.660)%
  --(4.276,7.660)--(4.343,7.660)--(4.411,7.660)--(4.479,7.660)--(4.546,7.659)--(4.614,7.659)%
  --(4.681,7.659)--(4.749,7.658)--(4.817,7.656)--(4.884,7.653)--(4.952,7.646)--(5.020,7.633)%
  --(5.087,7.609)--(5.155,7.561)--(5.222,7.471)--(5.290,7.303)--(5.358,7.001)--(5.425,6.490)%
  --(5.493,5.711)--(5.560,4.693)--(5.628,3.599)--(5.696,2.648)--(5.763,1.960)--(5.831,1.526)%
  --(5.898,1.275)--(5.966,1.138)--(6.034,1.065)--(6.101,1.026)--(6.169,1.006)--(6.236,0.996)%
  --(6.304,0.991)--(6.372,0.988)--(6.439,0.987)--(6.507,0.986)--(6.574,0.985)--(6.642,0.985)%
  --(6.710,0.985)--(6.777,0.985)--(6.845,0.985)--(6.912,0.985)--(6.980,0.985)--(7.048,0.985)%
  --(7.115,0.985)--(7.183,0.985)--(7.251,0.985)--(7.318,0.985)--(7.386,0.985)--(7.453,0.985)%
  --(7.521,0.985)--(7.589,0.985)--(7.656,0.985)--(7.724,0.985)--(7.791,0.985)--(7.859,0.985)%
  --(7.927,0.985)--(7.994,0.985)--(8.062,0.985)--(8.129,0.985)--(8.197,0.985);
\gpcolor{color=gp lt color border}
\gpsetlinetype{gp lt border}
\draw[gp path] (1.504,8.381)--(1.504,0.985)--(8.197,0.985)--(8.197,8.381)--cycle;
%% coordinates of the plot area
\gpdefrectangularnode{gp plot 1}{\pgfpoint{1.504cm}{0.985cm}}{\pgfpoint{8.197cm}{8.381cm}}
\end{tikzpicture}
%% gnuplot variables

      \end{tikzpicture}
    }
    \subfloat[][]{
      \label{fig:tanh_sharpness}
      \begin{tikzpicture}
  %% \begin{tikzpicture}[gnuplot]
%% generated with GNUPLOT 4.6p0 (Lua 5.1; terminal rev. 99, script rev. 100)
%% Thu 20 Dec 2012 03:14:05 PM CST
\path (0.000,0.000) rectangle (10.000,8.750);
\gpcolor{color=gp lt color border}
\gpsetlinetype{gp lt border}
\gpsetlinewidth{1.00}
\draw[gp path] (1.688,0.985)--(1.868,0.985);
\node[gp node right] at (1.504,0.985) { 0.98};
\draw[gp path] (1.688,1.725)--(1.868,1.725);
\node[gp node right] at (1.504,1.725) { 1};
\draw[gp path] (1.688,2.464)--(1.868,2.464);
\node[gp node right] at (1.504,2.464) { 1.02};
\draw[gp path] (1.688,3.204)--(1.868,3.204);
\node[gp node right] at (1.504,3.204) { 1.04};
\draw[gp path] (1.688,3.943)--(1.868,3.943);
\node[gp node right] at (1.504,3.943) { 1.06};
\draw[gp path] (1.688,4.683)--(1.868,4.683);
\node[gp node right] at (1.504,4.683) { 1.08};
\draw[gp path] (1.688,5.423)--(1.868,5.423);
\node[gp node right] at (1.504,5.423) { 1.1};
\draw[gp path] (1.688,6.162)--(1.868,6.162);
\node[gp node right] at (1.504,6.162) { 1.12};
\draw[gp path] (1.688,6.902)--(1.868,6.902);
\node[gp node right] at (1.504,6.902) { 1.14};
\draw[gp path] (1.688,7.641)--(1.868,7.641);
\node[gp node right] at (1.504,7.641) { 1.16};
\draw[gp path] (1.688,8.381)--(1.868,8.381);
\node[gp node right] at (1.504,8.381) { 1.18};
\draw[gp path] (1.688,0.985)--(1.688,1.165);
\draw[gp path] (1.688,8.381)--(1.688,8.201);
\node[gp node center] at (1.688,0.677) { 0};
\draw[gp path] (2.994,0.985)--(2.994,1.165);
\draw[gp path] (2.994,8.381)--(2.994,8.201);
\node[gp node center] at (2.994,0.677) { 2};
\draw[gp path] (4.300,0.985)--(4.300,1.165);
\draw[gp path] (4.300,8.381)--(4.300,8.201);
\node[gp node center] at (4.300,0.677) { 4};
\draw[gp path] (5.607,0.985)--(5.607,1.165);
\draw[gp path] (5.607,8.381)--(5.607,8.201);
\node[gp node center] at (5.607,0.677) { 6};
\draw[gp path] (6.913,0.985)--(6.913,1.165);
\draw[gp path] (6.913,8.381)--(6.913,8.201);
\node[gp node center] at (6.913,0.677) { 8};
\draw[gp path] (8.219,0.985)--(8.219,1.165);
\draw[gp path] (8.219,8.381)--(8.219,8.201);
\node[gp node center] at (8.219,0.677) { 10};
\draw[gp path] (8.219,0.985)--(8.039,0.985);
\node[gp node left] at (8.403,0.985) { 0};
\draw[gp path] (8.219,1.971)--(8.039,1.971);
\node[gp node left] at (8.403,1.971) { 10};
\draw[gp path] (8.219,2.957)--(8.039,2.957);
\node[gp node left] at (8.403,2.957) { 20};
\draw[gp path] (8.219,3.943)--(8.039,3.943);
\node[gp node left] at (8.403,3.943) { 30};
\draw[gp path] (8.219,4.930)--(8.039,4.930);
\node[gp node left] at (8.403,4.930) { 40};
\draw[gp path] (8.219,5.916)--(8.039,5.916);
\node[gp node left] at (8.403,5.916) { 50};
\draw[gp path] (8.219,6.902)--(8.039,6.902);
\node[gp node left] at (8.403,6.902) { 60};
\draw[gp path] (8.219,7.888)--(8.039,7.888);
\node[gp node left] at (8.403,7.888) { 70};
\draw[gp path] (1.688,8.381)--(1.688,0.985)--(8.219,0.985)--(8.219,8.381)--cycle;
\node[gp node center,rotate=-270] at (0.246,4.683) {HW1/eM pulse width (mm) (solid)};
\node[gp node center,rotate=-270] at (9.292,4.683) {HW1/eM pulse length ($\mu$m) (dashed)};
\node[gp node center] at (4.953,0.215) {Position in column (mm)};
\gpcolor{rgb color={1.000,0.000,0.000}}
\gpsetlinetype{gp lt plot 0}
\draw[gp path] (1.688,1.725)--(1.689,1.725)--(1.691,1.725)--(1.695,1.725)--(1.701,1.725)%
  --(1.709,1.725)--(1.718,1.725)--(1.729,1.725)--(1.741,1.725)--(1.755,1.725)--(1.771,1.725)%
  --(1.789,1.725)--(1.808,1.726)--(1.829,1.726)--(1.851,1.726)--(1.875,1.726)--(1.901,1.726)%
  --(1.928,1.727)--(1.957,1.727)--(1.988,1.727)--(2.021,1.727)--(2.055,1.728)--(2.091,1.728)%
  --(2.128,1.728)--(2.167,1.729)--(2.208,1.729)--(2.250,1.730)--(2.294,1.730)--(2.340,1.731)%
  --(2.387,1.731)--(2.436,1.732)--(2.487,1.732)--(2.540,1.733)--(2.594,1.733)--(2.649,1.734)%
  --(2.707,1.735)--(2.766,1.735)--(2.826,1.736)--(2.889,1.737)--(2.953,1.738)--(3.019,1.739)%
  --(3.086,1.740)--(3.155,1.741)--(3.226,1.743)--(3.298,1.744)--(3.372,1.746)--(3.448,1.748)%
  --(3.525,1.750)--(3.604,1.752)--(3.685,1.755)--(3.767,1.758)--(3.851,1.762)--(3.936,1.766)%
  --(4.024,1.771)--(4.112,1.777)--(4.203,1.785)--(4.295,1.793)--(4.389,1.803)--(4.484,1.816)%
  --(4.581,1.831)--(4.680,1.850)--(4.780,1.872)--(4.882,1.900)--(4.985,1.934)--(5.089,1.975)%
  --(5.195,2.025)--(5.303,2.087)--(5.412,2.161)--(5.522,2.250)--(5.633,2.355)--(5.745,2.477)%
  --(5.858,2.619)--(5.972,2.779)--(6.086,2.958)--(6.201,3.154)--(6.317,3.367)--(6.433,3.593)%
  --(6.550,3.831)--(6.666,4.080)--(6.783,4.337)--(6.900,4.600)--(7.017,4.869)--(7.134,5.142)%
  --(7.251,5.418)--(7.369,5.697)--(7.486,5.978)--(7.603,6.260)--(7.721,6.543)--(7.838,6.827)%
  --(7.956,7.112)--(8.073,7.398)--(8.190,7.684)--(8.219,7.753);
\gpsetlinetype{gp lt plot 1}
\draw[gp path] (1.688,1.027)--(1.689,1.123)--(1.691,1.220)--(1.695,1.317)--(1.701,1.414)%
  --(1.709,1.511)--(1.718,1.609)--(1.729,1.706)--(1.741,1.803)--(1.755,1.900)--(1.771,1.997)%
  --(1.789,2.094)--(1.808,2.191)--(1.829,2.288)--(1.851,2.385)--(1.875,2.482)--(1.901,2.579)%
  --(1.928,2.677)--(1.957,2.774)--(1.988,2.871)--(2.021,2.968)--(2.055,3.065)--(2.091,3.162)%
  --(2.128,3.259)--(2.167,3.356)--(2.208,3.453)--(2.250,3.550)--(2.294,3.647)--(2.340,3.744)%
  --(2.387,3.841)--(2.436,3.939)--(2.487,4.036)--(2.540,4.133)--(2.594,4.230)--(2.649,4.327)%
  --(2.707,4.424)--(2.766,4.521)--(2.826,4.618)--(2.889,4.715)--(2.953,4.812)--(3.019,4.909)%
  --(3.086,5.006)--(3.155,5.103)--(3.226,5.200)--(3.298,5.297)--(3.372,5.394)--(3.448,5.490)%
  --(3.525,5.587)--(3.604,5.684)--(3.685,5.780)--(3.767,5.877)--(3.851,5.973)--(3.936,6.070)%
  --(4.024,6.165)--(4.112,6.261)--(4.203,6.357)--(4.295,6.452)--(4.389,6.546)--(4.484,6.640)%
  --(4.581,6.733)--(4.680,6.825)--(4.780,6.916)--(4.882,7.005)--(4.985,7.092)--(5.089,7.177)%
  --(5.195,7.258)--(5.303,7.336)--(5.412,7.410)--(5.522,7.478)--(5.633,7.541)--(5.745,7.598)%
  --(5.858,7.648)--(5.972,7.692)--(6.086,7.729)--(6.201,7.760)--(6.317,7.785)--(6.433,7.805)%
  --(6.550,7.821)--(6.666,7.834)--(6.783,7.844)--(6.900,7.851)--(7.017,7.857)--(7.134,7.861)%
  --(7.251,7.864)--(7.369,7.866)--(7.486,7.867)--(7.603,7.868)--(7.721,7.869)--(7.838,7.869)%
  --(7.956,7.869)--(8.073,7.869)--(8.190,7.869)--(8.219,7.869);
\gpcolor{rgb color={0.000,1.000,0.000}}
\gpsetlinetype{gp lt plot 0}
\draw[gp path] (1.688,1.725)--(1.689,1.725)--(1.691,1.725)--(1.695,1.725)--(1.701,1.725)%
  --(1.709,1.725)--(1.718,1.725)--(1.729,1.725)--(1.741,1.725)--(1.755,1.725)--(1.771,1.725)%
  --(1.789,1.725)--(1.808,1.725)--(1.829,1.725)--(1.851,1.726)--(1.875,1.726)--(1.901,1.726)%
  --(1.928,1.726)--(1.957,1.726)--(1.988,1.727)--(2.021,1.727)--(2.055,1.727)--(2.091,1.727)%
  --(2.128,1.727)--(2.167,1.728)--(2.208,1.728)--(2.250,1.728)--(2.294,1.728)--(2.340,1.729)%
  --(2.387,1.729)--(2.437,1.729)--(2.487,1.730)--(2.540,1.730)--(2.594,1.730)--(2.649,1.731)%
  --(2.707,1.731)--(2.766,1.731)--(2.827,1.732)--(2.889,1.732)--(2.953,1.733)--(3.019,1.733)%
  --(3.086,1.733)--(3.155,1.734)--(3.226,1.734)--(3.298,1.735)--(3.372,1.735)--(3.448,1.736)%
  --(3.525,1.736)--(3.604,1.737)--(3.685,1.737)--(3.767,1.738)--(3.851,1.738)--(3.937,1.739)%
  --(4.024,1.740)--(4.113,1.740)--(4.204,1.741)--(4.296,1.742)--(4.390,1.742)--(4.486,1.743)%
  --(4.583,1.745)--(4.682,1.746)--(4.783,1.749)--(4.885,1.752)--(4.989,1.756)--(5.094,1.763)%
  --(5.202,1.775)--(5.310,1.792)--(5.421,1.820)--(5.532,1.864)--(5.646,1.931)--(5.760,2.028)%
  --(5.875,2.160)--(5.991,2.330)--(6.108,2.532)--(6.225,2.759)--(6.342,3.004)--(6.459,3.260)%
  --(6.576,3.524)--(6.694,3.792)--(6.811,4.062)--(6.929,4.333)--(7.046,4.606)--(7.163,4.879)%
  --(7.281,5.152)--(7.398,5.426)--(7.516,5.699)--(7.633,5.973)--(7.751,6.246)--(7.868,6.520)%
  --(7.986,6.794)--(8.103,7.067)--(8.219,7.337);
\gpsetlinetype{gp lt plot 1}
\draw[gp path] (1.688,1.027)--(1.689,1.123)--(1.691,1.220)--(1.695,1.317)--(1.701,1.414)%
  --(1.709,1.511)--(1.718,1.609)--(1.729,1.706)--(1.741,1.803)--(1.755,1.900)--(1.771,1.997)%
  --(1.789,2.094)--(1.808,2.191)--(1.829,2.288)--(1.851,2.385)--(1.875,2.482)--(1.901,2.579)%
  --(1.928,2.677)--(1.957,2.774)--(1.988,2.871)--(2.021,2.968)--(2.055,3.065)--(2.091,3.162)%
  --(2.128,3.259)--(2.167,3.356)--(2.208,3.453)--(2.250,3.550)--(2.294,3.647)--(2.340,3.745)%
  --(2.387,3.842)--(2.437,3.939)--(2.487,4.036)--(2.540,4.133)--(2.594,4.230)--(2.649,4.327)%
  --(2.707,4.424)--(2.766,4.521)--(2.827,4.618)--(2.889,4.715)--(2.953,4.813)--(3.019,4.910)%
  --(3.086,5.007)--(3.155,5.104)--(3.226,5.201)--(3.298,5.298)--(3.372,5.395)--(3.448,5.492)%
  --(3.525,5.589)--(3.604,5.686)--(3.685,5.783)--(3.767,5.881)--(3.851,5.978)--(3.937,6.075)%
  --(4.024,6.172)--(4.113,6.269)--(4.204,6.366)--(4.296,6.463)--(4.390,6.560)--(4.486,6.657)%
  --(4.583,6.754)--(4.682,6.851)--(4.783,6.947)--(4.885,7.044)--(4.989,7.139)--(5.094,7.234)%
  --(5.202,7.328)--(5.310,7.419)--(5.421,7.507)--(5.532,7.590)--(5.646,7.664)--(5.760,7.727)%
  --(5.875,7.777)--(5.991,7.814)--(6.108,7.839)--(6.225,7.855)--(6.342,7.865)--(6.459,7.871)%
  --(6.576,7.874)--(6.694,7.875)--(6.811,7.876)--(6.929,7.876)--(7.046,7.875)--(7.163,7.875)%
  --(7.281,7.875)--(7.398,7.874)--(7.516,7.874)--(7.633,7.873)--(7.751,7.872)--(7.868,7.872)%
  --(7.986,7.871)--(8.103,7.871)--(8.219,7.870);
\gpcolor{rgb color={0.000,0.000,1.000}}
\gpsetlinetype{gp lt plot 0}
\draw[gp path] (1.688,1.725)--(1.689,1.725)--(1.691,1.725)--(1.695,1.725)--(1.701,1.725)%
  --(1.709,1.725)--(1.718,1.725)--(1.729,1.725)--(1.741,1.725)--(1.755,1.725)--(1.771,1.725)%
  --(1.789,1.725)--(1.808,1.725)--(1.829,1.725)--(1.851,1.726)--(1.875,1.726)--(1.901,1.726)%
  --(1.928,1.726)--(1.957,1.726)--(1.988,1.727)--(2.021,1.727)--(2.055,1.727)--(2.091,1.727)%
  --(2.128,1.727)--(2.167,1.728)--(2.208,1.728)--(2.250,1.728)--(2.294,1.728)--(2.340,1.729)%
  --(2.387,1.729)--(2.437,1.729)--(2.487,1.730)--(2.540,1.730)--(2.594,1.730)--(2.649,1.731)%
  --(2.707,1.731)--(2.766,1.731)--(2.827,1.732)--(2.889,1.732)--(2.953,1.733)--(3.019,1.733)%
  --(3.086,1.733)--(3.155,1.734)--(3.226,1.734)--(3.298,1.735)--(3.372,1.735)--(3.448,1.736)%
  --(3.525,1.736)--(3.604,1.737)--(3.685,1.737)--(3.767,1.738)--(3.851,1.738)--(3.937,1.739)%
  --(4.024,1.739)--(4.113,1.740)--(4.204,1.741)--(4.296,1.741)--(4.390,1.742)--(4.486,1.742)%
  --(4.583,1.743)--(4.682,1.744)--(4.783,1.744)--(4.885,1.745)--(4.989,1.746)--(5.095,1.746)%
  --(5.202,1.747)--(5.311,1.749)--(5.421,1.753)--(5.534,1.763)--(5.648,1.788)--(5.763,1.850)%
  --(5.879,1.976)--(5.996,2.170)--(6.113,2.408)--(6.231,2.666)--(6.348,2.930)--(6.466,3.197)%
  --(6.583,3.465)--(6.701,3.732)--(6.818,4.000)--(6.935,4.268)--(7.053,4.536)--(7.170,4.804)%
  --(7.288,5.072)--(7.405,5.340)--(7.523,5.608)--(7.640,5.876)--(7.758,6.144)--(7.875,6.412)%
  --(7.993,6.681)--(8.110,6.949)--(8.219,7.197);
\gpsetlinetype{gp lt plot 1}
\draw[gp path] (1.688,1.027)--(1.689,1.123)--(1.691,1.220)--(1.695,1.317)--(1.701,1.414)%
  --(1.709,1.511)--(1.718,1.609)--(1.729,1.706)--(1.741,1.803)--(1.755,1.900)--(1.771,1.997)%
  --(1.789,2.094)--(1.808,2.191)--(1.829,2.288)--(1.851,2.385)--(1.875,2.482)--(1.901,2.579)%
  --(1.928,2.677)--(1.957,2.774)--(1.988,2.871)--(2.021,2.968)--(2.055,3.065)--(2.091,3.162)%
  --(2.128,3.259)--(2.167,3.356)--(2.208,3.453)--(2.250,3.550)--(2.294,3.647)--(2.340,3.745)%
  --(2.387,3.842)--(2.437,3.939)--(2.487,4.036)--(2.540,4.133)--(2.594,4.230)--(2.649,4.327)%
  --(2.707,4.424)--(2.766,4.521)--(2.827,4.618)--(2.889,4.715)--(2.953,4.813)--(3.019,4.910)%
  --(3.086,5.007)--(3.155,5.104)--(3.226,5.201)--(3.298,5.298)--(3.372,5.395)--(3.448,5.492)%
  --(3.525,5.589)--(3.604,5.686)--(3.685,5.783)--(3.767,5.881)--(3.851,5.978)--(3.937,6.075)%
  --(4.024,6.172)--(4.113,6.269)--(4.204,6.366)--(4.296,6.463)--(4.390,6.560)--(4.486,6.657)%
  --(4.583,6.754)--(4.682,6.852)--(4.783,6.949)--(4.885,7.046)--(4.989,7.143)--(5.095,7.240)%
  --(5.202,7.337)--(5.311,7.433)--(5.421,7.529)--(5.534,7.623)--(5.648,7.712)--(5.763,7.787)%
  --(5.879,7.838)--(5.996,7.865)--(6.113,7.875)--(6.231,7.878)--(6.348,7.879)--(6.466,7.879)%
  --(6.583,7.878)--(6.701,7.878)--(6.818,7.877)--(6.935,7.877)--(7.053,7.876)--(7.170,7.875)%
  --(7.288,7.875)--(7.405,7.874)--(7.523,7.874)--(7.640,7.873)--(7.758,7.873)--(7.875,7.872)%
  --(7.993,7.871)--(8.110,7.871)--(8.219,7.870);
\gpcolor{color=gp lt color border}
\gpsetlinetype{gp lt border}
\draw[gp path] (1.688,8.381)--(1.688,0.985)--(8.219,0.985)--(8.219,8.381)--cycle;
%% coordinates of the plot area
\gpdefrectangularnode{gp plot 1}{\pgfpoint{1.688cm}{0.985cm}}{\pgfpoint{8.219cm}{8.381cm}}
%% \end{tikzpicture}
%% gnuplot variables

\end{tikzpicture}

    }
  }
  \caption[An example of a tanh form acceleration region for three different ``sharpness'' values]{
    An example of a tahn form acceleration region (overall length 6.35mm) for three different ``sharpness'' values.
    The colored electric fields in figure \protect\subref{fig:field_on_axis} correspond to the pulse width (solid) and length (dashed) in figure \protect\subref{fig:tanh_sharpness}.
    This form is compared to FEM simulated electric field data in \ref{fig:field_on_axis_fem}.
  }
  \label{fig:gun-sharpness}
\end{figure}

\ref{fig:gun-sharpness} shows an example acceleration region for three values of the ``sharpness'' $s$.
What is clear from this figure is that the dynamics are not highly coupled to the value of this hard-to-quantify parameter.
For all values of $s$ the divergence of the beam is equal (though the size may vary slightly).
Interestingly the length of the pulse is stabilized on exiting the anode, coming very nearly to a constant length (not expanding nor contracting).
This effect is due to the slower electrons remaining in the cavity longer and thus receiving an additional kick, as described in \cite{oudheusden_electron_2007}.
This longitudinal electric field shape is compared to the electric field computed by finite-element (FEM) simulations of our acceleration cavity in Section \ref{sec:gun_design}.


