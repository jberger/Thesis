\subsection{Acceleration Region}

Obviously an overall acceleration occurs due to the applied electic field in the acceleration gap.
The anode must contain an aperture to allow the electrons to continue down the column.
The electric field that nears this aperture turns outward as expected, causing the anode itself to act as a negative lens. %TODO reference
What may not be so obvious is that the falling field longitudinally acts as a pulse compression element.

To model an acceleration region, one must assume a form for the accelerating field.
From that form one may derive the terms governing the pulse dynamics. Initially, assume that the on-axis field is purely in the longitudinal direction, which may be thought of as an implication of radial symmetry.
Mathematically, that is
\begin{equation}
  \vec{F}(r^{\prime}=0,z^{\prime}) = F_z (0,z^{\prime}) \hat{z}
\end{equation}
where as before, the primed coordinates are the location of the center of the pulse.

To first model the longitudinal (compression) action at the anode %TODO ref
on electrons that are a distance $z$ ahead or behind of the center of the pulse feel a slightly different field, 
\begin{equation}
  F_z(0,z^{\prime} - z) \approx F_z(0,z^{\prime} - z) + \frac{\partial F_z}{\partial z^{\prime}} z
\end{equation}
The leading term does not contribute to the intrapulse dynamic (see Section \ref{sec:const_force}), however the second term is of the proper form to say that
\begin{equation}
  M_z = \frac{\partial F_z}{\partial z^{\prime}}
\end{equation}

To model the radial action (negative lensing), we can expand for electrons that are a distance $r$ radially outward from the center of the pulse,
\begin{equation}
  F_r(r,z^{\prime}) \approx F_r(r,z^{\prime}) + \frac{\partial F_r}{\partial r^{\prime}} r
\end{equation}
without an analytic form for $F$, we can appeal to Gauss' law (in absense of external charge)
\begin{equation}
  \frac{\partial F_x}{\partial x^{\prime}} + \frac{\partial F_y}{\partial y^{\prime}} + \frac{\partial F_z}{\partial z^{\prime}} = 2 \frac{\partial F_{\smallT}}{\partial r^{\prime}} + \frac{\partial F_z}{\partial z^{\prime}} = 0
\end{equation}
to see that
\begin{equation}
  M_{\smallT} = -\frac{1}{2}\frac{\partial F_z}{\partial z^{\prime}}
\end{equation}

We may choose to assume that the on-axis field is defined as a smoothed step down function
\begin{equation}
  F_z(0,z^{\prime}) = \frac{qV}{2z^{\prime}_A} \left( 1 - \tanh \left( \frac{ z^{\prime} - z^{\prime}_A }{ z^{\prime}_A / s } \right) \right)
\end{equation}
where $V$ is the accelerating potential, $z^{\prime}_A$ is the position of the anode (the cathode assumed to be at $z^{\prime} = 0$) and $s$ is a constant quantifying the ``sharpness'' of the fall-off of the electric field.
Indeed, this form is analgous to the super-Gaussian envalope (\ref{eq:reg_of_influence}), however it is open ended on the cathode side, which is necessary for the electrons to be accelerated away from the photocathode.
Given this form (justified in Section \ref{sec:gun_design}) the lensing constants are then
\begin{gather}
  M_{\smallT} = -\frac{qVs}{4z^{\prime 2}_A} \operatorname{sech}^2 \left( \frac{z^{\prime} - z^{\prime}_A }{ z^{\prime}_A / s } \right) \\
  M_z = \frac{qVs}{2z^{\prime 2}_A} \operatorname{sech}^2 \left( \frac{z^{\prime} - z^{\prime}_A }{ z^{\prime}_A / s } \right)
\end{gather}
