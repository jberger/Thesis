%This work is licensed under the Creative Commons Attribution-NonCommercial-NoDerivs 3.0 United States License. To view a copy of this license, visit http://creativecommons.org/licenses/by-nc-nd/3.0/us/ or send a letter to Creative Commons, 444 Castro Street, Suite 900, Mountain View, California, 94041, USA.

In Chapters \ref{chap:previous_models} through \ref{chap:model_results}, I have explored the mathematical side of designing an ultrafast electron microscope; modeling the behavior of a pulse and the effect that column elements have upon it.
However, microscopy is an experimental science, and designing a UEM is no less hands-on.

When designing a UEM system, special care must be given to the creation and propagation of the electron beam.
In the case of single-shot UEM, several factors lead to two major considerations: first, that a large area photoemission source is preferred to smaller area or even single point emitters, and second, the transverse emittance of the beam must be minimized.

These considerations drive the remainder of this thesis.
In this chapter, I will first convince the reader of these considerations.
As a direct consequence of these considerations, it is our belief that a successful UEM project cannot simply stem from small modifications to the platform of a conventional electron microscope.
In the following chapters I will show how we are addressing these considerations for the prototype system at UIC.

%Ultrafast Electron Microscopy (UEM) is a subclass of a more general concept, Dynamic Transmission Electron Microscopy (DTEM).
%DTEM is a more general term for any electron microscope with higher time resolution, typically employing laser-stimulated photoemission.


\section{Pulse Charge Requirements}

It is easy to believe that in order to make a useful image in an electron microcope, the number of electrons collected must be above some minimum number.
The Rose criterion \cite{rose_television_1948} indicates that an average detector signal strength of $\sim$100 electrons/pixel is required for adequate gray scaling of the image.
For a 1k x 1k CCD camera, this implies an incident electron pulse with $N=10^8$ \cite{armstrong_prospects_2007}.

In a conventional electron microscope the electrons are generated by one of several processes, often by heating the cathode or applying a large field to the cathode.
Although these processes produce a very large number of electrons, far more than required by the Rose criterion, they are essentially non-directional.
A pinhole is used to eliminate most of the electrons with the exception of those which will travel down the column, reducing the beam current but producing a high-quality beam. %TODO ref(s)

In fact, for conventional electron microscopes without time resolution, the Rose criterion may be met for even very low beam current by simply allowing the image to acquire for a long enough time to collect sufficent electrons.
Further, high beam current may be damaging to the sample over time, so often a lower beam current is preferred.

By definition, for single-shot imaging with a DTEM, all of these electrons must be contained in a single pulse.
Although sample damage concerns may be neglected, destructive measurements will result in useful images, the use of a pinhole is likely to reduce the pulse charge below the useful level.
This means that the designer of a DTEM must endevor to keep all of the created electrons.

It should be noted that for diffraction studies with a DTEM, many fewer electrons per pulse are required for single-shot experiments; typically, about $10^6$ electrons/pulse \cite{armstrong_practical_2007} as significantly less signal is required to determine the position of discrete single-crystal Bragg diffraction spots or diffraction rings for polycrystalline specimens.

