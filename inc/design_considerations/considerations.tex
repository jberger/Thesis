%This work is licensed under the Creative Commons Attribution-NonCommercial-NoDerivs 3.0 United States License. To view a copy of this license, visit http://creativecommons.org/licenses/by-nc-nd/3.0/us/ or send a letter to Creative Commons, 444 Castro Street, Suite 900, Mountain View, California, 94041, USA.

In Chapters \ref{chap:ms_model} through \ref{chap:model_results}, I have explored the mathematical side of designing an ultrafast electron microscope; modeling the behavior of a pulse and the effects that column elements have upon it.
However, since microscopy is an experimental science, we are also designing and building a prototype UEM system at UIC.
In doing so we have decided that special care must be given to the creation and propagation of the electron beam.
In the case of single-shot UEM, in which the image of the sample is acquired using only one electron pulse, several factors lead to two major considerations: first, that a large area photoemission source is required to smaller area or even single point emitters, and second, that the ``normalized transverse emittance,'' a beam quality figure of merit which will be defined later, must be minimized.

These considerations drive the remainder of this thesis.
In this chapter, I will first convince the reader of these considerations.
As a direct consequence thereof, it is our belief that a successful UEM project cannot simply stem from small modifications to the platform of a conventional electron microscope.
In the following chapters, I will show how we are addressing these considerations for the prototype system at UIC, both in column design (Chapter \ref{chap:prototype}) and photocathode engineering (Chapter \ref{chap:photocathode}).

\section{Pulse Charge Requirement}

It is straightforward to show that in order to make a useful image in an electron microcope, the number of electrons collected must be above some minimum number.
The Rose criterion~\cite{rose_television_1948} indicates that an average detector signal strength of $\sim$100 electrons/pixel is required for adequate gray scaling of the image.
For a 1k $\times$ 1k CCD camera, this implies an incident electron pulse with $N=10^8$~\cite{armstrong_prospects_2007}.

In a conventional electron microscope the electrons are generated by one of several processes, often by heating the cathode or applying a large field to the cathode, or combinations thereof.
Although these processes produce a very large number of electrons, far more than required by the Rose criterion, they are essentially not highly directional.
A pinhole is used to eliminate most of the electrons, with the exception of those which will travel down the column axis, thereby reducing the beam current but producing a high-quality beam.
%TODO ref(s)
For conventional electron microscopes without time resolution, the Rose criterion may be met for arbitrarily low beam current by simply allowing the image to acquire for a long enough time to collect sufficent electrons.
Further, high beam current may be damaging to the sample over time, so a lower beam current is often preferred.

By definition, for single-shot imaging with a DTEM, all of the electrons used to form an image must be contained in a single pulse.
In a single-shot DTEM, the dynamics being studied are often destructive, regardless of the knock-on damage cause by the probe pulse; what matters is that the system be able to probe the sample before it is destroyed.
Nevertheless, although sample damage concerns may be neglected, the use of a pinhole is likely to reduce the pulse charge below the useful level.
This means that the designer of a DTEM must endeavor to retain and utilize all of the generated electrons.

It should be noted that for diffraction studies performed in a DTEM, many fewer electrons per pulse are required for single-shot experiments; typically, about $10^6$ electrons/pulse~\cite{armstrong_practical_2007} as significantly less signal is required to determine the position of discrete single-crystal Bragg diffraction spots or diffraction rings for polycrystalline specimens.

\section{Generation Current Limitations} \label{sec:childs_law}

Since we are now aware that there is a large minimum number of electrons needed per pulse, it is only natural to ask if there is the practical limit on how much charge (i.e., the number of electrons/pulse, $N$) can be extracted from a pulsed laser-driven DC photoelectron gun.
When the incident laser pulse duration $\tau$ is longer than the time-of-flight (TOF) $t_{gun}$ of a single electron across the acceleration gap $d$ of the DC gun with a potential $V_{DC}$, $\tau > t_{gun} = d \sqrt{2m/eV_{DC}} \approx 0.1-1$ns, typically, then the maximum photocurrent density $J_{CL}$ is given by the well-known Child-Langmuir law~\cite{child_discharge_1911,langmuir_effect_1923}:
\begin{equation}
  J_{CL} = \frac{4 \varepsilon_0}{9} \sqrt{\frac{2e}{m}} \frac{V_{DC}^{3/2}}{d^2} \,\text{,}
\end{equation}
where $e$ is the electron charge and $\varepsilon_0$ is the vacuum permittivity.

This steady-state law for the maximum current density that can be achieved is not valid in the regime of interest for UEM (and UED) where the use of picosecond and femtosecond photocathode drive laser pulses implies that $t_{gun}$ is usually much greater than the electron pulse duration $\tau_e$.
Though there is no clear delineation in the literature or the community between moniker of Ultrafast Electron Microscopy (UEM) and the more general term Dynamic Transmission Electron Microcopy (DTEM), we propose that this is useful delimination, reserving the term UEM for those systems where the pulse can be completely contained inside the accelerator, i.e. $t_{gun} > \tau_e$.
Pulses of this type are subject to a different ``short pulse analog'' of the Child-Langmuir limit.

In this short pulse case, an equivalent diode approximation~\cite{valfells_effects_2002} allows for the definition of a critical current density associated with virtual cathode formation in terms of the steady-state Child-Langmuir limit:
\begin{equation}
  J_{crit} = \frac{27}{4} \left ( \frac{ 1 - \sqrt{1-\tfrac{1}{3} \chi^2} }{ \chi^3 } \right ) J_{CL} \,\text{,}
\end{equation}
where $\chi = \tau_e / t_{gun}$.
When $\chi \ll 1$, the usual case in UEM, the critical current density is simply given by
\begin{equation}
  J_{crit} = \frac{\varepsilon_0 E_{DC}}{\tau_e} \,\text{,}
\end{equation}
where the DC gun field $E_{DC} = V_{DC}/d$.
This relation may be rewritten to yield a critical number of photoemitted electrons/pulse before virtual cathode formation:
\begin{equation} \label{eq:n_crit}
  N_{crit} = \frac{\pi \varepsilon_0 w^2 V_{DC}}{e d} \,\text{.}
\end{equation}
In obtaining the above equation, we have used an irradiated photocathode area of $\pi w^2$ associated with an incident Gaussian laser pulse of the form of \ref{eq:laser_form}.
The relation clearly indicates that $N_{crit}$ is only dependent upon the initial DC surface field (i.e., $E_{DC}$) at the photocathode and the illuminated area --- as expected, because a virtual cathode forms when the initial photocathode surface charge density associated with $E_{DC}$ is photoemitted, effectively screening the cathode from the acceleration field.

For a typical DC gun with $E_{DC} \approx $ 30 kV/cm (e.g., $V_{DC} = $ 100 kV and $d \approx $ 3 cm), \ref{eq:n_crit} reveals that the generation of $10^8$ electrons/pulse needed for single-shot imaging applications in a DTEM with a 1k $\times$ 1k CCD detector requires an incident laser beam area $\pi w^2$ of about 1 mm$^2$ --- the virtual cathode being formed for the extraction of more than $1.7 \times 10^8$ electrons/mm$^2$ in this short pulse regime.
We note that \ref{eq:n_crit} does not express a limit on the number of electrons/pulse $N$ that can be emitted from a pulsed laser-driven DC photoelectron gun, it merely indicates a critical number before virtual cathode formation occurs.
More electrons can be produced, at the expense of a severely distorted temporal electron pulse profile~\cite{valfells_effects_2002}.

This finally leads to the first consideration for UEM design mentioned in the introduction: for the clean pulsed emission of sufficient to create a useful image, the generated beam should have a HW1/eM width 
\begin{equation} \label{eq:minimum_w}
  w \gtrsim \sqrt{\frac{N e d}{\pi \varepsilon_0 V_{DC}}} \,\text{.}
\end{equation}
For our example case above, to cleanly extract $N \ge 10^8$ electrons/pulse we require $w \gtrsim 0.56 \text{ mm}$.
The effect that accomodating a large transverse pulse size has on designing an Ultrafast Electron Microscope will be discussed in Chapter \ref{chap:prototype}. 

\section{Beam Quality and Resolution Limit} \label{sec:must_reduce_transverse_momentum}

A common figure of merit for electron beams is its normalized transverse emittance~\cite{jensen_emittance_2010}, $\varepsilon_{\smallT}$, defined as 
\begin{equation}
  \varepsilon_{\smallT} = \frac{1}{m c} \sqrt{ \left < x^2 \right > \left < p_{\smallT}^2 \right > - \left < x p_{\smallT} \right >^2 } \,\text{,}
\end{equation}
where $m$ is the electron mass, $c$ is the speed of light in vacuum and $x$ and $p_{\smallT}$ are the transverse position and momentum respectively.
For the usual case of axially symmetric electron pulses, we may use the simplified form
\begin{equation} \label{eq:emittance_definition}
  \varepsilon_{\smallT} = \frac{1}{m c} \sqrt{ \left < x^2 \right > \left < p_{\smallT}^2 \right > } \,\text{.}
\end{equation}
By Liouville's Theorem (Section \ref{sec:liouville}), we know that this quantity is conserved throughout pulse propagation (for free-space and perfect electron-optical elements), and therefore is related directly to the initial photoemission properties of the system.
For a laser pulse of the form of \ref{eq:laser_form} we have simply
\begin{equation} \label{eq:emittance_definition_w}
  \varepsilon_{\smallT} = \frac{w}{m c} \sqrt{ \frac{\left < p_{\smallT}^2 \right >}{2} } \,\text{.}
\end{equation}
Note that imperfect electron-optical elements may cause ``emittance growth''~\cite{oshea_reversible_1998}, increasing the pulse's emittance after these elements.

This normalized transverse emittance limits the resolving capablility of the beam, so this quantity can be thought of as a measure of the beam quality, where a lower number is preferred~\cite{berger_dc_2009}.
Since \ref{eq:minimum_w} specifies a lower bound on the emission spot size $w$, knowing a UEM will exhibit higher resolution by employing an electron source with the minimum possible transverse emittance directly implies that every effort must be made to reduce the initial transverse momentum spread.
Though deriving an absolute limit on the allowable transverse emittance for imaging in a UEM would depend on many column-specific factors, since diffraction is expected to impose less stringent requirements on the beam than imaging, we may consider it when searching for an upper bound on transverse emittance.

In Ref.~\cite{berger_dc_2009}, we explore the impact of these limitations for an ultrafast electron diffraction (UED) experiment.
We find that for a desired spatial resolution per pixel $\Delta X$ (related to the magnification) using a $N_p$ x $N_p$ detector, one may only resolve a fractional change $\Delta a / a$ in a crystalline material's Bragg plane spacing $a$ in the $M^{th}$ diffraction order if
\begin{equation}
  \varepsilon_{\smallT} \le \frac{5 \hbar}{m c} N_p \, \Delta X \frac{M}{a} \left ( \frac{\Delta a}{a} \right ) \,\text{.}
\end{equation}
However, because of lower limit that exists on $w$ from \ref{eq:minimum_w}, we can say that these requirements impose an upper limit on the initial transverse momentum spread
\begin{equation}
  \sqrt{ \left < p_{\smallT}^2 \right > } = \sqrt{ \eta_{\smallT} (0) } \le \frac{5\sqrt{2} \hbar}{w} N_p \, \Delta X \frac{M}{a} \left ( \frac{\Delta a}{a} \right ) \,\text{.}
\end{equation}

To have $\Delta X = $1 nm spatial resolution, and be able to measure a 10\% change ($\Delta a / a = 0.1$) in a Bragg spacing $a \sim$ 1-2 \AA{} in the 1$^{st}$ diffraction order, the example DTEM from Section \ref{sec:childs_law}, equipped with a 1k x 1k CCD will require $\varepsilon_{\smallT} \lesssim$ 2 nm.
For comparison, using the simplistic initial conditions in \ref{eq:summary}, exciting carriers from a HW1/eM $w = 0.56 \text{ mm}$ Gaussian spot-size on a Tantalum metal photocathode (work function $\Phi =$ 4.25 eV) using our laser (photon energy $\hbar \omega =$ 4.75 eV, further described in Section \ref{sec:laser}), for an excess photoemission energy $\Delta E = \hbar \omega - \Phi =$ 0.5 eV, the normalized transverse emittance is about 200 nm.

One might think to narrow this discrepancy by simply better matching the photocathode work function and the laser photon energy, thereby reducing the excess photoemission energy $\Delta E$.
Unfortunately, since the photoemission efficiency is proportional to $\Delta E^2$~\cite{shalaev_electron_1994}, though this would indeed reduce the transverse emittance, it would do so at the cost of a much reduced pulse charge.
In Chapter \ref{chap:photocathode}, I will explore several more exotic photoemission processes in an attempt to reduce $\varepsilon_{\smallT}$ without this penalty.
