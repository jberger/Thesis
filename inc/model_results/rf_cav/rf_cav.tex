%This work is licensed under the Creative Commons Attribution-NonCommercial-NoDerivs 3.0 United States License. To view a copy of this license, visit http://creativecommons.org/licenses/by-nc-nd/3.0/us/ or send a letter to Creative Commons, 444 Castro Street, Suite 900, Mountain View, California, 94041, USA.

\section{Compensation of Temporal Space-charge Effects: RF Cavities} \label{sec:rf_cav_charge}

In comparison to using magnetic lenses to compensate for transverse pulse broadening by using magnetic lenses, RF cavities are being employed in UED systems \cite{oudheusden_electron_2007}, and are expected to be employed in UEM systems as well, to compensate for longitudinal broadening in the ultrashort pulses.
\ref{fig:rf_cav_num} illustrates the operation of a $ \Omega / 2 \pi = $ 3GHz $\text{TM}_{010}$ RF cavity of axial length $ d = \pi v_{\smallzero} / \Omega $ as an electron pulse compressor tuned to produce the shortest pulse at a distance $ z^{\prime} = $ 10cm behind the cavity for $N=1$ and $N=10^5$ pulse charges.
The mathematical model of the RF cavity (see Section \ref{rf_cav_model}) employs $L_{RF} = d \approx $ 14mm, $ z^{\prime}_{RF} = 0 $, and a region of influence super-Gaussian of order 2 ($n=2$ in \ref{eq:reg_of_influence}) to reflect the sharper ends of the RF cavity.
In both cases, the idealized initial conditions are used, starting the pulse at $ z^{\prime} = -10$ cm, an excess photoemission energy $\Delta E = 0.5$ eV, velocity related to a $V_{DC} = $ 20kV acceleration potential, HW1/eM $ w = $ 0.1mm and duration $\tau = $ 100fs ($v_{\smallzero} \tau \approx 8.4 \mu$m).
This electron pulse is therefore initially in the preferred highly oblate regime ($ \sigma_{\smallT}(0) \gg \sigma_{z} (0) $), for which space-charge effects predominantly act in the longitudinal direction, broadening the pulse in time (see \ref{fig:compare_shape}).

\begin{figure}
  \centerline {
    \subfloat[][]{
      \label{fig:rf_cav_N1}
      \begin{tikzpicture}
        %% \begin{tikzpicture}[gnuplot]
%% generated with GNUPLOT 4.6p0 (Lua 5.1; terminal rev. 99, script rev. 100)
%% Mon 01 Apr 2013 01:33:40 PM CDT
\path (0.000,0.000) rectangle (8.750,6.125);
\gpcolor{color=gp lt color border}
\gpsetlinetype{gp lt border}
\gpsetlinewidth{1.00}
\draw[gp path] (1.504,0.985)--(1.684,0.985);
\node[gp node right] at (1.320,0.985) { 0};
\draw[gp path] (1.504,1.667)--(1.684,1.667);
\node[gp node right] at (1.320,1.667) { 0.2};
\draw[gp path] (1.504,2.348)--(1.684,2.348);
\node[gp node right] at (1.320,2.348) { 0.4};
\draw[gp path] (1.504,3.030)--(1.684,3.030);
\node[gp node right] at (1.320,3.030) { 0.6};
\draw[gp path] (1.504,3.711)--(1.684,3.711);
\node[gp node right] at (1.320,3.711) { 0.8};
\draw[gp path] (1.504,4.393)--(1.684,4.393);
\node[gp node right] at (1.320,4.393) { 1};
\draw[gp path] (1.504,5.074)--(1.684,5.074);
\node[gp node right] at (1.320,5.074) { 1.2};
\draw[gp path] (1.504,5.756)--(1.684,5.756);
\node[gp node right] at (1.320,5.756) { 1.4};
\draw[gp path] (1.504,0.985)--(1.504,1.165);
\draw[gp path] (1.504,5.756)--(1.504,5.576);
\node[gp node center] at (1.504,0.677) {-10};
\draw[gp path] (2.560,0.985)--(2.560,1.165);
\draw[gp path] (2.560,5.756)--(2.560,5.576);
\node[gp node center] at (2.560,0.677) {-5};
\draw[gp path] (3.616,0.985)--(3.616,1.165);
\draw[gp path] (3.616,5.756)--(3.616,5.576);
\node[gp node center] at (3.616,0.677) { 0};
\draw[gp path] (4.673,0.985)--(4.673,1.165);
\draw[gp path] (4.673,5.756)--(4.673,5.576);
\node[gp node center] at (4.673,0.677) { 5};
\draw[gp path] (5.729,0.985)--(5.729,1.165);
\draw[gp path] (5.729,5.756)--(5.729,5.576);
\node[gp node center] at (5.729,0.677) { 10};
\draw[gp path] (6.785,0.985)--(6.785,1.165);
\draw[gp path] (6.785,5.756)--(6.785,5.576);
\node[gp node center] at (6.785,0.677) { 15};
\draw[gp path] (6.785,0.985)--(6.605,0.985);
\node[gp node left] at (6.969,0.985) { 0};
\draw[gp path] (6.785,1.581)--(6.605,1.581);
\node[gp node left] at (6.969,1.581) { 20};
\draw[gp path] (6.785,2.178)--(6.605,2.178);
\node[gp node left] at (6.969,2.178) { 40};
\draw[gp path] (6.785,2.774)--(6.605,2.774);
\node[gp node left] at (6.969,2.774) { 60};
\draw[gp path] (6.785,3.371)--(6.605,3.371);
\node[gp node left] at (6.969,3.371) { 80};
\draw[gp path] (6.785,3.967)--(6.605,3.967);
\node[gp node left] at (6.969,3.967) { 100};
\draw[gp path] (6.785,4.563)--(6.605,4.563);
\node[gp node left] at (6.969,4.563) { 120};
\draw[gp path] (6.785,5.160)--(6.605,5.160);
\node[gp node left] at (6.969,5.160) { 140};
\draw[gp path] (6.785,5.756)--(6.605,5.756);
\node[gp node left] at (6.969,5.756) { 160};
\draw[gp path] (1.504,5.756)--(1.504,0.985)--(6.785,0.985)--(6.785,5.756)--cycle;
\node[gp node center,rotate=-270] at (0.246,3.370) {HW1/eM Beam Width (mm)};
\node[gp node center,rotate=-270] at (8.042,3.370) {HW1/eM Beam Length ($\mu$m)};
\node[gp node center] at (4.144,0.215) {Pulse Location rel. to RF Cav. (cm)};
\gpcolor{color=gp lt color 0}
\gpsetlinetype{gp lt plot 0}
\draw[gp path] (1.504,1.326)--(1.533,1.326)--(1.562,1.327)--(1.591,1.328)--(1.620,1.330)%
  --(1.649,1.332)--(1.678,1.335)--(1.707,1.339)--(1.736,1.343)--(1.765,1.347)--(1.794,1.352)%
  --(1.824,1.357)--(1.853,1.363)--(1.882,1.369)--(1.911,1.375)--(1.940,1.382)--(1.969,1.389)%
  --(1.998,1.396)--(2.027,1.404)--(2.056,1.412)--(2.085,1.420)--(2.114,1.429)--(2.143,1.438)%
  --(2.172,1.447)--(2.201,1.456)--(2.230,1.465)--(2.259,1.475)--(2.288,1.485)--(2.317,1.495)%
  --(2.346,1.505)--(2.375,1.515)--(2.404,1.526)--(2.433,1.536)--(2.463,1.547)--(2.492,1.558)%
  --(2.521,1.569)--(2.550,1.580)--(2.579,1.591)--(2.608,1.602)--(2.637,1.613)--(2.666,1.625)%
  --(2.695,1.636)--(2.724,1.648)--(2.753,1.659)--(2.782,1.671)--(2.811,1.683)--(2.840,1.695)%
  --(2.869,1.707)--(2.898,1.719)--(2.927,1.731)--(2.956,1.743)--(2.985,1.755)--(3.014,1.767)%
  --(3.043,1.779)--(3.072,1.791)--(3.102,1.804)--(3.131,1.816)--(3.160,1.828)--(3.189,1.841)%
  --(3.218,1.853)--(3.247,1.866)--(3.276,1.878)--(3.305,1.891)--(3.334,1.903)--(3.363,1.916)%
  --(3.392,1.929)--(3.421,1.941)--(3.450,1.955)--(3.479,1.970)--(3.508,1.987)--(3.537,2.006)%
  --(3.566,2.026)--(3.595,2.046)--(3.624,2.067)--(3.653,2.088)--(3.682,2.110)--(3.711,2.132)%
  --(3.741,2.156)--(3.770,2.183)--(3.799,2.212)--(3.828,2.242)--(3.857,2.272)--(3.886,2.303)%
  --(3.915,2.334)--(3.944,2.364)--(3.973,2.395)--(4.002,2.426)--(4.031,2.456)--(4.060,2.487)%
  --(4.089,2.518)--(4.118,2.548)--(4.147,2.579)--(4.176,2.610)--(4.205,2.640)--(4.234,2.671)%
  --(4.263,2.702)--(4.292,2.732)--(4.321,2.763)--(4.350,2.794)--(4.380,2.824)--(4.409,2.855)%
  --(4.438,2.886)--(4.467,2.917)--(4.496,2.947)--(4.525,2.978)--(4.554,3.009)--(4.583,3.040)%
  --(4.612,3.070)--(4.641,3.101)--(4.670,3.132)--(4.699,3.163)--(4.728,3.193)--(4.757,3.224)%
  --(4.786,3.255)--(4.815,3.286)--(4.844,3.316)--(4.873,3.347)--(4.902,3.378)--(4.931,3.409)%
  --(4.960,3.439)--(4.989,3.470)--(5.019,3.501)--(5.048,3.532)--(5.077,3.562)--(5.106,3.593)%
  --(5.135,3.624)--(5.164,3.655)--(5.193,3.685)--(5.222,3.716)--(5.251,3.747)--(5.280,3.778)%
  --(5.309,3.809)--(5.338,3.839)--(5.367,3.870)--(5.396,3.901)--(5.425,3.932)--(5.454,3.962)%
  --(5.483,3.993)--(5.512,4.024)--(5.541,4.055)--(5.570,4.086)--(5.599,4.116)--(5.628,4.147)%
  --(5.658,4.178)--(5.687,4.209)--(5.716,4.239)--(5.745,4.270)--(5.774,4.301)--(5.803,4.332)%
  --(5.832,4.363)--(5.861,4.393)--(5.890,4.424)--(5.919,4.455)--(5.948,4.486)--(5.977,4.516)%
  --(6.006,4.547)--(6.035,4.578)--(6.064,4.609)--(6.093,4.640)--(6.122,4.670)--(6.151,4.701)%
  --(6.180,4.732)--(6.209,4.763)--(6.238,4.794)--(6.267,4.824)--(6.297,4.855)--(6.326,4.886)%
  --(6.355,4.917)--(6.384,4.948)--(6.413,4.978)--(6.442,5.009)--(6.471,5.040)--(6.500,5.071)%
  --(6.529,5.102)--(6.558,5.132)--(6.587,5.163)--(6.616,5.194)--(6.645,5.225)--(6.674,5.256)%
  --(6.703,5.286)--(6.732,5.317)--(6.761,5.348)--(6.785,5.373);
\gpcolor{rgb color={0.000,1.000,0.000}}
\draw[gp path] (1.504,1.235)--(1.533,1.242)--(1.562,1.262)--(1.591,1.292)--(1.620,1.329)%
  --(1.649,1.373)--(1.678,1.419)--(1.707,1.469)--(1.736,1.521)--(1.765,1.574)--(1.794,1.628)%
  --(1.824,1.683)--(1.853,1.738)--(1.882,1.794)--(1.911,1.851)--(1.940,1.908)--(1.969,1.965)%
  --(1.998,2.022)--(2.027,2.080)--(2.056,2.138)--(2.085,2.196)--(2.114,2.254)--(2.143,2.312)%
  --(2.172,2.370)--(2.201,2.428)--(2.230,2.486)--(2.259,2.545)--(2.288,2.603)--(2.317,2.662)%
  --(2.346,2.721)--(2.375,2.779)--(2.404,2.838)--(2.433,2.897)--(2.463,2.955)--(2.492,3.014)%
  --(2.521,3.073)--(2.550,3.132)--(2.579,3.190)--(2.608,3.249)--(2.637,3.308)--(2.666,3.367)%
  --(2.695,3.426)--(2.724,3.485)--(2.753,3.544)--(2.782,3.603)--(2.811,3.662)--(2.840,3.721)%
  --(2.869,3.780)--(2.898,3.839)--(2.927,3.898)--(2.956,3.957)--(2.985,4.016)--(3.014,4.075)%
  --(3.043,4.134)--(3.072,4.193)--(3.102,4.252)--(3.131,4.311)--(3.160,4.370)--(3.189,4.429)%
  --(3.218,4.488)--(3.247,4.547)--(3.276,4.606)--(3.305,4.665)--(3.334,4.724)--(3.363,4.783)%
  --(3.392,4.843)--(3.421,4.902)--(3.450,4.961)--(3.479,5.022)--(3.508,5.081)--(3.537,5.133)%
  --(3.566,5.174)--(3.595,5.198)--(3.624,5.203)--(3.653,5.187)--(3.682,5.154)--(3.711,5.106)%
  --(3.741,5.049)--(3.770,4.989)--(3.799,4.929)--(3.828,4.870)--(3.857,4.811)--(3.886,4.752)%
  --(3.915,4.692)--(3.944,4.633)--(3.973,4.574)--(4.002,4.515)--(4.031,4.456)--(4.060,4.397)%
  --(4.089,4.338)--(4.118,4.279)--(4.147,4.220)--(4.176,4.161)--(4.205,4.102)--(4.234,4.043)%
  --(4.263,3.984)--(4.292,3.925)--(4.321,3.866)--(4.350,3.807)--(4.380,3.748)--(4.409,3.689)%
  --(4.438,3.630)--(4.467,3.572)--(4.496,3.513)--(4.525,3.454)--(4.554,3.395)--(4.583,3.336)%
  --(4.612,3.277)--(4.641,3.218)--(4.670,3.160)--(4.699,3.101)--(4.728,3.042)--(4.757,2.983)%
  --(4.786,2.925)--(4.815,2.866)--(4.844,2.807)--(4.873,2.749)--(4.902,2.690)--(4.931,2.632)%
  --(4.960,2.573)--(4.989,2.515)--(5.019,2.456)--(5.048,2.398)--(5.077,2.340)--(5.106,2.282)%
  --(5.135,2.224)--(5.164,2.166)--(5.193,2.108)--(5.222,2.050)--(5.251,1.993)--(5.280,1.936)%
  --(5.309,1.879)--(5.338,1.822)--(5.367,1.766)--(5.396,1.710)--(5.425,1.655)--(5.454,1.600)%
  --(5.483,1.547)--(5.512,1.494)--(5.541,1.444)--(5.570,1.395)--(5.599,1.350)--(5.628,1.310)%
  --(5.658,1.276)--(5.687,1.250)--(5.716,1.237)--(5.745,1.237)--(5.774,1.251)--(5.803,1.276)%
  --(5.832,1.310)--(5.861,1.351)--(5.890,1.396)--(5.919,1.444)--(5.948,1.495)--(5.977,1.547)%
  --(6.006,1.601)--(6.035,1.655)--(6.064,1.710)--(6.093,1.766)--(6.122,1.823)--(6.151,1.879)%
  --(6.180,1.936)--(6.209,1.993)--(6.238,2.051)--(6.267,2.109)--(6.297,2.166)--(6.326,2.224)%
  --(6.355,2.282)--(6.384,2.340)--(6.413,2.399)--(6.442,2.457)--(6.471,2.515)--(6.500,2.574)%
  --(6.529,2.632)--(6.558,2.691)--(6.587,2.749)--(6.616,2.808)--(6.645,2.866)--(6.674,2.925)%
  --(6.703,2.984)--(6.732,3.043)--(6.761,3.101)--(6.785,3.149);
\gpcolor{color=gp lt color border}
\gpsetlinetype{gp lt border}
\draw[gp path] (1.504,5.756)--(1.504,0.985)--(6.785,0.985)--(6.785,5.756)--cycle;
%% coordinates of the plot area
\gpdefrectangularnode{gp plot 1}{\pgfpoint{1.504cm}{0.985cm}}{\pgfpoint{6.785cm}{5.756cm}}
%% \end{tikzpicture}
%% gnuplot variables

      \end{tikzpicture}
    }
    \subfloat[][]{
      \label{fig:rf_cav_N1e5}
      \begin{tikzpicture}
        %% \begin{tikzpicture}[gnuplot]
%% generated with GNUPLOT 4.6p0 (Lua 5.1; terminal rev. 99, script rev. 100)
%% Mon 01 Apr 2013 01:33:40 PM CDT
\path (0.000,0.000) rectangle (8.750,6.125);
\gpcolor{color=gp lt color border}
\gpsetlinetype{gp lt border}
\gpsetlinewidth{1.00}
\draw[gp path] (1.504,0.985)--(1.684,0.985);
\node[gp node right] at (1.320,0.985) { 0};
\draw[gp path] (1.504,1.780)--(1.684,1.780);
\node[gp node right] at (1.320,1.780) { 0.5};
\draw[gp path] (1.504,2.575)--(1.684,2.575);
\node[gp node right] at (1.320,2.575) { 1};
\draw[gp path] (1.504,3.371)--(1.684,3.371);
\node[gp node right] at (1.320,3.371) { 1.5};
\draw[gp path] (1.504,4.166)--(1.684,4.166);
\node[gp node right] at (1.320,4.166) { 2};
\draw[gp path] (1.504,4.961)--(1.684,4.961);
\node[gp node right] at (1.320,4.961) { 2.5};
\draw[gp path] (1.504,5.756)--(1.684,5.756);
\node[gp node right] at (1.320,5.756) { 3};
\draw[gp path] (1.504,0.985)--(1.504,1.165);
\draw[gp path] (1.504,5.756)--(1.504,5.576);
\node[gp node center] at (1.504,0.677) {-10};
\draw[gp path] (2.560,0.985)--(2.560,1.165);
\draw[gp path] (2.560,5.756)--(2.560,5.576);
\node[gp node center] at (2.560,0.677) {-5};
\draw[gp path] (3.616,0.985)--(3.616,1.165);
\draw[gp path] (3.616,5.756)--(3.616,5.576);
\node[gp node center] at (3.616,0.677) { 0};
\draw[gp path] (4.673,0.985)--(4.673,1.165);
\draw[gp path] (4.673,5.756)--(4.673,5.576);
\node[gp node center] at (4.673,0.677) { 5};
\draw[gp path] (5.729,0.985)--(5.729,1.165);
\draw[gp path] (5.729,5.756)--(5.729,5.576);
\node[gp node center] at (5.729,0.677) { 10};
\draw[gp path] (6.785,0.985)--(6.785,1.165);
\draw[gp path] (6.785,5.756)--(6.785,5.576);
\node[gp node center] at (6.785,0.677) { 15};
\draw[gp path] (6.785,0.985)--(6.605,0.985);
\node[gp node left] at (6.969,0.985) { 0};
\draw[gp path] (6.785,1.462)--(6.605,1.462);
\node[gp node left] at (6.969,1.462) { 50};
\draw[gp path] (6.785,1.939)--(6.605,1.939);
\node[gp node left] at (6.969,1.939) { 100};
\draw[gp path] (6.785,2.416)--(6.605,2.416);
\node[gp node left] at (6.969,2.416) { 150};
\draw[gp path] (6.785,2.893)--(6.605,2.893);
\node[gp node left] at (6.969,2.893) { 200};
\draw[gp path] (6.785,3.371)--(6.605,3.371);
\node[gp node left] at (6.969,3.371) { 250};
\draw[gp path] (6.785,3.848)--(6.605,3.848);
\node[gp node left] at (6.969,3.848) { 300};
\draw[gp path] (6.785,4.325)--(6.605,4.325);
\node[gp node left] at (6.969,4.325) { 350};
\draw[gp path] (6.785,4.802)--(6.605,4.802);
\node[gp node left] at (6.969,4.802) { 400};
\draw[gp path] (6.785,5.279)--(6.605,5.279);
\node[gp node left] at (6.969,5.279) { 450};
\draw[gp path] (6.785,5.756)--(6.605,5.756);
\node[gp node left] at (6.969,5.756) { 500};
\draw[gp path] (1.504,5.756)--(1.504,0.985)--(6.785,0.985)--(6.785,5.756)--cycle;
\node[gp node center,rotate=-270] at (0.246,3.370) {HW1/eM Beam Width (mm)};
\node[gp node center,rotate=-270] at (8.042,3.370) {HW1/eM Beam Length ($\mu$m)};
\node[gp node center] at (4.144,0.215) {Pulse Location rel. to RF Cav. (cm)};
\gpcolor{color=gp lt color 0}
\gpsetlinetype{gp lt plot 0}
\draw[gp path] (1.504,1.144)--(1.533,1.144)--(1.562,1.146)--(1.591,1.148)--(1.620,1.151)%
  --(1.649,1.154)--(1.678,1.159)--(1.707,1.164)--(1.736,1.169)--(1.765,1.175)--(1.794,1.182)%
  --(1.824,1.189)--(1.853,1.197)--(1.882,1.204)--(1.911,1.213)--(1.940,1.221)--(1.969,1.230)%
  --(1.998,1.239)--(2.027,1.248)--(2.056,1.257)--(2.085,1.267)--(2.114,1.277)--(2.143,1.287)%
  --(2.172,1.297)--(2.201,1.307)--(2.230,1.317)--(2.259,1.328)--(2.288,1.338)--(2.317,1.349)%
  --(2.346,1.359)--(2.375,1.370)--(2.404,1.381)--(2.433,1.392)--(2.463,1.403)--(2.492,1.414)%
  --(2.521,1.425)--(2.550,1.436)--(2.579,1.448)--(2.608,1.459)--(2.637,1.470)--(2.666,1.482)%
  --(2.695,1.493)--(2.724,1.505)--(2.753,1.516)--(2.782,1.528)--(2.811,1.539)--(2.840,1.551)%
  --(2.869,1.563)--(2.898,1.574)--(2.927,1.586)--(2.956,1.598)--(2.985,1.610)--(3.014,1.621)%
  --(3.043,1.633)--(3.072,1.645)--(3.102,1.657)--(3.131,1.669)--(3.160,1.681)--(3.189,1.693)%
  --(3.218,1.704)--(3.247,1.716)--(3.276,1.728)--(3.305,1.740)--(3.334,1.752)--(3.363,1.764)%
  --(3.392,1.777)--(3.421,1.789)--(3.450,1.801)--(3.479,1.816)--(3.508,1.832)--(3.537,1.850)%
  --(3.566,1.868)--(3.595,1.888)--(3.624,1.907)--(3.653,1.927)--(3.682,1.948)--(3.711,1.969)%
  --(3.741,1.991)--(3.770,2.016)--(3.799,2.043)--(3.828,2.072)--(3.857,2.101)--(3.886,2.129)%
  --(3.915,2.158)--(3.944,2.187)--(3.973,2.216)--(4.002,2.245)--(4.031,2.274)--(4.060,2.302)%
  --(4.089,2.331)--(4.118,2.360)--(4.147,2.389)--(4.176,2.418)--(4.205,2.447)--(4.234,2.476)%
  --(4.263,2.505)--(4.292,2.533)--(4.321,2.562)--(4.350,2.591)--(4.380,2.620)--(4.409,2.649)%
  --(4.438,2.678)--(4.467,2.707)--(4.496,2.736)--(4.525,2.765)--(4.554,2.794)--(4.583,2.823)%
  --(4.612,2.852)--(4.641,2.881)--(4.670,2.910)--(4.699,2.939)--(4.728,2.968)--(4.757,2.997)%
  --(4.786,3.025)--(4.815,3.054)--(4.844,3.083)--(4.873,3.112)--(4.902,3.141)--(4.931,3.170)%
  --(4.960,3.199)--(4.989,3.228)--(5.019,3.257)--(5.048,3.286)--(5.077,3.315)--(5.106,3.344)%
  --(5.135,3.373)--(5.164,3.402)--(5.193,3.431)--(5.222,3.460)--(5.251,3.490)--(5.280,3.519)%
  --(5.309,3.548)--(5.338,3.577)--(5.367,3.606)--(5.396,3.635)--(5.425,3.664)--(5.454,3.693)%
  --(5.483,3.722)--(5.512,3.751)--(5.541,3.780)--(5.570,3.809)--(5.599,3.838)--(5.628,3.867)%
  --(5.658,3.896)--(5.687,3.925)--(5.716,3.954)--(5.745,3.983)--(5.774,4.012)--(5.803,4.041)%
  --(5.832,4.070)--(5.861,4.100)--(5.890,4.129)--(5.919,4.158)--(5.948,4.187)--(5.977,4.216)%
  --(6.006,4.245)--(6.035,4.274)--(6.064,4.303)--(6.093,4.332)--(6.122,4.361)--(6.151,4.390)%
  --(6.180,4.419)--(6.209,4.448)--(6.238,4.478)--(6.267,4.507)--(6.297,4.536)--(6.326,4.565)%
  --(6.355,4.594)--(6.384,4.623)--(6.413,4.652)--(6.442,4.681)--(6.471,4.710)--(6.500,4.739)%
  --(6.529,4.768)--(6.558,4.798)--(6.587,4.827)--(6.616,4.856)--(6.645,4.885)--(6.674,4.914)%
  --(6.703,4.943)--(6.732,4.972)--(6.761,5.001)--(6.785,5.025);
\gpcolor{rgb color={0.000,1.000,0.000}}
\draw[gp path] (1.504,1.065)--(1.533,1.069)--(1.562,1.082)--(1.591,1.102)--(1.620,1.127)%
  --(1.649,1.157)--(1.678,1.191)--(1.707,1.228)--(1.736,1.269)--(1.765,1.312)--(1.794,1.358)%
  --(1.824,1.406)--(1.853,1.456)--(1.882,1.508)--(1.911,1.562)--(1.940,1.617)--(1.969,1.673)%
  --(1.998,1.731)--(2.027,1.789)--(2.056,1.849)--(2.085,1.910)--(2.114,1.971)--(2.143,2.033)%
  --(2.172,2.096)--(2.201,2.159)--(2.230,2.223)--(2.259,2.288)--(2.288,2.353)--(2.317,2.419)%
  --(2.346,2.485)--(2.375,2.551)--(2.404,2.618)--(2.433,2.685)--(2.463,2.753)--(2.492,2.821)%
  --(2.521,2.889)--(2.550,2.958)--(2.579,3.026)--(2.608,3.095)--(2.637,3.165)--(2.666,3.234)%
  --(2.695,3.304)--(2.724,3.374)--(2.753,3.444)--(2.782,3.514)--(2.811,3.585)--(2.840,3.655)%
  --(2.869,3.726)--(2.898,3.797)--(2.927,3.868)--(2.956,3.939)--(2.985,4.011)--(3.014,4.082)%
  --(3.043,4.154)--(3.072,4.226)--(3.102,4.298)--(3.131,4.370)--(3.160,4.442)--(3.189,4.514)%
  --(3.218,4.586)--(3.247,4.659)--(3.276,4.731)--(3.305,4.804)--(3.334,4.877)--(3.363,4.949)%
  --(3.392,5.022)--(3.421,5.095)--(3.450,5.169)--(3.479,5.243)--(3.508,5.316)--(3.537,5.382)%
  --(3.566,5.434)--(3.595,5.467)--(3.624,5.477)--(3.653,5.463)--(3.682,5.428)--(3.711,5.377)%
  --(3.741,5.315)--(3.770,5.250)--(3.799,5.184)--(3.828,5.120)--(3.857,5.056)--(3.886,4.991)%
  --(3.915,4.927)--(3.944,4.863)--(3.973,4.799)--(4.002,4.735)--(4.031,4.671)--(4.060,4.608)%
  --(4.089,4.544)--(4.118,4.480)--(4.147,4.416)--(4.176,4.353)--(4.205,4.289)--(4.234,4.225)%
  --(4.263,4.162)--(4.292,4.098)--(4.321,4.035)--(4.350,3.971)--(4.380,3.908)--(4.409,3.844)%
  --(4.438,3.781)--(4.467,3.717)--(4.496,3.654)--(4.525,3.591)--(4.554,3.528)--(4.583,3.464)%
  --(4.612,3.401)--(4.641,3.338)--(4.670,3.275)--(4.699,3.212)--(4.728,3.148)--(4.757,3.085)%
  --(4.786,3.022)--(4.815,2.959)--(4.844,2.896)--(4.873,2.833)--(4.902,2.770)--(4.931,2.707)%
  --(4.960,2.645)--(4.989,2.582)--(5.019,2.519)--(5.048,2.456)--(5.077,2.393)--(5.106,2.330)%
  --(5.135,2.268)--(5.164,2.205)--(5.193,2.142)--(5.222,2.079)--(5.251,2.017)--(5.280,1.954)%
  --(5.309,1.891)--(5.338,1.829)--(5.367,1.766)--(5.396,1.704)--(5.425,1.641)--(5.454,1.578)%
  --(5.483,1.516)--(5.512,1.453)--(5.541,1.391)--(5.570,1.329)--(5.599,1.266)--(5.628,1.204)%
  --(5.658,1.142)--(5.687,1.081)--(5.716,1.024)--(5.745,1.025)--(5.774,1.083)--(5.803,1.144)%
  --(5.832,1.206)--(5.861,1.268)--(5.890,1.330)--(5.919,1.393)--(5.948,1.455)--(5.977,1.518)%
  --(6.006,1.580)--(6.035,1.643)--(6.064,1.705)--(6.093,1.768)--(6.122,1.830)--(6.151,1.893)%
  --(6.180,1.955)--(6.209,2.018)--(6.238,2.081)--(6.267,2.143)--(6.297,2.206)--(6.326,2.268)%
  --(6.355,2.331)--(6.384,2.394)--(6.413,2.456)--(6.442,2.519)--(6.471,2.582)--(6.500,2.645)%
  --(6.529,2.707)--(6.558,2.770)--(6.587,2.833)--(6.616,2.895)--(6.645,2.958)--(6.674,3.021)%
  --(6.703,3.084)--(6.732,3.147)--(6.761,3.209)--(6.785,3.261);
\gpcolor{color=gp lt color border}
\gpsetlinetype{gp lt border}
\draw[gp path] (1.504,5.756)--(1.504,0.985)--(6.785,0.985)--(6.785,5.756)--cycle;
%% coordinates of the plot area
\gpdefrectangularnode{gp plot 1}{\pgfpoint{1.504cm}{0.985cm}}{\pgfpoint{6.785cm}{5.756cm}}
%% \end{tikzpicture}
%% gnuplot variables

      \end{tikzpicture}
    }
  }
  \caption[Dynamics of the pulse length and width under the influence of an RF cavity]{
    Dynamics of the pulse length (green) and width (red) under the influence of an RF cavity placed at $z' = 0$.
    The employed idealized pulses start at $z'= -100$ cm with velocity $v_{\smallzero} = \sim 84 \times 10^{6}$ m/s (20kV) and initial HW1/eM width of 100 $\mu$m and length $\sim 8.4 \mu$m ($v_{\smallzero} \cdot $ 100fs).
    \subref{fig:rf_cav_N1} has $N=1$ electron and demonstrates one-to-one imaging of the initial pulse length at the focus at $ z' = 100 $ cm.
    \subref{fig:rf_cav_N1e5} has $ N = 1 \times 10^5 $ electrons, whose added broadening allows for the factor of two reduction in pulse length at the focus ($\sim 48$ fs).
  }
  \label{fig:rf_cav_num}
\end{figure}


\ref{fig:rf_cav_num}\subref{fig:rf_cav_N1} depicts the optimized RF cavity performance with $N=1$ electrons in the pulse.
The internal pulse space-charge effects are negligible under these conditions, so that the RF cavity only compensates for the `dispersive' electron pulse broadening to $\sim$1.67ps at the RF cavity entrance that is due to the initial longitudinal momentum spread caused by the $\Delta E = $ 0.5eV excess photoemission energy.
For an RF field amplitude $E_{0} = $ 450 kV/m and phase $\phi = 0$, the dispersion-generated momentum chirp $\gamma_{z}$ is exactly reversed upon propagation through the RF cavity to produce a compressed pulse at $ z^{\prime} = $ 10cm behind the RF cavity.
In this case, the pulse is returned to its original temporal duration --- an exact 1-to-1 `temporal imaging' of the photo-generated electron pulse at $z' = -10$cm by the RF cavity to $ z' = $ 10cm.
On the other hand, the dynamics of the spatial size of the electron pulse are clearly affected by the negative compound transverse lens of the RF cavity (the magnetic contribution of the TM$_{010}$-mode plus the axial entrance and exit aperture electric fields, see Section \ref{rf_cav_model}), increasing at the temporal focus to a factor of $\sim 10\times$ its initial value.

In analogy to optical imaging, the compressed pulse duration in this dispersive limit is determined by the image-to-object distance (magnification) ratio, with pulses shorter than the original laser pulse duration being produced when the distance from the RF cavity to the temporal focus is less than that from the effective photocathode to the RF cavity; specifically, the ratio of the time-of-flight between the RF cavity and the temporal focus to that between the photo-gun and the RF cavity.

%TODO why is this even worse than before?
The RF electric field amplitude $E_{0}$ of about 450kV/m required to achieve the temporal focusing 10cm behind the RF cavity in the dispersive limit is also consistent with expectations.
By equating the differential impulse applied by the RF cavity across the electron pulse duration to that required to compensate for dispersive pulse broadening before the RF cavity and then compress the pulse thereafter, one can show that the required RF field amplitude for a $\text{TM}_{010}$ cavity with $ d = \pi v_{\smallzero} / \Omega $ is given to a good approximation by
\begin{equation} \label{eq:RF_field_required}
  E_{0} \approx V_{DC} \left ( \frac{1}{ L_{1} } + \frac{1}{ L_{2} } \right ) \text{,}
\end{equation}
where $L_{1}$ is the distance from the pulse source (the DC photoelectron gun) to the RF cavity (the object distance) and $L_{2}$ is the distance to the temporal pulse focus behind the RF cavity (the image distance).
For the considered case, $V_{DC} = $ 20kV and $ L_{1} = L_{2} = $ 10cm, giving $ E_{0} \approx $ 400kV/m, which is in reasonably good agreement with the optimum value of 450kV/m employed for the simulation displayed in \ref{fig:rf_cav_num}\subref{fig:rf_cav_N1}.
\ref{eq:RF_field_required} also clearly indicates that the DC photoelectron gun and $\text{TM}_{010}$ RF pulse compression cavity should be considered as a single system, since the temporal `focal length' of the RF cavity, $ V_{DC} / E_{0} $, is dependent upon the acceleration voltage.

The optimized RF cavity performance for the temporal compression of a pulse with the same initial photoemission conditions, but with $N = 10^{5}$ electrons, is shown in \ref{fig:rf_cav_num}\subref{fig:rf_cav_N1e5}.
In this case, intra-pulse space-charge effects act to further broaden the initial 100fs pulse duration to $\sim$6ps after the $\sim$1ns time-of-flight over the 10cm distance to the RF cavity.
The fact that almost the same RF field strength (495 kV/m) is required to generate a temporal focus at $z^{\prime} = $ 10cm is a direct consequence of the linear dependence of the longitudinal RF cavity force on $z$ (\ref{eq:RF_Force_z}), which means that the required restorative force is not dependent on $\sigma_{z}$ --- just as the focal length of a perfect optical lens is independent of its aperture size.
However, unlike in the dispersion limited case (\ref{fig:rf_cav_num}\subref{fig:rf_cav_N1}), the compressed pulse duration is now a factor of $\sim$2 less ($\sim 48$ fs) than the original 100fs laser pulse duration.
This additional pulse compression for the employed 1-to-1 image-to-object distance (magnification) ratio is made possible by the additional pulse bandwidth (and momentum chirp) generated by the space-charge effects --- in analogy to self-phase modulation (and dispersion) in fiber-grating laser pulse compression.\cite{strickland_compression_1985}
In this case, the $\sim$20-fold increase in the transverse pulse size (corresponding to a $\sim400\times$ reduced peak pulse charge density) aids the temporal compression by greatly diminishing Coulomb effects that counteract the reduction in electron pulse duration.
Again, in analogy to optical imaging, a stronger (and suitably phased) RF field will result in an even shorter compressed pulse duration as the distance to the temporal focus behind the RF cavity is reduced.

It is important to note that the example illustrated in \ref{fig:rf_cav_num}\subref{fig:rf_cav_N1e5} approaches the limit imposed in obtaining \ref{eq:RF_Efield_approx}; namely that the longitudinal electron pulse length be much shorter than the axial length $d$ of the RF cavity, or $ \Omega z \ll v_{\smallzero} $.
For a $\sim$10ps electron pulse entering the 3GHz RF cavity we have $ \Omega z / v_{\smallzero} = \Omega \tau \approx 0.19 $, for which the small angle approximation $ \sin ( \Omega z / v_{\smallzero} ) = \Omega z / v_{\smallzero} $ is 0.6\% inaccurate.
In principle, the AG model could simulate the action of the RF cavity on longer pulse by including additional terms $O( z^{3} )$ and higher in \ref{eq:RF_Force_z} resulting from the expansion of $ \sin \left ( \Omega z / v_{0} \right ) $, but at the expense of voiding its self-similar Gaussian approximation.

