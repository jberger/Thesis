%This work is licensed under the Creative Commons Attribution-NonCommercial-NoDerivs 3.0 United States License. To view a copy of this license, visit http://creativecommons.org/licenses/by-nc-nd/3.0/us/ or send a letter to Creative Commons, 444 Castro Street, Suite 900, Mountain View, California, 94041, USA.

\section{Effect of Pulse Eccentricity on Dynamics} \label{sec:initial_shapes}

The initial `space-time' eccentricity \ref{eq:define_xi}, $\xi ( 0 ) = v_{\smallzero} \tau / w $ , of the photo-generated electron pulse plays a major role in the bunch propagation dynamics, particularly when space-charge effects dominate.
\ref{fig:compare_shape} illustrates the typical dynamics of initially spherical ($ \sigma_{ \smallT }(0) = \sigma_{z} ( 0 ) $), oblate or `disk-like' ($ \sigma_{ \smallT } ( 0 ) > \sigma_{ z } ( 0 ) $), and prolate or `cigar-like' ($ \sigma_{ z } ( 0 ) > \sigma_{ \smallT } ( 0 ) $) pulses.
%TODO rewrite using w?
For simplicity, and unlike most simulations in this paper, the pulses in this simulation are not calculated from laser and photocathode parameters, but are rather defined directly.
%TODO should I remove this? Should I just specify the creation types on each instance?
The initial bunch dimensions are given by the ellipticities $ \xi(0) = \{ 1/3 , 1 , 3 \} $ while in each case the product of the initial HW1/eM bunch widths $w_{\smallT}^2 w_z = $ 1mm$^3$.
Since the initial pulse volume is consistent between all runs, initial peak pulse charge density is six orders of magnitude greater in \ref{fig:compare_shape}\subref{fig:compare_shape_N1e6} ($ N = 10^{6} $ electrons in the bunch) than in \ref{fig:compare_shape}\subref{fig:compare_shape_N1} ($ N = 10^{0} $).
The pulse velocity is $ v_{\smallzero} \approx c/3 $, i.e. a DC photogun with $V = 30\text{kV} $.

\begin{figure}
  \centering
  \subfloat[][]{
    \label{fig:compare_shape_N1}
    \begin{tikzpicture}[gnuplot]
%% generated with GNUPLOT 4.6p0 (Lua 5.1; terminal rev. 99, script rev. 100)
%% Tue 05 Feb 2013 10:52:10 AM CST
\path (0.000,0.000) rectangle (12.500,8.750);
\gpcolor{color=gp lt color border}
\gpsetlinetype{gp lt border}
\gpsetlinewidth{1.00}
\draw[gp path] (1.688,0.985)--(1.868,0.985);
\draw[gp path] (11.947,0.985)--(11.767,0.985);
\node[gp node right] at (1.504,0.985) { 1};
\draw[gp path] (1.688,2.042)--(1.868,2.042);
\draw[gp path] (11.947,2.042)--(11.767,2.042);
\node[gp node right] at (1.504,2.042) { 1.02};
\draw[gp path] (1.688,3.098)--(1.868,3.098);
\draw[gp path] (11.947,3.098)--(11.767,3.098);
\node[gp node right] at (1.504,3.098) { 1.04};
\draw[gp path] (1.688,4.155)--(1.868,4.155);
\draw[gp path] (11.947,4.155)--(11.767,4.155);
\node[gp node right] at (1.504,4.155) { 1.06};
\draw[gp path] (1.688,5.211)--(1.868,5.211);
\draw[gp path] (11.947,5.211)--(11.767,5.211);
\node[gp node right] at (1.504,5.211) { 1.08};
\draw[gp path] (1.688,6.268)--(1.868,6.268);
\draw[gp path] (11.947,6.268)--(11.767,6.268);
\node[gp node right] at (1.504,6.268) { 1.1};
\draw[gp path] (1.688,7.324)--(1.868,7.324);
\draw[gp path] (11.947,7.324)--(11.767,7.324);
\node[gp node right] at (1.504,7.324) { 1.12};
\draw[gp path] (1.688,8.381)--(1.868,8.381);
\draw[gp path] (11.947,8.381)--(11.767,8.381);
\node[gp node right] at (1.504,8.381) { 1.14};
\draw[gp path] (1.688,0.985)--(1.688,1.165);
\draw[gp path] (1.688,8.381)--(1.688,8.201);
\node[gp node center] at (1.688,0.677) { 0};
\draw[gp path] (3.056,0.985)--(3.056,1.165);
\draw[gp path] (3.056,8.381)--(3.056,8.201);
\node[gp node center] at (3.056,0.677) { 2};
\draw[gp path] (4.424,0.985)--(4.424,1.165);
\draw[gp path] (4.424,8.381)--(4.424,8.201);
\node[gp node center] at (4.424,0.677) { 4};
\draw[gp path] (5.792,0.985)--(5.792,1.165);
\draw[gp path] (5.792,8.381)--(5.792,8.201);
\node[gp node center] at (5.792,0.677) { 6};
\draw[gp path] (7.159,0.985)--(7.159,1.165);
\draw[gp path] (7.159,8.381)--(7.159,8.201);
\node[gp node center] at (7.159,0.677) { 8};
\draw[gp path] (8.527,0.985)--(8.527,1.165);
\draw[gp path] (8.527,8.381)--(8.527,8.201);
\node[gp node center] at (8.527,0.677) { 10};
\draw[gp path] (9.895,0.985)--(9.895,1.165);
\draw[gp path] (9.895,8.381)--(9.895,8.201);
\node[gp node center] at (9.895,0.677) { 12};
\draw[gp path] (11.263,0.985)--(11.263,1.165);
\draw[gp path] (11.263,8.381)--(11.263,8.201);
\node[gp node center] at (11.263,0.677) { 14};
\draw[gp path] (1.688,8.381)--(1.688,0.985)--(11.947,0.985)--(11.947,8.381)--cycle;
\node[gp node center,rotate=-270] at (0.246,4.683) {Normalized Pulse Width, Length (A.U.)};
\node[gp node center] at (6.817,0.215) {Position in Column (m)};
\gpcolor{rgb color={1.000,0.000,0.000}}
\gpsetlinetype{gp lt plot 0}
\draw[gp path] (1.688,0.985)--(1.801,0.985)--(1.914,0.986)--(2.027,0.987)--(2.139,0.988)%
  --(2.252,0.990)--(2.365,0.992)--(2.478,0.995)--(2.591,0.998)--(2.704,1.001)--(2.816,1.005)%
  --(2.929,1.010)--(3.042,1.014)--(3.155,1.019)--(3.268,1.025)--(3.381,1.031)--(3.494,1.037)%
  --(3.606,1.044)--(3.719,1.051)--(3.832,1.058)--(3.945,1.066)--(4.058,1.074)--(4.171,1.083)%
  --(4.284,1.092)--(4.396,1.102)--(4.509,1.111)--(4.622,1.122)--(4.735,1.133)--(4.848,1.144)%
  --(4.961,1.155)--(5.073,1.167)--(5.186,1.179)--(5.299,1.192)--(5.412,1.205)--(5.525,1.219)%
  --(5.638,1.233)--(5.751,1.247)--(5.863,1.262)--(5.976,1.277)--(6.089,1.292)--(6.202,1.308)%
  --(6.315,1.325)--(6.428,1.341)--(6.541,1.358)--(6.653,1.376)--(6.766,1.394)--(6.879,1.412)%
  --(6.992,1.431)--(7.105,1.450)--(7.218,1.469)--(7.330,1.489)--(7.443,1.509)--(7.556,1.530)%
  --(7.669,1.551)--(7.782,1.573)--(7.895,1.594)--(8.008,1.617)--(8.120,1.639)--(8.233,1.662)%
  --(8.346,1.686)--(8.459,1.709)--(8.572,1.734)--(8.685,1.758)--(8.797,1.783)--(8.910,1.809)%
  --(9.023,1.834)--(9.136,1.860)--(9.249,1.887)--(9.362,1.914)--(9.475,1.941)--(9.587,1.969)%
  --(9.700,1.997)--(9.813,2.025)--(9.926,2.054)--(10.039,2.083)--(10.152,2.113)--(10.265,2.143)%
  --(10.377,2.173)--(10.490,2.204)--(10.603,2.235)--(10.716,2.266)--(10.829,2.298)--(10.942,2.330)%
  --(11.054,2.363)--(11.167,2.396)--(11.280,2.429)--(11.393,2.463)--(11.506,2.497)--(11.619,2.531)%
  --(11.732,2.566)--(11.844,2.601)--(11.947,2.634);
\gpsetlinetype{gp lt plot 1}
\draw[gp path] (1.688,0.985)--(1.801,0.985)--(1.914,0.987)--(2.027,0.989)--(2.139,0.992)%
  --(2.252,0.996)--(2.365,1.001)--(2.478,1.007)--(2.591,1.014)--(2.704,1.022)--(2.816,1.031)%
  --(2.929,1.040)--(3.042,1.051)--(3.155,1.062)--(3.268,1.074)--(3.381,1.087)--(3.494,1.102)%
  --(3.606,1.117)--(3.719,1.133)--(3.832,1.149)--(3.945,1.167)--(4.058,1.186)--(4.171,1.205)%
  --(4.284,1.226)--(4.396,1.247)--(4.509,1.269)--(4.622,1.292)--(4.735,1.316)--(4.848,1.341)%
  --(4.961,1.367)--(5.073,1.394)--(5.186,1.421)--(5.299,1.450)--(5.412,1.479)--(5.525,1.509)%
  --(5.638,1.541)--(5.751,1.573)--(5.863,1.605)--(5.976,1.639)--(6.089,1.674)--(6.202,1.709)%
  --(6.315,1.746)--(6.428,1.783)--(6.541,1.821)--(6.653,1.860)--(6.766,1.900)--(6.879,1.941)%
  --(6.992,1.983)--(7.105,2.025)--(7.218,2.068)--(7.330,2.113)--(7.443,2.158)--(7.556,2.204)%
  --(7.669,2.250)--(7.782,2.298)--(7.895,2.347)--(8.008,2.396)--(8.120,2.446)--(8.233,2.497)%
  --(8.346,2.549)--(8.459,2.601)--(8.572,2.655)--(8.685,2.709)--(8.797,2.764)--(8.910,2.820)%
  --(9.023,2.877)--(9.136,2.935)--(9.249,2.993)--(9.362,3.053)--(9.475,3.113)--(9.587,3.174)%
  --(9.700,3.235)--(9.813,3.298)--(9.926,3.361)--(10.039,3.425)--(10.152,3.490)--(10.265,3.556)%
  --(10.377,3.622)--(10.490,3.689)--(10.603,3.758)--(10.716,3.826)--(10.829,3.896)--(10.942,3.966)%
  --(11.054,4.037)--(11.167,4.109)--(11.280,4.182)--(11.393,4.256)--(11.506,4.330)--(11.619,4.405)%
  --(11.732,4.481)--(11.844,4.557)--(11.947,4.627);
\gpcolor{rgb color={0.000,1.000,0.000}}
\gpsetlinetype{gp lt plot 0}
\draw[gp path] (1.688,0.985)--(1.801,0.985)--(1.914,0.987)--(2.027,0.989)--(2.139,0.992)%
  --(2.252,0.996)--(2.365,1.000)--(2.478,1.006)--(2.591,1.012)--(2.704,1.019)--(2.816,1.027)%
  --(2.929,1.036)--(3.042,1.046)--(3.155,1.056)--(3.268,1.068)--(3.381,1.080)--(3.494,1.093)%
  --(3.606,1.107)--(3.719,1.121)--(3.832,1.137)--(3.945,1.153)--(4.058,1.171)--(4.171,1.189)%
  --(4.284,1.208)--(4.396,1.227)--(4.509,1.248)--(4.622,1.269)--(4.735,1.291)--(4.848,1.314)%
  --(4.961,1.338)--(5.073,1.363)--(5.186,1.389)--(5.299,1.415)--(5.412,1.442)--(5.525,1.470)%
  --(5.638,1.499)--(5.751,1.528)--(5.863,1.559)--(5.976,1.590)--(6.089,1.622)--(6.202,1.655)%
  --(6.315,1.689)--(6.428,1.723)--(6.541,1.759)--(6.653,1.795)--(6.766,1.832)--(6.879,1.869)%
  --(6.992,1.908)--(7.105,1.947)--(7.218,1.988)--(7.330,2.028)--(7.443,2.070)--(7.556,2.113)%
  --(7.669,2.156)--(7.782,2.200)--(7.895,2.245)--(8.008,2.291)--(8.120,2.337)--(8.233,2.384)%
  --(8.346,2.432)--(8.459,2.481)--(8.572,2.531)--(8.685,2.581)--(8.797,2.632)--(8.910,2.684)%
  --(9.023,2.737)--(9.136,2.790)--(9.249,2.844)--(9.362,2.899)--(9.475,2.955)--(9.587,3.011)%
  --(9.700,3.069)--(9.813,3.127)--(9.926,3.185)--(10.039,3.245)--(10.152,3.305)--(10.265,3.366)%
  --(10.377,3.428)--(10.490,3.490)--(10.603,3.553)--(10.716,3.617)--(10.829,3.682)--(10.942,3.747)%
  --(11.054,3.813)--(11.167,3.880)--(11.280,3.947)--(11.393,4.015)--(11.506,4.084)--(11.619,4.154)%
  --(11.732,4.224)--(11.844,4.295)--(11.947,4.361);
\gpsetlinetype{gp lt plot 1}
\draw[gp path] (1.688,0.985)--(1.801,0.985)--(1.914,0.985)--(2.027,0.986)--(2.139,0.987)%
  --(2.252,0.988)--(2.365,0.989)--(2.478,0.990)--(2.591,0.992)--(2.704,0.994)--(2.816,0.996)%
  --(2.929,0.998)--(3.042,1.000)--(3.155,1.003)--(3.268,1.006)--(3.381,1.009)--(3.494,1.012)%
  --(3.606,1.015)--(3.719,1.019)--(3.832,1.023)--(3.945,1.027)--(4.058,1.031)--(4.171,1.036)%
  --(4.284,1.041)--(4.396,1.046)--(4.509,1.051)--(4.622,1.056)--(4.735,1.062)--(4.848,1.068)%
  --(4.961,1.074)--(5.073,1.080)--(5.186,1.086)--(5.299,1.093)--(5.412,1.100)--(5.525,1.107)%
  --(5.638,1.114)--(5.751,1.121)--(5.863,1.129)--(5.976,1.137)--(6.089,1.145)--(6.202,1.153)%
  --(6.315,1.162)--(6.428,1.171)--(6.541,1.179)--(6.653,1.189)--(6.766,1.198)--(6.879,1.207)%
  --(6.992,1.217)--(7.105,1.227)--(7.218,1.237)--(7.330,1.248)--(7.443,1.258)--(7.556,1.269)%
  --(7.669,1.280)--(7.782,1.291)--(7.895,1.303)--(8.008,1.314)--(8.120,1.326)--(8.233,1.338)%
  --(8.346,1.350)--(8.459,1.363)--(8.572,1.376)--(8.685,1.388)--(8.797,1.402)--(8.910,1.415)%
  --(9.023,1.428)--(9.136,1.442)--(9.249,1.456)--(9.362,1.470)--(9.475,1.484)--(9.587,1.499)%
  --(9.700,1.513)--(9.813,1.528)--(9.926,1.543)--(10.039,1.559)--(10.152,1.574)--(10.265,1.590)%
  --(10.377,1.606)--(10.490,1.622)--(10.603,1.638)--(10.716,1.655)--(10.829,1.672)--(10.942,1.689)%
  --(11.054,1.706)--(11.167,1.723)--(11.280,1.741)--(11.393,1.759)--(11.506,1.776)--(11.619,1.795)%
  --(11.732,1.813)--(11.844,1.832)--(11.947,1.849);
\gpcolor{rgb color={0.000,0.000,1.000}}
\gpsetlinetype{gp lt plot 0}
\draw[gp path] (1.688,0.985)--(1.801,0.986)--(1.914,0.989)--(2.027,0.993)--(2.139,0.999)%
  --(2.252,1.007)--(2.365,1.017)--(2.478,1.028)--(2.591,1.041)--(2.704,1.056)--(2.816,1.073)%
  --(2.929,1.091)--(3.042,1.111)--(3.155,1.133)--(3.268,1.157)--(3.381,1.182)--(3.494,1.209)%
  --(3.606,1.238)--(3.719,1.268)--(3.832,1.301)--(3.945,1.335)--(4.058,1.370)--(4.171,1.408)%
  --(4.284,1.447)--(4.396,1.488)--(4.509,1.530)--(4.622,1.574)--(4.735,1.620)--(4.848,1.668)%
  --(4.961,1.717)--(5.073,1.768)--(5.186,1.821)--(5.299,1.875)--(5.412,1.931)--(5.525,1.989)%
  --(5.638,2.048)--(5.751,2.109)--(5.863,2.172)--(5.976,2.236)--(6.089,2.302)--(6.202,2.370)%
  --(6.315,2.439)--(6.428,2.509)--(6.541,2.582)--(6.653,2.656)--(6.766,2.731)--(6.879,2.809)%
  --(6.992,2.887)--(7.105,2.968)--(7.218,3.050)--(7.330,3.133)--(7.443,3.218)--(7.556,3.305)%
  --(7.669,3.393)--(7.782,3.482)--(7.895,3.574)--(8.008,3.666)--(8.120,3.760)--(8.233,3.856)%
  --(8.346,3.953)--(8.459,4.052)--(8.572,4.152)--(8.685,4.254)--(8.797,4.357)--(8.910,4.462)%
  --(9.023,4.568)--(9.136,4.675)--(9.249,4.784)--(9.362,4.894)--(9.475,5.006)--(9.587,5.119)%
  --(9.700,5.234)--(9.813,5.349)--(9.926,5.467)--(10.039,5.585)--(10.152,5.705)--(10.265,5.827)%
  --(10.377,5.950)--(10.490,6.074)--(10.603,6.199)--(10.716,6.326)--(10.829,6.454)--(10.942,6.583)%
  --(11.054,6.714)--(11.167,6.846)--(11.280,6.979)--(11.393,7.114)--(11.506,7.249)--(11.619,7.386)%
  --(11.732,7.524)--(11.844,7.664)--(11.947,7.792);
\gpsetlinetype{gp lt plot 1}
\draw[gp path] (1.688,0.985)--(1.801,0.985)--(1.914,0.985)--(2.027,0.985)--(2.139,0.985)%
  --(2.252,0.986)--(2.365,0.986)--(2.478,0.986)--(2.591,0.987)--(2.704,0.987)--(2.816,0.987)%
  --(2.929,0.988)--(3.042,0.989)--(3.155,0.989)--(3.268,0.990)--(3.381,0.990)--(3.494,0.991)%
  --(3.606,0.992)--(3.719,0.993)--(3.832,0.994)--(3.945,0.995)--(4.058,0.996)--(4.171,0.997)%
  --(4.284,0.998)--(4.396,0.999)--(4.509,1.000)--(4.622,1.001)--(4.735,1.003)--(4.848,1.004)%
  --(4.961,1.005)--(5.073,1.007)--(5.186,1.008)--(5.299,1.010)--(5.412,1.012)--(5.525,1.013)%
  --(5.638,1.015)--(5.751,1.017)--(5.863,1.018)--(5.976,1.020)--(6.089,1.022)--(6.202,1.024)%
  --(6.315,1.026)--(6.428,1.028)--(6.541,1.030)--(6.653,1.032)--(6.766,1.034)--(6.879,1.037)%
  --(6.992,1.039)--(7.105,1.041)--(7.218,1.043)--(7.330,1.046)--(7.443,1.048)--(7.556,1.051)%
  --(7.669,1.053)--(7.782,1.056)--(7.895,1.059)--(8.008,1.061)--(8.120,1.064)--(8.233,1.067)%
  --(8.346,1.070)--(8.459,1.073)--(8.572,1.076)--(8.685,1.079)--(8.797,1.082)--(8.910,1.085)%
  --(9.023,1.088)--(9.136,1.091)--(9.249,1.094)--(9.362,1.097)--(9.475,1.101)--(9.587,1.104)%
  --(9.700,1.108)--(9.813,1.111)--(9.926,1.115)--(10.039,1.118)--(10.152,1.122)--(10.265,1.125)%
  --(10.377,1.129)--(10.490,1.133)--(10.603,1.137)--(10.716,1.141)--(10.829,1.145)--(10.942,1.148)%
  --(11.054,1.152)--(11.167,1.157)--(11.280,1.161)--(11.393,1.165)--(11.506,1.169)--(11.619,1.173)%
  --(11.732,1.178)--(11.844,1.182)--(11.947,1.186);
\gpcolor{color=gp lt color border}
\gpsetlinetype{gp lt border}
\draw[gp path] (1.688,8.381)--(1.688,0.985)--(11.947,0.985)--(11.947,8.381)--cycle;
%% coordinates of the plot area
\gpdefrectangularnode{gp plot 1}{\pgfpoint{1.688cm}{0.985cm}}{\pgfpoint{11.947cm}{8.381cm}}
\end{tikzpicture}
%% gnuplot variables

  }
  \\
  \subfloat[][]{
    \label{fig:compare_shape_N1e6}
    \begin{tikzpicture}[gnuplot]
%% generated with GNUPLOT 4.6p0 (Lua 5.1; terminal rev. 99, script rev. 100)
%% Tue 05 Feb 2013 10:54:38 AM CST
\path (0.000,0.000) rectangle (12.500,8.750);
\gpcolor{color=gp lt color border}
\gpsetlinetype{gp lt border}
\gpsetlinewidth{1.00}
\draw[gp path] (1.688,0.985)--(1.868,0.985);
\draw[gp path] (11.947,0.985)--(11.767,0.985);
\node[gp node right] at (1.504,0.985) { 1};
\draw[gp path] (1.688,2.042)--(1.868,2.042);
\draw[gp path] (11.947,2.042)--(11.767,2.042);
\node[gp node right] at (1.504,2.042) { 1.02};
\draw[gp path] (1.688,3.098)--(1.868,3.098);
\draw[gp path] (11.947,3.098)--(11.767,3.098);
\node[gp node right] at (1.504,3.098) { 1.04};
\draw[gp path] (1.688,4.155)--(1.868,4.155);
\draw[gp path] (11.947,4.155)--(11.767,4.155);
\node[gp node right] at (1.504,4.155) { 1.06};
\draw[gp path] (1.688,5.211)--(1.868,5.211);
\draw[gp path] (11.947,5.211)--(11.767,5.211);
\node[gp node right] at (1.504,5.211) { 1.08};
\draw[gp path] (1.688,6.268)--(1.868,6.268);
\draw[gp path] (11.947,6.268)--(11.767,6.268);
\node[gp node right] at (1.504,6.268) { 1.1};
\draw[gp path] (1.688,7.324)--(1.868,7.324);
\draw[gp path] (11.947,7.324)--(11.767,7.324);
\node[gp node right] at (1.504,7.324) { 1.12};
\draw[gp path] (1.688,8.381)--(1.868,8.381);
\draw[gp path] (11.947,8.381)--(11.767,8.381);
\node[gp node right] at (1.504,8.381) { 1.14};
\draw[gp path] (1.688,0.985)--(1.688,1.165);
\draw[gp path] (1.688,8.381)--(1.688,8.201);
\node[gp node center] at (1.688,0.677) { 0};
\draw[gp path] (3.056,0.985)--(3.056,1.165);
\draw[gp path] (3.056,8.381)--(3.056,8.201);
\node[gp node center] at (3.056,0.677) { 2};
\draw[gp path] (4.424,0.985)--(4.424,1.165);
\draw[gp path] (4.424,8.381)--(4.424,8.201);
\node[gp node center] at (4.424,0.677) { 4};
\draw[gp path] (5.792,0.985)--(5.792,1.165);
\draw[gp path] (5.792,8.381)--(5.792,8.201);
\node[gp node center] at (5.792,0.677) { 6};
\draw[gp path] (7.159,0.985)--(7.159,1.165);
\draw[gp path] (7.159,8.381)--(7.159,8.201);
\node[gp node center] at (7.159,0.677) { 8};
\draw[gp path] (8.527,0.985)--(8.527,1.165);
\draw[gp path] (8.527,8.381)--(8.527,8.201);
\node[gp node center] at (8.527,0.677) { 10};
\draw[gp path] (9.895,0.985)--(9.895,1.165);
\draw[gp path] (9.895,8.381)--(9.895,8.201);
\node[gp node center] at (9.895,0.677) { 12};
\draw[gp path] (11.263,0.985)--(11.263,1.165);
\draw[gp path] (11.263,8.381)--(11.263,8.201);
\node[gp node center] at (11.263,0.677) { 14};
\draw[gp path] (1.688,8.381)--(1.688,0.985)--(11.947,0.985)--(11.947,8.381)--cycle;
\node[gp node center,rotate=-270] at (0.246,4.683) {Normalized Pulse Width, Length (A.U.)};
\node[gp node center] at (6.817,0.215) {Position in Column (m)};
\gpcolor{rgb color={1.000,0.000,0.000}}
\gpsetlinetype{gp lt plot 0}
\draw[gp path] (1.688,0.985)--(1.801,0.985)--(1.914,0.987)--(2.027,0.989)--(2.139,0.992)%
  --(2.252,0.997)--(2.365,1.002)--(2.478,1.008)--(2.591,1.015)--(2.704,1.023)--(2.816,1.032)%
  --(2.929,1.042)--(3.042,1.052)--(3.155,1.064)--(3.268,1.077)--(3.381,1.090)--(3.494,1.104)%
  --(3.606,1.120)--(3.719,1.136)--(3.832,1.153)--(3.945,1.172)--(4.058,1.191)--(4.171,1.211)%
  --(4.284,1.232)--(4.396,1.253)--(4.509,1.276)--(4.622,1.300)--(4.735,1.325)--(4.848,1.350)%
  --(4.961,1.377)--(5.073,1.404)--(5.186,1.432)--(5.299,1.461)--(5.412,1.492)--(5.525,1.523)%
  --(5.638,1.555)--(5.751,1.587)--(5.863,1.621)--(5.976,1.656)--(6.089,1.691)--(6.202,1.728)%
  --(6.315,1.765)--(6.428,1.803)--(6.541,1.842)--(6.653,1.882)--(6.766,1.923)--(6.879,1.965)%
  --(6.992,2.008)--(7.105,2.052)--(7.218,2.096)--(7.330,2.141)--(7.443,2.188)--(7.556,2.235)%
  --(7.669,2.283)--(7.782,2.332)--(7.895,2.381)--(8.008,2.432)--(8.120,2.483)--(8.233,2.536)%
  --(8.346,2.589)--(8.459,2.643)--(8.572,2.698)--(8.685,2.754)--(8.797,2.810)--(8.910,2.868)%
  --(9.023,2.926)--(9.136,2.985)--(9.249,3.045)--(9.362,3.106)--(9.475,3.168)--(9.587,3.230)%
  --(9.700,3.294)--(9.813,3.358)--(9.926,3.423)--(10.039,3.489)--(10.152,3.555)--(10.265,3.623)%
  --(10.377,3.691)--(10.490,3.760)--(10.603,3.830)--(10.716,3.901)--(10.829,3.973)--(10.942,4.045)%
  --(11.054,4.118)--(11.167,4.192)--(11.280,4.267)--(11.393,4.342)--(11.506,4.419)--(11.619,4.496)%
  --(11.732,4.574)--(11.844,4.652)--(11.947,4.725);
\gpsetlinetype{gp lt plot 1}
\draw[gp path] (1.688,0.985)--(1.801,0.986)--(1.914,0.991)--(2.027,0.997)--(2.139,1.007)%
  --(2.252,1.019)--(2.365,1.035)--(2.478,1.053)--(2.591,1.073)--(2.704,1.097)--(2.816,1.123)%
  --(2.929,1.152)--(3.042,1.183)--(3.155,1.218)--(3.268,1.255)--(3.381,1.295)--(3.494,1.337)%
  --(3.606,1.382)--(3.719,1.430)--(3.832,1.481)--(3.945,1.535)--(4.058,1.591)--(4.171,1.650)%
  --(4.284,1.711)--(4.396,1.776)--(4.509,1.843)--(4.622,1.912)--(4.735,1.985)--(4.848,2.060)%
  --(4.961,2.137)--(5.073,2.218)--(5.186,2.301)--(5.299,2.386)--(5.412,2.474)--(5.525,2.565)%
  --(5.638,2.659)--(5.751,2.755)--(5.863,2.854)--(5.976,2.955)--(6.089,3.059)--(6.202,3.166)%
  --(6.315,3.275)--(6.428,3.387)--(6.541,3.501)--(6.653,3.618)--(6.766,3.737)--(6.879,3.859)%
  --(6.992,3.984)--(7.105,4.111)--(7.218,4.240)--(7.330,4.372)--(7.443,4.507)--(7.556,4.644)%
  --(7.669,4.783)--(7.782,4.925)--(7.895,5.069)--(8.008,5.216)--(8.120,5.366)--(8.233,5.517)%
  --(8.346,5.671)--(8.459,5.828)--(8.572,5.987)--(8.685,6.148)--(8.797,6.312)--(8.910,6.478)%
  --(9.023,6.646)--(9.136,6.817)--(9.249,6.990)--(9.362,7.165)--(9.475,7.343)--(9.587,7.523)%
  --(9.700,7.705)--(9.813,7.890)--(9.926,8.077)--(10.039,8.266)--(10.107,8.381);
\gpcolor{rgb color={0.000,1.000,0.000}}
\gpsetlinetype{gp lt plot 0}
\draw[gp path] (1.688,0.985)--(1.801,0.986)--(1.914,0.989)--(2.027,0.993)--(2.139,0.999)%
  --(2.252,1.008)--(2.365,1.018)--(2.478,1.029)--(2.591,1.043)--(2.704,1.058)--(2.816,1.075)%
  --(2.929,1.094)--(3.042,1.115)--(3.155,1.138)--(3.268,1.162)--(3.381,1.188)--(3.494,1.216)%
  --(3.606,1.246)--(3.719,1.278)--(3.832,1.311)--(3.945,1.346)--(4.058,1.383)--(4.171,1.422)%
  --(4.284,1.462)--(4.396,1.504)--(4.509,1.548)--(4.622,1.594)--(4.735,1.642)--(4.848,1.691)%
  --(4.961,1.742)--(5.073,1.795)--(5.186,1.849)--(5.299,1.906)--(5.412,1.964)--(5.525,2.024)%
  --(5.638,2.085)--(5.751,2.148)--(5.863,2.213)--(5.976,2.280)--(6.089,2.348)--(6.202,2.418)%
  --(6.315,2.490)--(6.428,2.564)--(6.541,2.639)--(6.653,2.716)--(6.766,2.794)--(6.879,2.874)%
  --(6.992,2.956)--(7.105,3.040)--(7.218,3.125)--(7.330,3.212)--(7.443,3.300)--(7.556,3.390)%
  --(7.669,3.482)--(7.782,3.575)--(7.895,3.670)--(8.008,3.767)--(8.120,3.865)--(8.233,3.965)%
  --(8.346,4.066)--(8.459,4.169)--(8.572,4.274)--(8.685,4.380)--(8.797,4.487)--(8.910,4.597)%
  --(9.023,4.708)--(9.136,4.820)--(9.249,4.934)--(9.362,5.049)--(9.475,5.166)--(9.587,5.284)%
  --(9.700,5.404)--(9.813,5.526)--(9.926,5.649)--(10.039,5.773)--(10.152,5.899)--(10.265,6.026)%
  --(10.377,6.155)--(10.490,6.286)--(10.603,6.417)--(10.716,6.551)--(10.829,6.685)--(10.942,6.821)%
  --(11.054,6.959)--(11.167,7.097)--(11.280,7.238)--(11.393,7.379)--(11.506,7.522)--(11.619,7.667)%
  --(11.732,7.813)--(11.844,7.960)--(11.947,8.095);
\gpsetlinetype{gp lt plot 1}
\draw[gp path] (1.688,0.985)--(1.801,0.986)--(1.914,0.987)--(2.027,0.990)--(2.139,0.994)%
  --(2.252,1.000)--(2.365,1.006)--(2.478,1.014)--(2.591,1.023)--(2.704,1.033)--(2.816,1.044)%
  --(2.929,1.056)--(3.042,1.070)--(3.155,1.084)--(3.268,1.100)--(3.381,1.117)--(3.494,1.136)%
  --(3.606,1.155)--(3.719,1.176)--(3.832,1.197)--(3.945,1.220)--(4.058,1.244)--(4.171,1.269)%
  --(4.284,1.296)--(4.396,1.323)--(4.509,1.352)--(4.622,1.382)--(4.735,1.413)--(4.848,1.445)%
  --(4.961,1.478)--(5.073,1.513)--(5.186,1.548)--(5.299,1.585)--(5.412,1.623)--(5.525,1.662)%
  --(5.638,1.702)--(5.751,1.744)--(5.863,1.786)--(5.976,1.830)--(6.089,1.874)--(6.202,1.920)%
  --(6.315,1.967)--(6.428,2.015)--(6.541,2.064)--(6.653,2.115)--(6.766,2.166)--(6.879,2.219)%
  --(6.992,2.272)--(7.105,2.327)--(7.218,2.383)--(7.330,2.440)--(7.443,2.498)--(7.556,2.557)%
  --(7.669,2.617)--(7.782,2.678)--(7.895,2.741)--(8.008,2.804)--(8.120,2.869)--(8.233,2.934)%
  --(8.346,3.001)--(8.459,3.069)--(8.572,3.138)--(8.685,3.208)--(8.797,3.278)--(8.910,3.350)%
  --(9.023,3.423)--(9.136,3.498)--(9.249,3.573)--(9.362,3.649)--(9.475,3.726)--(9.587,3.804)%
  --(9.700,3.883)--(9.813,3.964)--(9.926,4.045)--(10.039,4.127)--(10.152,4.210)--(10.265,4.295)%
  --(10.377,4.380)--(10.490,4.466)--(10.603,4.554)--(10.716,4.642)--(10.829,4.731)--(10.942,4.821)%
  --(11.054,4.913)--(11.167,5.005)--(11.280,5.098)--(11.393,5.192)--(11.506,5.287)--(11.619,5.383)%
  --(11.732,5.480)--(11.844,5.578)--(11.947,5.668);
\gpcolor{rgb color={0.000,0.000,1.000}}
\gpsetlinetype{gp lt plot 0}
\draw[gp path] (1.688,0.985)--(1.801,0.987)--(1.914,0.991)--(2.027,0.999)--(2.139,1.009)%
  --(2.252,1.023)--(2.365,1.040)--(2.478,1.060)--(2.591,1.082)--(2.704,1.108)--(2.816,1.137)%
  --(2.929,1.169)--(3.042,1.204)--(3.155,1.242)--(3.268,1.283)--(3.381,1.327)--(3.494,1.374)%
  --(3.606,1.424)--(3.719,1.477)--(3.832,1.533)--(3.945,1.592)--(4.058,1.654)--(4.171,1.719)%
  --(4.284,1.786)--(4.396,1.857)--(4.509,1.931)--(4.622,2.007)--(4.735,2.087)--(4.848,2.169)%
  --(4.961,2.255)--(5.073,2.343)--(5.186,2.434)--(5.299,2.528)--(5.412,2.625)--(5.525,2.725)%
  --(5.638,2.827)--(5.751,2.933)--(5.863,3.041)--(5.976,3.152)--(6.089,3.265)--(6.202,3.382)%
  --(6.315,3.501)--(6.428,3.623)--(6.541,3.748)--(6.653,3.875)--(6.766,4.006)--(6.879,4.138)%
  --(6.992,4.274)--(7.105,4.412)--(7.218,4.553)--(7.330,4.696)--(7.443,4.842)--(7.556,4.991)%
  --(7.669,5.142)--(7.782,5.296)--(7.895,5.453)--(8.008,5.612)--(8.120,5.773)--(8.233,5.937)%
  --(8.346,6.104)--(8.459,6.273)--(8.572,6.444)--(8.685,6.618)--(8.797,6.794)--(8.910,6.973)%
  --(9.023,7.154)--(9.136,7.337)--(9.249,7.523)--(9.362,7.712)--(9.475,7.902)--(9.587,8.095)%
  --(9.700,8.290)--(9.752,8.381);
\gpsetlinetype{gp lt plot 1}
\draw[gp path] (1.688,0.985)--(1.801,0.985)--(1.914,0.986)--(2.027,0.987)--(2.139,0.988)%
  --(2.252,0.990)--(2.365,0.992)--(2.478,0.994)--(2.591,0.997)--(2.704,1.000)--(2.816,1.003)%
  --(2.929,1.007)--(3.042,1.011)--(3.155,1.016)--(3.268,1.021)--(3.381,1.026)--(3.494,1.032)%
  --(3.606,1.038)--(3.719,1.044)--(3.832,1.051)--(3.945,1.058)--(4.058,1.065)--(4.171,1.073)%
  --(4.284,1.081)--(4.396,1.090)--(4.509,1.099)--(4.622,1.108)--(4.735,1.117)--(4.848,1.127)%
  --(4.961,1.138)--(5.073,1.148)--(5.186,1.159)--(5.299,1.171)--(5.412,1.182)--(5.525,1.195)%
  --(5.638,1.207)--(5.751,1.220)--(5.863,1.233)--(5.976,1.246)--(6.089,1.260)--(6.202,1.275)%
  --(6.315,1.289)--(6.428,1.304)--(6.541,1.319)--(6.653,1.335)--(6.766,1.351)--(6.879,1.367)%
  --(6.992,1.384)--(7.105,1.401)--(7.218,1.418)--(7.330,1.436)--(7.443,1.454)--(7.556,1.472)%
  --(7.669,1.491)--(7.782,1.510)--(7.895,1.529)--(8.008,1.549)--(8.120,1.569)--(8.233,1.590)%
  --(8.346,1.610)--(8.459,1.631)--(8.572,1.653)--(8.685,1.675)--(8.797,1.697)--(8.910,1.719)%
  --(9.023,1.742)--(9.136,1.765)--(9.249,1.788)--(9.362,1.812)--(9.475,1.836)--(9.587,1.861)%
  --(9.700,1.885)--(9.813,1.910)--(9.926,1.936)--(10.039,1.961)--(10.152,1.988)--(10.265,2.014)%
  --(10.377,2.041)--(10.490,2.068)--(10.603,2.095)--(10.716,2.122)--(10.829,2.150)--(10.942,2.179)%
  --(11.054,2.207)--(11.167,2.236)--(11.280,2.265)--(11.393,2.295)--(11.506,2.325)--(11.619,2.355)%
  --(11.732,2.385)--(11.844,2.416)--(11.947,2.444);
\gpcolor{color=gp lt color border}
\gpsetlinetype{gp lt border}
\draw[gp path] (1.688,8.381)--(1.688,0.985)--(11.947,0.985)--(11.947,8.381)--cycle;
%% coordinates of the plot area
\gpdefrectangularnode{gp plot 1}{\pgfpoint{1.688cm}{0.985cm}}{\pgfpoint{11.947cm}{8.381cm}}
\end{tikzpicture}
%% gnuplot variables

  }
  \caption[AG simulation of free-space pulse expansion]{
    Transverse (solid) and longitudinal (dashed) dynamics of three example pulses for three length/width ratios ($\xi$). 
    The red lines represent a oblate pulse (a ``pancake'') with $\xi=1/3$, green lines for a spherical pulse with $\xi=1$, and blue for an prolate pulse (a ``cigar'') having $\xi=3$.
    The pulses shown in \ref{fig:compare_shape_N1} all have total charge $N=1$, while in \ref{fig:compare_shape_N1e6} the pulses have $N=10^6$.
  }
  \label{fig:compare_shape}
\end{figure}

The bunch propagation dynamics for negligible pulse charge density ($ N = 10^{0} $) shown in \ref{fig:compare_shape}\subref{fig:compare_shape_N1} display the trends expected from the initial photoemission conditions.
In particular, in the initially spherical case ($ \sigma_{\smallT} (0) = \sigma_{z} (0) $), the transverse bunch width $ \sqrt{2 \sigma_{\smallT}} $ expands approximately twice as rapidly as the longitudinal length $ \sqrt{2 \sigma_{z}} $ due to the factor of two difference in the initial momentum variances $ \Delta p_{\alpha} $ and the artificially zero initial pulse chirp ($ \gamma_{\alpha} = 0 $).
Naturally, for initially non-spherical electron bunches, the pulse propagation model also shows that the relative bunch broadening $ \sqrt{ \sigma_{\alpha} / \sigma_{\alpha} (0) } $ for comparable $ \eta_{\alpha} $ is greatest for the smallest initial $ \sigma_{\alpha} (0) $; that is both oblate and prolate pulses are initially driven towards the spherical regime.
In all cases, this intrinsic electron pulse broadening due to non-zero $ \eta_{\alpha} $ generates a momentum chirp across the electron bunch; in the longitudinal dimension the faster electrons photoemitted with larger excess energy leading the slower electrons.

For an electron pulse with a higher initial space-charge density, intra-pulse Coulomb effects can dominate the propagation dynamics.
\ref{fig:compare_shape}\subref{fig:compare_shape_N1e6}, shows the predicted temporal dynamics of electron pulses with the same initial size and photoemission conditions as \ref{fig:compare_shape}\subref{fig:compare_shape_N1}, but with $ N = 10^{6} $ electrons --- the number required for single-shot diffraction pattern measurements in a DTEM.\cite{berger_dc_2009,armstrong_practical_2007}
The significant influence of space-charge effects for the higher initial peak pulse charge density is clearly evident.
%TODO revisit this next sentance, the data doesn't seem to bear it out
In the spherical case, space-charge induced bunch broadening overwhelms the initial difference between $\eta_{\smallT}$ and $\eta_{z}$, producing an almost uniform three-dimensional `Coulomb explosion' of the electron pulse with a one-dimensional bunch broadening rate roughly one order of magnitude greater than that for $ N = 10^{0} $ (\ref{fig:compare_shape}\subref{fig:compare_shape_N1}).
For $ \xi (0) > 1 $ (or $ \xi (0) < 1 $), the smaller initial transverse (longitudinal) bunch dimension experiences the greatest relative expansion in free-space propagation. This is to be expected, because the initial $\eta_{\alpha}$ are non-zero (but comparable in magnitude) while Coulomb effects will produce differential pulse broadening in the dimension with the largest internal space-charge field (i.e., the dimension with the largest electrostatic potential gradient).
%TODO the following sentance appears not to be true, still perhaps should include something like it
%In fact, both the $ N = 10^{6} $ oblate and prolate pulse examples shown in \ref{fig:compare_shape}\subref{fig:compare_shape_N1e6} transition through a spherical bunch shape ($ \xi \approx 1 $) after a time of flight of $\sim$0.5ns, at which point the peak pulse charge density has already been reduced by a factor of $\sim$100.
The initial impulse induced by the strong space-charge field then provides for continued linear expansion (in time), resulting in a linear pulse chirp in the dimension associated with the smallest initial HW1/eM bunch width\cite{michalik_analytic_2006,berger_dc_2009} $ \sqrt{ 2 \sigma_{\alpha} (0) } $.

The data presented in \ref{fig:compare_shape} using the AG model of Michalik and Sipe \cite{michalik_analytic_2006} are entirely consistent with prior works.\cite{reed_femtosecond_2006,siwick_ultrafast_2002}
In femtosecond UED experiments, significant efforts are made to offset space-charge induced temporal (longitudinal, $z$) pulse broadening by reducing the time of flight from the gun photocathode to the sample.\cite{siwick_ultrafast_2002,reed_evolution_2009}
Moreover, efforts are now underway to compensate for temporal electron pulse broadening using RF cavity pulse compression techniques.\cite{oudheusden_electron_2007}
In UEM, employing a standard retro-fitted electron microscope column driven by MHz repetition-rate sub-picosecond lasers,\cite{lobastov_four-dimensional_2005} deleterious space-charge effects have obliged operation in the $\sim$1 electron/pulse regime to maintain high spatial resolution.
Even for nanosecond DTEM (the extreme prolate pulse ($\xi \gg 1 $)), transverse space-charge induced beam broadening has necessitated the insertion of an additional magnetic electron lens in a microscope column to allow more efficient delivery of electrons to the specimen when $ N > 10^{6} $.\cite{lagrange_nanosecond_2008}
In this thesis, however, I will restrict the discussion to ultrashort (ps and sub-ps) electron pulses which (i) allow for the temporal resolution of fundamental events occurring on ultrafast time scales in materials science, biology and chemistry \cite{king_ultrafast_2005} and (ii) can be generated in the oblate regime ($ \xi < 1 $) where transverse bunch broadening effects are minimized (\ref{fig:compare_shape})(enabling more efficient beam propagation) and RF cavities can be used to compensate and reverse longitudinal (temporal) pulse broadening.\cite{veisz_hybrid_2007}

