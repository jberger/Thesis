\subsection{Trends for Initial Shapes}

The initial `space-time' eccentricity \ref{eq:define_xi}, $\xi ( 0 ) = v_{{ \scriptscriptstyle 0}} \tau / w $ , of the photo-generated electron pulse plays a major role in the bunch propagation dynamics, particularly when space-charge effects dominate.
\ref{fig:compare_shape} illustrates the typical dynamics of initially spherical ($ \sigma_{ \smallT }(0) = \sigma_{z} ( 0 ) $), `disk-like' ($ \sigma_{ \smallT } ( 0 ) > \sigma_{ z } ( 0 ) $), and `cigar-like' ($ \sigma_{ z } ( 0 ) > \sigma_{ \smallT } ( 0 ) $) pulses.
%TODO rewrite using w?
Unlike most simulations in this paper, the pulses in this simulation are not ``generated'' from laser parameters, but are defined directly.
The initial bunch dimensions are given by the ellipticities $ \xi(0) = \{ 1/3 , 1 , 3 \} $ while in each case the product of the initial HW1/eM bunch widths $w_{\smallT}^2 w_z = $ 1mm$^3$.
Since the initial pulse volumn is consistent between all runs, initial peak pulse charge density is six orders of magnitude greater in \ref{fig:compare_shape}\subref{fig:compare_N106} ($ N = 10^{6} $ electrons in the bunch) than in \ref{fig:compare_shape}\subref{fig:compare_N1} ($ N = 10^{0} $).
The pulse velocity is $ v_{{ \scriptscriptstyle 0}} \approx c/3 $, i.e. a DC photogun with $V = 30\text{kV} $.

\begin{figure}
  \centering
  \begin{tikzpicture}[gnuplot]
%% generated with GNUPLOT 4.6p0 (Lua 5.1; terminal rev. 99, script rev. 100)
%% Mon 04 Feb 2013 02:19:03 PM CST
\path (0.000,0.000) rectangle (12.500,8.750);
\gpcolor{color=gp lt color border}
\gpsetlinetype{gp lt border}
\gpsetlinewidth{1.00}
\draw[gp path] (1.380,0.616)--(1.560,0.616);
\draw[gp path] (11.947,0.616)--(11.767,0.616);
\node[gp node right] at (1.196,0.616) { 0.98};
\draw[gp path] (1.380,1.587)--(1.560,1.587);
\draw[gp path] (11.947,1.587)--(11.767,1.587);
\node[gp node right] at (1.196,1.587) { 1};
\draw[gp path] (1.380,2.557)--(1.560,2.557);
\draw[gp path] (11.947,2.557)--(11.767,2.557);
\node[gp node right] at (1.196,2.557) { 1.02};
\draw[gp path] (1.380,3.528)--(1.560,3.528);
\draw[gp path] (11.947,3.528)--(11.767,3.528);
\node[gp node right] at (1.196,3.528) { 1.04};
\draw[gp path] (1.380,4.499)--(1.560,4.499);
\draw[gp path] (11.947,4.499)--(11.767,4.499);
\node[gp node right] at (1.196,4.499) { 1.06};
\draw[gp path] (1.380,5.469)--(1.560,5.469);
\draw[gp path] (11.947,5.469)--(11.767,5.469);
\node[gp node right] at (1.196,5.469) { 1.08};
\draw[gp path] (1.380,6.440)--(1.560,6.440);
\draw[gp path] (11.947,6.440)--(11.767,6.440);
\node[gp node right] at (1.196,6.440) { 1.1};
\draw[gp path] (1.380,7.410)--(1.560,7.410);
\draw[gp path] (11.947,7.410)--(11.767,7.410);
\node[gp node right] at (1.196,7.410) { 1.12};
\draw[gp path] (1.380,8.381)--(1.560,8.381);
\draw[gp path] (11.947,8.381)--(11.767,8.381);
\node[gp node right] at (1.196,8.381) { 1.14};
\draw[gp path] (1.380,0.616)--(1.380,0.796);
\draw[gp path] (1.380,8.381)--(1.380,8.201);
\node[gp node center] at (1.380,0.308) { 0};
\draw[gp path] (2.701,0.616)--(2.701,0.796);
\draw[gp path] (2.701,8.381)--(2.701,8.201);
\node[gp node center] at (2.701,0.308) { 0.02};
\draw[gp path] (4.022,0.616)--(4.022,0.796);
\draw[gp path] (4.022,8.381)--(4.022,8.201);
\node[gp node center] at (4.022,0.308) { 0.04};
\draw[gp path] (5.343,0.616)--(5.343,0.796);
\draw[gp path] (5.343,8.381)--(5.343,8.201);
\node[gp node center] at (5.343,0.308) { 0.06};
\draw[gp path] (6.664,0.616)--(6.664,0.796);
\draw[gp path] (6.664,8.381)--(6.664,8.201);
\node[gp node center] at (6.664,0.308) { 0.08};
\draw[gp path] (7.984,0.616)--(7.984,0.796);
\draw[gp path] (7.984,8.381)--(7.984,8.201);
\node[gp node center] at (7.984,0.308) { 0.1};
\draw[gp path] (9.305,0.616)--(9.305,0.796);
\draw[gp path] (9.305,8.381)--(9.305,8.201);
\node[gp node center] at (9.305,0.308) { 0.12};
\draw[gp path] (10.626,0.616)--(10.626,0.796);
\draw[gp path] (10.626,8.381)--(10.626,8.201);
\node[gp node center] at (10.626,0.308) { 0.14};
\draw[gp path] (11.947,0.616)--(11.947,0.796);
\draw[gp path] (11.947,8.381)--(11.947,8.201);
\node[gp node center] at (11.947,0.308) { 0.16};
\draw[gp path] (1.380,8.381)--(1.380,0.616)--(11.947,0.616)--(11.947,8.381)--cycle;
\node[gp node right] at (10.479,8.047) {Prolate};
\gpcolor{rgb color={1.000,0.000,0.000}}
\gpsetlinetype{gp lt plot 0}
\draw[gp path] (10.663,8.047)--(11.579,8.047);
\draw[gp path] (1.380,1.587)--(1.489,1.587)--(1.598,1.587)--(1.707,1.588)--(1.816,1.590)%
  --(1.925,1.591)--(2.034,1.593)--(2.143,1.596)--(2.252,1.599)--(2.361,1.602)--(2.470,1.605)%
  --(2.579,1.609)--(2.688,1.613)--(2.797,1.618)--(2.906,1.623)--(3.015,1.628)--(3.124,1.634)%
  --(3.233,1.640)--(3.341,1.647)--(3.450,1.654)--(3.559,1.661)--(3.668,1.669)--(3.777,1.677)%
  --(3.886,1.685)--(3.995,1.694)--(4.104,1.703)--(4.213,1.712)--(4.322,1.722)--(4.431,1.732)%
  --(4.540,1.743)--(4.649,1.754)--(4.758,1.765)--(4.867,1.777)--(4.976,1.789)--(5.085,1.801)%
  --(5.194,1.814)--(5.303,1.827)--(5.412,1.841)--(5.521,1.855)--(5.630,1.869)--(5.739,1.884)%
  --(5.848,1.899)--(5.957,1.914)--(6.066,1.930)--(6.175,1.946)--(6.284,1.962)--(6.393,1.979)%
  --(6.502,1.996)--(6.611,2.014)--(6.720,2.032)--(6.829,2.050)--(6.938,2.068)--(7.047,2.087)%
  --(7.156,2.107)--(7.264,2.126)--(7.373,2.146)--(7.482,2.167)--(7.591,2.188)--(7.700,2.209)%
  --(7.809,2.230)--(7.918,2.252)--(8.027,2.274)--(8.136,2.297)--(8.245,2.320)--(8.354,2.343)%
  --(8.463,2.367)--(8.572,2.391)--(8.681,2.415)--(8.790,2.440)--(8.899,2.465)--(9.008,2.490)%
  --(9.117,2.516)--(9.226,2.542)--(9.335,2.569)--(9.444,2.595)--(9.553,2.623)--(9.662,2.650)%
  --(9.771,2.678)--(9.880,2.706)--(9.989,2.735)--(10.098,2.764)--(10.207,2.793)--(10.316,2.822)%
  --(10.425,2.852)--(10.534,2.883)--(10.643,2.913)--(10.752,2.944)--(10.861,2.976)--(10.970,3.007)%
  --(11.079,3.039)--(11.187,3.072)--(11.296,3.104);
\gpsetlinetype{gp lt plot 1}
\draw[gp path] (1.380,1.587)--(1.489,1.587)--(1.598,1.588)--(1.707,1.590)--(1.816,1.593)%
  --(1.925,1.597)--(2.034,1.602)--(2.143,1.607)--(2.252,1.613)--(2.361,1.621)--(2.470,1.628)%
  --(2.579,1.637)--(2.688,1.647)--(2.797,1.657)--(2.906,1.669)--(3.015,1.681)--(3.124,1.694)%
  --(3.233,1.708)--(3.341,1.722)--(3.450,1.738)--(3.559,1.754)--(3.668,1.771)--(3.777,1.789)%
  --(3.886,1.808)--(3.995,1.827)--(4.104,1.848)--(4.213,1.869)--(4.322,1.891)--(4.431,1.914)%
  --(4.540,1.938)--(4.649,1.962)--(4.758,1.987)--(4.867,2.014)--(4.976,2.041)--(5.085,2.068)%
  --(5.194,2.097)--(5.303,2.126)--(5.412,2.157)--(5.521,2.188)--(5.630,2.220)--(5.739,2.252)%
  --(5.848,2.286)--(5.957,2.320)--(6.066,2.355)--(6.175,2.391)--(6.284,2.427)--(6.393,2.465)%
  --(6.502,2.503)--(6.611,2.542)--(6.720,2.582)--(6.829,2.623)--(6.938,2.664)--(7.047,2.706)%
  --(7.156,2.749)--(7.264,2.793)--(7.373,2.837)--(7.482,2.883)--(7.591,2.929)--(7.700,2.976)%
  --(7.809,3.023)--(7.918,3.072)--(8.027,3.121)--(8.136,3.171)--(8.245,3.221)--(8.354,3.273)%
  --(8.463,3.325)--(8.572,3.378)--(8.681,3.432)--(8.790,3.486)--(8.899,3.541)--(9.008,3.597)%
  --(9.117,3.654)--(9.226,3.711)--(9.335,3.769)--(9.444,3.828)--(9.553,3.888)--(9.662,3.948)%
  --(9.771,4.009)--(9.880,4.071)--(9.989,4.134)--(10.098,4.197)--(10.207,4.261)--(10.316,4.325)%
  --(10.425,4.391)--(10.534,4.457)--(10.643,4.524)--(10.752,4.591)--(10.861,4.659)--(10.970,4.728)%
  --(11.079,4.798)--(11.187,4.868)--(11.296,4.939);
\gpcolor{color=gp lt color border}
\node[gp node right] at (10.479,7.739) {Sphere};
\gpcolor{rgb color={0.000,1.000,0.000}}
\gpsetlinetype{gp lt plot 0}
\draw[gp path] (10.663,7.739)--(11.579,7.739);
\draw[gp path] (1.380,1.587)--(1.489,1.587)--(1.598,1.588)--(1.707,1.590)--(1.816,1.593)%
  --(1.925,1.596)--(2.034,1.601)--(2.143,1.606)--(2.252,1.611)--(2.361,1.618)--(2.470,1.625)%
  --(2.579,1.633)--(2.688,1.642)--(2.797,1.652)--(2.906,1.662)--(3.015,1.674)--(3.124,1.686)%
  --(3.233,1.698)--(3.341,1.712)--(3.450,1.726)--(3.559,1.741)--(3.668,1.757)--(3.777,1.774)%
  --(3.886,1.791)--(3.995,1.809)--(4.104,1.828)--(4.213,1.848)--(4.322,1.868)--(4.431,1.889)%
  --(4.540,1.911)--(4.649,1.934)--(4.758,1.957)--(4.867,1.982)--(4.976,2.006)--(5.085,2.032)%
  --(5.194,2.059)--(5.303,2.086)--(5.412,2.114)--(5.521,2.143)--(5.630,2.172)--(5.739,2.202)%
  --(5.848,2.233)--(5.957,2.265)--(6.066,2.297)--(6.175,2.331)--(6.284,2.364)--(6.393,2.399)%
  --(6.502,2.435)--(6.611,2.471)--(6.720,2.508)--(6.829,2.545)--(6.938,2.584)--(7.047,2.623)%
  --(7.156,2.662)--(7.264,2.703)--(7.373,2.744)--(7.482,2.786)--(7.591,2.829)--(7.700,2.872)%
  --(7.809,2.916)--(7.918,2.961)--(8.027,3.007)--(8.136,3.053)--(8.245,3.100)--(8.354,3.148)%
  --(8.463,3.196)--(8.572,3.245)--(8.681,3.295)--(8.790,3.345)--(8.899,3.396)--(9.008,3.448)%
  --(9.117,3.501)--(9.226,3.554)--(9.335,3.608)--(9.444,3.663)--(9.553,3.718)--(9.662,3.774)%
  --(9.771,3.831)--(9.880,3.888)--(9.989,3.946)--(10.098,4.005)--(10.207,4.064)--(10.316,4.124)%
  --(10.425,4.185)--(10.534,4.246)--(10.643,4.308)--(10.752,4.371)--(10.861,4.434)--(10.970,4.498)%
  --(11.079,4.562)--(11.187,4.628)--(11.296,4.694);
\gpsetlinetype{gp lt plot 1}
\draw[gp path] (1.380,1.587)--(1.489,1.587)--(1.598,1.587)--(1.707,1.587)--(1.816,1.588)%
  --(1.925,1.589)--(2.034,1.590)--(2.143,1.591)--(2.252,1.593)--(2.361,1.594)--(2.470,1.596)%
  --(2.579,1.598)--(2.688,1.601)--(2.797,1.603)--(2.906,1.606)--(3.015,1.608)--(3.124,1.611)%
  --(3.233,1.615)--(3.341,1.618)--(3.450,1.622)--(3.559,1.625)--(3.668,1.629)--(3.777,1.633)%
  --(3.886,1.638)--(3.995,1.642)--(4.104,1.647)--(4.213,1.652)--(4.322,1.657)--(4.431,1.662)%
  --(4.540,1.668)--(4.649,1.674)--(4.758,1.680)--(4.867,1.686)--(4.976,1.692)--(5.085,1.698)%
  --(5.194,1.705)--(5.303,1.712)--(5.412,1.719)--(5.521,1.726)--(5.630,1.734)--(5.739,1.741)%
  --(5.848,1.749)--(5.957,1.757)--(6.066,1.765)--(6.175,1.774)--(6.284,1.782)--(6.393,1.791)%
  --(6.502,1.800)--(6.611,1.809)--(6.720,1.818)--(6.829,1.828)--(6.938,1.838)--(7.047,1.848)%
  --(7.156,1.858)--(7.264,1.868)--(7.373,1.879)--(7.482,1.889)--(7.591,1.900)--(7.700,1.911)%
  --(7.809,1.922)--(7.918,1.934)--(8.027,1.945)--(8.136,1.957)--(8.245,1.969)--(8.354,1.981)%
  --(8.463,1.994)--(8.572,2.006)--(8.681,2.019)--(8.790,2.032)--(8.899,2.045)--(9.008,2.059)%
  --(9.117,2.072)--(9.226,2.086)--(9.335,2.100)--(9.444,2.114)--(9.553,2.128)--(9.662,2.142)%
  --(9.771,2.157)--(9.880,2.172)--(9.989,2.187)--(10.098,2.202)--(10.207,2.218)--(10.316,2.233)%
  --(10.425,2.249)--(10.534,2.265)--(10.643,2.281)--(10.752,2.297)--(10.861,2.314)--(10.970,2.330)%
  --(11.079,2.347)--(11.187,2.364)--(11.296,2.382);
\gpcolor{color=gp lt color border}
\node[gp node right] at (10.479,7.431) {Oblate};
\gpcolor{rgb color={0.000,0.000,1.000}}
\gpsetlinetype{gp lt plot 0}
\draw[gp path] (10.663,7.431)--(11.579,7.431);
\draw[gp path] (1.380,1.587)--(1.489,1.587)--(1.598,1.590)--(1.707,1.594)--(1.816,1.600)%
  --(1.925,1.607)--(2.034,1.616)--(2.143,1.626)--(2.252,1.638)--(2.361,1.652)--(2.470,1.667)%
  --(2.579,1.684)--(2.688,1.702)--(2.797,1.723)--(2.906,1.744)--(3.015,1.767)--(3.124,1.792)%
  --(3.233,1.819)--(3.341,1.847)--(3.450,1.876)--(3.559,1.908)--(3.668,1.941)--(3.777,1.975)%
  --(3.886,2.011)--(3.995,2.048)--(4.104,2.087)--(4.213,2.128)--(4.322,2.170)--(4.431,2.214)%
  --(4.540,2.259)--(4.649,2.306)--(4.758,2.355)--(4.867,2.404)--(4.976,2.456)--(5.085,2.509)%
  --(5.194,2.563)--(5.303,2.619)--(5.412,2.677)--(5.521,2.736)--(5.630,2.797)--(5.739,2.859)%
  --(5.848,2.922)--(5.957,2.987)--(6.066,3.054)--(6.175,3.122)--(6.284,3.191)--(6.393,3.262)%
  --(6.502,3.334)--(6.611,3.408)--(6.720,3.483)--(6.829,3.560)--(6.938,3.638)--(7.047,3.718)%
  --(7.156,3.799)--(7.264,3.881)--(7.373,3.965)--(7.482,4.050)--(7.591,4.136)--(7.700,4.224)%
  --(7.809,4.314)--(7.918,4.404)--(8.027,4.496)--(8.136,4.590)--(8.245,4.684)--(8.354,4.780)%
  --(8.463,4.878)--(8.572,4.976)--(8.681,5.076)--(8.790,5.178)--(8.899,5.280)--(9.008,5.384)%
  --(9.117,5.490)--(9.226,5.596)--(9.335,5.704)--(9.444,5.813)--(9.553,5.923)--(9.662,6.035)%
  --(9.771,6.147)--(9.880,6.261)--(9.989,6.377)--(10.098,6.493)--(10.207,6.611)--(10.316,6.729)%
  --(10.425,6.849)--(10.534,6.971)--(10.643,7.093)--(10.752,7.217)--(10.861,7.341)--(10.970,7.467)%
  --(11.079,7.594)--(11.187,7.722)--(11.296,7.852);
\gpsetlinetype{gp lt plot 1}
\draw[gp path] (1.380,1.587)--(1.489,1.587)--(1.598,1.587)--(1.707,1.587)--(1.816,1.587)%
  --(1.925,1.587)--(2.034,1.587)--(2.143,1.588)--(2.252,1.588)--(2.361,1.588)--(2.470,1.589)%
  --(2.579,1.589)--(2.688,1.590)--(2.797,1.590)--(2.906,1.591)--(3.015,1.592)--(3.124,1.592)%
  --(3.233,1.593)--(3.341,1.594)--(3.450,1.595)--(3.559,1.596)--(3.668,1.596)--(3.777,1.597)%
  --(3.886,1.598)--(3.995,1.600)--(4.104,1.601)--(4.213,1.602)--(4.322,1.603)--(4.431,1.604)%
  --(4.540,1.605)--(4.649,1.607)--(4.758,1.608)--(4.867,1.610)--(4.976,1.611)--(5.085,1.612)%
  --(5.194,1.614)--(5.303,1.616)--(5.412,1.617)--(5.521,1.619)--(5.630,1.621)--(5.739,1.622)%
  --(5.848,1.624)--(5.957,1.626)--(6.066,1.628)--(6.175,1.630)--(6.284,1.632)--(6.393,1.634)%
  --(6.502,1.636)--(6.611,1.638)--(6.720,1.640)--(6.829,1.643)--(6.938,1.645)--(7.047,1.647)%
  --(7.156,1.649)--(7.264,1.652)--(7.373,1.654)--(7.482,1.657)--(7.591,1.659)--(7.700,1.662)%
  --(7.809,1.664)--(7.918,1.667)--(8.027,1.670)--(8.136,1.673)--(8.245,1.675)--(8.354,1.678)%
  --(8.463,1.681)--(8.572,1.684)--(8.681,1.687)--(8.790,1.690)--(8.899,1.693)--(9.008,1.696)%
  --(9.117,1.699)--(9.226,1.702)--(9.335,1.706)--(9.444,1.709)--(9.553,1.712)--(9.662,1.716)%
  --(9.771,1.719)--(9.880,1.723)--(9.989,1.726)--(10.098,1.730)--(10.207,1.733)--(10.316,1.737)%
  --(10.425,1.740)--(10.534,1.744)--(10.643,1.748)--(10.752,1.752)--(10.861,1.756)--(10.970,1.760)%
  --(11.079,1.764)--(11.187,1.767)--(11.296,1.772);
\gpcolor{color=gp lt color border}
\gpsetlinetype{gp lt border}
\draw[gp path] (1.380,8.381)--(1.380,0.616)--(11.947,0.616)--(11.947,8.381)--cycle;
%% coordinates of the plot area
\gpdefrectangularnode{gp plot 1}{\pgfpoint{1.380cm}{0.616cm}}{\pgfpoint{11.947cm}{8.381cm}}
\end{tikzpicture}
%% gnuplot variables

  \caption{
    Transverse (solid) and longitudinal (dashed) dynamics of three example pulses for three length/width ratios ($\xi$). 
    The red lines represent a prolate pulse (a ``pancake'') with $\xi=1/3$, green lines for a spherical pulse with $\xi=1$, and blue for oblate (a ``cigar'') having $\xi=3$.
  }
  \label{fig:compare_shape}
\end{figure}

%\begin{figure}[t]
%\centering
%\subfloat[][]
%{
%
%\label{fig:compare_N1}
%\includegraphics{compare_N1}
%}
%\\
%\subfloat[][]
%{
%
%\label{fig:compare_N106}
%\includegraphics{compare_N106}
%}
%\caption
%{
%Comparison of the propagation dynamics of disk-like ($ \xi (0) = 0.1 $), spherical ($ \xi (0) = 1 $) and cigar-like ($ \xi (0) = 10 $) pulses, generated from a 261nm laser driven Ta photocathode in a 20kV DC electron gun, at \subref{fig:compare_N1} negligible charge densities ($ N = 10^{0} $) and \subref{fig:compare_N106} charge densities suitable for single shot UED studies ($ N = 10^{6} $): the normalized transverse beam width ($ \sqrt{ \sigma_{\smallT} / \sigma_{\smallT} ( 0 ) } $) (solid line) and normalized longitudinal pulse width ($ \sqrt{ \sigma_{z} / \sigma_{z} ( 0 ) } $) (dashed line).
%}
%\label{fig:compare_shape}
%\end{figure}

%The bunch propagation dynamics for negligible pulse charge density ($ N = 10^{0} $) shown in \Fig \ref{fig:compare_shape}\subref{fig:compare_N1} display the trends expected from the initial photoemission conditions. In particular, in the initially spherical case ($ \sigma_{\smallT} (0) = \sigma_{z} (0) $), the transverse bunch width $ \sqrt{2 \sigma_{\smallT}} $ expands twice as rapidly as the longitudinal length $ \sqrt{2 \sigma_{z}} $ due to the factor of two difference in the initial momentum variances $ \Delta p_{i} $ and the zero initial pulse chirp ($ \gamma_{i} = 0 $). Naturally, for initially non-spherical electron bunches, the pulse propagation model also shows that the relative bunch broadening $ \sqrt{ \sigma_{i} / \sigma_{i} (0) } $ for comparable $ \eta_{i} $ is greatest for the smallest initial $ \sigma_{i} (0) $; that is both disk- and cigar-like pulses are initially driven towards the spherical regime. In all cases, this intrinsic electron pulse broadening due to non-zero $ \eta_{i} $ generates a momentum chirp across the electron bunch; in the longitudinal dimension the faster electrons photoemitted with larger excess energy leading the slower electrons.

%For an electron pulse with a higher initial space-charge density, intra-pulse Coulomb effects can dominate the propagation dynamics. In \Fig \ref{fig:compare_shape}\subref{fig:compare_N106}, we display the predicted temporal dynamics of electron pulses with the same initial size and photoemission conditions as \Fig \ref{fig:compare_shape}\subref{fig:compare_N1}, but with $ N = 10^{6} $ electrons --- the number required for single-shot diffraction pattern measurements in a DTEM.\cite{berger_dc_2009,armstrong_practical_2007}
%Above is probably true for UED but never done
%The significant influence of space-charge effects for the initial peak pulse charge density of $0.0287\text{C}/\text{m}^{3}$ is clearly evident. In the spherical case, space-charge induced bunch broadening overwhelms the initial difference between $\eta_{\smallT}$ and $\eta_{z}$, producing an almost uniform three-dimensional `Coulomb explosion' of the electron pulse with a one-dimensional bunch broadening rate roughly one order of magnitude greater than that for $ N = 10^{0} $ (\Fig \ref{fig:compare_shape}\subref{fig:compare_N1}). For $ \xi (0) > 1 $ (or $ \xi (0) < 1 $), the smaller initial transverse (longitudinal) bunch dimension experiences the greatest relative expansion in free-space propagation. This is to be expected, because the initial $\eta_{i}$ are non-zero (but comparable in magnitude) while Coulomb effects will produce differential pulse broadening in the dimension with the largest internal space-charge field (i.e., the dimension with the largest electrostatic potential gradient). In fact, both the $ N = 10^{6} $ disk- and cigar-like pulse examples shown in \Fig \ref{fig:compare_shape}\subref{fig:compare_N106} transition through a spherical bunch shape ($ \xi \approx 1 $) after a time of flight of $\sim$0.5ns, at which point the peak pulse charge density has already been reduced by a factor of $\sim$100. The initial impulse induced by the strong space-charge field then provides for continued linear expansion (in time), resulting in a linear pulse chirp in the dimension associated with the smallest initial HW1/eM bunch width\cite{michalik_analytic_2006,berger_dc_2009} $ \sqrt{ 2 \sigma_{i} (0) } $.

%The data presented in \Fig \ref{fig:compare_shape} using the AG model of Michalik and Sipe \cite{michalik_analytic_2006} are entirely consistent with prior works.\cite{reed_femtosecond_2006,siwick_ultrafast_2002} In femtosecond UED experiments, significant efforts are made to offset space-charge induced temporal (longitudinal, $z$) pulse broadening by reducing the time of flight from the gun photocathode to the sample.\cite{siwick_ultrafast_2002,reed_evolution_2009} Moreover, efforts are now underway to compensate for temporal electron pulse broadening using RF cavity pulse compression techniques.\cite{oudheusden_electron_2007} In UEM, employing a standard retro-fitted electron microscope column driven by MHz repetition-rate sub-picosecond lasers,\cite{lobastov_four-dimensional_2005} deleterious space-charge effects have obliged operation in the $\sim$1 electron/pulse regime to maintain high spatial resolution. Even for nanosecond DTEM (the extreme cigar-like pulse ($\xi \gg 1 $)), transverse space-charge induced beam broadening has necessitated the insertion of an additional magnetic electron lens in a microscope column to allow more efficient delivery of electrons to the specimen when $ N > 10^{6} $.\cite{lagrange_nanosecond_2008} In this paper, however, we will restrict the discussion to ultrashort (ps and sub-ps) electron pulses which (i) allow for the temporal resolution of fundamental events occurring on ultrafast time scales in materials science, biology and chemistry \cite{king_ultrafast_2005} and (ii) can be generated in the disk-like regime ($ \xi < 1 $) where transverse bunch broadening effects are minimized (\Fig \ref{fig:compare_shape})(enabling more efficient beam propagation) and RF cavities can be used to compensate and reverse longitudinal (temporal) pulse broadening.\cite{veisz_hybrid_2007}

