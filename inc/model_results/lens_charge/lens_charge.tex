%This work is licensed under the Creative Commons Attribution-NonCommercial-NoDerivs 3.0 United States License. To view a copy of this license, visit http://creativecommons.org/licenses/by-nc-nd/3.0/us/ or send a letter to Creative Commons, 444 Castro Street, Suite 900, Mountain View, California, 94041, USA.

\subsection{Space-charge Effects in Magnetic Lenses} \label{sec:mag_lens_charge}

In the transverse spatial dimension, space-charge effects that increase $ \sigma_{ \smallT } $ can be compensated for by using magnetic electron lenses.\cite{oudheusden_electron_2007,lagrange_nanosecond_2008}
In an UEM, of course, the electron pulse will also need to be focused onto the specimen --- the focal spot size, post specimen image magnification, and CCD detector pixelation determining the spatial resolution in a TEM geometry.\cite{berger_dc_2009}
The extent to which space-charge effects limit the focal fidelity is expected to be dependent upon both $N$ and the pulse shape (i.e., the pulse's space-time eccentricity).

In the preceding sections it has already been established that oblate eccentricities are preferred for UEM in general.
When focusing the pulse it can be shown that oblate pulses will also have markedly better performance.

\begin{figure}
  \centering
  \centerline{
    \subfloat[][] {
      \label{fig:focus_long_oblate}
      \begin{tikzpicture}
        \input{lens_60mm_oblate_data}
        \node at (2.7,4.9) [align=left] {$\xi=0.1$\\$f=$ 60mm};
        \node at (5.9,1.65) {$N=10^5$};
        \node at (4.5,2.7) {$N=10^6$};
        \node at (4,3.9) {$N=10^7$};
      \end{tikzpicture}
    }
    \subfloat[][] {
      \label{fig:focus_long_prolate}
      \begin{tikzpicture}
        %% \begin{tikzpicture}[gnuplot]
%% generated with GNUPLOT 4.6p0 (Lua 5.1; terminal rev. 99, script rev. 100)
%% Sat 06 Apr 2013 10:51:01 AM CDT
\gpsolidlines
\path (0.000,0.000) rectangle (8.750,6.125);
\gpcolor{color=gp lt color border}
\gpsetlinetype{gp lt border}
\gpsetlinewidth{1.00}
\draw[gp path] (1.504,0.985)--(1.684,0.985);
\draw[gp path] (8.197,0.985)--(8.017,0.985);
\node[gp node right] at (1.320,0.985) { 0.1};
\draw[gp path] (1.504,2.421)--(1.594,2.421);
\draw[gp path] (8.197,2.421)--(8.107,2.421);
\draw[gp path] (1.504,3.261)--(1.594,3.261);
\draw[gp path] (8.197,3.261)--(8.107,3.261);
\draw[gp path] (1.504,3.857)--(1.594,3.857);
\draw[gp path] (8.197,3.857)--(8.107,3.857);
\draw[gp path] (1.504,4.320)--(1.594,4.320);
\draw[gp path] (8.197,4.320)--(8.107,4.320);
\draw[gp path] (1.504,4.698)--(1.594,4.698);
\draw[gp path] (8.197,4.698)--(8.107,4.698);
\draw[gp path] (1.504,5.017)--(1.594,5.017);
\draw[gp path] (8.197,5.017)--(8.107,5.017);
\draw[gp path] (1.504,5.294)--(1.594,5.294);
\draw[gp path] (8.197,5.294)--(8.107,5.294);
\draw[gp path] (1.504,5.538)--(1.594,5.538);
\draw[gp path] (8.197,5.538)--(8.107,5.538);
\draw[gp path] (1.504,5.756)--(1.684,5.756);
\draw[gp path] (8.197,5.756)--(8.017,5.756);
\node[gp node right] at (1.320,5.756) { 1};
\draw[gp path] (1.504,0.985)--(1.504,1.165);
\draw[gp path] (1.504,5.756)--(1.504,5.576);
\node[gp node center] at (1.504,0.677) {-1};
\draw[gp path] (2.248,0.985)--(2.248,1.165);
\draw[gp path] (2.248,5.756)--(2.248,5.576);
\node[gp node center] at (2.248,0.677) {-0.5};
\draw[gp path] (2.991,0.985)--(2.991,1.165);
\draw[gp path] (2.991,5.756)--(2.991,5.576);
\node[gp node center] at (2.991,0.677) { 0};
\draw[gp path] (3.735,0.985)--(3.735,1.165);
\draw[gp path] (3.735,5.756)--(3.735,5.576);
\node[gp node center] at (3.735,0.677) { 0.5};
\draw[gp path] (4.479,0.985)--(4.479,1.165);
\draw[gp path] (4.479,5.756)--(4.479,5.576);
\node[gp node center] at (4.479,0.677) { 1};
\draw[gp path] (5.222,0.985)--(5.222,1.165);
\draw[gp path] (5.222,5.756)--(5.222,5.576);
\node[gp node center] at (5.222,0.677) { 1.5};
\draw[gp path] (5.966,0.985)--(5.966,1.165);
\draw[gp path] (5.966,5.756)--(5.966,5.576);
\node[gp node center] at (5.966,0.677) { 2};
\draw[gp path] (6.710,0.985)--(6.710,1.165);
\draw[gp path] (6.710,5.756)--(6.710,5.576);
\node[gp node center] at (6.710,0.677) { 2.5};
\draw[gp path] (7.453,0.985)--(7.453,1.165);
\draw[gp path] (7.453,5.756)--(7.453,5.576);
\node[gp node center] at (7.453,0.677) { 3};
\draw[gp path] (8.197,0.985)--(8.197,1.165);
\draw[gp path] (8.197,5.756)--(8.197,5.576);
\node[gp node center] at (8.197,0.677) { 3.5};
\draw[gp path] (1.504,5.756)--(1.504,0.985)--(8.197,0.985)--(8.197,5.756)--cycle;
\node[gp node center,rotate=-270] at (0.246,3.370) {HW1/eM beam width (mm)};
\node[gp node center] at (4.850,0.215) {Pulse location ($z^{\prime}/f)$};
\gpcolor{color=gp lt color 0}
\gpsetlinetype{gp lt plot 0}
\draw[gp path] (1.504,1.139)--(1.537,1.140)--(1.569,1.142)--(1.602,1.147)--(1.635,1.153)%
  --(1.668,1.161)--(1.700,1.171)--(1.733,1.183)--(1.766,1.196)--(1.798,1.210)--(1.831,1.226)%
  --(1.864,1.244)--(1.897,1.263)--(1.929,1.283)--(1.962,1.304)--(1.995,1.326)--(2.028,1.350)%
  --(2.060,1.374)--(2.093,1.399)--(2.126,1.425)--(2.158,1.452)--(2.191,1.479)--(2.224,1.507)%
  --(2.257,1.536)--(2.289,1.565)--(2.322,1.594)--(2.355,1.624)--(2.387,1.653)--(2.420,1.684)%
  --(2.453,1.714)--(2.486,1.745)--(2.518,1.775)--(2.551,1.806)--(2.584,1.837)--(2.617,1.867)%
  --(2.649,1.898)--(2.682,1.929)--(2.715,1.960)--(2.747,1.990)--(2.780,2.021)--(2.813,2.051)%
  --(2.846,2.081)--(2.878,2.111)--(2.911,2.141)--(2.944,2.171)--(2.976,2.195)--(3.009,2.192)%
  --(3.042,2.164)--(3.075,2.131)--(3.107,2.099)--(3.140,2.065)--(3.173,2.032)--(3.206,1.998)%
  --(3.238,1.964)--(3.271,1.930)--(3.304,1.896)--(3.336,1.862)--(3.369,1.827)--(3.402,1.793)%
  --(3.435,1.758)--(3.467,1.724)--(3.500,1.689)--(3.533,1.654)--(3.565,1.620)--(3.598,1.586)%
  --(3.631,1.551)--(3.664,1.518)--(3.696,1.484)--(3.729,1.451)--(3.762,1.418)--(3.794,1.386)%
  --(3.827,1.355)--(3.860,1.324)--(3.893,1.294)--(3.925,1.265)--(3.958,1.238)--(3.991,1.211)%
  --(4.024,1.185)--(4.056,1.161)--(4.089,1.139)--(4.122,1.118)--(4.154,1.099)--(4.187,1.081)%
  --(4.220,1.066)--(4.253,1.052)--(4.285,1.041)--(4.318,1.031)--(4.351,1.024)--(4.383,1.019)%
  --(4.416,1.017)--(4.449,1.017)--(4.482,1.019)--(4.514,1.023)--(4.547,1.030)--(4.580,1.039)%
  --(4.613,1.050)--(4.645,1.063)--(4.678,1.078)--(4.711,1.095)--(4.743,1.114)--(4.776,1.135)%
  --(4.809,1.157)--(4.842,1.181)--(4.874,1.206)--(4.907,1.233)--(4.940,1.260)--(4.972,1.289)%
  --(5.005,1.319)--(5.038,1.349)--(5.071,1.380)--(5.103,1.412)--(5.136,1.445)--(5.169,1.478)%
  --(5.202,1.511)--(5.234,1.545)--(5.267,1.579)--(5.300,1.614)--(5.332,1.648)--(5.365,1.683)%
  --(5.398,1.717)--(5.431,1.752)--(5.463,1.786)--(5.496,1.821)--(5.529,1.855)--(5.561,1.890)%
  --(5.594,1.924)--(5.627,1.958)--(5.660,1.992)--(5.692,2.026)--(5.725,2.059)--(5.758,2.092)%
  --(5.790,2.125)--(5.823,2.158)--(5.856,2.191)--(5.889,2.223)--(5.921,2.255)--(5.954,2.287)%
  --(5.987,2.318)--(6.020,2.349)--(6.052,2.380)--(6.085,2.410)--(6.118,2.441)--(6.150,2.471)%
  --(6.183,2.500)--(6.216,2.530)--(6.249,2.559)--(6.281,2.588)--(6.314,2.616)--(6.347,2.644)%
  --(6.379,2.672)--(6.412,2.700)--(6.445,2.728)--(6.478,2.755)--(6.510,2.782)--(6.543,2.808)%
  --(6.576,2.834)--(6.609,2.861)--(6.641,2.886)--(6.674,2.912)--(6.707,2.937)--(6.739,2.962)%
  --(6.772,2.987)--(6.805,3.012)--(6.838,3.036)--(6.870,3.060)--(6.903,3.084)--(6.936,3.108)%
  --(6.968,3.131)--(7.001,3.154)--(7.034,3.177)--(7.067,3.200)--(7.099,3.222)--(7.132,3.245)%
  --(7.165,3.267)--(7.198,3.289)--(7.230,3.310)--(7.263,3.332)--(7.296,3.353)--(7.328,3.374)%
  --(7.361,3.395)--(7.394,3.416)--(7.427,3.437)--(7.459,3.457)--(7.492,3.477)--(7.525,3.497)%
  --(7.557,3.517)--(7.590,3.537)--(7.623,3.556)--(7.656,3.576)--(7.688,3.595)--(7.721,3.614)%
  --(7.754,3.633)--(7.786,3.651)--(7.819,3.670)--(7.852,3.688)--(7.885,3.707)--(7.917,3.725)%
  --(7.950,3.743)--(7.983,3.761)--(8.016,3.778)--(8.048,3.796);
\gpcolor{color=gp lt color 1}
\gpsetlinetype{gp lt plot 1}
\draw[gp path] (1.504,1.139)--(1.537,1.140)--(1.569,1.142)--(1.602,1.147)--(1.635,1.153)%
  --(1.668,1.162)--(1.700,1.172)--(1.733,1.183)--(1.766,1.197)--(1.798,1.211)--(1.831,1.228)%
  --(1.864,1.245)--(1.897,1.265)--(1.929,1.285)--(1.962,1.306)--(1.995,1.329)--(2.028,1.353)%
  --(2.060,1.378)--(2.093,1.403)--(2.126,1.429)--(2.158,1.456)--(2.191,1.484)--(2.224,1.512)%
  --(2.257,1.541)--(2.289,1.570)--(2.322,1.600)--(2.355,1.630)--(2.387,1.660)--(2.420,1.691)%
  --(2.453,1.722)--(2.486,1.752)--(2.518,1.783)--(2.551,1.814)--(2.584,1.846)--(2.617,1.877)%
  --(2.649,1.908)--(2.682,1.939)--(2.715,1.970)--(2.747,2.001)--(2.780,2.032)--(2.813,2.062)%
  --(2.846,2.093)--(2.878,2.123)--(2.911,2.154)--(2.944,2.183)--(2.976,2.207)--(3.009,2.205)%
  --(3.042,2.177)--(3.075,2.145)--(3.107,2.112)--(3.140,2.080)--(3.173,2.046)--(3.206,2.013)%
  --(3.238,1.980)--(3.271,1.946)--(3.304,1.912)--(3.336,1.878)--(3.369,1.844)--(3.402,1.810)%
  --(3.435,1.776)--(3.467,1.742)--(3.500,1.708)--(3.533,1.674)--(3.565,1.640)--(3.598,1.606)%
  --(3.631,1.572)--(3.664,1.539)--(3.696,1.506)--(3.729,1.473)--(3.762,1.441)--(3.794,1.410)%
  --(3.827,1.379)--(3.860,1.349)--(3.893,1.319)--(3.925,1.291)--(3.958,1.263)--(3.991,1.237)%
  --(4.024,1.212)--(4.056,1.188)--(4.089,1.165)--(4.122,1.145)--(4.154,1.125)--(4.187,1.108)%
  --(4.220,1.092)--(4.253,1.079)--(4.285,1.067)--(4.318,1.057)--(4.351,1.050)--(4.383,1.045)%
  --(4.416,1.041)--(4.449,1.041)--(4.482,1.042)--(4.514,1.046)--(4.547,1.051)--(4.580,1.059)%
  --(4.613,1.070)--(4.645,1.082)--(4.678,1.096)--(4.711,1.112)--(4.743,1.130)--(4.776,1.150)%
  --(4.809,1.171)--(4.842,1.193)--(4.874,1.218)--(4.907,1.243)--(4.940,1.270)--(4.972,1.297)%
  --(5.005,1.326)--(5.038,1.356)--(5.071,1.386)--(5.103,1.417)--(5.136,1.449)--(5.169,1.481)%
  --(5.202,1.514)--(5.234,1.547)--(5.267,1.580)--(5.300,1.614)--(5.332,1.648)--(5.365,1.682)%
  --(5.398,1.716)--(5.431,1.750)--(5.463,1.784)--(5.496,1.818)--(5.529,1.852)--(5.561,1.886)%
  --(5.594,1.920)--(5.627,1.954)--(5.660,1.988)--(5.692,2.021)--(5.725,2.054)--(5.758,2.087)%
  --(5.790,2.120)--(5.823,2.153)--(5.856,2.185)--(5.889,2.217)--(5.921,2.249)--(5.954,2.281)%
  --(5.987,2.312)--(6.020,2.343)--(6.052,2.374)--(6.085,2.404)--(6.118,2.434)--(6.150,2.464)%
  --(6.183,2.494)--(6.216,2.523)--(6.249,2.553)--(6.281,2.581)--(6.314,2.610)--(6.347,2.638)%
  --(6.379,2.666)--(6.412,2.694)--(6.445,2.721)--(6.478,2.748)--(6.510,2.775)--(6.543,2.802)%
  --(6.576,2.828)--(6.609,2.855)--(6.641,2.880)--(6.674,2.906)--(6.707,2.931)--(6.739,2.957)%
  --(6.772,2.981)--(6.805,3.006)--(6.838,3.031)--(6.870,3.055)--(6.903,3.079)--(6.936,3.102)%
  --(6.968,3.126)--(7.001,3.149)--(7.034,3.172)--(7.067,3.195)--(7.099,3.218)--(7.132,3.240)%
  --(7.165,3.262)--(7.198,3.284)--(7.230,3.306)--(7.263,3.328)--(7.296,3.349)--(7.328,3.370)%
  --(7.361,3.392)--(7.394,3.412)--(7.427,3.433)--(7.459,3.454)--(7.492,3.474)--(7.525,3.494)%
  --(7.557,3.514)--(7.590,3.534)--(7.623,3.553)--(7.656,3.573)--(7.688,3.592)--(7.721,3.611)%
  --(7.754,3.630)--(7.786,3.649)--(7.819,3.668)--(7.852,3.686)--(7.885,3.705)--(7.917,3.723)%
  --(7.950,3.741)--(7.983,3.759)--(8.016,3.777)--(8.048,3.795);
\gpcolor{color=gp lt color 2}
\gpsetlinetype{gp lt plot 2}
\draw[gp path] (1.504,1.139)--(1.537,1.140)--(1.569,1.143)--(1.602,1.148)--(1.635,1.155)%
  --(1.668,1.165)--(1.700,1.176)--(1.733,1.189)--(1.766,1.204)--(1.798,1.221)--(1.831,1.240)%
  --(1.864,1.260)--(1.897,1.281)--(1.929,1.304)--(1.962,1.328)--(1.995,1.354)--(2.028,1.380)%
  --(2.060,1.408)--(2.093,1.437)--(2.126,1.466)--(2.158,1.496)--(2.191,1.527)--(2.224,1.558)%
  --(2.257,1.590)--(2.289,1.622)--(2.322,1.655)--(2.355,1.688)--(2.387,1.721)--(2.420,1.755)%
  --(2.453,1.789)--(2.486,1.823)--(2.518,1.856)--(2.551,1.890)--(2.584,1.924)--(2.617,1.958)%
  --(2.649,1.992)--(2.682,2.026)--(2.715,2.060)--(2.747,2.093)--(2.780,2.127)--(2.813,2.160)%
  --(2.846,2.193)--(2.878,2.226)--(2.911,2.259)--(2.944,2.291)--(2.976,2.317)--(3.009,2.317)%
  --(3.042,2.292)--(3.075,2.262)--(3.107,2.232)--(3.140,2.202)--(3.173,2.172)--(3.206,2.142)%
  --(3.238,2.112)--(3.271,2.081)--(3.304,2.051)--(3.336,2.020)--(3.369,1.990)--(3.402,1.959)%
  --(3.435,1.929)--(3.467,1.899)--(3.500,1.869)--(3.533,1.839)--(3.565,1.809)--(3.598,1.779)%
  --(3.631,1.750)--(3.664,1.721)--(3.696,1.692)--(3.729,1.664)--(3.762,1.636)--(3.794,1.609)%
  --(3.827,1.582)--(3.860,1.556)--(3.893,1.530)--(3.925,1.505)--(3.958,1.481)--(3.991,1.458)%
  --(4.024,1.436)--(4.056,1.415)--(4.089,1.395)--(4.122,1.376)--(4.154,1.359)--(4.187,1.342)%
  --(4.220,1.327)--(4.253,1.314)--(4.285,1.302)--(4.318,1.291)--(4.351,1.282)--(4.383,1.274)%
  --(4.416,1.269)--(4.449,1.264)--(4.482,1.262)--(4.514,1.261)--(4.547,1.262)--(4.580,1.264)%
  --(4.613,1.269)--(4.645,1.274)--(4.678,1.282)--(4.711,1.291)--(4.743,1.302)--(4.776,1.314)%
  --(4.809,1.327)--(4.842,1.342)--(4.874,1.359)--(4.907,1.376)--(4.940,1.395)--(4.972,1.415)%
  --(5.005,1.436)--(5.038,1.458)--(5.071,1.481)--(5.103,1.505)--(5.136,1.530)--(5.169,1.556)%
  --(5.202,1.582)--(5.234,1.608)--(5.267,1.636)--(5.300,1.664)--(5.332,1.692)--(5.365,1.720)%
  --(5.398,1.749)--(5.431,1.779)--(5.463,1.808)--(5.496,1.838)--(5.529,1.868)--(5.561,1.898)%
  --(5.594,1.928)--(5.627,1.959)--(5.660,1.989)--(5.692,2.019)--(5.725,2.050)--(5.758,2.080)%
  --(5.790,2.110)--(5.823,2.141)--(5.856,2.171)--(5.889,2.201)--(5.921,2.231)--(5.954,2.261)%
  --(5.987,2.291)--(6.020,2.320)--(6.052,2.350)--(6.085,2.379)--(6.118,2.408)--(6.150,2.437)%
  --(6.183,2.466)--(6.216,2.494)--(6.249,2.523)--(6.281,2.551)--(6.314,2.579)--(6.347,2.607)%
  --(6.379,2.634)--(6.412,2.662)--(6.445,2.689)--(6.478,2.716)--(6.510,2.743)--(6.543,2.769)%
  --(6.576,2.796)--(6.609,2.822)--(6.641,2.848)--(6.674,2.874)--(6.707,2.899)--(6.739,2.925)%
  --(6.772,2.950)--(6.805,2.975)--(6.838,2.999)--(6.870,3.024)--(6.903,3.048)--(6.936,3.072)%
  --(6.968,3.096)--(7.001,3.120)--(7.034,3.144)--(7.067,3.167)--(7.099,3.190)--(7.132,3.213)%
  --(7.165,3.236)--(7.198,3.259)--(7.230,3.281)--(7.263,3.303)--(7.296,3.325)--(7.328,3.347)%
  --(7.361,3.369)--(7.394,3.390)--(7.427,3.412)--(7.459,3.433)--(7.492,3.454)--(7.525,3.475)%
  --(7.557,3.496)--(7.590,3.516)--(7.623,3.537)--(7.656,3.557)--(7.688,3.577)--(7.721,3.597)%
  --(7.754,3.617)--(7.786,3.636)--(7.819,3.656)--(7.852,3.675)--(7.885,3.694)--(7.917,3.713)%
  --(7.950,3.732)--(7.983,3.751)--(8.016,3.769)--(8.048,3.788);
\gpcolor{color=gp lt color 3}
\gpsetlinetype{gp lt plot 3}
\draw[gp path] (1.504,1.139)--(1.537,1.141)--(1.569,1.148)--(1.602,1.160)--(1.635,1.176)%
  --(1.668,1.196)--(1.700,1.220)--(1.733,1.249)--(1.766,1.281)--(1.798,1.316)--(1.831,1.354)%
  --(1.864,1.395)--(1.897,1.439)--(1.929,1.484)--(1.962,1.532)--(1.995,1.581)--(2.028,1.632)%
  --(2.060,1.684)--(2.093,1.737)--(2.126,1.790)--(2.158,1.845)--(2.191,1.899)--(2.224,1.954)%
  --(2.257,2.009)--(2.289,2.065)--(2.322,2.120)--(2.355,2.175)--(2.387,2.230)--(2.420,2.284)%
  --(2.453,2.339)--(2.486,2.393)--(2.518,2.446)--(2.551,2.499)--(2.584,2.552)--(2.617,2.604)%
  --(2.649,2.655)--(2.682,2.706)--(2.715,2.756)--(2.747,2.806)--(2.780,2.855)--(2.813,2.904)%
  --(2.846,2.952)--(2.878,3.000)--(2.911,3.046)--(2.944,3.092)--(2.976,3.132)--(3.009,3.145)%
  --(3.042,3.134)--(3.075,3.118)--(3.107,3.102)--(3.140,3.087)--(3.173,3.072)--(3.206,3.057)%
  --(3.238,3.043)--(3.271,3.029)--(3.304,3.016)--(3.336,3.002)--(3.369,2.990)--(3.402,2.977)%
  --(3.435,2.965)--(3.467,2.953)--(3.500,2.942)--(3.533,2.932)--(3.565,2.921)--(3.598,2.911)%
  --(3.631,2.902)--(3.664,2.893)--(3.696,2.885)--(3.729,2.877)--(3.762,2.869)--(3.794,2.862)%
  --(3.827,2.856)--(3.860,2.850)--(3.893,2.845)--(3.925,2.840)--(3.958,2.835)--(3.991,2.832)%
  --(4.024,2.828)--(4.056,2.826)--(4.089,2.824)--(4.122,2.822)--(4.154,2.821)--(4.187,2.820)%
  --(4.220,2.820)--(4.253,2.821)--(4.285,2.822)--(4.318,2.824)--(4.351,2.826)--(4.383,2.828)%
  --(4.416,2.832)--(4.449,2.835)--(4.482,2.840)--(4.514,2.844)--(4.547,2.850)--(4.580,2.855)%
  --(4.613,2.862)--(4.645,2.868)--(4.678,2.876)--(4.711,2.883)--(4.743,2.892)--(4.776,2.900)%
  --(4.809,2.909)--(4.842,2.919)--(4.874,2.928)--(4.907,2.939)--(4.940,2.949)--(4.972,2.960)%
  --(5.005,2.972)--(5.038,2.984)--(5.071,2.996)--(5.103,3.008)--(5.136,3.021)--(5.169,3.034)%
  --(5.202,3.048)--(5.234,3.062)--(5.267,3.076)--(5.300,3.090)--(5.332,3.105)--(5.365,3.119)%
  --(5.398,3.135)--(5.431,3.150)--(5.463,3.166)--(5.496,3.181)--(5.529,3.197)--(5.561,3.213)%
  --(5.594,3.230)--(5.627,3.246)--(5.660,3.263)--(5.692,3.280)--(5.725,3.297)--(5.758,3.314)%
  --(5.790,3.332)--(5.823,3.349)--(5.856,3.367)--(5.889,3.384)--(5.921,3.402)--(5.954,3.420)%
  --(5.987,3.438)--(6.020,3.456)--(6.052,3.474)--(6.085,3.492)--(6.118,3.510)--(6.150,3.529)%
  --(6.183,3.547)--(6.216,3.565)--(6.249,3.584)--(6.281,3.602)--(6.314,3.621)--(6.347,3.639)%
  --(6.379,3.658)--(6.412,3.676)--(6.445,3.695)--(6.478,3.713)--(6.510,3.732)--(6.543,3.750)%
  --(6.576,3.769)--(6.609,3.787)--(6.641,3.806)--(6.674,3.824)--(6.707,3.843)--(6.739,3.861)%
  --(6.772,3.880)--(6.805,3.898)--(6.838,3.917)--(6.870,3.935)--(6.903,3.953)--(6.936,3.972)%
  --(6.968,3.990)--(7.001,4.008)--(7.034,4.026)--(7.067,4.044)--(7.099,4.062)--(7.132,4.080)%
  --(7.165,4.098)--(7.198,4.116)--(7.230,4.134)--(7.263,4.152)--(7.296,4.170)--(7.328,4.188)%
  --(7.361,4.205)--(7.394,4.223)--(7.427,4.240)--(7.459,4.258)--(7.492,4.275)--(7.525,4.293)%
  --(7.557,4.310)--(7.590,4.327)--(7.623,4.344)--(7.656,4.361)--(7.688,4.378)--(7.721,4.395)%
  --(7.754,4.412)--(7.786,4.429)--(7.819,4.446)--(7.852,4.463)--(7.885,4.479)--(7.917,4.496)%
  --(7.950,4.512)--(7.983,4.529)--(8.016,4.545)--(8.048,4.561);
\gpcolor{color=gp lt color border}
\gpsetlinetype{gp lt border}
\draw[gp path] (1.504,5.756)--(1.504,0.985)--(8.197,0.985)--(8.197,5.756)--cycle;
%% coordinates of the plot area
\gpdefrectangularnode{gp plot 1}{\pgfpoint{1.504cm}{0.985cm}}{\pgfpoint{8.197cm}{5.756cm}}
%% \end{tikzpicture}
%% gnuplot variables

        \node at (2.7,4.9) [align=left] {$\xi=10$\\$f=$ 60mm};
        \node at (4.2,2.3) {$N=10^5$};
        \node at (3.3,3.5) {$N=10^6$};
      \end{tikzpicture}
    }
  }
  \centerline{
    \subfloat[][] { 
      \label{fig:focus_short_oblate}
      \begin{tikzpicture}
        \input{lens_6mm_oblate_data}
        \node at (2.7,4.9) [align=left] {$\xi=0.1$\\$f=$ 6mm};
        \node at (6.5,3) {$N=10^7$};
      \end{tikzpicture}
    }
    \subfloat[][] { 
      \label{fig:focus_short_prolate}
      \begin{tikzpicture}
        %% \begin{tikzpicture}[gnuplot]
%% generated with GNUPLOT 4.6p0 (Lua 5.1; terminal rev. 99, script rev. 100)
%% Tue 26 Mar 2013 10:54:33 AM CDT
\gpsolidlines
\path (0.000,0.000) rectangle (8.750,6.125);
\gpcolor{color=gp lt color border}
\gpsetlinetype{gp lt border}
\gpsetlinewidth{1.00}
\draw[gp path] (1.688,0.985)--(1.868,0.985);
\draw[gp path] (8.197,0.985)--(8.017,0.985);
\node[gp node right] at (1.504,0.985) { 0.01};
\draw[gp path] (1.688,1.464)--(1.778,1.464);
\draw[gp path] (8.197,1.464)--(8.107,1.464);
\draw[gp path] (1.688,1.744)--(1.778,1.744);
\draw[gp path] (8.197,1.744)--(8.107,1.744);
\draw[gp path] (1.688,1.942)--(1.778,1.942);
\draw[gp path] (8.197,1.942)--(8.107,1.942);
\draw[gp path] (1.688,2.097)--(1.778,2.097);
\draw[gp path] (8.197,2.097)--(8.107,2.097);
\draw[gp path] (1.688,2.223)--(1.778,2.223);
\draw[gp path] (8.197,2.223)--(8.107,2.223);
\draw[gp path] (1.688,2.329)--(1.778,2.329);
\draw[gp path] (8.197,2.329)--(8.107,2.329);
\draw[gp path] (1.688,2.421)--(1.778,2.421);
\draw[gp path] (8.197,2.421)--(8.107,2.421);
\draw[gp path] (1.688,2.503)--(1.778,2.503);
\draw[gp path] (8.197,2.503)--(8.107,2.503);
\draw[gp path] (1.688,2.575)--(1.868,2.575);
\draw[gp path] (8.197,2.575)--(8.017,2.575);
\node[gp node right] at (1.504,2.575) { 0.1};
\draw[gp path] (1.688,3.054)--(1.778,3.054);
\draw[gp path] (8.197,3.054)--(8.107,3.054);
\draw[gp path] (1.688,3.334)--(1.778,3.334);
\draw[gp path] (8.197,3.334)--(8.107,3.334);
\draw[gp path] (1.688,3.533)--(1.778,3.533);
\draw[gp path] (8.197,3.533)--(8.107,3.533);
\draw[gp path] (1.688,3.687)--(1.778,3.687);
\draw[gp path] (8.197,3.687)--(8.107,3.687);
\draw[gp path] (1.688,3.813)--(1.778,3.813);
\draw[gp path] (8.197,3.813)--(8.107,3.813);
\draw[gp path] (1.688,3.919)--(1.778,3.919);
\draw[gp path] (8.197,3.919)--(8.107,3.919);
\draw[gp path] (1.688,4.012)--(1.778,4.012);
\draw[gp path] (8.197,4.012)--(8.107,4.012);
\draw[gp path] (1.688,4.093)--(1.778,4.093);
\draw[gp path] (8.197,4.093)--(8.107,4.093);
\draw[gp path] (1.688,4.166)--(1.868,4.166);
\draw[gp path] (8.197,4.166)--(8.017,4.166);
\node[gp node right] at (1.504,4.166) { 1};
\draw[gp path] (1.688,4.644)--(1.778,4.644);
\draw[gp path] (8.197,4.644)--(8.107,4.644);
\draw[gp path] (1.688,4.924)--(1.778,4.924);
\draw[gp path] (8.197,4.924)--(8.107,4.924);
\draw[gp path] (1.688,5.123)--(1.778,5.123);
\draw[gp path] (8.197,5.123)--(8.107,5.123);
\draw[gp path] (1.688,5.277)--(1.778,5.277);
\draw[gp path] (8.197,5.277)--(8.107,5.277);
\draw[gp path] (1.688,5.403)--(1.778,5.403);
\draw[gp path] (8.197,5.403)--(8.107,5.403);
\draw[gp path] (1.688,5.510)--(1.778,5.510);
\draw[gp path] (8.197,5.510)--(8.107,5.510);
\draw[gp path] (1.688,5.602)--(1.778,5.602);
\draw[gp path] (8.197,5.602)--(8.107,5.602);
\draw[gp path] (1.688,5.683)--(1.778,5.683);
\draw[gp path] (8.197,5.683)--(8.107,5.683);
\draw[gp path] (1.688,5.756)--(1.868,5.756);
\draw[gp path] (8.197,5.756)--(8.017,5.756);
\node[gp node right] at (1.504,5.756) { 10};
\draw[gp path] (1.688,0.985)--(1.688,1.165);
\draw[gp path] (1.688,5.756)--(1.688,5.576);
\node[gp node center] at (1.688,0.677) {-1};
\draw[gp path] (2.411,0.985)--(2.411,1.165);
\draw[gp path] (2.411,5.756)--(2.411,5.576);
\node[gp node center] at (2.411,0.677) {-0.5};
\draw[gp path] (3.134,0.985)--(3.134,1.165);
\draw[gp path] (3.134,5.756)--(3.134,5.576);
\node[gp node center] at (3.134,0.677) { 0};
\draw[gp path] (3.858,0.985)--(3.858,1.165);
\draw[gp path] (3.858,5.756)--(3.858,5.576);
\node[gp node center] at (3.858,0.677) { 0.5};
\draw[gp path] (4.581,0.985)--(4.581,1.165);
\draw[gp path] (4.581,5.756)--(4.581,5.576);
\node[gp node center] at (4.581,0.677) { 1};
\draw[gp path] (5.304,0.985)--(5.304,1.165);
\draw[gp path] (5.304,5.756)--(5.304,5.576);
\node[gp node center] at (5.304,0.677) { 1.5};
\draw[gp path] (6.027,0.985)--(6.027,1.165);
\draw[gp path] (6.027,5.756)--(6.027,5.576);
\node[gp node center] at (6.027,0.677) { 2};
\draw[gp path] (6.751,0.985)--(6.751,1.165);
\draw[gp path] (6.751,5.756)--(6.751,5.576);
\node[gp node center] at (6.751,0.677) { 2.5};
\draw[gp path] (7.474,0.985)--(7.474,1.165);
\draw[gp path] (7.474,5.756)--(7.474,5.576);
\node[gp node center] at (7.474,0.677) { 3};
\draw[gp path] (8.197,0.985)--(8.197,1.165);
\draw[gp path] (8.197,5.756)--(8.197,5.576);
\node[gp node center] at (8.197,0.677) { 3.5};
\draw[gp path] (1.688,5.756)--(1.688,0.985)--(8.197,0.985)--(8.197,5.756)--cycle;
\node[gp node center,rotate=-270] at (0.246,3.370) {HW1/eM beam width (mm)};
\node[gp node center] at (4.942,0.215) {Pulse location rel. to $f$=6mm lens};
\gpcolor{color=gp lt color 0}
\gpsetlinetype{gp lt plot 0}
\draw[gp path] (1.688,3.687)--(1.720,3.687)--(1.752,3.687)--(1.783,3.687)--(1.815,3.687)%
  --(1.847,3.687)--(1.879,3.687)--(1.911,3.687)--(1.943,3.687)--(1.974,3.687)--(2.006,3.687)%
  --(2.038,3.687)--(2.070,3.687)--(2.102,3.687)--(2.134,3.687)--(2.165,3.687)--(2.197,3.687)%
  --(2.229,3.687)--(2.261,3.687)--(2.293,3.687)--(2.324,3.687)--(2.356,3.687)--(2.388,3.687)%
  --(2.420,3.687)--(2.452,3.687)--(2.484,3.687)--(2.515,3.687)--(2.547,3.687)--(2.579,3.687)%
  --(2.611,3.687)--(2.643,3.687)--(2.674,3.686)--(2.706,3.686)--(2.738,3.686)--(2.770,3.685)%
  --(2.802,3.684)--(2.834,3.683)--(2.865,3.681)--(2.897,3.679)--(2.929,3.677)--(2.961,3.674)%
  --(2.993,3.670)--(3.025,3.665)--(3.056,3.659)--(3.088,3.653)--(3.120,3.645)--(3.152,3.637)%
  --(3.184,3.627)--(3.215,3.616)--(3.247,3.604)--(3.279,3.592)--(3.311,3.578)--(3.343,3.563)%
  --(3.375,3.547)--(3.406,3.530)--(3.438,3.512)--(3.470,3.494)--(3.502,3.474)--(3.534,3.454)%
  --(3.565,3.433)--(3.597,3.412)--(3.629,3.389)--(3.661,3.366)--(3.693,3.341)--(3.725,3.316)%
  --(3.756,3.290)--(3.788,3.263)--(3.820,3.235)--(3.852,3.205)--(3.884,3.175)--(3.916,3.142)%
  --(3.947,3.109)--(3.979,3.073)--(4.011,3.036)--(4.043,2.996)--(4.075,2.954)--(4.106,2.909)%
  --(4.138,2.862)--(4.170,2.810)--(4.202,2.755)--(4.234,2.695)--(4.266,2.629)--(4.297,2.557)%
  --(4.329,2.476)--(4.361,2.385)--(4.393,2.281)--(4.425,2.160)--(4.456,2.015)--(4.488,1.837)%
  --(4.520,1.616)--(4.552,1.367)--(4.584,1.256)--(4.616,1.439)--(4.647,1.686)--(4.679,1.893)%
  --(4.711,2.060)--(4.743,2.198)--(4.775,2.314)--(4.807,2.413)--(4.838,2.501)--(4.870,2.579)%
  --(4.902,2.649)--(4.934,2.713)--(4.966,2.772)--(4.997,2.826)--(5.029,2.876)--(5.061,2.923)%
  --(5.093,2.967)--(5.125,3.008)--(5.157,3.047)--(5.188,3.084)--(5.220,3.119)--(5.252,3.152)%
  --(5.284,3.184)--(5.316,3.214)--(5.348,3.243)--(5.379,3.271)--(5.411,3.298)--(5.443,3.324)%
  --(5.475,3.349)--(5.507,3.373)--(5.538,3.396)--(5.570,3.419)--(5.602,3.440)--(5.634,3.462)%
  --(5.666,3.482)--(5.698,3.502)--(5.729,3.521)--(5.761,3.540)--(5.793,3.558)--(5.825,3.576)%
  --(5.857,3.594)--(5.888,3.611)--(5.920,3.627)--(5.952,3.643)--(5.984,3.659)--(6.016,3.675)%
  --(6.048,3.690)--(6.079,3.705)--(6.111,3.719)--(6.143,3.733)--(6.175,3.747)--(6.207,3.761)%
  --(6.239,3.774)--(6.270,3.787)--(6.302,3.800)--(6.334,3.813)--(6.366,3.825)--(6.398,3.837)%
  --(6.429,3.849)--(6.461,3.861)--(6.493,3.873)--(6.525,3.884)--(6.557,3.895)--(6.589,3.906)%
  --(6.620,3.917)--(6.652,3.928)--(6.684,3.938)--(6.716,3.949)--(6.748,3.959)--(6.779,3.969)%
  --(6.811,3.979)--(6.843,3.989)--(6.875,3.998)--(6.907,4.008)--(6.939,4.017)--(6.970,4.026)%
  --(7.002,4.035)--(7.034,4.045)--(7.066,4.053)--(7.098,4.062)--(7.130,4.071)--(7.161,4.079)%
  --(7.193,4.088)--(7.225,4.096)--(7.257,4.104)--(7.289,4.113)--(7.320,4.121)--(7.352,4.129)%
  --(7.384,4.137)--(7.416,4.144)--(7.448,4.152)--(7.480,4.160)--(7.511,4.167)--(7.543,4.175)%
  --(7.575,4.182)--(7.607,4.189)--(7.639,4.197)--(7.670,4.204)--(7.702,4.211)--(7.734,4.218)%
  --(7.766,4.225)--(7.798,4.232)--(7.830,4.238)--(7.861,4.245)--(7.893,4.252)--(7.925,4.258)%
  --(7.957,4.265)--(7.989,4.271)--(8.021,4.278)--(8.052,4.284);
\gpcolor{color=gp lt color 1}
\gpsetlinetype{gp lt plot 1}
\draw[gp path] (1.688,3.687)--(1.720,3.687)--(1.752,3.687)--(1.783,3.687)--(1.815,3.687)%
  --(1.847,3.687)--(1.879,3.687)--(1.911,3.687)--(1.943,3.687)--(1.974,3.687)--(2.006,3.687)%
  --(2.038,3.687)--(2.070,3.687)--(2.102,3.687)--(2.134,3.687)--(2.165,3.687)--(2.197,3.687)%
  --(2.229,3.687)--(2.261,3.687)--(2.293,3.687)--(2.324,3.687)--(2.356,3.687)--(2.388,3.687)%
  --(2.420,3.687)--(2.452,3.687)--(2.484,3.687)--(2.515,3.687)--(2.547,3.687)--(2.579,3.687)%
  --(2.611,3.687)--(2.643,3.687)--(2.674,3.686)--(2.706,3.686)--(2.738,3.686)--(2.770,3.685)%
  --(2.802,3.684)--(2.834,3.683)--(2.865,3.681)--(2.897,3.679)--(2.929,3.677)--(2.961,3.674)%
  --(2.993,3.670)--(3.025,3.665)--(3.056,3.660)--(3.088,3.653)--(3.120,3.645)--(3.152,3.637)%
  --(3.184,3.627)--(3.215,3.616)--(3.247,3.604)--(3.279,3.592)--(3.311,3.578)--(3.343,3.563)%
  --(3.375,3.547)--(3.406,3.530)--(3.438,3.512)--(3.470,3.494)--(3.502,3.474)--(3.534,3.454)%
  --(3.565,3.433)--(3.597,3.412)--(3.629,3.389)--(3.661,3.366)--(3.693,3.341)--(3.725,3.316)%
  --(3.756,3.290)--(3.788,3.263)--(3.820,3.235)--(3.852,3.205)--(3.884,3.175)--(3.916,3.142)%
  --(3.947,3.109)--(3.979,3.073)--(4.011,3.036)--(4.043,2.996)--(4.075,2.954)--(4.106,2.909)%
  --(4.138,2.862)--(4.170,2.811)--(4.202,2.755)--(4.234,2.695)--(4.266,2.630)--(4.297,2.557)%
  --(4.329,2.477)--(4.361,2.386)--(4.393,2.282)--(4.425,2.160)--(4.456,2.015)--(4.488,1.838)%
  --(4.520,1.618)--(4.552,1.369)--(4.584,1.256)--(4.616,1.438)--(4.647,1.685)--(4.679,1.892)%
  --(4.711,2.059)--(4.743,2.197)--(4.775,2.313)--(4.807,2.413)--(4.838,2.501)--(4.870,2.579)%
  --(4.902,2.649)--(4.934,2.713)--(4.966,2.771)--(4.997,2.825)--(5.029,2.876)--(5.061,2.922)%
  --(5.093,2.966)--(5.125,3.008)--(5.157,3.046)--(5.188,3.083)--(5.220,3.118)--(5.252,3.152)%
  --(5.284,3.183)--(5.316,3.214)--(5.348,3.243)--(5.379,3.271)--(5.411,3.298)--(5.443,3.324)%
  --(5.475,3.349)--(5.507,3.373)--(5.538,3.396)--(5.570,3.418)--(5.602,3.440)--(5.634,3.461)%
  --(5.666,3.482)--(5.698,3.502)--(5.729,3.521)--(5.761,3.540)--(5.793,3.558)--(5.825,3.576)%
  --(5.857,3.594)--(5.888,3.610)--(5.920,3.627)--(5.952,3.643)--(5.984,3.659)--(6.016,3.675)%
  --(6.048,3.690)--(6.079,3.704)--(6.111,3.719)--(6.143,3.733)--(6.175,3.747)--(6.207,3.761)%
  --(6.239,3.774)--(6.270,3.787)--(6.302,3.800)--(6.334,3.813)--(6.366,3.825)--(6.398,3.837)%
  --(6.429,3.849)--(6.461,3.861)--(6.493,3.873)--(6.525,3.884)--(6.557,3.895)--(6.589,3.906)%
  --(6.620,3.917)--(6.652,3.928)--(6.684,3.938)--(6.716,3.949)--(6.748,3.959)--(6.779,3.969)%
  --(6.811,3.979)--(6.843,3.989)--(6.875,3.998)--(6.907,4.008)--(6.939,4.017)--(6.970,4.026)%
  --(7.002,4.035)--(7.034,4.044)--(7.066,4.053)--(7.098,4.062)--(7.130,4.071)--(7.161,4.079)%
  --(7.193,4.088)--(7.225,4.096)--(7.257,4.104)--(7.289,4.113)--(7.320,4.121)--(7.352,4.129)%
  --(7.384,4.136)--(7.416,4.144)--(7.448,4.152)--(7.480,4.160)--(7.511,4.167)--(7.543,4.175)%
  --(7.575,4.182)--(7.607,4.189)--(7.639,4.196)--(7.670,4.204)--(7.702,4.211)--(7.734,4.218)%
  --(7.766,4.225)--(7.798,4.231)--(7.830,4.238)--(7.861,4.245)--(7.893,4.252)--(7.925,4.258)%
  --(7.957,4.265)--(7.989,4.271)--(8.021,4.278)--(8.052,4.284);
\gpcolor{color=gp lt color 2}
\gpsetlinetype{gp lt plot 2}
\draw[gp path] (1.688,3.687)--(1.720,3.687)--(1.752,3.687)--(1.783,3.687)--(1.815,3.687)%
  --(1.847,3.687)--(1.879,3.687)--(1.911,3.687)--(1.943,3.687)--(1.974,3.687)--(2.006,3.687)%
  --(2.038,3.687)--(2.070,3.687)--(2.102,3.687)--(2.134,3.687)--(2.165,3.687)--(2.197,3.687)%
  --(2.229,3.687)--(2.261,3.687)--(2.293,3.687)--(2.324,3.687)--(2.356,3.687)--(2.388,3.687)%
  --(2.420,3.687)--(2.452,3.687)--(2.484,3.687)--(2.515,3.687)--(2.547,3.687)--(2.579,3.687)%
  --(2.611,3.687)--(2.643,3.687)--(2.674,3.686)--(2.706,3.686)--(2.738,3.686)--(2.770,3.685)%
  --(2.802,3.684)--(2.834,3.683)--(2.865,3.681)--(2.897,3.679)--(2.929,3.677)--(2.961,3.674)%
  --(2.993,3.670)--(3.025,3.665)--(3.056,3.660)--(3.088,3.653)--(3.120,3.646)--(3.152,3.637)%
  --(3.184,3.627)--(3.215,3.616)--(3.247,3.605)--(3.279,3.592)--(3.311,3.578)--(3.343,3.563)%
  --(3.375,3.547)--(3.406,3.530)--(3.438,3.512)--(3.470,3.494)--(3.502,3.475)--(3.534,3.454)%
  --(3.565,3.434)--(3.597,3.412)--(3.629,3.389)--(3.661,3.366)--(3.693,3.342)--(3.725,3.317)%
  --(3.756,3.291)--(3.788,3.263)--(3.820,3.235)--(3.852,3.206)--(3.884,3.175)--(3.916,3.143)%
  --(3.947,3.109)--(3.979,3.074)--(4.011,3.036)--(4.043,2.997)--(4.075,2.955)--(4.106,2.910)%
  --(4.138,2.863)--(4.170,2.812)--(4.202,2.757)--(4.234,2.697)--(4.266,2.632)--(4.297,2.559)%
  --(4.329,2.479)--(4.361,2.389)--(4.393,2.285)--(4.425,2.165)--(4.456,2.021)--(4.488,1.846)%
  --(4.520,1.629)--(4.552,1.382)--(4.584,1.257)--(4.616,1.427)--(4.647,1.673)--(4.679,1.882)%
  --(4.711,2.050)--(4.743,2.189)--(4.775,2.306)--(4.807,2.407)--(4.838,2.495)--(4.870,2.574)%
  --(4.902,2.644)--(4.934,2.708)--(4.966,2.767)--(4.997,2.822)--(5.029,2.872)--(5.061,2.919)%
  --(5.093,2.963)--(5.125,3.004)--(5.157,3.043)--(5.188,3.080)--(5.220,3.116)--(5.252,3.149)%
  --(5.284,3.181)--(5.316,3.211)--(5.348,3.241)--(5.379,3.269)--(5.411,3.296)--(5.443,3.322)%
  --(5.475,3.347)--(5.507,3.371)--(5.538,3.394)--(5.570,3.417)--(5.602,3.438)--(5.634,3.460)%
  --(5.666,3.480)--(5.698,3.500)--(5.729,3.519)--(5.761,3.538)--(5.793,3.557)--(5.825,3.575)%
  --(5.857,3.592)--(5.888,3.609)--(5.920,3.626)--(5.952,3.642)--(5.984,3.658)--(6.016,3.673)%
  --(6.048,3.688)--(6.079,3.703)--(6.111,3.718)--(6.143,3.732)--(6.175,3.746)--(6.207,3.759)%
  --(6.239,3.773)--(6.270,3.786)--(6.302,3.799)--(6.334,3.811)--(6.366,3.824)--(6.398,3.836)%
  --(6.429,3.848)--(6.461,3.860)--(6.493,3.871)--(6.525,3.883)--(6.557,3.894)--(6.589,3.905)%
  --(6.620,3.916)--(6.652,3.927)--(6.684,3.937)--(6.716,3.948)--(6.748,3.958)--(6.779,3.968)%
  --(6.811,3.978)--(6.843,3.988)--(6.875,3.997)--(6.907,4.007)--(6.939,4.016)--(6.970,4.025)%
  --(7.002,4.035)--(7.034,4.044)--(7.066,4.052)--(7.098,4.061)--(7.130,4.070)--(7.161,4.079)%
  --(7.193,4.087)--(7.225,4.095)--(7.257,4.104)--(7.289,4.112)--(7.320,4.120)--(7.352,4.128)%
  --(7.384,4.136)--(7.416,4.144)--(7.448,4.151)--(7.480,4.159)--(7.511,4.166)--(7.543,4.174)%
  --(7.575,4.181)--(7.607,4.189)--(7.639,4.196)--(7.670,4.203)--(7.702,4.210)--(7.734,4.217)%
  --(7.766,4.224)--(7.798,4.231)--(7.830,4.238)--(7.861,4.244)--(7.893,4.251)--(7.925,4.258)%
  --(7.957,4.264)--(7.989,4.271)--(8.021,4.277)--(8.052,4.283);
\gpcolor{color=gp lt color 3}
\gpsetlinetype{gp lt plot 3}
\draw[gp path] (1.688,3.687)--(1.720,3.687)--(1.752,3.687)--(1.783,3.687)--(1.815,3.687)%
  --(1.847,3.687)--(1.879,3.687)--(1.911,3.687)--(1.943,3.687)--(1.974,3.687)--(2.006,3.687)%
  --(2.038,3.687)--(2.070,3.687)--(2.102,3.687)--(2.134,3.687)--(2.165,3.687)--(2.197,3.687)%
  --(2.229,3.687)--(2.261,3.687)--(2.293,3.687)--(2.324,3.687)--(2.356,3.687)--(2.388,3.687)%
  --(2.420,3.687)--(2.452,3.687)--(2.484,3.687)--(2.515,3.687)--(2.547,3.687)--(2.579,3.687)%
  --(2.611,3.687)--(2.643,3.687)--(2.674,3.687)--(2.706,3.687)--(2.738,3.686)--(2.770,3.686)%
  --(2.802,3.685)--(2.834,3.684)--(2.865,3.682)--(2.897,3.680)--(2.929,3.678)--(2.961,3.675)%
  --(2.993,3.671)--(3.025,3.666)--(3.056,3.661)--(3.088,3.654)--(3.120,3.647)--(3.152,3.638)%
  --(3.184,3.629)--(3.215,3.618)--(3.247,3.606)--(3.279,3.593)--(3.311,3.579)--(3.343,3.565)%
  --(3.375,3.549)--(3.406,3.532)--(3.438,3.514)--(3.470,3.496)--(3.502,3.477)--(3.534,3.457)%
  --(3.565,3.436)--(3.597,3.415)--(3.629,3.392)--(3.661,3.369)--(3.693,3.345)--(3.725,3.320)%
  --(3.756,3.294)--(3.788,3.268)--(3.820,3.240)--(3.852,3.211)--(3.884,3.180)--(3.916,3.149)%
  --(3.947,3.115)--(3.979,3.080)--(4.011,3.044)--(4.043,3.005)--(4.075,2.964)--(4.106,2.920)%
  --(4.138,2.874)--(4.170,2.824)--(4.202,2.770)--(4.234,2.713)--(4.266,2.650)--(4.297,2.580)%
  --(4.329,2.504)--(4.361,2.418)--(4.393,2.321)--(4.425,2.209)--(4.456,2.078)--(4.488,1.922)%
  --(4.520,1.731)--(4.552,1.505)--(4.584,1.311)--(4.616,1.343)--(4.647,1.558)--(4.679,1.777)%
  --(4.711,1.959)--(4.743,2.109)--(4.775,2.236)--(4.807,2.344)--(4.838,2.438)--(4.870,2.521)%
  --(4.902,2.596)--(4.934,2.663)--(4.966,2.725)--(4.997,2.782)--(5.029,2.834)--(5.061,2.883)%
  --(5.093,2.929)--(5.125,2.972)--(5.157,3.013)--(5.188,3.051)--(5.220,3.087)--(5.252,3.122)%
  --(5.284,3.154)--(5.316,3.186)--(5.348,3.216)--(5.379,3.245)--(5.411,3.272)--(5.443,3.299)%
  --(5.475,3.325)--(5.507,3.349)--(5.538,3.373)--(5.570,3.396)--(5.602,3.419)--(5.634,3.440)%
  --(5.666,3.461)--(5.698,3.482)--(5.729,3.501)--(5.761,3.521)--(5.793,3.539)--(5.825,3.558)%
  --(5.857,3.575)--(5.888,3.593)--(5.920,3.610)--(5.952,3.626)--(5.984,3.642)--(6.016,3.658)%
  --(6.048,3.673)--(6.079,3.688)--(6.111,3.703)--(6.143,3.718)--(6.175,3.732)--(6.207,3.746)%
  --(6.239,3.759)--(6.270,3.772)--(6.302,3.786)--(6.334,3.798)--(6.366,3.811)--(6.398,3.823)%
  --(6.429,3.836)--(6.461,3.847)--(6.493,3.859)--(6.525,3.871)--(6.557,3.882)--(6.589,3.893)%
  --(6.620,3.904)--(6.652,3.915)--(6.684,3.926)--(6.716,3.936)--(6.748,3.947)--(6.779,3.957)%
  --(6.811,3.967)--(6.843,3.977)--(6.875,3.987)--(6.907,3.996)--(6.939,4.006)--(6.970,4.015)%
  --(7.002,4.024)--(7.034,4.033)--(7.066,4.042)--(7.098,4.051)--(7.130,4.060)--(7.161,4.069)%
  --(7.193,4.077)--(7.225,4.086)--(7.257,4.094)--(7.289,4.102)--(7.320,4.110)--(7.352,4.119)%
  --(7.384,4.126)--(7.416,4.134)--(7.448,4.142)--(7.480,4.150)--(7.511,4.157)--(7.543,4.165)%
  --(7.575,4.172)--(7.607,4.180)--(7.639,4.187)--(7.670,4.194)--(7.702,4.201)--(7.734,4.209)%
  --(7.766,4.216)--(7.798,4.222)--(7.830,4.229)--(7.861,4.236)--(7.893,4.243)--(7.925,4.250)%
  --(7.957,4.256)--(7.989,4.263)--(8.021,4.269)--(8.052,4.276);
\gpcolor{color=gp lt color 4}
\gpsetlinetype{gp lt plot 4}
\draw[gp path] (1.688,3.687)--(1.720,3.687)--(1.752,3.687)--(1.783,3.687)--(1.815,3.687)%
  --(1.847,3.687)--(1.879,3.687)--(1.911,3.687)--(1.943,3.687)--(1.974,3.687)--(2.006,3.688)%
  --(2.038,3.688)--(2.070,3.688)--(2.102,3.688)--(2.134,3.688)--(2.165,3.688)--(2.197,3.689)%
  --(2.229,3.689)--(2.261,3.689)--(2.293,3.689)--(2.324,3.690)--(2.356,3.690)--(2.388,3.690)%
  --(2.420,3.691)--(2.452,3.691)--(2.484,3.691)--(2.515,3.692)--(2.547,3.692)--(2.579,3.692)%
  --(2.611,3.692)--(2.643,3.693)--(2.674,3.693)--(2.706,3.693)--(2.738,3.693)--(2.770,3.693)%
  --(2.802,3.692)--(2.834,3.692)--(2.865,3.691)--(2.897,3.689)--(2.929,3.687)--(2.961,3.684)%
  --(2.993,3.681)--(3.025,3.677)--(3.056,3.672)--(3.088,3.666)--(3.120,3.659)--(3.152,3.651)%
  --(3.184,3.642)--(3.215,3.632)--(3.247,3.621)--(3.279,3.609)--(3.311,3.596)--(3.343,3.582)%
  --(3.375,3.567)--(3.406,3.551)--(3.438,3.535)--(3.470,3.517)--(3.502,3.499)--(3.534,3.481)%
  --(3.565,3.461)--(3.597,3.441)--(3.629,3.421)--(3.661,3.399)--(3.693,3.377)--(3.725,3.354)%
  --(3.756,3.331)--(3.788,3.306)--(3.820,3.281)--(3.852,3.255)--(3.884,3.228)--(3.916,3.200)%
  --(3.947,3.171)--(3.979,3.141)--(4.011,3.109)--(4.043,3.076)--(4.075,3.042)--(4.106,3.006)%
  --(4.138,2.968)--(4.170,2.928)--(4.202,2.886)--(4.234,2.842)--(4.266,2.795)--(4.297,2.745)%
  --(4.329,2.692)--(4.361,2.635)--(4.393,2.574)--(4.425,2.508)--(4.456,2.436)--(4.488,2.358)%
  --(4.520,2.272)--(4.552,2.177)--(4.584,2.073)--(4.616,1.959)--(4.647,1.837)--(4.679,1.713)%
  --(4.711,1.606)--(4.743,1.551)--(4.775,1.575)--(4.807,1.664)--(4.838,1.783)--(4.870,1.905)%
  --(4.902,2.021)--(4.934,2.127)--(4.966,2.224)--(4.997,2.311)--(5.029,2.390)--(5.061,2.463)%
  --(5.093,2.529)--(5.125,2.591)--(5.157,2.648)--(5.188,2.701)--(5.220,2.750)--(5.252,2.797)%
  --(5.284,2.841)--(5.316,2.883)--(5.348,2.922)--(5.379,2.959)--(5.411,2.995)--(5.443,3.029)%
  --(5.475,3.062)--(5.507,3.093)--(5.538,3.122)--(5.570,3.151)--(5.602,3.179)--(5.634,3.205)%
  --(5.666,3.231)--(5.698,3.256)--(5.729,3.280)--(5.761,3.303)--(5.793,3.325)--(5.825,3.347)%
  --(5.857,3.368)--(5.888,3.389)--(5.920,3.409)--(5.952,3.428)--(5.984,3.447)--(6.016,3.466)%
  --(6.048,3.484)--(6.079,3.501)--(6.111,3.518)--(6.143,3.535)--(6.175,3.551)--(6.207,3.567)%
  --(6.239,3.583)--(6.270,3.598)--(6.302,3.613)--(6.334,3.628)--(6.366,3.642)--(6.398,3.656)%
  --(6.429,3.670)--(6.461,3.684)--(6.493,3.697)--(6.525,3.710)--(6.557,3.723)--(6.589,3.736)%
  --(6.620,3.748)--(6.652,3.760)--(6.684,3.772)--(6.716,3.784)--(6.748,3.795)--(6.779,3.807)%
  --(6.811,3.818)--(6.843,3.829)--(6.875,3.840)--(6.907,3.850)--(6.939,3.861)--(6.970,3.871)%
  --(7.002,3.881)--(7.034,3.892)--(7.066,3.901)--(7.098,3.911)--(7.130,3.921)--(7.161,3.930)%
  --(7.193,3.940)--(7.225,3.949)--(7.257,3.958)--(7.289,3.967)--(7.320,3.976)--(7.352,3.985)%
  --(7.384,3.994)--(7.416,4.002)--(7.448,4.011)--(7.480,4.019)--(7.511,4.027)--(7.543,4.036)%
  --(7.575,4.044)--(7.607,4.052)--(7.639,4.060)--(7.670,4.068)--(7.702,4.075)--(7.734,4.083)%
  --(7.766,4.091)--(7.798,4.098)--(7.830,4.105)--(7.861,4.113)--(7.893,4.120)--(7.925,4.127)%
  --(7.957,4.134)--(7.989,4.141)--(8.021,4.148)--(8.052,4.155);
\gpcolor{color=gp lt color border}
\gpsetlinetype{gp lt border}
\draw[gp path] (1.688,5.756)--(1.688,0.985)--(8.197,0.985)--(8.197,5.756)--cycle;
%% coordinates of the plot area
\gpdefrectangularnode{gp plot 1}{\pgfpoint{1.688cm}{0.985cm}}{\pgfpoint{8.197cm}{5.756cm}}
%% \end{tikzpicture}
%% gnuplot variables

        \node at (2.7,4.9) [align=left] {$\xi=10$\\$f=$ 6mm};
        \node at (6.9,2.6) {$N=10^7$};
        \node at (6,2) {$N=10^6$};
      \end{tikzpicture}
    }
  }
  \caption[AG model simulated focal behavior of perfect magnetic lenses as a function of pulse charge $N$]{
    AG model simulated focal behavior of perfect magnetic lenses as a function of pulse charge $N$.
    Each plot's pulses begin at the lens' front focal plane at velocity $c/3$ (related to 20 kV energy) and have the same volume.
    The left plots (\subref{fig:focus_long_oblate} and \subref{fig:focus_short_oblate}) are initially oblate shaped, HW1/eM width $ 500 \mu \text{m}$ and ellipticiy $ \xi ( 0 ) = 0.1 $.
    The right plots (\subref{fig:focus_long_prolate} and \subref{fig:focus_short_prolate}) are initially prolate shaped, HW1/eM width $ 107.7 \mu \text{m}$ and ellipticiy $ \xi ( 0 ) = 10 $.
    Distance traveled in column is measured relative to the lens at $z=0$, in units of the focal lenth.
    The top plots (\subref{fig:focus_long_oblate} and \subref{fig:focus_long_prolate}) have a longer focal length $f = $ 60mm compared to the bottom plots (\subref{fig:focus_short_oblate} and \subref{fig:focus_short_prolate}) $ f = $ 6.0mm.
    A logarithmic scale is used for comparative clarity near the foci.
    Clearly the best performance is achieved for shorter focal lengths and oblate pulses \subref{fig:focus_short_oblate} even at higher initial charge densities.
  }
  \label{fig:focus_lens_charge}
\end{figure}


\ref{fig:focus_lens_charge} shows the predicted dynamics of the spatial electron pulse waist (HW1/eM) for intially oblate ($ \xi ( 0 ) = 0.1 $) or prolate ($ \xi ( 0 ) = 10 $) pulses under perfect focusing for lenses of two focal lengths; a `long' focal length of 60mm and a `short' focal length of 6.0mm.
For both lens strengths, the deleterious effects of space-charge are evident as the number of electrons/pulse $ N $ is increased.
To make an effective comparison, highlighting the effect of space-charge alone, the pulses used are at 20kV velocity with $\Delta E = $ 0.5eV but are not simulated using an accelerator.
The length of the accelerator and the anode lensing would make useful comparison difficult; as such the simulation presented approximates a beam which has been perfectly collimated at the front-focal plane of the lens.

The single-electron ($ N = 1 $) focal spot sizes of $\sim$150 microns for the $ f = 60 \text{mm} $ lens and $\sim$15 microns for the shorter $ f = 6.0 \text{mm} $ lens show marked increase starting around $ N \approx 10^{ 5 } $ and $ N \approx 10^{ 6 }$, respectively, and worsening dramatically above these thresholds.
This order of magnitude difference is fundamentally related to the greater time of flight before the focus of the $ f = 60\text{mm} $ lens; it increases the impulse that the internal Coulomb force can exert on the pulse as its charge density is increased under focusing.
A consequent further effect is a shift in the focal position to $ z > f $.
A more subtle effect is that disk-like pulses are more readily focused than cigar-like pulses --- the rate of transverse pulse broadening due to space-charge effects already being intrinsically greater in the latter (\ref{fig:compare_shape}).

%TODO? include simulations for different voltages to establish TOF argument?

Thus, the extended AG electron pulse propagation model predicts that higher fidelity focusing will be achieved with shorter focal length magnetic lenses, higher acceleration energy electrons (time of flight reduction), and pulses with $ \sigma_{ \smallT } \gg \sigma_{ z } $ at the entrance aperture of the lens.

