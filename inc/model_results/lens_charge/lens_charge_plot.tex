%TODO fix axes to be more consistent
%TODO add N=1e8 for oblate?

\begin{figure}
  \centering
  \centerline{
    \subfloat[][] {
      \label{fig:focus_long_oblate}
      \begin{tikzpicture}
        %% \begin{tikzpicture}[gnuplot]
%% generated with GNUPLOT 4.6p0 (Lua 5.1; terminal rev. 99, script rev. 100)
%% Sat 06 Apr 2013 10:51:01 AM CDT
\gpsolidlines
\path (0.000,0.000) rectangle (8.750,6.125);
\gpcolor{color=gp lt color border}
\gpsetlinetype{gp lt border}
\gpsetlinewidth{1.00}
\draw[gp path] (1.504,0.985)--(1.684,0.985);
\draw[gp path] (8.197,0.985)--(8.017,0.985);
\node[gp node right] at (1.320,0.985) { 0.1};
\draw[gp path] (1.504,1.703)--(1.594,1.703);
\draw[gp path] (8.197,1.703)--(8.107,1.703);
\draw[gp path] (1.504,2.123)--(1.594,2.123);
\draw[gp path] (8.197,2.123)--(8.107,2.123);
\draw[gp path] (1.504,2.421)--(1.594,2.421);
\draw[gp path] (8.197,2.421)--(8.107,2.421);
\draw[gp path] (1.504,2.652)--(1.594,2.652);
\draw[gp path] (8.197,2.652)--(8.107,2.652);
\draw[gp path] (1.504,2.841)--(1.594,2.841);
\draw[gp path] (8.197,2.841)--(8.107,2.841);
\draw[gp path] (1.504,3.001)--(1.594,3.001);
\draw[gp path] (8.197,3.001)--(8.107,3.001);
\draw[gp path] (1.504,3.139)--(1.594,3.139);
\draw[gp path] (8.197,3.139)--(8.107,3.139);
\draw[gp path] (1.504,3.261)--(1.594,3.261);
\draw[gp path] (8.197,3.261)--(8.107,3.261);
\draw[gp path] (1.504,3.371)--(1.684,3.371);
\draw[gp path] (8.197,3.371)--(8.017,3.371);
\node[gp node right] at (1.320,3.371) { 1};
\draw[gp path] (1.504,4.089)--(1.594,4.089);
\draw[gp path] (8.197,4.089)--(8.107,4.089);
\draw[gp path] (1.504,4.509)--(1.594,4.509);
\draw[gp path] (8.197,4.509)--(8.107,4.509);
\draw[gp path] (1.504,4.807)--(1.594,4.807);
\draw[gp path] (8.197,4.807)--(8.107,4.807);
\draw[gp path] (1.504,5.038)--(1.594,5.038);
\draw[gp path] (8.197,5.038)--(8.107,5.038);
\draw[gp path] (1.504,5.227)--(1.594,5.227);
\draw[gp path] (8.197,5.227)--(8.107,5.227);
\draw[gp path] (1.504,5.386)--(1.594,5.386);
\draw[gp path] (8.197,5.386)--(8.107,5.386);
\draw[gp path] (1.504,5.525)--(1.594,5.525);
\draw[gp path] (8.197,5.525)--(8.107,5.525);
\draw[gp path] (1.504,5.647)--(1.594,5.647);
\draw[gp path] (8.197,5.647)--(8.107,5.647);
\draw[gp path] (1.504,5.756)--(1.684,5.756);
\draw[gp path] (8.197,5.756)--(8.017,5.756);
\node[gp node right] at (1.320,5.756) { 10};
\draw[gp path] (1.504,0.985)--(1.504,1.165);
\draw[gp path] (1.504,5.756)--(1.504,5.576);
\node[gp node center] at (1.504,0.677) {-1};
\draw[gp path] (2.248,0.985)--(2.248,1.165);
\draw[gp path] (2.248,5.756)--(2.248,5.576);
\node[gp node center] at (2.248,0.677) {-0.5};
\draw[gp path] (2.991,0.985)--(2.991,1.165);
\draw[gp path] (2.991,5.756)--(2.991,5.576);
\node[gp node center] at (2.991,0.677) { 0};
\draw[gp path] (3.735,0.985)--(3.735,1.165);
\draw[gp path] (3.735,5.756)--(3.735,5.576);
\node[gp node center] at (3.735,0.677) { 0.5};
\draw[gp path] (4.479,0.985)--(4.479,1.165);
\draw[gp path] (4.479,5.756)--(4.479,5.576);
\node[gp node center] at (4.479,0.677) { 1};
\draw[gp path] (5.222,0.985)--(5.222,1.165);
\draw[gp path] (5.222,5.756)--(5.222,5.576);
\node[gp node center] at (5.222,0.677) { 1.5};
\draw[gp path] (5.966,0.985)--(5.966,1.165);
\draw[gp path] (5.966,5.756)--(5.966,5.576);
\node[gp node center] at (5.966,0.677) { 2};
\draw[gp path] (6.710,0.985)--(6.710,1.165);
\draw[gp path] (6.710,5.756)--(6.710,5.576);
\node[gp node center] at (6.710,0.677) { 2.5};
\draw[gp path] (7.453,0.985)--(7.453,1.165);
\draw[gp path] (7.453,5.756)--(7.453,5.576);
\node[gp node center] at (7.453,0.677) { 3};
\draw[gp path] (8.197,0.985)--(8.197,1.165);
\draw[gp path] (8.197,5.756)--(8.197,5.576);
\node[gp node center] at (8.197,0.677) { 3.5};
\draw[gp path] (1.504,5.756)--(1.504,0.985)--(8.197,0.985)--(8.197,5.756)--cycle;
\node[gp node center,rotate=-270] at (0.246,3.370) {HW1/eM beam width (mm)};
\node[gp node center] at (4.850,0.215) {Pulse location ($z^{\prime}/f)$};
\gpcolor{color=gp lt color 0}
\gpsetlinetype{gp lt plot 0}
\draw[gp path] (1.504,2.652)--(1.537,2.652)--(1.569,2.652)--(1.602,2.653)--(1.635,2.653)%
  --(1.668,2.653)--(1.700,2.653)--(1.733,2.653)--(1.766,2.654)--(1.798,2.654)--(1.831,2.655)%
  --(1.864,2.655)--(1.897,2.655)--(1.929,2.656)--(1.962,2.657)--(1.995,2.657)--(2.028,2.658)%
  --(2.060,2.658)--(2.093,2.659)--(2.126,2.660)--(2.158,2.661)--(2.191,2.662)--(2.224,2.663)%
  --(2.257,2.663)--(2.289,2.664)--(2.322,2.665)--(2.355,2.667)--(2.387,2.668)--(2.420,2.669)%
  --(2.453,2.670)--(2.486,2.671)--(2.518,2.672)--(2.551,2.674)--(2.584,2.675)--(2.617,2.676)%
  --(2.649,2.678)--(2.682,2.679)--(2.715,2.681)--(2.747,2.682)--(2.780,2.684)--(2.813,2.685)%
  --(2.846,2.687)--(2.878,2.689)--(2.911,2.690)--(2.944,2.692)--(2.976,2.691)--(3.009,2.681)%
  --(3.042,2.662)--(3.075,2.640)--(3.107,2.618)--(3.140,2.595)--(3.173,2.572)--(3.206,2.549)%
  --(3.238,2.525)--(3.271,2.500)--(3.304,2.475)--(3.336,2.450)--(3.369,2.424)--(3.402,2.397)%
  --(3.435,2.370)--(3.467,2.342)--(3.500,2.314)--(3.533,2.285)--(3.565,2.255)--(3.598,2.225)%
  --(3.631,2.194)--(3.664,2.162)--(3.696,2.129)--(3.729,2.096)--(3.762,2.062)--(3.794,2.027)%
  --(3.827,1.992)--(3.860,1.956)--(3.893,1.919)--(3.925,1.882)--(3.958,1.844)--(3.991,1.806)%
  --(4.024,1.767)--(4.056,1.728)--(4.089,1.690)--(4.122,1.651)--(4.154,1.614)--(4.187,1.577)%
  --(4.220,1.542)--(4.253,1.508)--(4.285,1.477)--(4.318,1.450)--(4.351,1.426)--(4.383,1.406)%
  --(4.416,1.392)--(4.449,1.382)--(4.482,1.379)--(4.514,1.381)--(4.547,1.389)--(4.580,1.402)%
  --(4.613,1.420)--(4.645,1.443)--(4.678,1.470)--(4.711,1.500)--(4.743,1.533)--(4.776,1.568)%
  --(4.809,1.604)--(4.842,1.642)--(4.874,1.680)--(4.907,1.719)--(4.940,1.758)--(4.972,1.796)%
  --(5.005,1.835)--(5.038,1.873)--(5.071,1.910)--(5.103,1.947)--(5.136,1.983)--(5.169,2.019)%
  --(5.202,2.053)--(5.234,2.088)--(5.267,2.121)--(5.300,2.154)--(5.332,2.186)--(5.365,2.217)%
  --(5.398,2.247)--(5.431,2.277)--(5.463,2.307)--(5.496,2.335)--(5.529,2.363)--(5.561,2.390)%
  --(5.594,2.417)--(5.627,2.443)--(5.660,2.469)--(5.692,2.494)--(5.725,2.519)--(5.758,2.543)%
  --(5.790,2.566)--(5.823,2.590)--(5.856,2.612)--(5.889,2.634)--(5.921,2.656)--(5.954,2.678)%
  --(5.987,2.699)--(6.020,2.719)--(6.052,2.740)--(6.085,2.759)--(6.118,2.779)--(6.150,2.798)%
  --(6.183,2.817)--(6.216,2.836)--(6.249,2.854)--(6.281,2.872)--(6.314,2.890)--(6.347,2.907)%
  --(6.379,2.924)--(6.412,2.941)--(6.445,2.958)--(6.478,2.974)--(6.510,2.990)--(6.543,3.006)%
  --(6.576,3.022)--(6.609,3.037)--(6.641,3.052)--(6.674,3.067)--(6.707,3.082)--(6.739,3.097)%
  --(6.772,3.111)--(6.805,3.125)--(6.838,3.139)--(6.870,3.153)--(6.903,3.167)--(6.936,3.180)%
  --(6.968,3.194)--(7.001,3.207)--(7.034,3.220)--(7.067,3.233)--(7.099,3.245)--(7.132,3.258)%
  --(7.165,3.270)--(7.198,3.282)--(7.230,3.295)--(7.263,3.307)--(7.296,3.318)--(7.328,3.330)%
  --(7.361,3.342)--(7.394,3.353)--(7.427,3.364)--(7.459,3.376)--(7.492,3.387)--(7.525,3.398)%
  --(7.557,3.409)--(7.590,3.419)--(7.623,3.430)--(7.656,3.441)--(7.688,3.451)--(7.721,3.461)%
  --(7.754,3.472)--(7.786,3.482)--(7.819,3.492)--(7.852,3.502)--(7.885,3.512)--(7.917,3.521)%
  --(7.950,3.531)--(7.983,3.541)--(8.016,3.550)--(8.048,3.560);
\gpcolor{color=gp lt color 1}
\gpsetlinetype{gp lt plot 1}
\draw[gp path] (1.504,2.652)--(1.537,2.652)--(1.569,2.652)--(1.602,2.653)--(1.635,2.653)%
  --(1.668,2.653)--(1.700,2.653)--(1.733,2.653)--(1.766,2.654)--(1.798,2.654)--(1.831,2.655)%
  --(1.864,2.655)--(1.897,2.656)--(1.929,2.656)--(1.962,2.657)--(1.995,2.657)--(2.028,2.658)%
  --(2.060,2.659)--(2.093,2.660)--(2.126,2.660)--(2.158,2.661)--(2.191,2.662)--(2.224,2.663)%
  --(2.257,2.664)--(2.289,2.665)--(2.322,2.666)--(2.355,2.667)--(2.387,2.668)--(2.420,2.670)%
  --(2.453,2.671)--(2.486,2.672)--(2.518,2.673)--(2.551,2.675)--(2.584,2.676)--(2.617,2.677)%
  --(2.649,2.679)--(2.682,2.680)--(2.715,2.682)--(2.747,2.684)--(2.780,2.685)--(2.813,2.687)%
  --(2.846,2.688)--(2.878,2.690)--(2.911,2.692)--(2.944,2.694)--(2.976,2.693)--(3.009,2.683)%
  --(3.042,2.664)--(3.075,2.642)--(3.107,2.620)--(3.140,2.598)--(3.173,2.575)--(3.206,2.551)%
  --(3.238,2.527)--(3.271,2.503)--(3.304,2.478)--(3.336,2.453)--(3.369,2.427)--(3.402,2.401)%
  --(3.435,2.374)--(3.467,2.346)--(3.500,2.318)--(3.533,2.289)--(3.565,2.259)--(3.598,2.229)%
  --(3.631,2.198)--(3.664,2.167)--(3.696,2.135)--(3.729,2.102)--(3.762,2.068)--(3.794,2.034)%
  --(3.827,1.999)--(3.860,1.963)--(3.893,1.927)--(3.925,1.890)--(3.958,1.853)--(3.991,1.815)%
  --(4.024,1.777)--(4.056,1.738)--(4.089,1.700)--(4.122,1.662)--(4.154,1.625)--(4.187,1.589)%
  --(4.220,1.554)--(4.253,1.520)--(4.285,1.490)--(4.318,1.462)--(4.351,1.438)--(4.383,1.418)%
  --(4.416,1.403)--(4.449,1.392)--(4.482,1.388)--(4.514,1.389)--(4.547,1.395)--(4.580,1.407)%
  --(4.613,1.424)--(4.645,1.446)--(4.678,1.471)--(4.711,1.500)--(4.743,1.532)--(4.776,1.565)%
  --(4.809,1.601)--(4.842,1.638)--(4.874,1.675)--(4.907,1.713)--(4.940,1.751)--(4.972,1.790)%
  --(5.005,1.828)--(5.038,1.865)--(5.071,1.902)--(5.103,1.939)--(5.136,1.975)--(5.169,2.011)%
  --(5.202,2.045)--(5.234,2.079)--(5.267,2.113)--(5.300,2.145)--(5.332,2.177)--(5.365,2.209)%
  --(5.398,2.239)--(5.431,2.269)--(5.463,2.298)--(5.496,2.327)--(5.529,2.355)--(5.561,2.382)%
  --(5.594,2.409)--(5.627,2.436)--(5.660,2.461)--(5.692,2.486)--(5.725,2.511)--(5.758,2.535)%
  --(5.790,2.559)--(5.823,2.582)--(5.856,2.605)--(5.889,2.627)--(5.921,2.649)--(5.954,2.670)%
  --(5.987,2.691)--(6.020,2.712)--(6.052,2.732)--(6.085,2.752)--(6.118,2.772)--(6.150,2.791)%
  --(6.183,2.810)--(6.216,2.829)--(6.249,2.847)--(6.281,2.865)--(6.314,2.883)--(6.347,2.900)%
  --(6.379,2.918)--(6.412,2.935)--(6.445,2.951)--(6.478,2.968)--(6.510,2.984)--(6.543,3.000)%
  --(6.576,3.015)--(6.609,3.031)--(6.641,3.046)--(6.674,3.061)--(6.707,3.076)--(6.739,3.091)%
  --(6.772,3.105)--(6.805,3.119)--(6.838,3.134)--(6.870,3.147)--(6.903,3.161)--(6.936,3.175)%
  --(6.968,3.188)--(7.001,3.201)--(7.034,3.214)--(7.067,3.227)--(7.099,3.240)--(7.132,3.253)%
  --(7.165,3.265)--(7.198,3.277)--(7.230,3.289)--(7.263,3.301)--(7.296,3.313)--(7.328,3.325)%
  --(7.361,3.337)--(7.394,3.348)--(7.427,3.360)--(7.459,3.371)--(7.492,3.382)--(7.525,3.393)%
  --(7.557,3.404)--(7.590,3.415)--(7.623,3.425)--(7.656,3.436)--(7.688,3.446)--(7.721,3.457)%
  --(7.754,3.467)--(7.786,3.477)--(7.819,3.487)--(7.852,3.497)--(7.885,3.507)--(7.917,3.517)%
  --(7.950,3.527)--(7.983,3.536)--(8.016,3.546)--(8.048,3.555);
\gpcolor{color=gp lt color 2}
\gpsetlinetype{gp lt plot 2}
\draw[gp path] (1.504,2.652)--(1.537,2.652)--(1.569,2.653)--(1.602,2.653)--(1.635,2.653)%
  --(1.668,2.653)--(1.700,2.654)--(1.733,2.654)--(1.766,2.654)--(1.798,2.655)--(1.831,2.656)%
  --(1.864,2.656)--(1.897,2.657)--(1.929,2.658)--(1.962,2.659)--(1.995,2.659)--(2.028,2.660)%
  --(2.060,2.661)--(2.093,2.662)--(2.126,2.664)--(2.158,2.665)--(2.191,2.666)--(2.224,2.667)%
  --(2.257,2.669)--(2.289,2.670)--(2.322,2.672)--(2.355,2.673)--(2.387,2.675)--(2.420,2.676)%
  --(2.453,2.678)--(2.486,2.680)--(2.518,2.682)--(2.551,2.684)--(2.584,2.685)--(2.617,2.687)%
  --(2.649,2.689)--(2.682,2.691)--(2.715,2.694)--(2.747,2.696)--(2.780,2.698)--(2.813,2.700)%
  --(2.846,2.702)--(2.878,2.705)--(2.911,2.707)--(2.944,2.710)--(2.976,2.710)--(3.009,2.700)%
  --(3.042,2.682)--(3.075,2.661)--(3.107,2.640)--(3.140,2.618)--(3.173,2.596)--(3.206,2.573)%
  --(3.238,2.551)--(3.271,2.527)--(3.304,2.504)--(3.336,2.480)--(3.369,2.455)--(3.402,2.430)%
  --(3.435,2.404)--(3.467,2.378)--(3.500,2.352)--(3.533,2.325)--(3.565,2.297)--(3.598,2.269)%
  --(3.631,2.240)--(3.664,2.211)--(3.696,2.181)--(3.729,2.150)--(3.762,2.120)--(3.794,2.088)%
  --(3.827,2.056)--(3.860,2.024)--(3.893,1.991)--(3.925,1.957)--(3.958,1.923)--(3.991,1.889)%
  --(4.024,1.855)--(4.056,1.821)--(4.089,1.786)--(4.122,1.752)--(4.154,1.718)--(4.187,1.685)%
  --(4.220,1.653)--(4.253,1.622)--(4.285,1.593)--(4.318,1.565)--(4.351,1.540)--(4.383,1.518)%
  --(4.416,1.499)--(4.449,1.484)--(4.482,1.472)--(4.514,1.465)--(4.547,1.462)--(4.580,1.464)%
  --(4.613,1.470)--(4.645,1.480)--(4.678,1.494)--(4.711,1.512)--(4.743,1.534)--(4.776,1.558)%
  --(4.809,1.584)--(4.842,1.613)--(4.874,1.643)--(4.907,1.675)--(4.940,1.708)--(4.972,1.741)%
  --(5.005,1.775)--(5.038,1.809)--(5.071,1.843)--(5.103,1.877)--(5.136,1.911)--(5.169,1.945)%
  --(5.202,1.978)--(5.234,2.011)--(5.267,2.044)--(5.300,2.076)--(5.332,2.107)--(5.365,2.138)%
  --(5.398,2.168)--(5.431,2.198)--(5.463,2.227)--(5.496,2.256)--(5.529,2.284)--(5.561,2.311)%
  --(5.594,2.339)--(5.627,2.365)--(5.660,2.391)--(5.692,2.417)--(5.725,2.442)--(5.758,2.466)%
  --(5.790,2.490)--(5.823,2.514)--(5.856,2.537)--(5.889,2.560)--(5.921,2.582)--(5.954,2.604)%
  --(5.987,2.626)--(6.020,2.647)--(6.052,2.667)--(6.085,2.688)--(6.118,2.708)--(6.150,2.728)%
  --(6.183,2.747)--(6.216,2.766)--(6.249,2.785)--(6.281,2.804)--(6.314,2.822)--(6.347,2.840)%
  --(6.379,2.857)--(6.412,2.875)--(6.445,2.892)--(6.478,2.909)--(6.510,2.925)--(6.543,2.942)%
  --(6.576,2.958)--(6.609,2.974)--(6.641,2.989)--(6.674,3.005)--(6.707,3.020)--(6.739,3.035)%
  --(6.772,3.050)--(6.805,3.064)--(6.838,3.079)--(6.870,3.093)--(6.903,3.107)--(6.936,3.121)%
  --(6.968,3.135)--(7.001,3.148)--(7.034,3.162)--(7.067,3.175)--(7.099,3.188)--(7.132,3.201)%
  --(7.165,3.214)--(7.198,3.226)--(7.230,3.239)--(7.263,3.251)--(7.296,3.263)--(7.328,3.275)%
  --(7.361,3.287)--(7.394,3.299)--(7.427,3.311)--(7.459,3.322)--(7.492,3.334)--(7.525,3.345)%
  --(7.557,3.356)--(7.590,3.367)--(7.623,3.378)--(7.656,3.389)--(7.688,3.400)--(7.721,3.410)%
  --(7.754,3.421)--(7.786,3.431)--(7.819,3.442)--(7.852,3.452)--(7.885,3.462)--(7.917,3.472)%
  --(7.950,3.482)--(7.983,3.492)--(8.016,3.502)--(8.048,3.511);
\gpcolor{color=gp lt color 3}
\gpsetlinetype{gp lt plot 3}
\draw[gp path] (1.504,2.652)--(1.537,2.653)--(1.569,2.653)--(1.602,2.653)--(1.635,2.654)%
  --(1.668,2.655)--(1.700,2.657)--(1.733,2.658)--(1.766,2.660)--(1.798,2.662)--(1.831,2.665)%
  --(1.864,2.667)--(1.897,2.670)--(1.929,2.673)--(1.962,2.676)--(1.995,2.679)--(2.028,2.683)%
  --(2.060,2.687)--(2.093,2.691)--(2.126,2.695)--(2.158,2.699)--(2.191,2.704)--(2.224,2.708)%
  --(2.257,2.713)--(2.289,2.718)--(2.322,2.724)--(2.355,2.729)--(2.387,2.735)--(2.420,2.740)%
  --(2.453,2.746)--(2.486,2.752)--(2.518,2.758)--(2.551,2.764)--(2.584,2.771)--(2.617,2.777)%
  --(2.649,2.784)--(2.682,2.790)--(2.715,2.797)--(2.747,2.804)--(2.780,2.811)--(2.813,2.818)%
  --(2.846,2.825)--(2.878,2.832)--(2.911,2.839)--(2.944,2.847)--(2.976,2.852)--(3.009,2.847)%
  --(3.042,2.834)--(3.075,2.818)--(3.107,2.803)--(3.140,2.787)--(3.173,2.771)--(3.206,2.755)%
  --(3.238,2.739)--(3.271,2.723)--(3.304,2.707)--(3.336,2.690)--(3.369,2.674)--(3.402,2.657)%
  --(3.435,2.641)--(3.467,2.624)--(3.500,2.607)--(3.533,2.590)--(3.565,2.574)--(3.598,2.557)%
  --(3.631,2.540)--(3.664,2.522)--(3.696,2.505)--(3.729,2.488)--(3.762,2.471)--(3.794,2.454)%
  --(3.827,2.437)--(3.860,2.419)--(3.893,2.402)--(3.925,2.385)--(3.958,2.368)--(3.991,2.351)%
  --(4.024,2.334)--(4.056,2.318)--(4.089,2.301)--(4.122,2.284)--(4.154,2.268)--(4.187,2.252)%
  --(4.220,2.236)--(4.253,2.220)--(4.285,2.205)--(4.318,2.190)--(4.351,2.175)--(4.383,2.161)%
  --(4.416,2.147)--(4.449,2.133)--(4.482,2.120)--(4.514,2.108)--(4.547,2.096)--(4.580,2.084)%
  --(4.613,2.073)--(4.645,2.063)--(4.678,2.053)--(4.711,2.044)--(4.743,2.036)--(4.776,2.029)%
  --(4.809,2.022)--(4.842,2.016)--(4.874,2.011)--(4.907,2.007)--(4.940,2.004)--(4.972,2.001)%
  --(5.005,2.000)--(5.038,1.999)--(5.071,1.999)--(5.103,2.000)--(5.136,2.002)--(5.169,2.004)%
  --(5.202,2.008)--(5.234,2.012)--(5.267,2.017)--(5.300,2.023)--(5.332,2.030)--(5.365,2.037)%
  --(5.398,2.045)--(5.431,2.053)--(5.463,2.063)--(5.496,2.072)--(5.529,2.083)--(5.561,2.094)%
  --(5.594,2.105)--(5.627,2.117)--(5.660,2.129)--(5.692,2.142)--(5.725,2.155)--(5.758,2.168)%
  --(5.790,2.181)--(5.823,2.195)--(5.856,2.209)--(5.889,2.223)--(5.921,2.238)--(5.954,2.252)%
  --(5.987,2.267)--(6.020,2.282)--(6.052,2.297)--(6.085,2.312)--(6.118,2.327)--(6.150,2.342)%
  --(6.183,2.357)--(6.216,2.372)--(6.249,2.387)--(6.281,2.403)--(6.314,2.418)--(6.347,2.433)%
  --(6.379,2.448)--(6.412,2.463)--(6.445,2.478)--(6.478,2.493)--(6.510,2.508)--(6.543,2.523)%
  --(6.576,2.538)--(6.609,2.552)--(6.641,2.567)--(6.674,2.581)--(6.707,2.596)--(6.739,2.610)%
  --(6.772,2.624)--(6.805,2.639)--(6.838,2.653)--(6.870,2.667)--(6.903,2.680)--(6.936,2.694)%
  --(6.968,2.708)--(7.001,2.721)--(7.034,2.735)--(7.067,2.748)--(7.099,2.762)--(7.132,2.775)%
  --(7.165,2.788)--(7.198,2.801)--(7.230,2.814)--(7.263,2.826)--(7.296,2.839)--(7.328,2.851)%
  --(7.361,2.864)--(7.394,2.876)--(7.427,2.888)--(7.459,2.901)--(7.492,2.913)--(7.525,2.925)%
  --(7.557,2.936)--(7.590,2.948)--(7.623,2.960)--(7.656,2.971)--(7.688,2.983)--(7.721,2.994)%
  --(7.754,3.005)--(7.786,3.017)--(7.819,3.028)--(7.852,3.039)--(7.885,3.050)--(7.917,3.060)%
  --(7.950,3.071)--(7.983,3.082)--(8.016,3.092)--(8.048,3.103);
\gpcolor{color=gp lt color 4}
\gpsetlinetype{gp lt plot 4}
\draw[gp path] (1.504,2.652)--(1.537,2.653)--(1.569,2.657)--(1.602,2.662)--(1.635,2.669)%
  --(1.668,2.678)--(1.700,2.688)--(1.733,2.701)--(1.766,2.715)--(1.798,2.730)--(1.831,2.747)%
  --(1.864,2.764)--(1.897,2.783)--(1.929,2.803)--(1.962,2.823)--(1.995,2.844)--(2.028,2.865)%
  --(2.060,2.887)--(2.093,2.910)--(2.126,2.932)--(2.158,2.955)--(2.191,2.978)--(2.224,3.001)%
  --(2.257,3.024)--(2.289,3.047)--(2.322,3.070)--(2.355,3.093)--(2.387,3.116)--(2.420,3.139)%
  --(2.453,3.161)--(2.486,3.184)--(2.518,3.206)--(2.551,3.228)--(2.584,3.250)--(2.617,3.271)%
  --(2.649,3.292)--(2.682,3.314)--(2.715,3.334)--(2.747,3.355)--(2.780,3.375)--(2.813,3.396)%
  --(2.846,3.415)--(2.878,3.435)--(2.911,3.455)--(2.944,3.474)--(2.976,3.490)--(3.009,3.497)%
  --(3.042,3.495)--(3.075,3.491)--(3.107,3.488)--(3.140,3.484)--(3.173,3.481)--(3.206,3.477)%
  --(3.238,3.474)--(3.271,3.470)--(3.304,3.467)--(3.336,3.464)--(3.369,3.461)--(3.402,3.458)%
  --(3.435,3.455)--(3.467,3.453)--(3.500,3.450)--(3.533,3.448)--(3.565,3.445)--(3.598,3.443)%
  --(3.631,3.440)--(3.664,3.438)--(3.696,3.436)--(3.729,3.434)--(3.762,3.432)--(3.794,3.430)%
  --(3.827,3.428)--(3.860,3.426)--(3.893,3.425)--(3.925,3.423)--(3.958,3.422)--(3.991,3.420)%
  --(4.024,3.419)--(4.056,3.418)--(4.089,3.417)--(4.122,3.416)--(4.154,3.415)--(4.187,3.414)%
  --(4.220,3.413)--(4.253,3.412)--(4.285,3.411)--(4.318,3.411)--(4.351,3.410)--(4.383,3.410)%
  --(4.416,3.409)--(4.449,3.409)--(4.482,3.409)--(4.514,3.409)--(4.547,3.409)--(4.580,3.409)%
  --(4.613,3.409)--(4.645,3.409)--(4.678,3.409)--(4.711,3.409)--(4.743,3.409)--(4.776,3.410)%
  --(4.809,3.410)--(4.842,3.411)--(4.874,3.411)--(4.907,3.412)--(4.940,3.413)--(4.972,3.413)%
  --(5.005,3.414)--(5.038,3.415)--(5.071,3.416)--(5.103,3.417)--(5.136,3.418)--(5.169,3.419)%
  --(5.202,3.420)--(5.234,3.422)--(5.267,3.423)--(5.300,3.424)--(5.332,3.426)--(5.365,3.427)%
  --(5.398,3.429)--(5.431,3.430)--(5.463,3.432)--(5.496,3.433)--(5.529,3.435)--(5.561,3.437)%
  --(5.594,3.439)--(5.627,3.440)--(5.660,3.442)--(5.692,3.444)--(5.725,3.446)--(5.758,3.448)%
  --(5.790,3.450)--(5.823,3.452)--(5.856,3.454)--(5.889,3.457)--(5.921,3.459)--(5.954,3.461)%
  --(5.987,3.463)--(6.020,3.466)--(6.052,3.468)--(6.085,3.470)--(6.118,3.473)--(6.150,3.475)%
  --(6.183,3.478)--(6.216,3.480)--(6.249,3.483)--(6.281,3.486)--(6.314,3.488)--(6.347,3.491)%
  --(6.379,3.494)--(6.412,3.496)--(6.445,3.499)--(6.478,3.502)--(6.510,3.505)--(6.543,3.508)%
  --(6.576,3.510)--(6.609,3.513)--(6.641,3.516)--(6.674,3.519)--(6.707,3.522)--(6.739,3.525)%
  --(6.772,3.528)--(6.805,3.531)--(6.838,3.534)--(6.870,3.537)--(6.903,3.540)--(6.936,3.544)%
  --(6.968,3.547)--(7.001,3.550)--(7.034,3.553)--(7.067,3.556)--(7.099,3.559)--(7.132,3.563)%
  --(7.165,3.566)--(7.198,3.569)--(7.230,3.573)--(7.263,3.576)--(7.296,3.579)--(7.328,3.582)%
  --(7.361,3.586)--(7.394,3.589)--(7.427,3.593)--(7.459,3.596)--(7.492,3.599)--(7.525,3.603)%
  --(7.557,3.606)--(7.590,3.610)--(7.623,3.613)--(7.656,3.617)--(7.688,3.620)--(7.721,3.624)%
  --(7.754,3.627)--(7.786,3.631)--(7.819,3.634)--(7.852,3.638)--(7.885,3.641)--(7.917,3.645)%
  --(7.950,3.648)--(7.983,3.652)--(8.016,3.655)--(8.048,3.659);
\gpcolor{color=gp lt color border}
\gpsetlinetype{gp lt border}
\draw[gp path] (1.504,5.756)--(1.504,0.985)--(8.197,0.985)--(8.197,5.756)--cycle;
%% coordinates of the plot area
\gpdefrectangularnode{gp plot 1}{\pgfpoint{1.504cm}{0.985cm}}{\pgfpoint{8.197cm}{5.756cm}}
%% \end{tikzpicture}
%% gnuplot variables

        \node at (5.9,1.65) {$N=10^5$};
        \node at (4,3) {$N=10^6$};
        \node at (3,4) {$N=10^7$};
      \end{tikzpicture}
    }
    \subfloat[][] {
      \label{fig:focus_long_prolate}
      \begin{tikzpicture}
        %% \begin{tikzpicture}[gnuplot]
%% generated with GNUPLOT 4.6p0 (Lua 5.1; terminal rev. 99, script rev. 100)
%% Tue 26 Mar 2013 03:01:41 PM CDT
\gpsolidlines
\path (0.000,0.000) rectangle (8.750,6.125);
\gpcolor{color=gp lt color border}
\gpsetlinetype{gp lt border}
\gpsetlinewidth{1.00}
\draw[gp path] (1.504,0.985)--(1.684,0.985);
\draw[gp path] (8.197,0.985)--(8.017,0.985);
\node[gp node right] at (1.320,0.985) { 0.1};
\draw[gp path] (1.504,1.703)--(1.594,1.703);
\draw[gp path] (8.197,1.703)--(8.107,1.703);
\draw[gp path] (1.504,2.123)--(1.594,2.123);
\draw[gp path] (8.197,2.123)--(8.107,2.123);
\draw[gp path] (1.504,2.421)--(1.594,2.421);
\draw[gp path] (8.197,2.421)--(8.107,2.421);
\draw[gp path] (1.504,2.652)--(1.594,2.652);
\draw[gp path] (8.197,2.652)--(8.107,2.652);
\draw[gp path] (1.504,2.841)--(1.594,2.841);
\draw[gp path] (8.197,2.841)--(8.107,2.841);
\draw[gp path] (1.504,3.001)--(1.594,3.001);
\draw[gp path] (8.197,3.001)--(8.107,3.001);
\draw[gp path] (1.504,3.139)--(1.594,3.139);
\draw[gp path] (8.197,3.139)--(8.107,3.139);
\draw[gp path] (1.504,3.261)--(1.594,3.261);
\draw[gp path] (8.197,3.261)--(8.107,3.261);
\draw[gp path] (1.504,3.371)--(1.684,3.371);
\draw[gp path] (8.197,3.371)--(8.017,3.371);
\node[gp node right] at (1.320,3.371) { 1};
\draw[gp path] (1.504,4.089)--(1.594,4.089);
\draw[gp path] (8.197,4.089)--(8.107,4.089);
\draw[gp path] (1.504,4.509)--(1.594,4.509);
\draw[gp path] (8.197,4.509)--(8.107,4.509);
\draw[gp path] (1.504,4.807)--(1.594,4.807);
\draw[gp path] (8.197,4.807)--(8.107,4.807);
\draw[gp path] (1.504,5.038)--(1.594,5.038);
\draw[gp path] (8.197,5.038)--(8.107,5.038);
\draw[gp path] (1.504,5.227)--(1.594,5.227);
\draw[gp path] (8.197,5.227)--(8.107,5.227);
\draw[gp path] (1.504,5.386)--(1.594,5.386);
\draw[gp path] (8.197,5.386)--(8.107,5.386);
\draw[gp path] (1.504,5.525)--(1.594,5.525);
\draw[gp path] (8.197,5.525)--(8.107,5.525);
\draw[gp path] (1.504,5.647)--(1.594,5.647);
\draw[gp path] (8.197,5.647)--(8.107,5.647);
\draw[gp path] (1.504,5.756)--(1.684,5.756);
\draw[gp path] (8.197,5.756)--(8.017,5.756);
\node[gp node right] at (1.320,5.756) { 10};
\draw[gp path] (1.504,0.985)--(1.504,1.165);
\draw[gp path] (1.504,5.756)--(1.504,5.576);
\node[gp node center] at (1.504,0.677) {-1};
\draw[gp path] (2.248,0.985)--(2.248,1.165);
\draw[gp path] (2.248,5.756)--(2.248,5.576);
\node[gp node center] at (2.248,0.677) {-0.5};
\draw[gp path] (2.991,0.985)--(2.991,1.165);
\draw[gp path] (2.991,5.756)--(2.991,5.576);
\node[gp node center] at (2.991,0.677) { 0};
\draw[gp path] (3.735,0.985)--(3.735,1.165);
\draw[gp path] (3.735,5.756)--(3.735,5.576);
\node[gp node center] at (3.735,0.677) { 0.5};
\draw[gp path] (4.479,0.985)--(4.479,1.165);
\draw[gp path] (4.479,5.756)--(4.479,5.576);
\node[gp node center] at (4.479,0.677) { 1};
\draw[gp path] (5.222,0.985)--(5.222,1.165);
\draw[gp path] (5.222,5.756)--(5.222,5.576);
\node[gp node center] at (5.222,0.677) { 1.5};
\draw[gp path] (5.966,0.985)--(5.966,1.165);
\draw[gp path] (5.966,5.756)--(5.966,5.576);
\node[gp node center] at (5.966,0.677) { 2};
\draw[gp path] (6.710,0.985)--(6.710,1.165);
\draw[gp path] (6.710,5.756)--(6.710,5.576);
\node[gp node center] at (6.710,0.677) { 2.5};
\draw[gp path] (7.453,0.985)--(7.453,1.165);
\draw[gp path] (7.453,5.756)--(7.453,5.576);
\node[gp node center] at (7.453,0.677) { 3};
\draw[gp path] (8.197,0.985)--(8.197,1.165);
\draw[gp path] (8.197,5.756)--(8.197,5.576);
\node[gp node center] at (8.197,0.677) { 3.5};
\draw[gp path] (1.504,5.756)--(1.504,0.985)--(8.197,0.985)--(8.197,5.756)--cycle;
\node[gp node center,rotate=-270] at (0.246,3.370) {HW1/eM beam width (mm)};
\node[gp node center] at (4.850,0.215) {Pulse location rel. to $f$=60mm lens};
\gpcolor{color=gp lt color 0}
\gpsetlinetype{gp lt plot 0}
\draw[gp path] (1.504,2.652)--(1.537,2.652)--(1.569,2.652)--(1.602,2.653)--(1.635,2.653)%
  --(1.668,2.653)--(1.700,2.653)--(1.733,2.653)--(1.766,2.654)--(1.798,2.654)--(1.831,2.655)%
  --(1.864,2.655)--(1.897,2.655)--(1.929,2.656)--(1.962,2.657)--(1.995,2.657)--(2.028,2.658)%
  --(2.060,2.658)--(2.093,2.659)--(2.126,2.660)--(2.158,2.661)--(2.191,2.662)--(2.224,2.663)%
  --(2.257,2.663)--(2.289,2.664)--(2.322,2.665)--(2.355,2.667)--(2.387,2.668)--(2.420,2.669)%
  --(2.453,2.670)--(2.486,2.671)--(2.518,2.672)--(2.551,2.674)--(2.584,2.675)--(2.617,2.676)%
  --(2.649,2.678)--(2.682,2.679)--(2.715,2.681)--(2.747,2.682)--(2.780,2.684)--(2.813,2.685)%
  --(2.846,2.687)--(2.878,2.689)--(2.911,2.690)--(2.944,2.692)--(2.976,2.691)--(3.009,2.681)%
  --(3.042,2.662)--(3.075,2.640)--(3.107,2.618)--(3.140,2.595)--(3.173,2.572)--(3.206,2.549)%
  --(3.238,2.525)--(3.271,2.500)--(3.304,2.475)--(3.336,2.450)--(3.369,2.424)--(3.402,2.397)%
  --(3.435,2.370)--(3.467,2.342)--(3.500,2.314)--(3.533,2.285)--(3.565,2.255)--(3.598,2.225)%
  --(3.631,2.194)--(3.664,2.162)--(3.696,2.129)--(3.729,2.096)--(3.762,2.062)--(3.794,2.027)%
  --(3.827,1.992)--(3.860,1.956)--(3.893,1.919)--(3.925,1.882)--(3.958,1.844)--(3.991,1.806)%
  --(4.024,1.767)--(4.056,1.728)--(4.089,1.690)--(4.122,1.651)--(4.154,1.614)--(4.187,1.577)%
  --(4.220,1.542)--(4.253,1.508)--(4.285,1.477)--(4.318,1.450)--(4.351,1.426)--(4.383,1.406)%
  --(4.416,1.392)--(4.449,1.382)--(4.482,1.379)--(4.514,1.381)--(4.547,1.389)--(4.580,1.402)%
  --(4.613,1.420)--(4.645,1.443)--(4.678,1.470)--(4.711,1.500)--(4.743,1.533)--(4.776,1.568)%
  --(4.809,1.604)--(4.842,1.642)--(4.874,1.680)--(4.907,1.719)--(4.940,1.758)--(4.972,1.796)%
  --(5.005,1.835)--(5.038,1.873)--(5.071,1.910)--(5.103,1.947)--(5.136,1.983)--(5.169,2.019)%
  --(5.202,2.053)--(5.234,2.088)--(5.267,2.121)--(5.300,2.154)--(5.332,2.186)--(5.365,2.217)%
  --(5.398,2.247)--(5.431,2.277)--(5.463,2.307)--(5.496,2.335)--(5.529,2.363)--(5.561,2.390)%
  --(5.594,2.417)--(5.627,2.443)--(5.660,2.469)--(5.692,2.494)--(5.725,2.519)--(5.758,2.543)%
  --(5.790,2.566)--(5.823,2.590)--(5.856,2.612)--(5.889,2.634)--(5.921,2.656)--(5.954,2.678)%
  --(5.987,2.699)--(6.020,2.719)--(6.052,2.740)--(6.085,2.759)--(6.118,2.779)--(6.150,2.798)%
  --(6.183,2.817)--(6.216,2.836)--(6.249,2.854)--(6.281,2.872)--(6.314,2.890)--(6.347,2.907)%
  --(6.379,2.924)--(6.412,2.941)--(6.445,2.958)--(6.478,2.974)--(6.510,2.990)--(6.543,3.006)%
  --(6.576,3.022)--(6.609,3.037)--(6.641,3.052)--(6.674,3.067)--(6.707,3.082)--(6.739,3.097)%
  --(6.772,3.111)--(6.805,3.125)--(6.838,3.139)--(6.870,3.153)--(6.903,3.167)--(6.936,3.180)%
  --(6.968,3.194)--(7.001,3.207)--(7.034,3.220)--(7.067,3.233)--(7.099,3.245)--(7.132,3.258)%
  --(7.165,3.270)--(7.198,3.282)--(7.230,3.295)--(7.263,3.307)--(7.296,3.318)--(7.328,3.330)%
  --(7.361,3.342)--(7.394,3.353)--(7.427,3.364)--(7.459,3.376)--(7.492,3.387)--(7.525,3.398)%
  --(7.557,3.409)--(7.590,3.419)--(7.623,3.430)--(7.656,3.441)--(7.688,3.451)--(7.721,3.461)%
  --(7.754,3.472)--(7.786,3.482)--(7.819,3.492)--(7.852,3.502)--(7.885,3.512)--(7.917,3.521)%
  --(7.950,3.531)--(7.983,3.541)--(8.016,3.550)--(8.048,3.560);
\gpcolor{color=gp lt color 1}
\gpsetlinetype{gp lt plot 1}
\draw[gp path] (1.504,2.652)--(1.537,2.652)--(1.569,2.652)--(1.602,2.653)--(1.635,2.653)%
  --(1.668,2.653)--(1.700,2.653)--(1.733,2.653)--(1.766,2.654)--(1.798,2.654)--(1.831,2.655)%
  --(1.864,2.655)--(1.897,2.656)--(1.929,2.656)--(1.962,2.657)--(1.995,2.657)--(2.028,2.658)%
  --(2.060,2.659)--(2.093,2.660)--(2.126,2.660)--(2.158,2.661)--(2.191,2.662)--(2.224,2.663)%
  --(2.257,2.664)--(2.289,2.665)--(2.322,2.666)--(2.355,2.667)--(2.387,2.668)--(2.420,2.670)%
  --(2.453,2.671)--(2.486,2.672)--(2.518,2.673)--(2.551,2.675)--(2.584,2.676)--(2.617,2.677)%
  --(2.649,2.679)--(2.682,2.680)--(2.715,2.682)--(2.747,2.684)--(2.780,2.685)--(2.813,2.687)%
  --(2.846,2.688)--(2.878,2.690)--(2.911,2.692)--(2.944,2.694)--(2.976,2.693)--(3.009,2.683)%
  --(3.042,2.664)--(3.075,2.642)--(3.107,2.620)--(3.140,2.598)--(3.173,2.575)--(3.206,2.551)%
  --(3.238,2.527)--(3.271,2.503)--(3.304,2.478)--(3.336,2.453)--(3.369,2.427)--(3.402,2.401)%
  --(3.435,2.374)--(3.467,2.346)--(3.500,2.318)--(3.533,2.289)--(3.565,2.259)--(3.598,2.229)%
  --(3.631,2.198)--(3.664,2.167)--(3.696,2.135)--(3.729,2.102)--(3.762,2.068)--(3.794,2.034)%
  --(3.827,1.999)--(3.860,1.963)--(3.893,1.927)--(3.925,1.890)--(3.958,1.853)--(3.991,1.815)%
  --(4.024,1.777)--(4.056,1.738)--(4.089,1.700)--(4.122,1.662)--(4.154,1.625)--(4.187,1.589)%
  --(4.220,1.554)--(4.253,1.520)--(4.285,1.490)--(4.318,1.462)--(4.351,1.438)--(4.383,1.418)%
  --(4.416,1.403)--(4.449,1.392)--(4.482,1.388)--(4.514,1.389)--(4.547,1.395)--(4.580,1.407)%
  --(4.613,1.424)--(4.645,1.446)--(4.678,1.471)--(4.711,1.500)--(4.743,1.532)--(4.776,1.565)%
  --(4.809,1.601)--(4.842,1.638)--(4.874,1.675)--(4.907,1.713)--(4.940,1.751)--(4.972,1.790)%
  --(5.005,1.828)--(5.038,1.865)--(5.071,1.902)--(5.103,1.939)--(5.136,1.975)--(5.169,2.011)%
  --(5.202,2.045)--(5.234,2.079)--(5.267,2.113)--(5.300,2.145)--(5.332,2.177)--(5.365,2.209)%
  --(5.398,2.239)--(5.431,2.269)--(5.463,2.298)--(5.496,2.327)--(5.529,2.355)--(5.561,2.382)%
  --(5.594,2.409)--(5.627,2.436)--(5.660,2.461)--(5.692,2.486)--(5.725,2.511)--(5.758,2.535)%
  --(5.790,2.559)--(5.823,2.582)--(5.856,2.605)--(5.889,2.627)--(5.921,2.649)--(5.954,2.670)%
  --(5.987,2.691)--(6.020,2.712)--(6.052,2.732)--(6.085,2.752)--(6.118,2.772)--(6.150,2.791)%
  --(6.183,2.810)--(6.216,2.829)--(6.249,2.847)--(6.281,2.865)--(6.314,2.883)--(6.347,2.900)%
  --(6.379,2.918)--(6.412,2.935)--(6.445,2.951)--(6.478,2.968)--(6.510,2.984)--(6.543,3.000)%
  --(6.576,3.015)--(6.609,3.031)--(6.641,3.046)--(6.674,3.061)--(6.707,3.076)--(6.739,3.091)%
  --(6.772,3.105)--(6.805,3.119)--(6.838,3.134)--(6.870,3.147)--(6.903,3.161)--(6.936,3.175)%
  --(6.968,3.188)--(7.001,3.201)--(7.034,3.214)--(7.067,3.227)--(7.099,3.240)--(7.132,3.253)%
  --(7.165,3.265)--(7.198,3.277)--(7.230,3.289)--(7.263,3.301)--(7.296,3.313)--(7.328,3.325)%
  --(7.361,3.337)--(7.394,3.348)--(7.427,3.360)--(7.459,3.371)--(7.492,3.382)--(7.525,3.393)%
  --(7.557,3.404)--(7.590,3.415)--(7.623,3.425)--(7.656,3.436)--(7.688,3.446)--(7.721,3.457)%
  --(7.754,3.467)--(7.786,3.477)--(7.819,3.487)--(7.852,3.497)--(7.885,3.507)--(7.917,3.517)%
  --(7.950,3.527)--(7.983,3.536)--(8.016,3.546)--(8.048,3.555);
\gpcolor{color=gp lt color 2}
\gpsetlinetype{gp lt plot 2}
\draw[gp path] (1.504,2.652)--(1.537,2.652)--(1.569,2.653)--(1.602,2.653)--(1.635,2.653)%
  --(1.668,2.653)--(1.700,2.654)--(1.733,2.654)--(1.766,2.654)--(1.798,2.655)--(1.831,2.656)%
  --(1.864,2.656)--(1.897,2.657)--(1.929,2.658)--(1.962,2.659)--(1.995,2.659)--(2.028,2.660)%
  --(2.060,2.661)--(2.093,2.662)--(2.126,2.664)--(2.158,2.665)--(2.191,2.666)--(2.224,2.667)%
  --(2.257,2.669)--(2.289,2.670)--(2.322,2.672)--(2.355,2.673)--(2.387,2.675)--(2.420,2.676)%
  --(2.453,2.678)--(2.486,2.680)--(2.518,2.682)--(2.551,2.684)--(2.584,2.685)--(2.617,2.687)%
  --(2.649,2.689)--(2.682,2.691)--(2.715,2.694)--(2.747,2.696)--(2.780,2.698)--(2.813,2.700)%
  --(2.846,2.702)--(2.878,2.705)--(2.911,2.707)--(2.944,2.710)--(2.976,2.710)--(3.009,2.700)%
  --(3.042,2.682)--(3.075,2.661)--(3.107,2.640)--(3.140,2.618)--(3.173,2.596)--(3.206,2.573)%
  --(3.238,2.551)--(3.271,2.527)--(3.304,2.504)--(3.336,2.480)--(3.369,2.455)--(3.402,2.430)%
  --(3.435,2.404)--(3.467,2.378)--(3.500,2.352)--(3.533,2.325)--(3.565,2.297)--(3.598,2.269)%
  --(3.631,2.240)--(3.664,2.211)--(3.696,2.181)--(3.729,2.150)--(3.762,2.120)--(3.794,2.088)%
  --(3.827,2.056)--(3.860,2.024)--(3.893,1.991)--(3.925,1.957)--(3.958,1.923)--(3.991,1.889)%
  --(4.024,1.855)--(4.056,1.821)--(4.089,1.786)--(4.122,1.752)--(4.154,1.718)--(4.187,1.685)%
  --(4.220,1.653)--(4.253,1.622)--(4.285,1.593)--(4.318,1.565)--(4.351,1.540)--(4.383,1.518)%
  --(4.416,1.499)--(4.449,1.484)--(4.482,1.472)--(4.514,1.465)--(4.547,1.462)--(4.580,1.464)%
  --(4.613,1.470)--(4.645,1.480)--(4.678,1.494)--(4.711,1.512)--(4.743,1.534)--(4.776,1.558)%
  --(4.809,1.584)--(4.842,1.613)--(4.874,1.643)--(4.907,1.675)--(4.940,1.708)--(4.972,1.741)%
  --(5.005,1.775)--(5.038,1.809)--(5.071,1.843)--(5.103,1.877)--(5.136,1.911)--(5.169,1.945)%
  --(5.202,1.978)--(5.234,2.011)--(5.267,2.044)--(5.300,2.076)--(5.332,2.107)--(5.365,2.138)%
  --(5.398,2.168)--(5.431,2.198)--(5.463,2.227)--(5.496,2.256)--(5.529,2.284)--(5.561,2.311)%
  --(5.594,2.339)--(5.627,2.365)--(5.660,2.391)--(5.692,2.417)--(5.725,2.442)--(5.758,2.466)%
  --(5.790,2.490)--(5.823,2.514)--(5.856,2.537)--(5.889,2.560)--(5.921,2.582)--(5.954,2.604)%
  --(5.987,2.626)--(6.020,2.647)--(6.052,2.667)--(6.085,2.688)--(6.118,2.708)--(6.150,2.728)%
  --(6.183,2.747)--(6.216,2.766)--(6.249,2.785)--(6.281,2.804)--(6.314,2.822)--(6.347,2.840)%
  --(6.379,2.857)--(6.412,2.875)--(6.445,2.892)--(6.478,2.909)--(6.510,2.925)--(6.543,2.942)%
  --(6.576,2.958)--(6.609,2.974)--(6.641,2.989)--(6.674,3.005)--(6.707,3.020)--(6.739,3.035)%
  --(6.772,3.050)--(6.805,3.064)--(6.838,3.079)--(6.870,3.093)--(6.903,3.107)--(6.936,3.121)%
  --(6.968,3.135)--(7.001,3.148)--(7.034,3.162)--(7.067,3.175)--(7.099,3.188)--(7.132,3.201)%
  --(7.165,3.214)--(7.198,3.226)--(7.230,3.239)--(7.263,3.251)--(7.296,3.263)--(7.328,3.275)%
  --(7.361,3.287)--(7.394,3.299)--(7.427,3.311)--(7.459,3.322)--(7.492,3.334)--(7.525,3.345)%
  --(7.557,3.356)--(7.590,3.367)--(7.623,3.378)--(7.656,3.389)--(7.688,3.400)--(7.721,3.410)%
  --(7.754,3.421)--(7.786,3.431)--(7.819,3.442)--(7.852,3.452)--(7.885,3.462)--(7.917,3.472)%
  --(7.950,3.482)--(7.983,3.492)--(8.016,3.502)--(8.048,3.511);
\gpcolor{color=gp lt color 3}
\gpsetlinetype{gp lt plot 3}
\draw[gp path] (1.504,2.652)--(1.537,2.653)--(1.569,2.653)--(1.602,2.653)--(1.635,2.654)%
  --(1.668,2.655)--(1.700,2.657)--(1.733,2.658)--(1.766,2.660)--(1.798,2.662)--(1.831,2.665)%
  --(1.864,2.667)--(1.897,2.670)--(1.929,2.673)--(1.962,2.676)--(1.995,2.679)--(2.028,2.683)%
  --(2.060,2.687)--(2.093,2.691)--(2.126,2.695)--(2.158,2.699)--(2.191,2.704)--(2.224,2.708)%
  --(2.257,2.713)--(2.289,2.718)--(2.322,2.724)--(2.355,2.729)--(2.387,2.735)--(2.420,2.740)%
  --(2.453,2.746)--(2.486,2.752)--(2.518,2.758)--(2.551,2.764)--(2.584,2.771)--(2.617,2.777)%
  --(2.649,2.784)--(2.682,2.790)--(2.715,2.797)--(2.747,2.804)--(2.780,2.811)--(2.813,2.818)%
  --(2.846,2.825)--(2.878,2.832)--(2.911,2.839)--(2.944,2.847)--(2.976,2.852)--(3.009,2.847)%
  --(3.042,2.834)--(3.075,2.818)--(3.107,2.803)--(3.140,2.787)--(3.173,2.771)--(3.206,2.755)%
  --(3.238,2.739)--(3.271,2.723)--(3.304,2.707)--(3.336,2.690)--(3.369,2.674)--(3.402,2.657)%
  --(3.435,2.641)--(3.467,2.624)--(3.500,2.607)--(3.533,2.590)--(3.565,2.574)--(3.598,2.557)%
  --(3.631,2.540)--(3.664,2.522)--(3.696,2.505)--(3.729,2.488)--(3.762,2.471)--(3.794,2.454)%
  --(3.827,2.437)--(3.860,2.419)--(3.893,2.402)--(3.925,2.385)--(3.958,2.368)--(3.991,2.351)%
  --(4.024,2.334)--(4.056,2.318)--(4.089,2.301)--(4.122,2.284)--(4.154,2.268)--(4.187,2.252)%
  --(4.220,2.236)--(4.253,2.220)--(4.285,2.205)--(4.318,2.190)--(4.351,2.175)--(4.383,2.161)%
  --(4.416,2.147)--(4.449,2.133)--(4.482,2.120)--(4.514,2.108)--(4.547,2.096)--(4.580,2.084)%
  --(4.613,2.073)--(4.645,2.063)--(4.678,2.053)--(4.711,2.044)--(4.743,2.036)--(4.776,2.029)%
  --(4.809,2.022)--(4.842,2.016)--(4.874,2.011)--(4.907,2.007)--(4.940,2.004)--(4.972,2.001)%
  --(5.005,2.000)--(5.038,1.999)--(5.071,1.999)--(5.103,2.000)--(5.136,2.002)--(5.169,2.004)%
  --(5.202,2.008)--(5.234,2.012)--(5.267,2.017)--(5.300,2.023)--(5.332,2.030)--(5.365,2.037)%
  --(5.398,2.045)--(5.431,2.053)--(5.463,2.063)--(5.496,2.072)--(5.529,2.083)--(5.561,2.094)%
  --(5.594,2.105)--(5.627,2.117)--(5.660,2.129)--(5.692,2.142)--(5.725,2.155)--(5.758,2.168)%
  --(5.790,2.181)--(5.823,2.195)--(5.856,2.209)--(5.889,2.223)--(5.921,2.238)--(5.954,2.252)%
  --(5.987,2.267)--(6.020,2.282)--(6.052,2.297)--(6.085,2.312)--(6.118,2.327)--(6.150,2.342)%
  --(6.183,2.357)--(6.216,2.372)--(6.249,2.387)--(6.281,2.403)--(6.314,2.418)--(6.347,2.433)%
  --(6.379,2.448)--(6.412,2.463)--(6.445,2.478)--(6.478,2.493)--(6.510,2.508)--(6.543,2.523)%
  --(6.576,2.538)--(6.609,2.552)--(6.641,2.567)--(6.674,2.581)--(6.707,2.596)--(6.739,2.610)%
  --(6.772,2.624)--(6.805,2.639)--(6.838,2.653)--(6.870,2.667)--(6.903,2.680)--(6.936,2.694)%
  --(6.968,2.708)--(7.001,2.721)--(7.034,2.735)--(7.067,2.748)--(7.099,2.762)--(7.132,2.775)%
  --(7.165,2.788)--(7.198,2.801)--(7.230,2.814)--(7.263,2.826)--(7.296,2.839)--(7.328,2.851)%
  --(7.361,2.864)--(7.394,2.876)--(7.427,2.888)--(7.459,2.901)--(7.492,2.913)--(7.525,2.925)%
  --(7.557,2.936)--(7.590,2.948)--(7.623,2.960)--(7.656,2.971)--(7.688,2.983)--(7.721,2.994)%
  --(7.754,3.005)--(7.786,3.017)--(7.819,3.028)--(7.852,3.039)--(7.885,3.050)--(7.917,3.060)%
  --(7.950,3.071)--(7.983,3.082)--(8.016,3.092)--(8.048,3.103);
\gpcolor{color=gp lt color 4}
\gpsetlinetype{gp lt plot 4}
\draw[gp path] (1.504,2.652)--(1.537,2.653)--(1.569,2.657)--(1.602,2.662)--(1.635,2.669)%
  --(1.668,2.678)--(1.700,2.688)--(1.733,2.701)--(1.766,2.715)--(1.798,2.730)--(1.831,2.747)%
  --(1.864,2.764)--(1.897,2.783)--(1.929,2.803)--(1.962,2.823)--(1.995,2.844)--(2.028,2.865)%
  --(2.060,2.887)--(2.093,2.910)--(2.126,2.932)--(2.158,2.955)--(2.191,2.978)--(2.224,3.001)%
  --(2.257,3.024)--(2.289,3.047)--(2.322,3.070)--(2.355,3.093)--(2.387,3.116)--(2.420,3.139)%
  --(2.453,3.161)--(2.486,3.184)--(2.518,3.206)--(2.551,3.228)--(2.584,3.250)--(2.617,3.271)%
  --(2.649,3.292)--(2.682,3.314)--(2.715,3.334)--(2.747,3.355)--(2.780,3.375)--(2.813,3.396)%
  --(2.846,3.415)--(2.878,3.435)--(2.911,3.455)--(2.944,3.474)--(2.976,3.490)--(3.009,3.497)%
  --(3.042,3.495)--(3.075,3.491)--(3.107,3.488)--(3.140,3.484)--(3.173,3.481)--(3.206,3.477)%
  --(3.238,3.474)--(3.271,3.470)--(3.304,3.467)--(3.336,3.464)--(3.369,3.461)--(3.402,3.458)%
  --(3.435,3.455)--(3.467,3.453)--(3.500,3.450)--(3.533,3.448)--(3.565,3.445)--(3.598,3.443)%
  --(3.631,3.440)--(3.664,3.438)--(3.696,3.436)--(3.729,3.434)--(3.762,3.432)--(3.794,3.430)%
  --(3.827,3.428)--(3.860,3.426)--(3.893,3.425)--(3.925,3.423)--(3.958,3.422)--(3.991,3.420)%
  --(4.024,3.419)--(4.056,3.418)--(4.089,3.417)--(4.122,3.416)--(4.154,3.415)--(4.187,3.414)%
  --(4.220,3.413)--(4.253,3.412)--(4.285,3.411)--(4.318,3.411)--(4.351,3.410)--(4.383,3.410)%
  --(4.416,3.409)--(4.449,3.409)--(4.482,3.409)--(4.514,3.409)--(4.547,3.409)--(4.580,3.409)%
  --(4.613,3.409)--(4.645,3.409)--(4.678,3.409)--(4.711,3.409)--(4.743,3.409)--(4.776,3.410)%
  --(4.809,3.410)--(4.842,3.411)--(4.874,3.411)--(4.907,3.412)--(4.940,3.413)--(4.972,3.413)%
  --(5.005,3.414)--(5.038,3.415)--(5.071,3.416)--(5.103,3.417)--(5.136,3.418)--(5.169,3.419)%
  --(5.202,3.420)--(5.234,3.422)--(5.267,3.423)--(5.300,3.424)--(5.332,3.426)--(5.365,3.427)%
  --(5.398,3.429)--(5.431,3.430)--(5.463,3.432)--(5.496,3.433)--(5.529,3.435)--(5.561,3.437)%
  --(5.594,3.439)--(5.627,3.440)--(5.660,3.442)--(5.692,3.444)--(5.725,3.446)--(5.758,3.448)%
  --(5.790,3.450)--(5.823,3.452)--(5.856,3.454)--(5.889,3.457)--(5.921,3.459)--(5.954,3.461)%
  --(5.987,3.463)--(6.020,3.466)--(6.052,3.468)--(6.085,3.470)--(6.118,3.473)--(6.150,3.475)%
  --(6.183,3.478)--(6.216,3.480)--(6.249,3.483)--(6.281,3.486)--(6.314,3.488)--(6.347,3.491)%
  --(6.379,3.494)--(6.412,3.496)--(6.445,3.499)--(6.478,3.502)--(6.510,3.505)--(6.543,3.508)%
  --(6.576,3.510)--(6.609,3.513)--(6.641,3.516)--(6.674,3.519)--(6.707,3.522)--(6.739,3.525)%
  --(6.772,3.528)--(6.805,3.531)--(6.838,3.534)--(6.870,3.537)--(6.903,3.540)--(6.936,3.544)%
  --(6.968,3.547)--(7.001,3.550)--(7.034,3.553)--(7.067,3.556)--(7.099,3.559)--(7.132,3.563)%
  --(7.165,3.566)--(7.198,3.569)--(7.230,3.573)--(7.263,3.576)--(7.296,3.579)--(7.328,3.582)%
  --(7.361,3.586)--(7.394,3.589)--(7.427,3.593)--(7.459,3.596)--(7.492,3.599)--(7.525,3.603)%
  --(7.557,3.606)--(7.590,3.610)--(7.623,3.613)--(7.656,3.617)--(7.688,3.620)--(7.721,3.624)%
  --(7.754,3.627)--(7.786,3.631)--(7.819,3.634)--(7.852,3.638)--(7.885,3.641)--(7.917,3.645)%
  --(7.950,3.648)--(7.983,3.652)--(8.016,3.655)--(8.048,3.659);
\gpcolor{color=gp lt color border}
\gpsetlinetype{gp lt border}
\draw[gp path] (1.504,5.756)--(1.504,0.985)--(8.197,0.985)--(8.197,5.756)--cycle;
%% coordinates of the plot area
\gpdefrectangularnode{gp plot 1}{\pgfpoint{1.504cm}{0.985cm}}{\pgfpoint{8.197cm}{5.756cm}}
%% \end{tikzpicture}
%% gnuplot variables

        \node at (4.2,2.3) {$N=10^5$};
        \node at (3.3,3.5) {$N=10^6$};
      \end{tikzpicture}
    }
  }
  \centerline{
    \subfloat[][] { 
      \label{fig:focus_short_oblate}
      \begin{tikzpicture}
        %% \begin{tikzpicture}[gnuplot]
%% generated with GNUPLOT 4.6p0 (Lua 5.1; terminal rev. 99, script rev. 100)
%% Tue 26 Mar 2013 08:38:13 PM CDT
\gpsolidlines
\path (0.000,0.000) rectangle (8.750,6.125);
\gpcolor{color=gp lt color border}
\gpsetlinetype{gp lt border}
\gpsetlinewidth{1.00}
\draw[gp path] (1.688,0.985)--(1.868,0.985);
\draw[gp path] (8.197,0.985)--(8.017,0.985);
\node[gp node right] at (1.504,0.985) { 0.01};
\draw[gp path] (1.688,1.464)--(1.778,1.464);
\draw[gp path] (8.197,1.464)--(8.107,1.464);
\draw[gp path] (1.688,1.744)--(1.778,1.744);
\draw[gp path] (8.197,1.744)--(8.107,1.744);
\draw[gp path] (1.688,1.942)--(1.778,1.942);
\draw[gp path] (8.197,1.942)--(8.107,1.942);
\draw[gp path] (1.688,2.097)--(1.778,2.097);
\draw[gp path] (8.197,2.097)--(8.107,2.097);
\draw[gp path] (1.688,2.223)--(1.778,2.223);
\draw[gp path] (8.197,2.223)--(8.107,2.223);
\draw[gp path] (1.688,2.329)--(1.778,2.329);
\draw[gp path] (8.197,2.329)--(8.107,2.329);
\draw[gp path] (1.688,2.421)--(1.778,2.421);
\draw[gp path] (8.197,2.421)--(8.107,2.421);
\draw[gp path] (1.688,2.503)--(1.778,2.503);
\draw[gp path] (8.197,2.503)--(8.107,2.503);
\draw[gp path] (1.688,2.575)--(1.868,2.575);
\draw[gp path] (8.197,2.575)--(8.017,2.575);
\node[gp node right] at (1.504,2.575) { 0.1};
\draw[gp path] (1.688,3.054)--(1.778,3.054);
\draw[gp path] (8.197,3.054)--(8.107,3.054);
\draw[gp path] (1.688,3.334)--(1.778,3.334);
\draw[gp path] (8.197,3.334)--(8.107,3.334);
\draw[gp path] (1.688,3.533)--(1.778,3.533);
\draw[gp path] (8.197,3.533)--(8.107,3.533);
\draw[gp path] (1.688,3.687)--(1.778,3.687);
\draw[gp path] (8.197,3.687)--(8.107,3.687);
\draw[gp path] (1.688,3.813)--(1.778,3.813);
\draw[gp path] (8.197,3.813)--(8.107,3.813);
\draw[gp path] (1.688,3.919)--(1.778,3.919);
\draw[gp path] (8.197,3.919)--(8.107,3.919);
\draw[gp path] (1.688,4.012)--(1.778,4.012);
\draw[gp path] (8.197,4.012)--(8.107,4.012);
\draw[gp path] (1.688,4.093)--(1.778,4.093);
\draw[gp path] (8.197,4.093)--(8.107,4.093);
\draw[gp path] (1.688,4.166)--(1.868,4.166);
\draw[gp path] (8.197,4.166)--(8.017,4.166);
\node[gp node right] at (1.504,4.166) { 1};
\draw[gp path] (1.688,4.644)--(1.778,4.644);
\draw[gp path] (8.197,4.644)--(8.107,4.644);
\draw[gp path] (1.688,4.924)--(1.778,4.924);
\draw[gp path] (8.197,4.924)--(8.107,4.924);
\draw[gp path] (1.688,5.123)--(1.778,5.123);
\draw[gp path] (8.197,5.123)--(8.107,5.123);
\draw[gp path] (1.688,5.277)--(1.778,5.277);
\draw[gp path] (8.197,5.277)--(8.107,5.277);
\draw[gp path] (1.688,5.403)--(1.778,5.403);
\draw[gp path] (8.197,5.403)--(8.107,5.403);
\draw[gp path] (1.688,5.510)--(1.778,5.510);
\draw[gp path] (8.197,5.510)--(8.107,5.510);
\draw[gp path] (1.688,5.602)--(1.778,5.602);
\draw[gp path] (8.197,5.602)--(8.107,5.602);
\draw[gp path] (1.688,5.683)--(1.778,5.683);
\draw[gp path] (8.197,5.683)--(8.107,5.683);
\draw[gp path] (1.688,5.756)--(1.868,5.756);
\draw[gp path] (8.197,5.756)--(8.017,5.756);
\node[gp node right] at (1.504,5.756) { 10};
\draw[gp path] (1.688,0.985)--(1.688,1.165);
\draw[gp path] (1.688,5.756)--(1.688,5.576);
\node[gp node center] at (1.688,0.677) {-1};
\draw[gp path] (2.411,0.985)--(2.411,1.165);
\draw[gp path] (2.411,5.756)--(2.411,5.576);
\node[gp node center] at (2.411,0.677) {-0.5};
\draw[gp path] (3.134,0.985)--(3.134,1.165);
\draw[gp path] (3.134,5.756)--(3.134,5.576);
\node[gp node center] at (3.134,0.677) { 0};
\draw[gp path] (3.858,0.985)--(3.858,1.165);
\draw[gp path] (3.858,5.756)--(3.858,5.576);
\node[gp node center] at (3.858,0.677) { 0.5};
\draw[gp path] (4.581,0.985)--(4.581,1.165);
\draw[gp path] (4.581,5.756)--(4.581,5.576);
\node[gp node center] at (4.581,0.677) { 1};
\draw[gp path] (5.304,0.985)--(5.304,1.165);
\draw[gp path] (5.304,5.756)--(5.304,5.576);
\node[gp node center] at (5.304,0.677) { 1.5};
\draw[gp path] (6.027,0.985)--(6.027,1.165);
\draw[gp path] (6.027,5.756)--(6.027,5.576);
\node[gp node center] at (6.027,0.677) { 2};
\draw[gp path] (6.751,0.985)--(6.751,1.165);
\draw[gp path] (6.751,5.756)--(6.751,5.576);
\node[gp node center] at (6.751,0.677) { 2.5};
\draw[gp path] (7.474,0.985)--(7.474,1.165);
\draw[gp path] (7.474,5.756)--(7.474,5.576);
\node[gp node center] at (7.474,0.677) { 3};
\draw[gp path] (8.197,0.985)--(8.197,1.165);
\draw[gp path] (8.197,5.756)--(8.197,5.576);
\node[gp node center] at (8.197,0.677) { 3.5};
\draw[gp path] (1.688,5.756)--(1.688,0.985)--(8.197,0.985)--(8.197,5.756)--cycle;
\node[gp node center,rotate=-270] at (0.246,3.370) {HW1/eM beam width (mm)};
\node[gp node center] at (4.942,0.215) {Pulse location rel. to $f$=6mm lens};
\gpcolor{color=gp lt color 0}
\gpsetlinetype{gp lt plot 0}
\draw[gp path] (1.688,3.687)--(1.720,3.687)--(1.752,3.687)--(1.783,3.687)--(1.815,3.687)%
  --(1.847,3.687)--(1.879,3.687)--(1.911,3.687)--(1.943,3.687)--(1.974,3.687)--(2.006,3.687)%
  --(2.038,3.687)--(2.070,3.687)--(2.102,3.687)--(2.134,3.687)--(2.165,3.687)--(2.197,3.687)%
  --(2.229,3.687)--(2.261,3.687)--(2.293,3.687)--(2.324,3.687)--(2.356,3.687)--(2.388,3.687)%
  --(2.420,3.687)--(2.452,3.687)--(2.484,3.687)--(2.515,3.687)--(2.547,3.687)--(2.579,3.687)%
  --(2.611,3.687)--(2.643,3.687)--(2.674,3.686)--(2.706,3.686)--(2.738,3.686)--(2.770,3.685)%
  --(2.802,3.684)--(2.834,3.683)--(2.865,3.681)--(2.897,3.679)--(2.929,3.677)--(2.961,3.674)%
  --(2.993,3.670)--(3.025,3.665)--(3.056,3.659)--(3.088,3.653)--(3.120,3.645)--(3.152,3.637)%
  --(3.184,3.627)--(3.215,3.616)--(3.247,3.604)--(3.279,3.592)--(3.311,3.578)--(3.343,3.563)%
  --(3.375,3.547)--(3.406,3.530)--(3.438,3.512)--(3.470,3.494)--(3.502,3.474)--(3.534,3.454)%
  --(3.565,3.433)--(3.597,3.412)--(3.629,3.389)--(3.661,3.366)--(3.693,3.341)--(3.725,3.316)%
  --(3.756,3.290)--(3.788,3.263)--(3.820,3.235)--(3.852,3.205)--(3.884,3.175)--(3.916,3.142)%
  --(3.947,3.109)--(3.979,3.073)--(4.011,3.036)--(4.043,2.996)--(4.075,2.954)--(4.106,2.909)%
  --(4.138,2.862)--(4.170,2.810)--(4.202,2.755)--(4.234,2.695)--(4.266,2.629)--(4.297,2.557)%
  --(4.329,2.476)--(4.361,2.385)--(4.393,2.281)--(4.425,2.160)--(4.456,2.015)--(4.488,1.837)%
  --(4.520,1.616)--(4.552,1.367)--(4.584,1.256)--(4.616,1.439)--(4.647,1.686)--(4.679,1.893)%
  --(4.711,2.060)--(4.743,2.198)--(4.775,2.314)--(4.807,2.413)--(4.838,2.501)--(4.870,2.579)%
  --(4.902,2.649)--(4.934,2.713)--(4.966,2.772)--(4.997,2.826)--(5.029,2.876)--(5.061,2.923)%
  --(5.093,2.967)--(5.125,3.008)--(5.157,3.047)--(5.188,3.084)--(5.220,3.119)--(5.252,3.152)%
  --(5.284,3.184)--(5.316,3.214)--(5.348,3.243)--(5.379,3.271)--(5.411,3.298)--(5.443,3.324)%
  --(5.475,3.349)--(5.507,3.373)--(5.538,3.396)--(5.570,3.419)--(5.602,3.440)--(5.634,3.462)%
  --(5.666,3.482)--(5.698,3.502)--(5.729,3.521)--(5.761,3.540)--(5.793,3.558)--(5.825,3.576)%
  --(5.857,3.594)--(5.888,3.611)--(5.920,3.627)--(5.952,3.643)--(5.984,3.659)--(6.016,3.675)%
  --(6.048,3.690)--(6.079,3.705)--(6.111,3.719)--(6.143,3.733)--(6.175,3.747)--(6.207,3.761)%
  --(6.239,3.774)--(6.270,3.787)--(6.302,3.800)--(6.334,3.813)--(6.366,3.825)--(6.398,3.837)%
  --(6.429,3.849)--(6.461,3.861)--(6.493,3.873)--(6.525,3.884)--(6.557,3.895)--(6.589,3.906)%
  --(6.620,3.917)--(6.652,3.928)--(6.684,3.938)--(6.716,3.949)--(6.748,3.959)--(6.779,3.969)%
  --(6.811,3.979)--(6.843,3.989)--(6.875,3.998)--(6.907,4.008)--(6.939,4.017)--(6.970,4.026)%
  --(7.002,4.035)--(7.034,4.045)--(7.066,4.053)--(7.098,4.062)--(7.130,4.071)--(7.161,4.079)%
  --(7.193,4.088)--(7.225,4.096)--(7.257,4.104)--(7.289,4.113)--(7.320,4.121)--(7.352,4.129)%
  --(7.384,4.137)--(7.416,4.144)--(7.448,4.152)--(7.480,4.160)--(7.511,4.167)--(7.543,4.175)%
  --(7.575,4.182)--(7.607,4.189)--(7.639,4.197)--(7.670,4.204)--(7.702,4.211)--(7.734,4.218)%
  --(7.766,4.225)--(7.798,4.232)--(7.830,4.238)--(7.861,4.245)--(7.893,4.252)--(7.925,4.258)%
  --(7.957,4.265)--(7.989,4.271)--(8.021,4.278)--(8.052,4.284);
\gpcolor{color=gp lt color 1}
\gpsetlinetype{gp lt plot 1}
\draw[gp path] (1.688,3.687)--(1.720,3.687)--(1.752,3.687)--(1.783,3.687)--(1.815,3.687)%
  --(1.847,3.687)--(1.879,3.687)--(1.911,3.687)--(1.943,3.687)--(1.974,3.687)--(2.006,3.687)%
  --(2.038,3.687)--(2.070,3.687)--(2.102,3.687)--(2.134,3.687)--(2.165,3.687)--(2.197,3.687)%
  --(2.229,3.687)--(2.261,3.687)--(2.293,3.687)--(2.324,3.687)--(2.356,3.687)--(2.388,3.687)%
  --(2.420,3.687)--(2.452,3.687)--(2.484,3.687)--(2.515,3.687)--(2.547,3.687)--(2.579,3.687)%
  --(2.611,3.687)--(2.643,3.687)--(2.674,3.686)--(2.706,3.686)--(2.738,3.686)--(2.770,3.685)%
  --(2.802,3.684)--(2.834,3.683)--(2.865,3.681)--(2.897,3.679)--(2.929,3.677)--(2.961,3.674)%
  --(2.993,3.670)--(3.025,3.665)--(3.056,3.660)--(3.088,3.653)--(3.120,3.645)--(3.152,3.637)%
  --(3.184,3.627)--(3.215,3.616)--(3.247,3.604)--(3.279,3.592)--(3.311,3.578)--(3.343,3.563)%
  --(3.375,3.547)--(3.406,3.530)--(3.438,3.512)--(3.470,3.494)--(3.502,3.474)--(3.534,3.454)%
  --(3.565,3.433)--(3.597,3.412)--(3.629,3.389)--(3.661,3.366)--(3.693,3.341)--(3.725,3.316)%
  --(3.756,3.290)--(3.788,3.263)--(3.820,3.235)--(3.852,3.205)--(3.884,3.175)--(3.916,3.142)%
  --(3.947,3.109)--(3.979,3.073)--(4.011,3.036)--(4.043,2.996)--(4.075,2.954)--(4.106,2.909)%
  --(4.138,2.862)--(4.170,2.811)--(4.202,2.755)--(4.234,2.695)--(4.266,2.630)--(4.297,2.557)%
  --(4.329,2.477)--(4.361,2.386)--(4.393,2.282)--(4.425,2.160)--(4.456,2.015)--(4.488,1.838)%
  --(4.520,1.618)--(4.552,1.369)--(4.584,1.256)--(4.616,1.438)--(4.647,1.685)--(4.679,1.892)%
  --(4.711,2.059)--(4.743,2.197)--(4.775,2.313)--(4.807,2.413)--(4.838,2.501)--(4.870,2.579)%
  --(4.902,2.649)--(4.934,2.713)--(4.966,2.771)--(4.997,2.825)--(5.029,2.876)--(5.061,2.922)%
  --(5.093,2.966)--(5.125,3.008)--(5.157,3.046)--(5.188,3.083)--(5.220,3.118)--(5.252,3.152)%
  --(5.284,3.183)--(5.316,3.214)--(5.348,3.243)--(5.379,3.271)--(5.411,3.298)--(5.443,3.324)%
  --(5.475,3.349)--(5.507,3.373)--(5.538,3.396)--(5.570,3.418)--(5.602,3.440)--(5.634,3.461)%
  --(5.666,3.482)--(5.698,3.502)--(5.729,3.521)--(5.761,3.540)--(5.793,3.558)--(5.825,3.576)%
  --(5.857,3.594)--(5.888,3.610)--(5.920,3.627)--(5.952,3.643)--(5.984,3.659)--(6.016,3.675)%
  --(6.048,3.690)--(6.079,3.704)--(6.111,3.719)--(6.143,3.733)--(6.175,3.747)--(6.207,3.761)%
  --(6.239,3.774)--(6.270,3.787)--(6.302,3.800)--(6.334,3.813)--(6.366,3.825)--(6.398,3.837)%
  --(6.429,3.849)--(6.461,3.861)--(6.493,3.873)--(6.525,3.884)--(6.557,3.895)--(6.589,3.906)%
  --(6.620,3.917)--(6.652,3.928)--(6.684,3.938)--(6.716,3.949)--(6.748,3.959)--(6.779,3.969)%
  --(6.811,3.979)--(6.843,3.989)--(6.875,3.998)--(6.907,4.008)--(6.939,4.017)--(6.970,4.026)%
  --(7.002,4.035)--(7.034,4.044)--(7.066,4.053)--(7.098,4.062)--(7.130,4.071)--(7.161,4.079)%
  --(7.193,4.088)--(7.225,4.096)--(7.257,4.104)--(7.289,4.113)--(7.320,4.121)--(7.352,4.129)%
  --(7.384,4.136)--(7.416,4.144)--(7.448,4.152)--(7.480,4.160)--(7.511,4.167)--(7.543,4.175)%
  --(7.575,4.182)--(7.607,4.189)--(7.639,4.196)--(7.670,4.204)--(7.702,4.211)--(7.734,4.218)%
  --(7.766,4.225)--(7.798,4.231)--(7.830,4.238)--(7.861,4.245)--(7.893,4.252)--(7.925,4.258)%
  --(7.957,4.265)--(7.989,4.271)--(8.021,4.278)--(8.052,4.284);
\gpcolor{color=gp lt color 2}
\gpsetlinetype{gp lt plot 2}
\draw[gp path] (1.688,3.687)--(1.720,3.687)--(1.752,3.687)--(1.783,3.687)--(1.815,3.687)%
  --(1.847,3.687)--(1.879,3.687)--(1.911,3.687)--(1.943,3.687)--(1.974,3.687)--(2.006,3.687)%
  --(2.038,3.687)--(2.070,3.687)--(2.102,3.687)--(2.134,3.687)--(2.165,3.687)--(2.197,3.687)%
  --(2.229,3.687)--(2.261,3.687)--(2.293,3.687)--(2.324,3.687)--(2.356,3.687)--(2.388,3.687)%
  --(2.420,3.687)--(2.452,3.687)--(2.484,3.687)--(2.515,3.687)--(2.547,3.687)--(2.579,3.687)%
  --(2.611,3.687)--(2.643,3.687)--(2.674,3.686)--(2.706,3.686)--(2.738,3.686)--(2.770,3.685)%
  --(2.802,3.684)--(2.834,3.683)--(2.865,3.681)--(2.897,3.679)--(2.929,3.677)--(2.961,3.674)%
  --(2.993,3.670)--(3.025,3.665)--(3.056,3.660)--(3.088,3.653)--(3.120,3.646)--(3.152,3.637)%
  --(3.184,3.627)--(3.215,3.616)--(3.247,3.605)--(3.279,3.592)--(3.311,3.578)--(3.343,3.563)%
  --(3.375,3.547)--(3.406,3.530)--(3.438,3.512)--(3.470,3.494)--(3.502,3.475)--(3.534,3.454)%
  --(3.565,3.434)--(3.597,3.412)--(3.629,3.389)--(3.661,3.366)--(3.693,3.342)--(3.725,3.317)%
  --(3.756,3.291)--(3.788,3.263)--(3.820,3.235)--(3.852,3.206)--(3.884,3.175)--(3.916,3.143)%
  --(3.947,3.109)--(3.979,3.074)--(4.011,3.036)--(4.043,2.997)--(4.075,2.955)--(4.106,2.910)%
  --(4.138,2.863)--(4.170,2.812)--(4.202,2.757)--(4.234,2.697)--(4.266,2.632)--(4.297,2.559)%
  --(4.329,2.479)--(4.361,2.389)--(4.393,2.285)--(4.425,2.165)--(4.456,2.021)--(4.488,1.846)%
  --(4.520,1.629)--(4.552,1.382)--(4.584,1.257)--(4.616,1.427)--(4.647,1.673)--(4.679,1.882)%
  --(4.711,2.050)--(4.743,2.189)--(4.775,2.306)--(4.807,2.407)--(4.838,2.495)--(4.870,2.574)%
  --(4.902,2.644)--(4.934,2.708)--(4.966,2.767)--(4.997,2.822)--(5.029,2.872)--(5.061,2.919)%
  --(5.093,2.963)--(5.125,3.004)--(5.157,3.043)--(5.188,3.080)--(5.220,3.116)--(5.252,3.149)%
  --(5.284,3.181)--(5.316,3.211)--(5.348,3.241)--(5.379,3.269)--(5.411,3.296)--(5.443,3.322)%
  --(5.475,3.347)--(5.507,3.371)--(5.538,3.394)--(5.570,3.417)--(5.602,3.438)--(5.634,3.460)%
  --(5.666,3.480)--(5.698,3.500)--(5.729,3.519)--(5.761,3.538)--(5.793,3.557)--(5.825,3.575)%
  --(5.857,3.592)--(5.888,3.609)--(5.920,3.626)--(5.952,3.642)--(5.984,3.658)--(6.016,3.673)%
  --(6.048,3.688)--(6.079,3.703)--(6.111,3.718)--(6.143,3.732)--(6.175,3.746)--(6.207,3.759)%
  --(6.239,3.773)--(6.270,3.786)--(6.302,3.799)--(6.334,3.811)--(6.366,3.824)--(6.398,3.836)%
  --(6.429,3.848)--(6.461,3.860)--(6.493,3.871)--(6.525,3.883)--(6.557,3.894)--(6.589,3.905)%
  --(6.620,3.916)--(6.652,3.927)--(6.684,3.937)--(6.716,3.948)--(6.748,3.958)--(6.779,3.968)%
  --(6.811,3.978)--(6.843,3.988)--(6.875,3.997)--(6.907,4.007)--(6.939,4.016)--(6.970,4.025)%
  --(7.002,4.035)--(7.034,4.044)--(7.066,4.052)--(7.098,4.061)--(7.130,4.070)--(7.161,4.079)%
  --(7.193,4.087)--(7.225,4.095)--(7.257,4.104)--(7.289,4.112)--(7.320,4.120)--(7.352,4.128)%
  --(7.384,4.136)--(7.416,4.144)--(7.448,4.151)--(7.480,4.159)--(7.511,4.166)--(7.543,4.174)%
  --(7.575,4.181)--(7.607,4.189)--(7.639,4.196)--(7.670,4.203)--(7.702,4.210)--(7.734,4.217)%
  --(7.766,4.224)--(7.798,4.231)--(7.830,4.238)--(7.861,4.244)--(7.893,4.251)--(7.925,4.258)%
  --(7.957,4.264)--(7.989,4.271)--(8.021,4.277)--(8.052,4.283);
\gpcolor{color=gp lt color 3}
\gpsetlinetype{gp lt plot 3}
\draw[gp path] (1.688,3.687)--(1.720,3.687)--(1.752,3.687)--(1.783,3.687)--(1.815,3.687)%
  --(1.847,3.687)--(1.879,3.687)--(1.911,3.687)--(1.943,3.687)--(1.974,3.687)--(2.006,3.687)%
  --(2.038,3.687)--(2.070,3.687)--(2.102,3.687)--(2.134,3.687)--(2.165,3.687)--(2.197,3.687)%
  --(2.229,3.687)--(2.261,3.687)--(2.293,3.687)--(2.324,3.687)--(2.356,3.687)--(2.388,3.687)%
  --(2.420,3.687)--(2.452,3.687)--(2.484,3.687)--(2.515,3.687)--(2.547,3.687)--(2.579,3.687)%
  --(2.611,3.687)--(2.643,3.687)--(2.674,3.687)--(2.706,3.687)--(2.738,3.686)--(2.770,3.686)%
  --(2.802,3.685)--(2.834,3.684)--(2.865,3.682)--(2.897,3.680)--(2.929,3.678)--(2.961,3.675)%
  --(2.993,3.671)--(3.025,3.666)--(3.056,3.661)--(3.088,3.654)--(3.120,3.647)--(3.152,3.638)%
  --(3.184,3.629)--(3.215,3.618)--(3.247,3.606)--(3.279,3.593)--(3.311,3.579)--(3.343,3.565)%
  --(3.375,3.549)--(3.406,3.532)--(3.438,3.514)--(3.470,3.496)--(3.502,3.477)--(3.534,3.457)%
  --(3.565,3.436)--(3.597,3.415)--(3.629,3.392)--(3.661,3.369)--(3.693,3.345)--(3.725,3.320)%
  --(3.756,3.294)--(3.788,3.268)--(3.820,3.240)--(3.852,3.211)--(3.884,3.180)--(3.916,3.149)%
  --(3.947,3.115)--(3.979,3.080)--(4.011,3.044)--(4.043,3.005)--(4.075,2.964)--(4.106,2.920)%
  --(4.138,2.874)--(4.170,2.824)--(4.202,2.770)--(4.234,2.713)--(4.266,2.650)--(4.297,2.580)%
  --(4.329,2.504)--(4.361,2.418)--(4.393,2.321)--(4.425,2.209)--(4.456,2.078)--(4.488,1.922)%
  --(4.520,1.731)--(4.552,1.505)--(4.584,1.311)--(4.616,1.343)--(4.647,1.558)--(4.679,1.777)%
  --(4.711,1.959)--(4.743,2.109)--(4.775,2.236)--(4.807,2.344)--(4.838,2.438)--(4.870,2.521)%
  --(4.902,2.596)--(4.934,2.663)--(4.966,2.725)--(4.997,2.782)--(5.029,2.834)--(5.061,2.883)%
  --(5.093,2.929)--(5.125,2.972)--(5.157,3.013)--(5.188,3.051)--(5.220,3.087)--(5.252,3.122)%
  --(5.284,3.154)--(5.316,3.186)--(5.348,3.216)--(5.379,3.245)--(5.411,3.272)--(5.443,3.299)%
  --(5.475,3.325)--(5.507,3.349)--(5.538,3.373)--(5.570,3.396)--(5.602,3.419)--(5.634,3.440)%
  --(5.666,3.461)--(5.698,3.482)--(5.729,3.501)--(5.761,3.521)--(5.793,3.539)--(5.825,3.558)%
  --(5.857,3.575)--(5.888,3.593)--(5.920,3.610)--(5.952,3.626)--(5.984,3.642)--(6.016,3.658)%
  --(6.048,3.673)--(6.079,3.688)--(6.111,3.703)--(6.143,3.718)--(6.175,3.732)--(6.207,3.746)%
  --(6.239,3.759)--(6.270,3.772)--(6.302,3.786)--(6.334,3.798)--(6.366,3.811)--(6.398,3.823)%
  --(6.429,3.836)--(6.461,3.847)--(6.493,3.859)--(6.525,3.871)--(6.557,3.882)--(6.589,3.893)%
  --(6.620,3.904)--(6.652,3.915)--(6.684,3.926)--(6.716,3.936)--(6.748,3.947)--(6.779,3.957)%
  --(6.811,3.967)--(6.843,3.977)--(6.875,3.987)--(6.907,3.996)--(6.939,4.006)--(6.970,4.015)%
  --(7.002,4.024)--(7.034,4.033)--(7.066,4.042)--(7.098,4.051)--(7.130,4.060)--(7.161,4.069)%
  --(7.193,4.077)--(7.225,4.086)--(7.257,4.094)--(7.289,4.102)--(7.320,4.110)--(7.352,4.119)%
  --(7.384,4.126)--(7.416,4.134)--(7.448,4.142)--(7.480,4.150)--(7.511,4.157)--(7.543,4.165)%
  --(7.575,4.172)--(7.607,4.180)--(7.639,4.187)--(7.670,4.194)--(7.702,4.201)--(7.734,4.209)%
  --(7.766,4.216)--(7.798,4.222)--(7.830,4.229)--(7.861,4.236)--(7.893,4.243)--(7.925,4.250)%
  --(7.957,4.256)--(7.989,4.263)--(8.021,4.269)--(8.052,4.276);
\gpcolor{color=gp lt color 4}
\gpsetlinetype{gp lt plot 4}
\draw[gp path] (1.688,3.687)--(1.720,3.687)--(1.752,3.687)--(1.783,3.687)--(1.815,3.687)%
  --(1.847,3.687)--(1.879,3.687)--(1.911,3.687)--(1.943,3.687)--(1.974,3.687)--(2.006,3.688)%
  --(2.038,3.688)--(2.070,3.688)--(2.102,3.688)--(2.134,3.688)--(2.165,3.688)--(2.197,3.689)%
  --(2.229,3.689)--(2.261,3.689)--(2.293,3.689)--(2.324,3.690)--(2.356,3.690)--(2.388,3.690)%
  --(2.420,3.691)--(2.452,3.691)--(2.484,3.691)--(2.515,3.692)--(2.547,3.692)--(2.579,3.692)%
  --(2.611,3.692)--(2.643,3.693)--(2.674,3.693)--(2.706,3.693)--(2.738,3.693)--(2.770,3.693)%
  --(2.802,3.692)--(2.834,3.692)--(2.865,3.691)--(2.897,3.689)--(2.929,3.687)--(2.961,3.684)%
  --(2.993,3.681)--(3.025,3.677)--(3.056,3.672)--(3.088,3.666)--(3.120,3.659)--(3.152,3.651)%
  --(3.184,3.642)--(3.215,3.632)--(3.247,3.621)--(3.279,3.609)--(3.311,3.596)--(3.343,3.582)%
  --(3.375,3.567)--(3.406,3.551)--(3.438,3.535)--(3.470,3.517)--(3.502,3.499)--(3.534,3.481)%
  --(3.565,3.461)--(3.597,3.441)--(3.629,3.421)--(3.661,3.399)--(3.693,3.377)--(3.725,3.354)%
  --(3.756,3.331)--(3.788,3.306)--(3.820,3.281)--(3.852,3.255)--(3.884,3.228)--(3.916,3.200)%
  --(3.947,3.171)--(3.979,3.141)--(4.011,3.109)--(4.043,3.076)--(4.075,3.042)--(4.106,3.006)%
  --(4.138,2.968)--(4.170,2.928)--(4.202,2.886)--(4.234,2.842)--(4.266,2.795)--(4.297,2.745)%
  --(4.329,2.692)--(4.361,2.635)--(4.393,2.574)--(4.425,2.508)--(4.456,2.436)--(4.488,2.358)%
  --(4.520,2.272)--(4.552,2.177)--(4.584,2.073)--(4.616,1.959)--(4.647,1.837)--(4.679,1.713)%
  --(4.711,1.606)--(4.743,1.551)--(4.775,1.575)--(4.807,1.664)--(4.838,1.783)--(4.870,1.905)%
  --(4.902,2.021)--(4.934,2.127)--(4.966,2.224)--(4.997,2.311)--(5.029,2.390)--(5.061,2.463)%
  --(5.093,2.529)--(5.125,2.591)--(5.157,2.648)--(5.188,2.701)--(5.220,2.750)--(5.252,2.797)%
  --(5.284,2.841)--(5.316,2.883)--(5.348,2.922)--(5.379,2.959)--(5.411,2.995)--(5.443,3.029)%
  --(5.475,3.062)--(5.507,3.093)--(5.538,3.122)--(5.570,3.151)--(5.602,3.179)--(5.634,3.205)%
  --(5.666,3.231)--(5.698,3.256)--(5.729,3.280)--(5.761,3.303)--(5.793,3.325)--(5.825,3.347)%
  --(5.857,3.368)--(5.888,3.389)--(5.920,3.409)--(5.952,3.428)--(5.984,3.447)--(6.016,3.466)%
  --(6.048,3.484)--(6.079,3.501)--(6.111,3.518)--(6.143,3.535)--(6.175,3.551)--(6.207,3.567)%
  --(6.239,3.583)--(6.270,3.598)--(6.302,3.613)--(6.334,3.628)--(6.366,3.642)--(6.398,3.656)%
  --(6.429,3.670)--(6.461,3.684)--(6.493,3.697)--(6.525,3.710)--(6.557,3.723)--(6.589,3.736)%
  --(6.620,3.748)--(6.652,3.760)--(6.684,3.772)--(6.716,3.784)--(6.748,3.795)--(6.779,3.807)%
  --(6.811,3.818)--(6.843,3.829)--(6.875,3.840)--(6.907,3.850)--(6.939,3.861)--(6.970,3.871)%
  --(7.002,3.881)--(7.034,3.892)--(7.066,3.901)--(7.098,3.911)--(7.130,3.921)--(7.161,3.930)%
  --(7.193,3.940)--(7.225,3.949)--(7.257,3.958)--(7.289,3.967)--(7.320,3.976)--(7.352,3.985)%
  --(7.384,3.994)--(7.416,4.002)--(7.448,4.011)--(7.480,4.019)--(7.511,4.027)--(7.543,4.036)%
  --(7.575,4.044)--(7.607,4.052)--(7.639,4.060)--(7.670,4.068)--(7.702,4.075)--(7.734,4.083)%
  --(7.766,4.091)--(7.798,4.098)--(7.830,4.105)--(7.861,4.113)--(7.893,4.120)--(7.925,4.127)%
  --(7.957,4.134)--(7.989,4.141)--(8.021,4.148)--(8.052,4.155);
\gpcolor{color=gp lt color border}
\gpsetlinetype{gp lt border}
\draw[gp path] (1.688,5.756)--(1.688,0.985)--(8.197,0.985)--(8.197,5.756)--cycle;
%% coordinates of the plot area
\gpdefrectangularnode{gp plot 1}{\pgfpoint{1.688cm}{0.985cm}}{\pgfpoint{8.197cm}{5.756cm}}
%% \end{tikzpicture}
%% gnuplot variables

        \node at (6.5,3) {$N=10^7$};
      \end{tikzpicture}
    }
    \subfloat[][] { 
      \label{fig:focus_short_prolate}
      \begin{tikzpicture}
        %% \begin{tikzpicture}[gnuplot]
%% generated with GNUPLOT 4.6p0 (Lua 5.1; terminal rev. 99, script rev. 100)
%% Tue 26 Mar 2013 10:54:33 AM CDT
\gpsolidlines
\path (0.000,0.000) rectangle (8.750,6.125);
\gpcolor{color=gp lt color border}
\gpsetlinetype{gp lt border}
\gpsetlinewidth{1.00}
\draw[gp path] (1.688,0.985)--(1.868,0.985);
\draw[gp path] (8.197,0.985)--(8.017,0.985);
\node[gp node right] at (1.504,0.985) { 0.01};
\draw[gp path] (1.688,1.464)--(1.778,1.464);
\draw[gp path] (8.197,1.464)--(8.107,1.464);
\draw[gp path] (1.688,1.744)--(1.778,1.744);
\draw[gp path] (8.197,1.744)--(8.107,1.744);
\draw[gp path] (1.688,1.942)--(1.778,1.942);
\draw[gp path] (8.197,1.942)--(8.107,1.942);
\draw[gp path] (1.688,2.097)--(1.778,2.097);
\draw[gp path] (8.197,2.097)--(8.107,2.097);
\draw[gp path] (1.688,2.223)--(1.778,2.223);
\draw[gp path] (8.197,2.223)--(8.107,2.223);
\draw[gp path] (1.688,2.329)--(1.778,2.329);
\draw[gp path] (8.197,2.329)--(8.107,2.329);
\draw[gp path] (1.688,2.421)--(1.778,2.421);
\draw[gp path] (8.197,2.421)--(8.107,2.421);
\draw[gp path] (1.688,2.503)--(1.778,2.503);
\draw[gp path] (8.197,2.503)--(8.107,2.503);
\draw[gp path] (1.688,2.575)--(1.868,2.575);
\draw[gp path] (8.197,2.575)--(8.017,2.575);
\node[gp node right] at (1.504,2.575) { 0.1};
\draw[gp path] (1.688,3.054)--(1.778,3.054);
\draw[gp path] (8.197,3.054)--(8.107,3.054);
\draw[gp path] (1.688,3.334)--(1.778,3.334);
\draw[gp path] (8.197,3.334)--(8.107,3.334);
\draw[gp path] (1.688,3.533)--(1.778,3.533);
\draw[gp path] (8.197,3.533)--(8.107,3.533);
\draw[gp path] (1.688,3.687)--(1.778,3.687);
\draw[gp path] (8.197,3.687)--(8.107,3.687);
\draw[gp path] (1.688,3.813)--(1.778,3.813);
\draw[gp path] (8.197,3.813)--(8.107,3.813);
\draw[gp path] (1.688,3.919)--(1.778,3.919);
\draw[gp path] (8.197,3.919)--(8.107,3.919);
\draw[gp path] (1.688,4.012)--(1.778,4.012);
\draw[gp path] (8.197,4.012)--(8.107,4.012);
\draw[gp path] (1.688,4.093)--(1.778,4.093);
\draw[gp path] (8.197,4.093)--(8.107,4.093);
\draw[gp path] (1.688,4.166)--(1.868,4.166);
\draw[gp path] (8.197,4.166)--(8.017,4.166);
\node[gp node right] at (1.504,4.166) { 1};
\draw[gp path] (1.688,4.644)--(1.778,4.644);
\draw[gp path] (8.197,4.644)--(8.107,4.644);
\draw[gp path] (1.688,4.924)--(1.778,4.924);
\draw[gp path] (8.197,4.924)--(8.107,4.924);
\draw[gp path] (1.688,5.123)--(1.778,5.123);
\draw[gp path] (8.197,5.123)--(8.107,5.123);
\draw[gp path] (1.688,5.277)--(1.778,5.277);
\draw[gp path] (8.197,5.277)--(8.107,5.277);
\draw[gp path] (1.688,5.403)--(1.778,5.403);
\draw[gp path] (8.197,5.403)--(8.107,5.403);
\draw[gp path] (1.688,5.510)--(1.778,5.510);
\draw[gp path] (8.197,5.510)--(8.107,5.510);
\draw[gp path] (1.688,5.602)--(1.778,5.602);
\draw[gp path] (8.197,5.602)--(8.107,5.602);
\draw[gp path] (1.688,5.683)--(1.778,5.683);
\draw[gp path] (8.197,5.683)--(8.107,5.683);
\draw[gp path] (1.688,5.756)--(1.868,5.756);
\draw[gp path] (8.197,5.756)--(8.017,5.756);
\node[gp node right] at (1.504,5.756) { 10};
\draw[gp path] (1.688,0.985)--(1.688,1.165);
\draw[gp path] (1.688,5.756)--(1.688,5.576);
\node[gp node center] at (1.688,0.677) {-1};
\draw[gp path] (2.411,0.985)--(2.411,1.165);
\draw[gp path] (2.411,5.756)--(2.411,5.576);
\node[gp node center] at (2.411,0.677) {-0.5};
\draw[gp path] (3.134,0.985)--(3.134,1.165);
\draw[gp path] (3.134,5.756)--(3.134,5.576);
\node[gp node center] at (3.134,0.677) { 0};
\draw[gp path] (3.858,0.985)--(3.858,1.165);
\draw[gp path] (3.858,5.756)--(3.858,5.576);
\node[gp node center] at (3.858,0.677) { 0.5};
\draw[gp path] (4.581,0.985)--(4.581,1.165);
\draw[gp path] (4.581,5.756)--(4.581,5.576);
\node[gp node center] at (4.581,0.677) { 1};
\draw[gp path] (5.304,0.985)--(5.304,1.165);
\draw[gp path] (5.304,5.756)--(5.304,5.576);
\node[gp node center] at (5.304,0.677) { 1.5};
\draw[gp path] (6.027,0.985)--(6.027,1.165);
\draw[gp path] (6.027,5.756)--(6.027,5.576);
\node[gp node center] at (6.027,0.677) { 2};
\draw[gp path] (6.751,0.985)--(6.751,1.165);
\draw[gp path] (6.751,5.756)--(6.751,5.576);
\node[gp node center] at (6.751,0.677) { 2.5};
\draw[gp path] (7.474,0.985)--(7.474,1.165);
\draw[gp path] (7.474,5.756)--(7.474,5.576);
\node[gp node center] at (7.474,0.677) { 3};
\draw[gp path] (8.197,0.985)--(8.197,1.165);
\draw[gp path] (8.197,5.756)--(8.197,5.576);
\node[gp node center] at (8.197,0.677) { 3.5};
\draw[gp path] (1.688,5.756)--(1.688,0.985)--(8.197,0.985)--(8.197,5.756)--cycle;
\node[gp node center,rotate=-270] at (0.246,3.370) {HW1/eM beam width (mm)};
\node[gp node center] at (4.942,0.215) {Pulse location rel. to $f$=6mm lens};
\gpcolor{color=gp lt color 0}
\gpsetlinetype{gp lt plot 0}
\draw[gp path] (1.688,3.687)--(1.720,3.687)--(1.752,3.687)--(1.783,3.687)--(1.815,3.687)%
  --(1.847,3.687)--(1.879,3.687)--(1.911,3.687)--(1.943,3.687)--(1.974,3.687)--(2.006,3.687)%
  --(2.038,3.687)--(2.070,3.687)--(2.102,3.687)--(2.134,3.687)--(2.165,3.687)--(2.197,3.687)%
  --(2.229,3.687)--(2.261,3.687)--(2.293,3.687)--(2.324,3.687)--(2.356,3.687)--(2.388,3.687)%
  --(2.420,3.687)--(2.452,3.687)--(2.484,3.687)--(2.515,3.687)--(2.547,3.687)--(2.579,3.687)%
  --(2.611,3.687)--(2.643,3.687)--(2.674,3.686)--(2.706,3.686)--(2.738,3.686)--(2.770,3.685)%
  --(2.802,3.684)--(2.834,3.683)--(2.865,3.681)--(2.897,3.679)--(2.929,3.677)--(2.961,3.674)%
  --(2.993,3.670)--(3.025,3.665)--(3.056,3.659)--(3.088,3.653)--(3.120,3.645)--(3.152,3.637)%
  --(3.184,3.627)--(3.215,3.616)--(3.247,3.604)--(3.279,3.592)--(3.311,3.578)--(3.343,3.563)%
  --(3.375,3.547)--(3.406,3.530)--(3.438,3.512)--(3.470,3.494)--(3.502,3.474)--(3.534,3.454)%
  --(3.565,3.433)--(3.597,3.412)--(3.629,3.389)--(3.661,3.366)--(3.693,3.341)--(3.725,3.316)%
  --(3.756,3.290)--(3.788,3.263)--(3.820,3.235)--(3.852,3.205)--(3.884,3.175)--(3.916,3.142)%
  --(3.947,3.109)--(3.979,3.073)--(4.011,3.036)--(4.043,2.996)--(4.075,2.954)--(4.106,2.909)%
  --(4.138,2.862)--(4.170,2.810)--(4.202,2.755)--(4.234,2.695)--(4.266,2.629)--(4.297,2.557)%
  --(4.329,2.476)--(4.361,2.385)--(4.393,2.281)--(4.425,2.160)--(4.456,2.015)--(4.488,1.837)%
  --(4.520,1.616)--(4.552,1.367)--(4.584,1.256)--(4.616,1.439)--(4.647,1.686)--(4.679,1.893)%
  --(4.711,2.060)--(4.743,2.198)--(4.775,2.314)--(4.807,2.413)--(4.838,2.501)--(4.870,2.579)%
  --(4.902,2.649)--(4.934,2.713)--(4.966,2.772)--(4.997,2.826)--(5.029,2.876)--(5.061,2.923)%
  --(5.093,2.967)--(5.125,3.008)--(5.157,3.047)--(5.188,3.084)--(5.220,3.119)--(5.252,3.152)%
  --(5.284,3.184)--(5.316,3.214)--(5.348,3.243)--(5.379,3.271)--(5.411,3.298)--(5.443,3.324)%
  --(5.475,3.349)--(5.507,3.373)--(5.538,3.396)--(5.570,3.419)--(5.602,3.440)--(5.634,3.462)%
  --(5.666,3.482)--(5.698,3.502)--(5.729,3.521)--(5.761,3.540)--(5.793,3.558)--(5.825,3.576)%
  --(5.857,3.594)--(5.888,3.611)--(5.920,3.627)--(5.952,3.643)--(5.984,3.659)--(6.016,3.675)%
  --(6.048,3.690)--(6.079,3.705)--(6.111,3.719)--(6.143,3.733)--(6.175,3.747)--(6.207,3.761)%
  --(6.239,3.774)--(6.270,3.787)--(6.302,3.800)--(6.334,3.813)--(6.366,3.825)--(6.398,3.837)%
  --(6.429,3.849)--(6.461,3.861)--(6.493,3.873)--(6.525,3.884)--(6.557,3.895)--(6.589,3.906)%
  --(6.620,3.917)--(6.652,3.928)--(6.684,3.938)--(6.716,3.949)--(6.748,3.959)--(6.779,3.969)%
  --(6.811,3.979)--(6.843,3.989)--(6.875,3.998)--(6.907,4.008)--(6.939,4.017)--(6.970,4.026)%
  --(7.002,4.035)--(7.034,4.045)--(7.066,4.053)--(7.098,4.062)--(7.130,4.071)--(7.161,4.079)%
  --(7.193,4.088)--(7.225,4.096)--(7.257,4.104)--(7.289,4.113)--(7.320,4.121)--(7.352,4.129)%
  --(7.384,4.137)--(7.416,4.144)--(7.448,4.152)--(7.480,4.160)--(7.511,4.167)--(7.543,4.175)%
  --(7.575,4.182)--(7.607,4.189)--(7.639,4.197)--(7.670,4.204)--(7.702,4.211)--(7.734,4.218)%
  --(7.766,4.225)--(7.798,4.232)--(7.830,4.238)--(7.861,4.245)--(7.893,4.252)--(7.925,4.258)%
  --(7.957,4.265)--(7.989,4.271)--(8.021,4.278)--(8.052,4.284);
\gpcolor{color=gp lt color 1}
\gpsetlinetype{gp lt plot 1}
\draw[gp path] (1.688,3.687)--(1.720,3.687)--(1.752,3.687)--(1.783,3.687)--(1.815,3.687)%
  --(1.847,3.687)--(1.879,3.687)--(1.911,3.687)--(1.943,3.687)--(1.974,3.687)--(2.006,3.687)%
  --(2.038,3.687)--(2.070,3.687)--(2.102,3.687)--(2.134,3.687)--(2.165,3.687)--(2.197,3.687)%
  --(2.229,3.687)--(2.261,3.687)--(2.293,3.687)--(2.324,3.687)--(2.356,3.687)--(2.388,3.687)%
  --(2.420,3.687)--(2.452,3.687)--(2.484,3.687)--(2.515,3.687)--(2.547,3.687)--(2.579,3.687)%
  --(2.611,3.687)--(2.643,3.687)--(2.674,3.686)--(2.706,3.686)--(2.738,3.686)--(2.770,3.685)%
  --(2.802,3.684)--(2.834,3.683)--(2.865,3.681)--(2.897,3.679)--(2.929,3.677)--(2.961,3.674)%
  --(2.993,3.670)--(3.025,3.665)--(3.056,3.660)--(3.088,3.653)--(3.120,3.645)--(3.152,3.637)%
  --(3.184,3.627)--(3.215,3.616)--(3.247,3.604)--(3.279,3.592)--(3.311,3.578)--(3.343,3.563)%
  --(3.375,3.547)--(3.406,3.530)--(3.438,3.512)--(3.470,3.494)--(3.502,3.474)--(3.534,3.454)%
  --(3.565,3.433)--(3.597,3.412)--(3.629,3.389)--(3.661,3.366)--(3.693,3.341)--(3.725,3.316)%
  --(3.756,3.290)--(3.788,3.263)--(3.820,3.235)--(3.852,3.205)--(3.884,3.175)--(3.916,3.142)%
  --(3.947,3.109)--(3.979,3.073)--(4.011,3.036)--(4.043,2.996)--(4.075,2.954)--(4.106,2.909)%
  --(4.138,2.862)--(4.170,2.811)--(4.202,2.755)--(4.234,2.695)--(4.266,2.630)--(4.297,2.557)%
  --(4.329,2.477)--(4.361,2.386)--(4.393,2.282)--(4.425,2.160)--(4.456,2.015)--(4.488,1.838)%
  --(4.520,1.618)--(4.552,1.369)--(4.584,1.256)--(4.616,1.438)--(4.647,1.685)--(4.679,1.892)%
  --(4.711,2.059)--(4.743,2.197)--(4.775,2.313)--(4.807,2.413)--(4.838,2.501)--(4.870,2.579)%
  --(4.902,2.649)--(4.934,2.713)--(4.966,2.771)--(4.997,2.825)--(5.029,2.876)--(5.061,2.922)%
  --(5.093,2.966)--(5.125,3.008)--(5.157,3.046)--(5.188,3.083)--(5.220,3.118)--(5.252,3.152)%
  --(5.284,3.183)--(5.316,3.214)--(5.348,3.243)--(5.379,3.271)--(5.411,3.298)--(5.443,3.324)%
  --(5.475,3.349)--(5.507,3.373)--(5.538,3.396)--(5.570,3.418)--(5.602,3.440)--(5.634,3.461)%
  --(5.666,3.482)--(5.698,3.502)--(5.729,3.521)--(5.761,3.540)--(5.793,3.558)--(5.825,3.576)%
  --(5.857,3.594)--(5.888,3.610)--(5.920,3.627)--(5.952,3.643)--(5.984,3.659)--(6.016,3.675)%
  --(6.048,3.690)--(6.079,3.704)--(6.111,3.719)--(6.143,3.733)--(6.175,3.747)--(6.207,3.761)%
  --(6.239,3.774)--(6.270,3.787)--(6.302,3.800)--(6.334,3.813)--(6.366,3.825)--(6.398,3.837)%
  --(6.429,3.849)--(6.461,3.861)--(6.493,3.873)--(6.525,3.884)--(6.557,3.895)--(6.589,3.906)%
  --(6.620,3.917)--(6.652,3.928)--(6.684,3.938)--(6.716,3.949)--(6.748,3.959)--(6.779,3.969)%
  --(6.811,3.979)--(6.843,3.989)--(6.875,3.998)--(6.907,4.008)--(6.939,4.017)--(6.970,4.026)%
  --(7.002,4.035)--(7.034,4.044)--(7.066,4.053)--(7.098,4.062)--(7.130,4.071)--(7.161,4.079)%
  --(7.193,4.088)--(7.225,4.096)--(7.257,4.104)--(7.289,4.113)--(7.320,4.121)--(7.352,4.129)%
  --(7.384,4.136)--(7.416,4.144)--(7.448,4.152)--(7.480,4.160)--(7.511,4.167)--(7.543,4.175)%
  --(7.575,4.182)--(7.607,4.189)--(7.639,4.196)--(7.670,4.204)--(7.702,4.211)--(7.734,4.218)%
  --(7.766,4.225)--(7.798,4.231)--(7.830,4.238)--(7.861,4.245)--(7.893,4.252)--(7.925,4.258)%
  --(7.957,4.265)--(7.989,4.271)--(8.021,4.278)--(8.052,4.284);
\gpcolor{color=gp lt color 2}
\gpsetlinetype{gp lt plot 2}
\draw[gp path] (1.688,3.687)--(1.720,3.687)--(1.752,3.687)--(1.783,3.687)--(1.815,3.687)%
  --(1.847,3.687)--(1.879,3.687)--(1.911,3.687)--(1.943,3.687)--(1.974,3.687)--(2.006,3.687)%
  --(2.038,3.687)--(2.070,3.687)--(2.102,3.687)--(2.134,3.687)--(2.165,3.687)--(2.197,3.687)%
  --(2.229,3.687)--(2.261,3.687)--(2.293,3.687)--(2.324,3.687)--(2.356,3.687)--(2.388,3.687)%
  --(2.420,3.687)--(2.452,3.687)--(2.484,3.687)--(2.515,3.687)--(2.547,3.687)--(2.579,3.687)%
  --(2.611,3.687)--(2.643,3.687)--(2.674,3.686)--(2.706,3.686)--(2.738,3.686)--(2.770,3.685)%
  --(2.802,3.684)--(2.834,3.683)--(2.865,3.681)--(2.897,3.679)--(2.929,3.677)--(2.961,3.674)%
  --(2.993,3.670)--(3.025,3.665)--(3.056,3.660)--(3.088,3.653)--(3.120,3.646)--(3.152,3.637)%
  --(3.184,3.627)--(3.215,3.616)--(3.247,3.605)--(3.279,3.592)--(3.311,3.578)--(3.343,3.563)%
  --(3.375,3.547)--(3.406,3.530)--(3.438,3.512)--(3.470,3.494)--(3.502,3.475)--(3.534,3.454)%
  --(3.565,3.434)--(3.597,3.412)--(3.629,3.389)--(3.661,3.366)--(3.693,3.342)--(3.725,3.317)%
  --(3.756,3.291)--(3.788,3.263)--(3.820,3.235)--(3.852,3.206)--(3.884,3.175)--(3.916,3.143)%
  --(3.947,3.109)--(3.979,3.074)--(4.011,3.036)--(4.043,2.997)--(4.075,2.955)--(4.106,2.910)%
  --(4.138,2.863)--(4.170,2.812)--(4.202,2.757)--(4.234,2.697)--(4.266,2.632)--(4.297,2.559)%
  --(4.329,2.479)--(4.361,2.389)--(4.393,2.285)--(4.425,2.165)--(4.456,2.021)--(4.488,1.846)%
  --(4.520,1.629)--(4.552,1.382)--(4.584,1.257)--(4.616,1.427)--(4.647,1.673)--(4.679,1.882)%
  --(4.711,2.050)--(4.743,2.189)--(4.775,2.306)--(4.807,2.407)--(4.838,2.495)--(4.870,2.574)%
  --(4.902,2.644)--(4.934,2.708)--(4.966,2.767)--(4.997,2.822)--(5.029,2.872)--(5.061,2.919)%
  --(5.093,2.963)--(5.125,3.004)--(5.157,3.043)--(5.188,3.080)--(5.220,3.116)--(5.252,3.149)%
  --(5.284,3.181)--(5.316,3.211)--(5.348,3.241)--(5.379,3.269)--(5.411,3.296)--(5.443,3.322)%
  --(5.475,3.347)--(5.507,3.371)--(5.538,3.394)--(5.570,3.417)--(5.602,3.438)--(5.634,3.460)%
  --(5.666,3.480)--(5.698,3.500)--(5.729,3.519)--(5.761,3.538)--(5.793,3.557)--(5.825,3.575)%
  --(5.857,3.592)--(5.888,3.609)--(5.920,3.626)--(5.952,3.642)--(5.984,3.658)--(6.016,3.673)%
  --(6.048,3.688)--(6.079,3.703)--(6.111,3.718)--(6.143,3.732)--(6.175,3.746)--(6.207,3.759)%
  --(6.239,3.773)--(6.270,3.786)--(6.302,3.799)--(6.334,3.811)--(6.366,3.824)--(6.398,3.836)%
  --(6.429,3.848)--(6.461,3.860)--(6.493,3.871)--(6.525,3.883)--(6.557,3.894)--(6.589,3.905)%
  --(6.620,3.916)--(6.652,3.927)--(6.684,3.937)--(6.716,3.948)--(6.748,3.958)--(6.779,3.968)%
  --(6.811,3.978)--(6.843,3.988)--(6.875,3.997)--(6.907,4.007)--(6.939,4.016)--(6.970,4.025)%
  --(7.002,4.035)--(7.034,4.044)--(7.066,4.052)--(7.098,4.061)--(7.130,4.070)--(7.161,4.079)%
  --(7.193,4.087)--(7.225,4.095)--(7.257,4.104)--(7.289,4.112)--(7.320,4.120)--(7.352,4.128)%
  --(7.384,4.136)--(7.416,4.144)--(7.448,4.151)--(7.480,4.159)--(7.511,4.166)--(7.543,4.174)%
  --(7.575,4.181)--(7.607,4.189)--(7.639,4.196)--(7.670,4.203)--(7.702,4.210)--(7.734,4.217)%
  --(7.766,4.224)--(7.798,4.231)--(7.830,4.238)--(7.861,4.244)--(7.893,4.251)--(7.925,4.258)%
  --(7.957,4.264)--(7.989,4.271)--(8.021,4.277)--(8.052,4.283);
\gpcolor{color=gp lt color 3}
\gpsetlinetype{gp lt plot 3}
\draw[gp path] (1.688,3.687)--(1.720,3.687)--(1.752,3.687)--(1.783,3.687)--(1.815,3.687)%
  --(1.847,3.687)--(1.879,3.687)--(1.911,3.687)--(1.943,3.687)--(1.974,3.687)--(2.006,3.687)%
  --(2.038,3.687)--(2.070,3.687)--(2.102,3.687)--(2.134,3.687)--(2.165,3.687)--(2.197,3.687)%
  --(2.229,3.687)--(2.261,3.687)--(2.293,3.687)--(2.324,3.687)--(2.356,3.687)--(2.388,3.687)%
  --(2.420,3.687)--(2.452,3.687)--(2.484,3.687)--(2.515,3.687)--(2.547,3.687)--(2.579,3.687)%
  --(2.611,3.687)--(2.643,3.687)--(2.674,3.687)--(2.706,3.687)--(2.738,3.686)--(2.770,3.686)%
  --(2.802,3.685)--(2.834,3.684)--(2.865,3.682)--(2.897,3.680)--(2.929,3.678)--(2.961,3.675)%
  --(2.993,3.671)--(3.025,3.666)--(3.056,3.661)--(3.088,3.654)--(3.120,3.647)--(3.152,3.638)%
  --(3.184,3.629)--(3.215,3.618)--(3.247,3.606)--(3.279,3.593)--(3.311,3.579)--(3.343,3.565)%
  --(3.375,3.549)--(3.406,3.532)--(3.438,3.514)--(3.470,3.496)--(3.502,3.477)--(3.534,3.457)%
  --(3.565,3.436)--(3.597,3.415)--(3.629,3.392)--(3.661,3.369)--(3.693,3.345)--(3.725,3.320)%
  --(3.756,3.294)--(3.788,3.268)--(3.820,3.240)--(3.852,3.211)--(3.884,3.180)--(3.916,3.149)%
  --(3.947,3.115)--(3.979,3.080)--(4.011,3.044)--(4.043,3.005)--(4.075,2.964)--(4.106,2.920)%
  --(4.138,2.874)--(4.170,2.824)--(4.202,2.770)--(4.234,2.713)--(4.266,2.650)--(4.297,2.580)%
  --(4.329,2.504)--(4.361,2.418)--(4.393,2.321)--(4.425,2.209)--(4.456,2.078)--(4.488,1.922)%
  --(4.520,1.731)--(4.552,1.505)--(4.584,1.311)--(4.616,1.343)--(4.647,1.558)--(4.679,1.777)%
  --(4.711,1.959)--(4.743,2.109)--(4.775,2.236)--(4.807,2.344)--(4.838,2.438)--(4.870,2.521)%
  --(4.902,2.596)--(4.934,2.663)--(4.966,2.725)--(4.997,2.782)--(5.029,2.834)--(5.061,2.883)%
  --(5.093,2.929)--(5.125,2.972)--(5.157,3.013)--(5.188,3.051)--(5.220,3.087)--(5.252,3.122)%
  --(5.284,3.154)--(5.316,3.186)--(5.348,3.216)--(5.379,3.245)--(5.411,3.272)--(5.443,3.299)%
  --(5.475,3.325)--(5.507,3.349)--(5.538,3.373)--(5.570,3.396)--(5.602,3.419)--(5.634,3.440)%
  --(5.666,3.461)--(5.698,3.482)--(5.729,3.501)--(5.761,3.521)--(5.793,3.539)--(5.825,3.558)%
  --(5.857,3.575)--(5.888,3.593)--(5.920,3.610)--(5.952,3.626)--(5.984,3.642)--(6.016,3.658)%
  --(6.048,3.673)--(6.079,3.688)--(6.111,3.703)--(6.143,3.718)--(6.175,3.732)--(6.207,3.746)%
  --(6.239,3.759)--(6.270,3.772)--(6.302,3.786)--(6.334,3.798)--(6.366,3.811)--(6.398,3.823)%
  --(6.429,3.836)--(6.461,3.847)--(6.493,3.859)--(6.525,3.871)--(6.557,3.882)--(6.589,3.893)%
  --(6.620,3.904)--(6.652,3.915)--(6.684,3.926)--(6.716,3.936)--(6.748,3.947)--(6.779,3.957)%
  --(6.811,3.967)--(6.843,3.977)--(6.875,3.987)--(6.907,3.996)--(6.939,4.006)--(6.970,4.015)%
  --(7.002,4.024)--(7.034,4.033)--(7.066,4.042)--(7.098,4.051)--(7.130,4.060)--(7.161,4.069)%
  --(7.193,4.077)--(7.225,4.086)--(7.257,4.094)--(7.289,4.102)--(7.320,4.110)--(7.352,4.119)%
  --(7.384,4.126)--(7.416,4.134)--(7.448,4.142)--(7.480,4.150)--(7.511,4.157)--(7.543,4.165)%
  --(7.575,4.172)--(7.607,4.180)--(7.639,4.187)--(7.670,4.194)--(7.702,4.201)--(7.734,4.209)%
  --(7.766,4.216)--(7.798,4.222)--(7.830,4.229)--(7.861,4.236)--(7.893,4.243)--(7.925,4.250)%
  --(7.957,4.256)--(7.989,4.263)--(8.021,4.269)--(8.052,4.276);
\gpcolor{color=gp lt color 4}
\gpsetlinetype{gp lt plot 4}
\draw[gp path] (1.688,3.687)--(1.720,3.687)--(1.752,3.687)--(1.783,3.687)--(1.815,3.687)%
  --(1.847,3.687)--(1.879,3.687)--(1.911,3.687)--(1.943,3.687)--(1.974,3.687)--(2.006,3.688)%
  --(2.038,3.688)--(2.070,3.688)--(2.102,3.688)--(2.134,3.688)--(2.165,3.688)--(2.197,3.689)%
  --(2.229,3.689)--(2.261,3.689)--(2.293,3.689)--(2.324,3.690)--(2.356,3.690)--(2.388,3.690)%
  --(2.420,3.691)--(2.452,3.691)--(2.484,3.691)--(2.515,3.692)--(2.547,3.692)--(2.579,3.692)%
  --(2.611,3.692)--(2.643,3.693)--(2.674,3.693)--(2.706,3.693)--(2.738,3.693)--(2.770,3.693)%
  --(2.802,3.692)--(2.834,3.692)--(2.865,3.691)--(2.897,3.689)--(2.929,3.687)--(2.961,3.684)%
  --(2.993,3.681)--(3.025,3.677)--(3.056,3.672)--(3.088,3.666)--(3.120,3.659)--(3.152,3.651)%
  --(3.184,3.642)--(3.215,3.632)--(3.247,3.621)--(3.279,3.609)--(3.311,3.596)--(3.343,3.582)%
  --(3.375,3.567)--(3.406,3.551)--(3.438,3.535)--(3.470,3.517)--(3.502,3.499)--(3.534,3.481)%
  --(3.565,3.461)--(3.597,3.441)--(3.629,3.421)--(3.661,3.399)--(3.693,3.377)--(3.725,3.354)%
  --(3.756,3.331)--(3.788,3.306)--(3.820,3.281)--(3.852,3.255)--(3.884,3.228)--(3.916,3.200)%
  --(3.947,3.171)--(3.979,3.141)--(4.011,3.109)--(4.043,3.076)--(4.075,3.042)--(4.106,3.006)%
  --(4.138,2.968)--(4.170,2.928)--(4.202,2.886)--(4.234,2.842)--(4.266,2.795)--(4.297,2.745)%
  --(4.329,2.692)--(4.361,2.635)--(4.393,2.574)--(4.425,2.508)--(4.456,2.436)--(4.488,2.358)%
  --(4.520,2.272)--(4.552,2.177)--(4.584,2.073)--(4.616,1.959)--(4.647,1.837)--(4.679,1.713)%
  --(4.711,1.606)--(4.743,1.551)--(4.775,1.575)--(4.807,1.664)--(4.838,1.783)--(4.870,1.905)%
  --(4.902,2.021)--(4.934,2.127)--(4.966,2.224)--(4.997,2.311)--(5.029,2.390)--(5.061,2.463)%
  --(5.093,2.529)--(5.125,2.591)--(5.157,2.648)--(5.188,2.701)--(5.220,2.750)--(5.252,2.797)%
  --(5.284,2.841)--(5.316,2.883)--(5.348,2.922)--(5.379,2.959)--(5.411,2.995)--(5.443,3.029)%
  --(5.475,3.062)--(5.507,3.093)--(5.538,3.122)--(5.570,3.151)--(5.602,3.179)--(5.634,3.205)%
  --(5.666,3.231)--(5.698,3.256)--(5.729,3.280)--(5.761,3.303)--(5.793,3.325)--(5.825,3.347)%
  --(5.857,3.368)--(5.888,3.389)--(5.920,3.409)--(5.952,3.428)--(5.984,3.447)--(6.016,3.466)%
  --(6.048,3.484)--(6.079,3.501)--(6.111,3.518)--(6.143,3.535)--(6.175,3.551)--(6.207,3.567)%
  --(6.239,3.583)--(6.270,3.598)--(6.302,3.613)--(6.334,3.628)--(6.366,3.642)--(6.398,3.656)%
  --(6.429,3.670)--(6.461,3.684)--(6.493,3.697)--(6.525,3.710)--(6.557,3.723)--(6.589,3.736)%
  --(6.620,3.748)--(6.652,3.760)--(6.684,3.772)--(6.716,3.784)--(6.748,3.795)--(6.779,3.807)%
  --(6.811,3.818)--(6.843,3.829)--(6.875,3.840)--(6.907,3.850)--(6.939,3.861)--(6.970,3.871)%
  --(7.002,3.881)--(7.034,3.892)--(7.066,3.901)--(7.098,3.911)--(7.130,3.921)--(7.161,3.930)%
  --(7.193,3.940)--(7.225,3.949)--(7.257,3.958)--(7.289,3.967)--(7.320,3.976)--(7.352,3.985)%
  --(7.384,3.994)--(7.416,4.002)--(7.448,4.011)--(7.480,4.019)--(7.511,4.027)--(7.543,4.036)%
  --(7.575,4.044)--(7.607,4.052)--(7.639,4.060)--(7.670,4.068)--(7.702,4.075)--(7.734,4.083)%
  --(7.766,4.091)--(7.798,4.098)--(7.830,4.105)--(7.861,4.113)--(7.893,4.120)--(7.925,4.127)%
  --(7.957,4.134)--(7.989,4.141)--(8.021,4.148)--(8.052,4.155);
\gpcolor{color=gp lt color border}
\gpsetlinetype{gp lt border}
\draw[gp path] (1.688,5.756)--(1.688,0.985)--(8.197,0.985)--(8.197,5.756)--cycle;
%% coordinates of the plot area
\gpdefrectangularnode{gp plot 1}{\pgfpoint{1.688cm}{0.985cm}}{\pgfpoint{8.197cm}{5.756cm}}
%% \end{tikzpicture}
%% gnuplot variables

        \node at (6.9,2.6) {$N=10^7$};
        \node at (6,2) {$N=10^6$};
      \end{tikzpicture}
    }
  }
  \caption[AG model simulated focal behavior of perfect magnetic lenses as a function of pulse charge $N$]{
    AG model simulated focal behavior of perfect magnetic lenses as a function of pulse charge $N$.
    Each plot's pulses begin at the lens' front focal plane at 20 kV velocity ($c/3$) and have the same volume.
    The left plots (\subref{fig:focus_long_oblate} and \subref{fig:focus_short_oblate}) are initially oblate shaped, HW1/eM width $ 500 \mu \text{m}$ and ellipticiy $ \xi ( 0 ) = 0.1 $.
    The right plots (\subref{fig:focus_long_prolate} and \subref{fig:focus_short_prolate}) are initially prolate shaped, HW1/eM width $ 107.7 \mu \text{m}$ and ellipticiy $ \xi ( 0 ) = 10 $.
    Distance traveled in column is measured relative to the lens at $z=0$, in units of the focal lenth.
    The top plots (\subref{fig:focus_long_oblate} and \subref{fig:focus_long_prolate}) have a longer focal length $f = $ 60mm compared to the bottom plots (\subref{fig:focus_short_oblate} and \subref{fig:focus_short_prolate}) $ f = $ 6.0mm.
    A logarithmic scale is used for comparative clarity near the foci.
    Clearly the best performance is achieved for shorter focal lengths and oblate pulses \subref{fig:focus_short_oblate} even at higher initial charge densities.
  }
  \label{fig:focus_lens_charge}
\end{figure}
