%This work is licensed under the Creative Commons Attribution-NonCommercial-NoDerivs 3.0 United States License. To view a copy of this license, visit http://creativecommons.org/licenses/by-nc-nd/3.0/us/ or send a letter to Creative Commons, 444 Castro Street, Suite 900, Mountain View, California, 94041, USA.

In Chapters \ref{chap:previous_models} and \ref{chap:extension}, I have introduced the extended Analytic Gaussian model.
In this chapter, I use this model to make general observations about pulse propagation for Ultrafast Electron Microscopy.

The examples presented will each use one of two initialization schemes.
The first represents an idealized pulse which is not generated by photocathode and then accelerated, but rather springs into existence at its initial position with its velocity already established.
This type of pulse is useful to demonstrate general pulse dynamics or the ideal influence of electron-optical elements independent of the pulse's creation conditions.
Simulations using pulses of this type will provide an initial velocity $v_{\smallzero}$ (or equivalent acceleration voltage $V_{DC}$), an excess photoemission energy $\Delta E$, and HW1/eM dimensions; width $w_{\smallT}$ and duration $\tau$ or length $ w_z = v_{\smallzero} \tau$.
The width and length will be used directly to calculate initial spatial variances $\sigma_{\alpha}$.
Both momentum chirp terms, $\gamma_{\alpha}$, will be initially zero, but since pulses must still be subject to Liouville's Theorem (discussed in Section \ref{sec:liouville}), \ref{eq:summary} will still be used to calculate meaningful local momentum widths $\eta_{\alpha}$.
This idealization only applies to the initialization, therefore these pulses will still be subject to intrinsic pulse broadening and internal space-charge repulsion which might cause pulse expansion before or during the demonstration, especially for high charge densities.

Other examples in this section will require --- or are markedly different when using --- a more realistic pulse resulting from photoemission and subsequent acceleration.
Simulations using pulses of this type will provide and HW1/eM width and duration of the laser, the length and voltage of the acceleration region, and an excess photoemission energy.
In these cases, all initial parameters will be calculated from \ref{eq:summary}.
The acceleration process, and more specifically the anode (see Section \ref{sec:gun_model}), will have an impact on the beam dynamics which will be evident in the pulse's evolution.

