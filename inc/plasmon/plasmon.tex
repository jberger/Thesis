%This work is licensed under the Creative Commons Attribution-NonCommercial-NoDerivs 3.0 United States License. To view a copy of this license, visit http://creativecommons.org/licenses/by-nc-nd/3.0/us/ or send a letter to Creative Commons, 444 Castro Street, Suite 900, Mountain View, California, 94041, USA.

\section{Introduction}

In the search for alternative mechanisms to reduce the intrinsic divergence of the electron beam upon photoemission, a possible area of interest was Plasmon-Assisted Photoemission (PAPE). 
This avenue of investigation was motivated by \fixme{cite} in which authors \fixme{authors} indicate that when driving a surface plasmon on gold they witnessed a reduction in the emission angle of photoemitted electrons.
In their geometry (called the Kretschmann geometry) the plasmon is excited by a laser through a glass substrate, however due to fears about heat transportation, it was decided to use the ``grating coupling'' geometry.

\subsection{Driving the Plasmon}

Every material has an intrinsic plasma frequency given by 
\begin{equation}
  \omega_{P} = \sqrt{\frac{n e^2}{\varepsilon_{0} m^*}}
\end{equation}
In order to drive a plasmon on the surface of a material, the portion of the laser wave-vector parallel to the plasmon ($\vec{k}_{laser,x}$) must be matched to the wave-vector of the surface plasmon.
