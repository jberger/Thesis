%This work is licensed under the Creative Commons Attribution-NonCommercial-NoDerivs 3.0 United States License. To view a copy of this license, visit http://creativecommons.org/licenses/by-nc-nd/3.0/us/ or send a letter to Creative Commons, 444 Castro Street, Suite 900, Mountain View, California, 94041, USA.

\documentclass{uicthesi}
\usepackage{amsmath,amsfonts,amssymb}
\usepackage{esint}
\usepackage{hyperref}
\usepackage{url}
\usepackage{ifthen}
\usepackage{pgffor}

%\usepackage{euler}
%\usepackage{breqn}

\usepackage[version=3]{mhchem}

\usepackage{import}
\usepackage{subfig}

\usepackage{graphicx}
\graphicspath{{./images/}}
\DeclareGraphicsExtensions{.png,.pdf,.jpg}

\usepackage[noexternal]{mytikz}

\usepackage{fancyvrb}
\input{pygments}

%% My convenience fuctions
\providecommand{\vecprime}[1]{\vec{#1}\,^{\prime}}

\providecommand{\smallT}{\ensuremath{{\scriptscriptstyle T}}}
\providecommand{\smallD}{\ensuremath{{\scriptscriptstyle D}}}
\providecommand{\smallzero}{\ensuremath{{\scriptscriptstyle 0}}}
\providecommand{\smallF}[0]{ { \scriptscriptstyle F } }
%\providecommand{\smallZ}{\ensuremath{{\scriptscriptstyle Z}}}

\providecommand{\punct}[1]{\ensuremath{\, \text{#1}}}

\providecommand{\abbr}[2]{\item[#1\hfill] #2}

%\usepackage{color}

%% Special handling for using biblatex vs bibtex ==
\newboolean{usebiblatex}
\setboolean{usebiblatex}{false}

\ifthenelse{\boolean{usebiblatex}}{
  \usepackage{biblatex}
  %\usepackage[backend=biber]{biblatex-chicago}
  \addbibresource{bibliography.bib}
  \addbibresource{biblaser.bib}
  \addbibresource{bibeste.bib}
}{}
%% ==

\title{Ultrafast Electron Microscopes: Design Criteria, Electron Sources, and Column Modeling}
\author{Joel Berger}
\pdegrees{B.S., University of Illinois at Chicago, 2005}
\degree{Doctor of Philosophy in Physics}
 
\begin{document}
\maketitle

%\copyrightpage
\newpage
\begin{figure}
  \centering
  \includegraphics{by-nc-nd}
\end{figure}
This work is licensed under the Creative Commons Attribution-NonCommercial-NoDerivs 3.0 United States License.
To view a copy of this license, visit \url{http://creativecommons.org/licenses/by-nc-nd/3.0/us/} or send a letter to
\begin{verbatim}
Creative Commons
444 Castro Street, Suite 900
Mountain View, CA, 94041, USA.
\end{verbatim}


\dedication
This thesis is dedicated to my parents, who in love and infinite patience always encouraged me, no matter how many questions I asked; and to my amazing wife, who has believed in me since the day we met. 
 
\acknowledgment
I would like to thank my advisor, W. Andreas Schroeder, and all of my committee members, Nestor Zaluzec, %of the Electron Microscopy Center at Argonne National Laboratory
Alan Nicholls, % of Research Resources Center at UIC
Christoph Grein and Robert Klie.
It has been my honor to learn from and work with each one of you.

I would like to thank all of my research group members, past and present, Stephanie Schieffer, Benjamin Rickman, John Hogan, Tuo Li, Daniel Brajkovic, Michael Greco, Michael Szotek, and Robin Taylor.
Through your professional efforts and personal interactions, you have made my time at UIC most enjoyable.
I also want to thank former member Nathan Rimington for laying the groundwork for the presented laser.

I want to thank my collaborators, Nigel Browning, now of Pacific Nortwest National Laboratory, and Dean Miller and Jon Hiller of the Electron Microscopy Center at Argonne National Laboratory.

Finally, I want to thank the funding agencies which have made this research possible.
It has been supported by: %TODO list grants

\initials{JAB}
 
%\preface
%The preface.
 
\tableofcontents
%\listoftables
\listoffigures
 
\listofabbreviations
\begin{list}
  {}
  {\setlength{\labelwidth}{1in}
   \setlength{\leftmargin}{1.5in}
   \setlength{\labelsep}{.5in}
   \setlength{\rightmargin}{\leftmargin}}

  \abbr{AG}{Analytic Gaussian (Model)}
  \abbr{AR}{Anti-Reflection}
  \abbr{BBO}{$\beta$-Barium Borate}
  \abbr{CF}{ConFlat}
  \abbr{DC}{Direct Current (i.e. Static Electric Field)}
  \abbr{DTEM}{Dynamic Transmission Electron Microscopy}
  \abbr{ESTE}{Excited State Thermionic Emission}
  \abbr{FIB}{Focused Ion Beam}
  \abbr{FWHM}{Full-Width Half Maximum}
  \abbr{GRIN}{Gradient Index (of Refraction)}
  \abbr{GVD}{Group Velocity Dispersion}
  \abbr{HW1/eM}{Half-Width 1/e Maximum}
  \abbr{KGW}{Potassium Gadolinium Tungstate}
  \abbr{KYW}{Potassium Yttrium Tungstate}
  \abbr{LBO}{Lithium Triborate}
  \abbr{LHS}{Left Hand Side}
  \abbr{ML}{Mode-Locked}
  \abbr{PAPE}{Plasmon-Assisted Photoemission}
  \abbr{Q}{Quality Factor}
  \abbr{RHS}{Right Hand Side}
  \abbr{RF}{Radio Frequency}
  \abbr{SBR}{Saturable Bragg Reflector}
  \abbr{TLS}{Thermal Lens Shaping/Shaped}
  \abbr{TEM}{Transmission Electron Microscopy}
  \abbr{TEM$_{00}$}{Tranvserse ElectroMagnetic 00 Mode}
  \abbr{TM$_{010}$}{Tranvserse Magnetic 010 Mode}
  \abbr{UED}{Ultrafast Electron Diffraction}
  \abbr{UEM}{Ultrafast Electron Microscope}
  \abbr{UIC}{University of Illinois at Chicago}
  \abbr{UV}{Ultraviolet}
\end{list}

 
\summary
Dynamic Transmission Electron Microscopy, and its picosecond/femtosecond subclass Ultrafast Electron Microscopy, is an emerging field in intrumentation science.
It attempts to combine the nanoscale spatial resolution of transmission electron microscopes with the temporal resolution of modern ultrafast lasers.
In this thesis I present my contributions to this young field. 
These include a novel model for simulating the dynamics os ultrafast electron pulses in electron microscope systems,
design critera for constructing such a system, and theoretical and experimental groundwork geared towards selecting a useful photocathode for electron generation.
I also present the prototype ultrafast electron microscope being built at UIC.

\chapter{Introduction}

  \subimport{inc/}{introduction}

\chapter{The Analytic Gaussian Model} \label{chap:ms_model}

  \subimport{inc/ms_model/}{ms_model}

\chapter{Extension of the Analytic Gaussian Model\\for UEM Column Modeling} \label{chap:extension}

  \subimport{inc/extension/}{introduction}

  \subimport{inc/extension/initial_conditions/}{initial_conditions}

  \subimport{inc/extension/liouvilles_theorem/}{liouvilles_theorem}

  \subimport{inc/extension/external_forces/}{formal}

  \subimport{inc/extension/external_forces/}{practical}

  \subimport{inc/gun/}{gun_model}

  \subimport{inc/extension/implementation/}{implementation}

\chapter{Model Results, Predictions and Implications} \label{chap:model_results}

  \subimport{inc/model_results/}{introduction}

  \subimport{inc/model_results/shapes/}{shapes}

  \subimport{inc/model_results/spacecharge/}{spacecharge}

  \subimport{inc/model_results/lens_charge/}{lens_charge}

  \subimport{inc/model_results/rf_cav/}{rf_cav}

  \subimport{inc/model_results/compound/}{compound}

\chapter{UEM Design Considerations} \label{chap:considerations}

  \subimport{inc/design_considerations/}{considerations}

\chapter{Prototype Instrument} \label{chap:prototype}

  \subimport{inc/hardware/}{introduction}

  \subimport{inc/hardware/laser/}{laser}

  \subimport{inc/hardware/}{column}

  \subimport{inc/gun/}{gun_design}

  \subimport{inc/hardware/magnetic_lenses/}{maglens}

  \subimport{inc/hardware/deflector/}{deflector}

\chapter{Photocathode Engineering} \label{chap:photocathode}

  \subimport{inc/photocathode/setup/}{setup}

  \subimport{inc/photocathode/single_photon/}{single_photon}

  \subimport{inc/photocathode/two_photon_thermionic/}{two_photon_thermionic}

  \subimport{inc/photocathode/plasmon/}{plasmon}

  \subimport{inc/photocathode/este/}{este}

  \subimport{inc/photocathode/}{future}

\chapter{Conclusions}

  \subimport{inc/}{conclusions}

%uncomment when multiple appedicies are used 
%\appendices
\appendix
  \subimport{inc/appendix/}{script}

\newpage

%% Special handling for using biblatex vs bibtex ==
\ifthenelse{\boolean{usebiblatex}}{
  \printbibliography
}{
  \bibformc
  \bibliography{bibliography,biblaser,bibeste}
}
%% ==

%\vita
%This is where the vita goes.  Its organization is left as an exercise.
%Hint: see the list of abbreviations.
\end{document}
