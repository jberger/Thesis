%This work is licensed under the Creative Commons Attribution-NonCommercial-NoDerivs 3.0 United States License. To view a copy of this license, visit http://creativecommons.org/licenses/by-nc-nd/3.0/us/ or send a letter to Creative Commons, 444 Castro Street, Suite 900, Mountain View, California, 94041, USA.

\documentclass{uicthesi}
\usepackage{amsmath,amsfonts,amssymb}
\usepackage{esint}
\usepackage{hyperref}
\usepackage{url}
\usepackage{ifthen}
\usepackage{pgffor}

%\usepackage{euler}
%\usepackage{breqn}

\usepackage[version=3]{mhchem}

\usepackage{import}
\usepackage{subfig}

\usepackage{graphicx}
\graphicspath{{./images/}}
\DeclareGraphicsExtensions{.png,.pdf,.jpg}
\usepackage{tikz}
\usetikzlibrary{calc,positioning,intersections,arrows,shapes.geometric,shapes.arrows}

%% handle externalizing images, probably makes problems for subimport
%\usetikzlibrary{external}
%\tikzexternalize
%\tikzsetexternalprefix{figure_cache/}

\pgfdeclarelayer{background}
\pgfsetlayers{background,main}

\usepackage{gnuplot-lua-tikz}

% when inputdata is true, show full figure, otherwise don't include data which takes a long time
\newboolean{inputdata}
\setboolean{inputdata}{false}

\providecommand{\inputdata}[1]{
  \ifthenelse{\boolean{inputdata}}{
    \input{#1}
  }{
    \node{Content hidden (#1)};
  }
}

\setboolean{inputdata}{true}
\usepackage{fancyvrb}
\input{pygments}

%% My convenience fuctions
\providecommand{\vecprime}[1]{\vec{#1}\,^{\prime}}

\providecommand{\smallT}{\ensuremath{{\scriptscriptstyle T}}}
\providecommand{\smallD}{\ensuremath{{\scriptscriptstyle D}}}
\providecommand{\smallzero}{\ensuremath{{\scriptscriptstyle 0}}}
\providecommand{\smallF}[0]{ { \scriptscriptstyle F } }
%\providecommand{\smallZ}{\ensuremath{{\scriptscriptstyle Z}}}

\providecommand{\punct}[1]{\ensuremath{\, \text{#1}}}

\providecommand{\abbr}[2]{\item[#1\hfill] #2}

\usepackage{color}
\providecommand{\fixme}[1]{\textcolor{red}{#1}}

%% Special handling for using biblatex vs bibtex ==
\newboolean{usebiblatex}
\setboolean{usebiblatex}{false}

\ifthenelse{\boolean{usebiblatex}}{
  \usepackage{biblatex}
  %\usepackage[backend=biber]{biblatex-chicago}
  \addbibresource{bibliography.bib}
  \addbibresource{biblaser.bib}
  \addbibresource{bibeste.bib}
}{}
%% ==

\title{Title}
\author{Joel Berger}
\pdegrees{of the\\University of Illinois at Chicago}
\degree{Doctor of Philosophy in Physics}
 
\begin{document}
\maketitle

%\copyrightpage
\newpage
\begin{figure}
  \centering
  \includegraphics{by-nc-nd}
\end{figure}
This work is licensed under the Creative Commons Attribution-NonCommercial-NoDerivs 3.0 United States License.
To view a copy of this license, visit \url{http://creativecommons.org/licenses/by-nc-nd/3.0/us/} or send a letter to
\begin{verbatim}
Creative Commons
444 Castro Street, Suite 900
Mountain View, CA, 94041, USA.
\end{verbatim}


\dedication
The dedication.
 
\acknowledgment
The acknowledgment.
\initials{JAB}
 
\preface
The preface.
 
\tableofcontents
\listoftables
\listoffigures
 
\listofabbreviations
\begin{list}
  {}
  {\setlength{\labelwidth}{1in}
   \setlength{\leftmargin}{1.5in}
   \setlength{\labelsep}{.5in}
   \setlength{\rightmargin}{\leftmargin}}

  \abbr{AG}{Analytic Gaussian (Model)}
  \abbr{AR}{Anti-Reflection}
  \abbr{BBO}{$\beta$-Barium Borate}
  \abbr{CF}{ConFlat}
  \abbr{DC}{Direct Current (i.e. Static Electric Field)}
  \abbr{DTEM}{Dynamic Transmission Electron Microscopy}
  \abbr{ESTE}{Excited State Thermionic Emission}
  \abbr{FIB}{Focused Ion Beam}
  \abbr{FWHM}{Full-Width Half Maximum}
  \abbr{GRIN}{Gradient Index (of Refraction)}
  \abbr{GVD}{Group Velocity Dispersion}
  \abbr{HW1/eM}{Half-Width 1/e Maximum}
  \abbr{KGW}{Potassium Gadolinium Tungstate}
  \abbr{KYW}{Potassium Yttrium Tungstate}
  \abbr{LBO}{Lithium Triborate}
  \abbr{LHS}{Left Hand Side}
  \abbr{ML}{Mode-Locked}
  \abbr{PAPE}{Plasmon-Assisted Photoemission}
  \abbr{Q}{Quality Factor}
  \abbr{RHS}{Right Hand Side}
  \abbr{RF}{Radio Frequency}
  \abbr{SBR}{Saturable Bragg Reflector}
  \abbr{TLS}{Thermal Lens Shaping/Shaped}
  \abbr{TEM}{Transmission Electron Microscopy}
  \abbr{TEM$_{00}$}{Tranvserse ElectroMagnetic 00 Mode}
  \abbr{TM$_{010}$}{Tranvserse Magnetic 010 Mode}
  \abbr{UED}{Ultrafast Electron Diffraction}
  \abbr{UEM}{Ultrafast Electron Microscope}
  \abbr{UIC}{University of Illinois at Chicago}
  \abbr{UV}{Ultraviolet}
\end{list}

 
\summary
A summary is required.

\chapter{Introduction}

\chapter{Review of Foundational Modeling Efforts} \label{chap:previous_models}

  \subimport{inc/ms_model/}{ms_model}

\chapter{Extension of the Analytic Gaussian Model\\for UEM Column Modeling} \label{chap:extension}

Modeling a system as complex as an Ultrafast Electron Microscope is an enourmous task.
In Chapter \ref{chap:previous_models}, I have highlighted some previous efforts at modeling electron pulse evolution.
While these are excellent works in their own right, none provide the complete picture of the evolution of an electron pulse through the entire column.
In the following chapter, I will present a model which builds on and extends those individual pieces to form a single model; one which is generic enough to model many different columns and is computationally fast enough to be used to design and optimize a column.

  \subimport{inc/extension/initial_conditions/}{initial_conditions}

  \subimport{inc/extension/liouvilles_theorem/}{liouvilles_theorem}

  %TODO refer from here to Free expansion results

  \subimport{inc/extension/external_forces/}{formal}

  \subimport{inc/extension/external_forces/}{practical}

  \subimport{inc/gun/}{gun_model}

  \subimport{inc/extension/implementation/}{implementation}

\chapter{Model Results, Predictions and Implications} \label{chap:model_results}

In Chapters \ref{chap:previous_models} and \ref{chap:extension}, I have introduced the extended Analytic Gaussian model.
In the following chapter I use this model to make general observations about pulse propagation for Ultrafast Electron Microscopy.

\section{Free Expansion Propagation}

  \subimport{inc/model_results/shapes/}{shapes}

  \subimport{inc/model_results/spacecharge/}{spacecharge}

\section{Propagtion under External Forces}

  \subimport{inc/model_results/lens_charge/}{lens_charge}

\chapter{UEM Design Considerations}

  \subimport{inc/design_considerations/}{considerations}

\chapter{Prototype Instrument}

  \subimport{inc/hardware/}{introduction}

  \subimport{inc/hardware/laser/}{laser}

  \subimport{inc/hardware/}{column}

  \subimport{inc/gun/}{gun_design}

  \subimport{inc/hardware/magnetic_lenses/}{maglens}

  \subimport{inc/hardware/deflector/}{deflector}

\chapter{Photocathode Engineering}

  \subimport{inc/photocathode/setup/}{setup}

\section{Flat Metal Photocathodes}

  \subimport{inc/photocathode/plasmon/}{plasmon}

  \subimport{inc/photocathode/este/}{este}

%uncomment when multiple appedicies are used 
%\appendices
\appendix
  \subimport{inc/extension/implementation/}{appendix-script}

\newpage

%% Special handling for using biblatex vs bibtex ==
\ifthenelse{\boolean{usebiblatex}}{
  \printbibliography
}{
  \bibformc
  \bibliography{bibliography,biblaser,bibeste}
}
%% ==

%\vita
%This is where the vita goes.  Its organization is left as an exercise.
%Hint: see the list of abbreviations.
\end{document}
